\documentclass{article}
\usepackage[utf8]{inputenc}

\usepackage{mathtools}
\usepackage{algpseudocode}
\usepackage{amsfonts}
\usepackage{amsmath}
\usepackage{amssymb}
\usepackage{amsthm}
\usepackage{bm}
\usepackage{listings}
\usepackage{float}
\usepackage{fancyvrb}
\usepackage{xcolor}
\usepackage{tikz-cd}

\definecolor{maccolor}{rgb}{0.3,0.3,0.8}
\newcommand\macoutput[1]{{\tt [Macaulay2 output o#1]}}%placeholder to compile w/o running M2
\lstdefinelanguage{Macaulay2}
{
basicstyle={\ttfamily},
keywordstyle={\color{maccolor!80!black}},
commentstyle={\color{gray}},
stringstyle={\color{red!40!black}},
rulecolor=\color{maccolor},
basewidth={1.2ex}, %workaround for prompts being same width as normal tt text
sensitive=false,
morecomment=[l]{--},
morecomment=[s]{-*}{*-},
morestring=[b]",
escapechar={`},
escapebegin={\rmfamily},
morekeywords={about,abs,AbstractToricVarieties,accumulate,Acknowledgement,acos,acosh,acot,addCancelTask,addDependencyTask,addEndFunction,addHook,AdditionalPaths,addStartFunction,addStartTask,Adjacent,adjoint,AdjointIdeal,AffineVariety,AfterEval,AfterNoPrint,AfterPrint,agm,AInfinity,alarm,AlgebraicSplines,Algorithm,Alignment,all,AllCodimensions,allowableThreads,ambient,analyticSpread,Analyzer,AnalyzeSheafOnP1,ancestor,ancestors,ANCHOR,and,andP,AngleBarList,ann,annihilator,antipode,any,append,applicationDirectory,applicationDirectorySuffix,apply,applyKeys,applyPairs,applyTable,applyValues,apropos,argument,Array,arXiv,Ascending,ascii,asin,asinh,ass,assert,associatedGradedRing,associatedPrimes,AssociativeAlgebras,AssociativeExpression,atan,atan2,atEndOfFile,Authors,autoload,AuxiliaryFiles,backtrace,Bag,Bareiss,baseFilename,BaseFunction,baseName,baseRing,baseRings,BaseRow,BasicList,basis,BasisElementLimit,Bayer,BeforePrint,beginDocumentation,BeginningMacaulay2,Benchmark,benchmark,Bertini,BesselJ,BesselY,betti,BettiCharacters,BettiTally,between,BGG,BIBasis,Binary,BinaryOperation,Binomial,binomial,BinomialEdgeIdeals,Binomials,BKZ,BlockMatrix,BLOCKQUOTE,BODY,Body,BoijSoederberg,BOLD,Book3264Examples,Boolean,BooleanGB,borel,Boxes,BR,break,Browse,Bruns,cache,CacheExampleOutput,CacheFunction,CacheTable,cacheValue,CallLimit,cancelTask,capture,catch,Caveat,CC,CDATA,ceiling,Center,centerString,Certification,ChainComplex,chainComplex,ChainComplexExtras,ChainComplexMap,ChainComplexOperations,ChangeMatrix,char,CharacteristicClasses,characters,charAnalyzer,check,CheckDocumentation,chi,Chordal,class,Classic,clean,clearAll,clearEcho,clearOutput,close,closeIn,closeOut,ClosestFit,CODE,code,codim,CodimensionLimit,coefficient,CoefficientRing,coefficientRing,coefficients,Cofactor,CohenEngine,CohenTopLevel,CoherentSheaf,CohomCalg,cohomology,coimage,CoincidentRootLoci,coker,cokernel,collectGarbage,columnAdd,columnate,columnMult,columnPermute,columnRankProfile,columnSwap,combine,Command,commandInterpreter,commandLine,COMMENT,commonest,commonRing,comodule,CompactMatrix,compactMatrixForm,CompiledFunction,CompiledFunctionBody,CompiledFunctionClosure,Complement,complement,complete,CompleteIntersection,CompleteIntersectionResolutions,Complexes,ComplexField,components,compose,compositions,compress,concatenate,conductor,ConductorElement,cone,Configuration,ConformalBlocks,conjugate,connectionCount,Consequences,Constant,Constants,constParser,content,continue,contract,Contributors,ConvexInterface,conwayPolynomial,ConwayPolynomials,copy,copyDirectory,copyFile,copyright,Core,CorrespondenceScrolls,cos,cosh,cot,CotangentSchubert,cotangentSheaf,coth,cover,coverMap,cpuTime,createTask,Cremona,csc,csch,current,currentColumnNumber,currentDirectory,currentFileDirectory,currentFileName,currentLayout,currentLineNumber,currentPackage,currentString,currentTime,Cyclotomic,Database,Date,DD,dd,deadParser,debug,debugError,DebuggingMode,debuggingMode,debugLevel,DecomposableSparseSystems,Decompose,decompose,deepSplice,Default,default,defaultPrecision,Degree,degree,degreeLength,DegreeLift,DegreeLimit,DegreeMap,DegreeOrder,DegreeRank,Degrees,degrees,degreesMonoid,degreesRing,delete,demark,denominator,Dense,Density,Depth,depth,Descending,Descent,Describe,describe,Description,det,determinant,DeterminantalRepresentations,DGAlgebras,diagonalMatrix,diameter,Dictionary,dictionary,dictionaryPath,diff,DiffAlg,difference,dim,directSum,disassemble,discriminant,dismiss,Dispatch,distinguished,DIV,Divide,divideByVariable,DivideConquer,DividedPowers,Divisor,DL,Dmodules,do,doc,docExample,docTemplate,document,DocumentTag,Down,drop,DT,dual,eagonNorthcott,EagonResolution,echoOff,echoOn,EdgeIdeals,edit,EigenSolver,eigenvalues,eigenvectors,eint,EisenbudHunekeVasconcelos,elapsedTime,elapsedTiming,elements,Eliminate,eliminate,Elimination,EliminationMatrices,EllipticCurves,EllipticIntegrals,else,EM,Email,End,end,endl,endPackage,Engine,engineDebugLevel,EngineRing,EngineTests,entries,EnumerationCurves,environment,Equation,EquivariantGB,erase,erf,erfc,error,errorDepth,euler,EulerConstant,eulers,even,EXAMPLE,ExampleFiles,ExampleItem,examples,ExampleSystems,Exclude,exec,exit,exp,expectedReesIdeal,expm1,exponents,export,exportFrom,exportMutable,Expression,expression,Ext,extend,ExteriorIdeals,ExteriorModules,exteriorPower,Factor,factor,false,Fano,FastMinors,FastNonminimal,FGLM,File,fileDictionaries,fileExecutable,fileExists,fileExitHooks,fileLength,fileMode,FileName,FilePosition,fileReadable,fileTime,fileWritable,fillMatrix,findFiles,findHeft,FindOne,findProgram,findSynonyms,FiniteFittingIdeals,First,first,firstkey,FirstPackage,fittingIdeal,flagLookup,FlatMonoid,flatten,flattenRing,Flexible,flip,floor,flush,fold,FollowLinks,for,forceGB,fork,FormalGroupLaws,Format,format,formation,FourierMotzkin,FourTiTwo,fpLLL,frac,fraction,FractionField,frames,FrobeniusThresholds,from,fromDividedPowers,fromDual,Function,FunctionApplication,FunctionBody,functionBody,FunctionClosure,FunctionFieldDesingularization,fusePairs,futureParser,GaloisField,Gamma,gb,GBDegrees,gbRemove,gbSnapshot,gbTrace,gcd,gcdCoefficients,gcdLLL,GCstats,genera,GeneralOrderedMonoid,GenerateAssertions,generateAssertions,generator,generators,Generic,GenericInitialIdeal,genericMatrix,genericSkewMatrix,genericSymmetricMatrix,gens,genus,get,getc,getChangeMatrix,getenv,getGlobalSymbol,getNetFile,getNonUnit,getPrimeWithRootOfUnity,getSymbol,getWWW,GF,gfanInterface,Givens,GKMVarieties,GLex,Global,global,globalAssign,globalAssignFunction,GlobalAssignHook,globalAssignment,globalAssignmentHooks,GlobalDictionary,GlobalHookStore,globalReleaseFunction,GlobalReleaseHook,Gorenstein,GradedLieAlgebras,GradedModule,gradedModule,GradedModuleMap,gradedModuleMap,gramm,GraphicalModels,GraphicalModelsMLE,Graphics,graphIdeal,graphRing,Graphs,Grassmannian,GRevLex,GroebnerBasis,groebnerBasis,GroebnerBasisOptions,GroebnerStrata,GroebnerWalk,groupID,GroupLex,GroupRevLex,GTZ,Hadamard,handleInterrupts,HardDegreeLimit,hash,HashTable,hashTable,HEAD,HEADER1,HEADER2,HEADER3,HEADER4,HEADER5,HEADER6,HeaderType,Heading,Headline,Heft,heft,Height,height,help,Hermite,hermite,Hermitian,HH,hh,HigherCIOperators,HighestWeights,Hilbert,hilbertFunction,hilbertPolynomial,hilbertSeries,HodgeIntegrals,hold,Holder,Hom,homeDirectory,HomePage,Homogeneous,Homogeneous2,homogenize,homology,homomorphism,HomotopyLieAlgebra,hooks,horizontalJoin,HorizontalSpace,HR,HREF,HTML,html,httpHeaders,Hybrid,HyperplaneArrangements,Hypertext,hypertext,HypertextContainer,HypertextParagraph,icFracP,icFractions,icMap,icPIdeal,id,Ideal,ideal,idealizer,identity,if,IgnoreExampleErrors,ii,image,imaginaryPart,IMG,ImmutableType,importFrom,in,incomparable,Increment,independentSets,indeterminate,IndeterminateNumber,Index,index,indexComponents,IndexedVariable,IndexedVariableTable,indices,inducedMap,inducesWellDefinedMap,InexactField,InexactFieldFamily,InexactNumber,InfiniteNumber,infinity,info,InfoDirSection,infoHelp,Inhomogeneous,input,Inputs,insert,installAssignmentMethod,installedPackages,installHilbertFunction,installMethod,installMinprimes,installPackage,InstallPrefix,instance,instances,IntegralClosure,integralClosure,integrate,IntermediateMarkUpType,interpreterDepth,intersect,intersectInP,Intersection,intersection,interval,InvariantRing,inverse,InverseMethod,inversePermutation,Inverses,inverseSystem,InverseSystems,Invertible,InvolutiveBases,irreducibleCharacteristicSeries,irreducibleDecomposition,isAffineRing,isANumber,isBorel,isCanceled,isCommutative,isConstant,isDirectory,isDirectSum,isEmpty,isField,isFinite,isFinitePrimeField,isFreeModule,isGlobalSymbol,isHomogeneous,isIdeal,isInfinite,isInjective,isInputFile,isIsomorphism,isLinearType,isListener,isLLL,isMember,isModule,isMonomialIdeal,isNormal,isOpen,isOutputFile,isPolynomialRing,isPrimary,isPrime,isPrimitive,isPseudoprime,isQuotientModule,isQuotientOf,isQuotientRing,isReady,isReal,isReduction,isRegularFile,isRing,isSkewCommutative,isSorted,isSquareFree,isStandardGradedPolynomialRing,isSubmodule,isSubquotient,isSubset,isSupportedInZeroLocus,isSurjective,isTable,isUnit,isWellDefined,isWeylAlgebra,ITALIC,Iterate,Jacobian,jacobian,jacobianDual,Jets,Join,join,Jupyter,K3Carpets,K3Surfaces,Keep,KeepFiles,KeepZeroes,ker,kernel,kernelLLL,kernelOfLocalization,Key,keys,Keyword,Keywords,kill,koszul,Kronecker,KustinMiller,LABEL,last,lastMatch,LATER,LatticePolytopes,Layout,lcm,leadCoefficient,leadComponent,leadMonomial,leadTerm,Left,left,length,LengthLimit,letterParser,Lex,LexIdeals,LI,Licenses,LieTypes,lift,liftable,Limit,limitFiles,limitProcesses,Linear,LinearAlgebra,LinearTruncations,lineNumber,lines,LINK,linkFile,List,list,listForm,listLocalSymbols,listSymbols,listUserSymbols,LITERAL,LLL,LLLBases,lngamma,load,loadDepth,LoadDocumentation,loadedFiles,loadedPackages,loadPackage,Local,local,localDictionaries,LocalDictionary,localize,LocalRings,locate,log,log1p,LongPolynomial,lookup,lookupCount,LowerBound,LUdecomposition,M0nbar,M2CODE,Macaulay2Doc,makeDirectory,MakeDocumentation,makeDocumentTag,MakeHTML,MakeInfo,MakeLinks,makePackageIndex,MakePDF,makeS2,Manipulator,map,MapExpression,MapleInterface,markedGB,Markov,MarkUpType,match,mathML,Matrix,matrix,MatrixExpression,Matroids,max,maxAllowableThreads,maxExponent,MaximalRank,maxPosition,MaxReductionCount,MCMApproximations,member,memoize,memoizeClear,memoizeValues,MENU,merge,mergePairs,META,method,MethodFunction,MethodFunctionBinary,MethodFunctionSingle,MethodFunctionWithOptions,methodOptions,methods,midpoint,min,minExponent,mingens,mingle,minimalBetti,MinimalGenerators,MinimalMatrix,minimalPresentation,minimalPresentationMap,minimalPresentationMapInv,MinimalPrimes,minimalPrimes,minimalReduction,Minimize,minimizeFilename,MinimumVersion,minors,minPosition,minPres,minprimes,Minus,minus,Miura,MixedMultiplicity,mkdir,mod,Module,module,ModuleDeformations,modulo,MonodromySolver,Monoid,monoid,MonoidElement,Monomial,MonomialAlgebras,monomialCurveIdeal,MonomialIdeal,monomialIdeal,MonomialIntegerPrograms,MonomialOrbits,MonomialOrder,Monomials,monomials,MonomialSize,monomialSubideal,moveFile,multidegree,multidoc,multigraded,MultigradedBettiTally,MultiGradedRationalMap,multiplicity,MultiplicitySequence,MultiplierIdeals,MultiplierIdealsDim2,MultiprojectiveVarieties,mutable,MutableHashTable,mutableIdentity,MutableList,MutableMatrix,mutableMatrix,NAGtypes,Name,nanosleep,Nauty,NautyGraphs,NCAlgebra,NCLex,needs,needsPackage,Net,net,NetFile,netList,new,newClass,newCoordinateSystem,NewFromMethod,newline,NewMethod,newNetFile,NewOfFromMethod,NewOfMethod,newPackage,newRing,nextkey,nextPrime,nil,NNParser,NoetherianOperators,NoetherNormalization,NonAssociativeProduct,NonminimalComplexes,nonspaceAnalyzer,NoPrint,norm,normalCone,Normaliz,NormalToricVarieties,not,Nothing,notify,notImplemented,NTL,null,nullaryMethods,nullhomotopy,nullParser,nullSpace,Number,number,NumberedVerticalList,numcols,numColumns,numerator,numeric,NumericalAlgebraicGeometry,NumericalCertification,NumericalImplicitization,NumericalLinearAlgebra,NumericalSchubertCalculus,numericInterval,NumericSolutions,numgens,numRows,numrows,odd,oeis,of,ofClass,OL,OldPolyhedra,OldToricVectorBundles,on,OneExpression,OnlineLookup,OO,oo,ooo,oooo,openDatabase,openDatabaseOut,openFiles,openIn,openInOut,openListener,OpenMath,openOut,openOutAppend,operatorAttributes,Option,OptionalComponentsPresent,optionalSignParser,Options,options,OptionTable,optP,or,Order,order,OrderedMonoid,orP,OutputDictionary,Outputs,override,pack,Package,package,PackageCitations,PackageDictionary,PackageExports,PackageImports,PackageTemplate,packageTemplate,pad,pager,PairLimit,pairs,PairsRemaining,PARA,Parametrization,parent,Parenthesize,Parser,Parsing,part,Partition,partition,partitions,parts,path,pdim,peek,PencilsOfQuadrics,Permanents,permanents,permutations,pfaffians,PHCpack,PhylogeneticTrees,pi,PieriMaps,pivots,PlaneCurveSingularities,plus,poincare,poincareN,Points,polarize,poly,Polyhedra,Polymake,PolynomialRing,Posets,Position,position,positions,PositivityToricBundles,POSIX,Postfix,Power,power,powermod,PRE,Precision,precision,Prefix,prefixDirectory,prefixPath,preimage,prepend,presentation,pretty,primaryComponent,PrimaryDecomposition,primaryDecomposition,PrimaryTag,PrimitiveElement,Print,print,printerr,printingAccuracy,printingLeadLimit,printingPrecision,printingSeparator,printingTimeLimit,printingTrailLimit,printString,printWidth,processID,Product,product,ProductOrder,profile,profileSummary,Program,programPaths,ProgramRun,Proj,Projective,ProjectiveHilbertPolynomial,projectiveHilbertPolynomial,ProjectiveVariety,promote,protect,Prune,prune,PruneComplex,pruningMap,Pseudocode,pseudocode,pseudoRemainder,Pullback,PushForward,pushForward,Python,QQ,QQParser,QRDecomposition,QthPower,Quasidegrees,QuaternaryQuartics,QuillenSuslin,quit,Quotient,quotient,quotientRemainder,QuotientRing,Radical,radical,RadicalCodim1,radicalContainment,RaiseError,random,RandomCanonicalCurves,RandomComplexes,RandomCurves,RandomCurvesOverVerySmallFiniteFields,RandomGenus14Curves,RandomIdeals,randomKRationalPoint,RandomMonomialIdeals,randomMutableMatrix,RandomObjects,RandomPlaneCurves,RandomPoints,RandomSpaceCurves,Range,rank,RationalMaps,RationalPoints,RationalPoints2,ReactionNetworks,read,readDirectory,readlink,readPackage,RealField,RealFP,realPart,realpath,RealQP,RealQP1,RealRoots,RealRR,RealXD,recursionDepth,recursionLimit,Reduce,reducedRowEchelonForm,reduceHilbert,reductionNumber,ReesAlgebra,reesAlgebra,reesAlgebraIdeal,reesIdeal,References,ReflexivePolytopesDB,regex,regexQuote,registerFinalizer,regSeqInIdeal,Regularity,regularity,relations,RelativeCanonicalResolution,relativizeFilename,Reload,remainder,RemakeAllDocumentation,remove,removeDirectory,removeFile,removeLowestDimension,reorganize,replace,RerunExamples,res,reshape,ResidualIntersections,ResLengthThree,Resolution,resolution,ResolutionsOfStanleyReisnerRings,restart,Result,resultant,Resultants,return,returnCode,Reverse,reverse,RevLex,Right,right,Ring,ring,RingElement,RingFamily,ringFromFractions,RingMap,rootPath,roots,rootURI,rotate,round,rowAdd,RowExpression,rowMult,rowPermute,rowRankProfile,rowSwap,RR,RRi,rsort,run,RunDirectory,RunExamples,RunExternalM2,runHooks,runLengthEncode,runProgram,same,saturate,Saturation,scan,scanKeys,scanLines,scanPairs,scanValues,schedule,schreyerOrder,Schubert,Schubert2,SchurComplexes,SchurFunctors,SchurRings,SCRIPT,scriptCommandLine,ScriptedFunctor,SCSCP,searchPath,sec,sech,SectionRing,SeeAlso,seeParsing,SegreClasses,select,selectInSubring,selectVariables,SelfInitializingType,SemidefiniteProgramming,Seminormalization,separate,SeparateExec,separateRegexp,Sequence,sequence,Serialization,serialNumber,Set,set,setEcho,setGroupID,setIOExclusive,setIOSynchronized,setIOUnSynchronized,setRandomSeed,setup,setupEmacs,sheaf,SheafExpression,sheafExt,sheafHom,SheafOfRings,shield,ShimoyamaYokoyama,short,show,showClassStructure,showHtml,showStructure,showTex,showUserStructure,SimpleDoc,simpleDocFrob,SimplicialComplexes,SimplicialDecomposability,SimplicialPosets,SimplifyFractions,sin,singularLocus,sinh,size,size2,SizeLimit,SkewCommutative,SlackIdeals,sleep,SLnEquivariantMatrices,SLPexpressions,SMALL,smithNormalForm,solve,someTerms,Sort,sort,sortColumns,SortStrategy,source,SourceCode,SourceRing,SPACE,SpaceCurves,SPAN,span,SparseMonomialVectorExpression,SparseResultants,SparseVectorExpression,Spec,SpechtModule,SpecialFanoFourfolds,specialFiber,specialFiberIdeal,SpectralSequences,splice,splitWWW,sqrt,SRdeformations,stack,stacksProject,Standard,standardForm,standardPairs,StartWithOneMinor,stashValue,StatePolytope,StatGraphs,status,stderr,stdio,step,StopBeforeComputation,stopIfError,StopWithMinimalGenerators,Strategy,String,STRONG,StronglyStableIdeals,STYLE,Style,style,SUB,sub,SubalgebraBases,sublists,submatrix,submatrixByDegrees,Subnodes,subquotient,SubringLimit,Subscript,subscript,SUBSECTION,subsets,substitute,substring,subtable,Sugarless,Sum,sum,SumOfTwists,SumsOfSquares,SUP,super,SuperLinearAlgebra,Superscript,superscript,support,SVD,SVDComplexes,switch,SwitchingFields,sylvesterMatrix,Symbol,symbol,SymbolBody,symbolBody,SymbolicPowers,symlinkDirectory,symlinkFile,symmetricAlgebra,symmetricAlgebraIdeal,symmetricKernel,SymmetricPolynomials,symmetricPower,synonym,SYNOPSIS,syz,Syzygies,SyzygyLimit,SyzygyMatrix,SyzygyRows,syzygyScheme,TABLE,Table,table,take,Tally,tally,tan,TangentCone,tangentCone,tangentSheaf,tanh,target,Task,taskResult,TateOnProducts,TD,temporaryFileName,tensor,tensorAssociativity,TensorComplexes,terminalParser,terms,TEST,Test,testExample,testHunekeQuestion,TestIdeals,TestInput,tests,TEX,tex,TeXmacs,texMath,Text,TH,then,Thing,ThinSincereQuivers,ThreadedGB,threadVariable,Threshold,throw,Time,time,times,timing,TITLE,TO,to,TO2,toAbsolutePath,toCC,toDividedPowers,toDual,toExternalString,toField,TOH,toList,toLower,top,top,topCoefficients,Topcom,topComponents,topLevelMode,Tor,TorAlgebra,Toric,ToricInvariants,ToricTopology,ToricVectorBundles,toRR,toRRi,toSequence,toString,TotalPairs,toUpper,TR,trace,transpose,TriangularSets,Tries,Trim,trim,Triplets,Tropical,true,Truncate,truncate,truncateOutput,Truncations,try,TSpreadIdeals,TT,tutorial,Type,TypicalValue,typicalValues,UL,ultimate,unbag,uncurry,Undo,undocumented,uniform,uninstallAllPackages,uninstallPackage,Unique,unique,Units,Unmixed,unsequence,unstack,Up,UpdateOnly,UpperTriangular,URL,urlEncode,Usage,use,UseCachedExampleOutput,UseHilbertFunction,UserMode,userSymbols,UseSyzygies,utf8,utf8check,validate,value,values,Variable,VariableBaseName,Variables,Variety,variety,vars,Vasconcelos,Vector,vector,VectorExpression,VectorFields,VectorGraphics,Verbose,Verbosity,Verify,VersalDeformations,versalEmbedding,Version,version,VerticalList,VerticalSpace,viewHelp,VirtualResolutions,VirtualTally,VisibleList,Visualize,wait,WebApp,wedgeProduct,weightRange,Weights,WeylAlgebra,WeylGroups,when,whichGm,while,width,wikipedia,Wrap,wrap,WrapperType,XML,xor,youngest,zero,ZeroExpression,zeta,ZZ,ZZParser}
}
\lstalias{Macaulay2output}{Macaulay2}


\hbadness = 10000
\vbadness = 10000

\newcommand\restr[2]{{% we make the whole thing an ordinary symbol
  \left.\kern-\nulldelimiterspace % automatically resize the bar with \right
  #1 % the function
  \vphantom{\big|} % pretend it's a little taller at normal size
  \right|_{#2} % this is the delimiter
  }}

% Default fixed font does not support bold face
\DeclareFixedFont{\ttb}{T1}{txtt}{bx}{n}{12} % for bold
\DeclareFixedFont{\ttm}{T1}{txtt}{m}{n}{12}  % for normal
% Custom colors

\usepackage{color}
\definecolor{deepblue}{rgb}{0,0,0.5}
\definecolor{deepred}{rgb}{0.6,0,0}
\definecolor{deepgreen}{rgb}{0,0.5,0}

% Python style for highlighting
\newcommand\pythonstyle{\lstset{
language=Python,
basicstyle=\ttm,
morekeywords={self},              % Add keywords here
keywordstyle=\ttb\color{deepblue},
emph={MyClass,__init__},          % Custom highlighting
emphstyle=\ttb\color{deepred},    % Custom highlighting style
stringstyle=\color{deepgreen},
frame=tb,                         % Any extra options here
showstringspaces=false
}}

\lstnewenvironment{python}[1][]
{
\pythonstyle
\lstset{#1}
}
{}

\theoremstyle{definition}

\newtheorem{theorem}{Theorem}[section]
\newtheorem{definition}[theorem]{Definition}
\newtheorem{corollary}[theorem]{Corollary}
\newtheorem{lemma}[theorem]{Lemma}

\newcommand{\Z}{\mathbb{Z}}
\newcommand{\Q}{\mathbb{Q}}
\newcommand{\R}{\mathbb{R}}
\newcommand{\C}{\mathbb{C}}
\newcommand{\K}{\mathbb{K}}
\renewcommand{\P}{\mathbb{P}}
\newcommand{\F}{\mathbb{F}}
\newcommand{\N}{\mathbb{N}}
\newcommand{\A}{\mathbb{A}}

\newcommand{\x}{\bm{x}}
\newcommand{\Kx}{\K[\bm{x}]}
\newcommand{\KP}[2]{\K[#1_1, #1_2, \ldots, #1_{#2}]}

\renewcommand{\AA}[1]{\A^{#1}}
\newcommand{\An}{\A^n}
\newcommand{\Am}{\A^m}

\newcommand{\PP}[1]{\P^{#1}}
\newcommand{\Pn}{\P^n}
\newcommand{\Pm}{\P^m}

\newcommand{\Hom}{\text{Hom}}
\newcommand{\Aut}{\text{Aut}}
\newcommand{\End}{\text{End}}
\newcommand{\Iso}{\text{Iso}}


\newcommand{\lm}{\text{lm}}
\newcommand{\nr}{\text{nilrad}}
\newcommand{\spec}{\text{spec}}
\newcommand{\spn}{\text{span}}
\newcommand{\codim}{\text{codim}}
\newcommand{\ann}{\text{ann}}
\newcommand{\im}{\text{im}}
\newcommand{\id}{\text{id}}
\newcommand{\height}{\text{height}}
\newcommand{\ini}{\text{in}}

\newcommand{\catname}[1]{{\normalfont\textbf{#1}}}
\newcommand{\Set}{\catname{Set}}
\newcommand{\CRing}{\catname{CRing}}
\newcommand{\Top}{\catname{Top}}
\newcommand{\op}{\catname{op}}

\setlength{\parindent}{0pt}




\begin{document}

\section*{Ch 1}

\subsection*{Ex 1.20}
$A(X)$ is a field if and only if $I(X)$ is maximal, which happens if and only
if $X$ is minimal, i.e a point.

\subsection*{Ex 1.21}
Let $J = \langle f_1 = x_1^{3} - x_2^{6}, f_2 = x_1x_2 - x_2^{3}\rangle$. Then
$f_2 = x_2(x_1 - x_2^{2})$ is zero either when $x_2 = 0$ or when $x_1 =
x_2^{2}$, meanwhile $f_1 = (x_1 - x_2^{2})(x_1^{2} + x_1x_2^{2} + x_2^{4})$ is
zero when $x_1 = x_2^{2}$, but not when $x_2 = 0, x_1 \not = 0$.
We see that $\sqrt{J} = I(V(J)) = \langle x_1 - x_2^{2} \rangle$.

\subsection*{Ex 1.22}
First, of the $i$-th coordinate axis can be written as the zero locust
of all but the $i$-the hyperplanes of codimension $1$. So, if $X_i$
is the $i$-th coordinate axis, we have
\begin{align*}
	X_i 
	&=
	\bigcap_{j \not = i} V(x_j)
	&=
	V\left(\sum_{j \not = i} (x_j)\right)
	&=
	V(x_1, x_2, \ldots, x_{i- 1}, x_{i + 1}, \ldots x_n).
\end{align*}
We are interested in the ideal of the union of all $X_i$ in the case of $n = 3$
\begin{align*}
	I(X)
	&=
	I\left(\bigcup X_i\right) \\
	&=
	\bigcap I(a_i) \\
	&=
	(x_1, x_2) \cap (x_2, x_3) \cap (x_1, x_3)\\
	&=
	(x_1x_2, x_2x_3, x_1x_3).
\end{align*} 

Another way to arrive at the same result, is to realise that the coordinate
ring on a given axis is equal to the set of univariate polynomials in the given
indeterminate, and the coordinate ring on all three axes is the vector space
generated by all univariate monomials. This is exactly what $\Kx / (x_1x_2,
x_2x_3, x_1x_3)$ is. \\

Now, since $x_1, x_2, x_3 \not \in I(X)$, but $x_1x_2, x_2x_3, x_1x_3 \in
I(X)$, we have that any generating set of $I(X)$ must have a linear span which
includes each of the three monomials $x_1x_2, x_2x_3, x_1x_3$. But these three
monomials are linearly independent, and thus we need atleast three generators.

\subsection*{Ex 1.23}
\subsubsection*{(a)}

Let $J$ be an ideal in $A(Y)$. Then if $f \in J$, we have that $f(x) = 0$
implies that all $g \in \pi^{-1}(f) = f + \ker(\pi)$ satisfy $g(x) = 0$, since
$x \in \ker(\pi)$. \\

For the other direction, if $f \in \pi^{-1}(J)$ and $f(x) = 0$, we have that 
$\pi(f)(x) = 0$. \\

So, the two sets on polynomials vanish on the same points, whence their
varieties are equal by definition. \\

\subsubsection*{(b)}

We have that $\pi^{-1}(I_Y(X))$ is the set of polynomials $f$, such that
$\pi(f)$ vanishes on $X$. But since all the polynomials in $\ker(\pi)$ vanish
on all of $Y$, and therefore $X$, we have that $\pi(f)$ vanishs on $X$ if and
only if $f$ does, which means that $\pi^{-1}(I_Y(X)) = I(X)$.

\subsubsection*{(c)}

We have 
\begin{align*}
	I_{Y}(V_{Y}(J))
	&=
	I_{Y}(V(\pi^{-1}(J))) \\
	&=
	\pi(I(V(\pi^{-1}(J)))) \\
	&\subseteq
	\pi(\sqrt{\pi^{-1}(J)}) \\
	&=
	\pi(\pi^{-1}(\sqrt{J})) \\
	&=
	\sqrt{J},
\end{align*}
where we used part (a), (b), the Nullstellensatz, and the fact that contraction
commutes with taking the radical. We show the remaining inclusions as well. \\

Let $X \subseteq Y$ be a variety and let $x \in X$. Then 
every polynomial function in $I_Y(Y)$ vanishes on $x$, and $x \in V_{Y}(I_{Y}(X))$. \\

Let $J \unlhd A(Y)$ and $f \in \sqrt{J}$. Then $f^m$ vanishes on $V_{Y}(J)$ for some $m$,
whence $f$ does as well. It follows that $f \in I_{Y}(V_{Y}(J))$. \\

Finally, let $x \in V_{Y}(I_Y(X))$. Every variety is the zero locust of some
ideal, so let $X = V_{Y}(J)$. Then $x \in V_Y(I_Y(V_Y(J))) = V_Y(\sqrt{J})
\subseteq V_{Y}(J) = X$.


\subsection*{Ex 2.17}

We have $X = V\left(f_1 = x_1 - x_2x_3, f_2 = x_1x_3 - x_2^{2}\right)$. In this ring we have
\begin{align*}
	x_1
	&\equiv
	x_2x_3, \\
	x_1x_3 
	&\equiv
	x_2^{2}, \\
\end{align*}
which in turn yields
\begin{align*}
	x_1
	&\equiv
	x_2x_3, \\
	x_2x_3^{2} 
	&\equiv
	x_2^{2}, \\
\end{align*}

So we see that $A(X)$ is isomorphic to the algebra $A' = \K[x_2,
x_3]/(x_2x_3^{2} - x_2^{2})$. Let $(x_2(x_3^{2} - x_2)) = J \unlhd \K[x_2, x_3]
= A(V(f_1))$. It's easy to see that $J$ isn't prime. It is radical though as it
is a principal ideal generated by a squarefree product of irreducibles. It
follows that $J = (x_2) \cap (x_3^{2} - x_2)$ since the intersection of two
radical ideals equals the radical of their product, and we see that the minimal
prime ideals belonging to $J$ are $(x_2)$ and $(x_3^{2} - x_2)$. Since prime
ideals in $A(V(f_1)) \cong \K[x_2, x_3]$ are in one to one correspondence with
prime ideals in $\K[x_1, x_2, x_3]$ which contain $(f_1)$, we have that the
minimal ideals belonging to $(f_1, f_2)$ are $(x_2, x_1 - x_2x_3) = (x_1, x_2)$
and $(x_3^{2} - x_2, x_1 - x_2x_3)$, so $X = V(x_1, x_2) \cup V(x_3^{2} - x_2,
x_1 - x_2x_3)$ is a decomposition of $X$ into irreducible componenets. \\

This can be verified with the following Sage code (from python3)

\begin{python}
import sage.all
from sage.rings.rational_field import QQ
from sage.rings.polynomial.polynomial_ring_constructor \
	import PolynomialRing

def ex2_17(): 
    R = PolynomialRing(QQ, ["x_1", "x_2", "x_3"])
    x1, x2, x3 = R.gens()

    f1 = x1 - x2 * x3
    f2 = x1 * x3 - x2 ** 2
    
    J = R * [f1, f2]

    for Q in J.primary_decomposition():
        print(Q.radical().gens())
\end{python}

\subsection*{Ex 2.18}

$I(X)$ is the set of all polynomials vanishing on $X$, I.e it is maximal among
sets of polynomials vanishing on $X$. Thus, $V(I(X))$ is minimal among the
varieties containing $X$, which is the same as saying that $V(I(X))$ is the
closure of $X$ in the Zariski topology.

\subsection*{Ex 2.19}

\subsubsection*{(a)}

Let $X$ be a non-empty topology such that it can't be written as a finite union
of non-empty connected closed sets. Then in particular, $X$ isn't connected, so
we can write $X = X_1 \cup X_1'$ where $X_1 \cap X_1' = \emptyset$ and $X_1,
X_1'$ are non-empty closed sets. But then atleast one of these sets must be
disconnected, say $X_1$, and we can write $X_1 = X_2 \cup X_2'$ like before.
Continuing this way yields an infinite chain 
\[
X_1 \supsetneq X_2 \supsetneq X_3 \supsetneq \ldots,
\] 
and $X$ can't be Noetherian.

\subsubsection*{(b)}

Let $X = \bigcup_{1 \leq i \leq r} X_i$. Then given any infinite strict
decreasing chain of closed subsetes $Y_i$ we get $r$ infinite (possibly
non-strict) decreasing chains from $Y_i \cap X_j$ for each $1 \leq j \leq r$.
Taking the pairwise union of all $r$ chains yields the original chain $Y_i$,
and it follows that atleast one of the chains must contain infinitely many
strict inclusions, and by removing duplicates from this chain, we get an
infinite strictly decreasing chain of closed subsets. In other words, if $X$ is
non-Noetherian, then one of the $X_i$ must be non-Noetherian.

\subsection*{Ex 2.20}

\subsubsection*{(a)}

$Y \cap A$ closed in the subspace topology by definition implies that there is
some closed set $Z$ in $X$ such that $Z \cap A = Y$. Then $Y \subseteq Z$.
Since $\overline{Y}$ is the intersection of all closeds set in $X$ which
contain $Y$, we have $\overline{Y} \subseteq Z$. It follows that $\overline{Y}
\cap A \subseteq Z \cap A = Y$, but we also have $Y \subset A, Y \subset
\overline{Y}$, so $\overline{Y} \cap A = Y$.

\subsubsection*{(b)}

If $\overline{A} = U_1 \cup U_2$ with $U_1, U_2$ closed in the $\overline{A}$
subspace topology, then there are closed subsets $X_i$ such that $U_i = X_i
\cap \overline{A}$ and $A = (X_1 \cap A) \cup (X_2 \cap A)$. Moreover, if $X_i
\cap A = A$, then we'd have $\overline{A} \subseteq X_i$, and $\overline{A} =
U_i$, so the union $A = (X_1 \cap A) \cup (X_2 \cap A)$ is a non-trivial
decomposition. \\

For the other direction, first note that $\overline{U_1} \cup \overline{U_2}
\supseteq \overline{U_1 \cup U_2}$ since $\overline{U_1} \cup \overline{U_2}$
is a closed set which covers both $U_1$ and $U_2$. \\

If $A = U_1 \cup U_2$ with $U_1, U_2$ closed in the $A$ subspace topology, then
there are closed subsets $X_i$ such that $U_i = X_i \cap A$. Then $X_i \cap
\overline{A}$ must cover $\overline{U_i}$, and 
\[
	\overline{A} 
	\subseteq 
	\overline{U_1} \cup \overline{U_2} 
	\subseteq 
	(X_1 \cap \overline{A}) \cup (X_2 \cap \overline{A}),
\] 
but it's clear that the inclusions must hold in the other direction as well
since everything on the RHS is intersected with the LHS, and we see that
$\overline{A}$ is reducible. Moreover, the two sets on the RHS are non-empty as
the $U_i = X_i \cap A \subseteq X_i \cap \overline{A}$ are.

\subsection*{Ex 2.21}

\subsubsection*{(a)}

Assume that the cover $U_i$ of $X$ consists only of open connected sets and
that they all pairwise intersect each other. Let $X_1, X_2$ be two closed
proper subets of $X$ such that $X_1 \cup X_2 = X$. Then $X_1, X_2$ are also
both open as they complement each other. Now consider some $U_1$. If $U_1$
intersects both $X_1$ and $X_2$, we have that $U_1 \cap X_i$ is closed, since
$U_1 \cap X_1 = (U_1^{c} \cup X_2)^{c}$, so $U_1 = (U_1 \cap X_1) \cup (U_1
\cap X_2)$ is a decomposition of $U_1$ into closed sets, whence $U_1 \cap
X_1$ must intersect $U_2 \cap X_2$ by hypothesis of the $U_i$ being connected.
It follows that $X_1 \cap X_2 \not = \emptyset$ in this case. \\

Now consider the case where $U_1$ only intersects one of the $X_i$, say $X_1$.
Let $U_2$ be a set which intersects $X_2$ (such exist since the $U_i$ cover
$X$). Then as $U_1$ and $U_2$ intersect each other, it follows that $U_2$ must
intersect both $X_1, X_2$, and we can conclude like before that $X_1$ must
intersect $X_2$ in this case as well. 

\subsubsection*{(b)}

Let $X, U_i$ be as above but with the further stipulation that each $U_i$ is
irreducible. Assume towards a contradiction that $X$ is reducible and let $X =
X'_1 \cup \ldots \cup X'_r$ where each $X_i$ is a closed proper subset of $X$
and no $X_i$ is contained in the union of all $X_j, j \not = i$. Let $X_1 =
X'_1, X_2 = X_2 \cup \ldots X'_r$. Then $X = X_1 \cup X_2$ is a decomposition
of $X$ into two closed proper subsets. \\

Like above, if a $U_i$ intersects both $X_1, X_2$ we get a decomposition of
$U_i = (U_i \cap X_1) \cup (U_i \cap X_2)$ into non-empty closed proper
subsets, which is imposible by our assumption that each $U_i$ is irreducible.
Thus all $U_i$ must be contained in either $X_1$ or $X_2$ but not both. This is
imposible since they all pairwise intersect each other (and cover both $X_1,
X_2$). 


\subsection*{Ex 2.22}

\subsubsection*{(a)}

If $f(X) = U_1 \cup U_2$ is a decomposition of $f(X)$ into non-intersecting
closed proper subsets, then $X = f^{-1}(U_1) \cup f^{-1}(U_2)$ is the same for
$X$. So $f(X)$ disconnected implies $X$ disconnected.

\subsubsection*{(b)}

Like above.

\subsection*{Ex 2.23}

\subsubsection*{(a)}

First note that $I(\overline{X}) = I(V(I(X)) = I(X)$, so
we'll prove that $I(Y_1 \setminus Y_2) = I(Y_1) : I(Y_2)$. \\

Let $f \in I(Y_1 \setminus Y_2)$. Then $f$ vanishes on $Y_1$, and we have that 
$f g \in I(Y_1)$ for all $g \in \Kx$, and in particular, for all $g \in I(Y_2)$,
so $f \in I(Y_1) : I(Y_2)$. \\

Now let $f \in I(Y_1) : I(Y_2)$. Then for all $g \in I(Y_2)$, we have that $fg
\in I(Y_1)$. Now, since $Y_2$ is a subvariety, it's the zero locust of
$I(Y_2)$, and for all points $a \in Y_1 \setminus Y_2$, there is some
polynomial $g_a$ such that $g_a(a) \not = 0$. But $fg_{a} \in I(Y_1)$, so it
must be that $f(a) = 0$, and we see that $f \in I(Y_1 \setminus Y_2)$.

\subsubsection*{(b)}
Using the Nullstellensatz, part (a), and the fact that the $J_i$ are radical,
we get
\begin{align*}
	\overline{V(J_1) \setminus V(J_2)}
	&= V(I(\overline{V(J_1) \setminus V(J_2)})) \\
	&= V(I(V(J_1)) : I(V(J_2))) \\
	&= V(\sqrt{J_1} : \sqrt{J_2}) \\
	&= V(J_1 : J_2).
\end{align*}


\subsection*{Ex 2.24} 

Let $X \subseteq \An, Y \subseteq \Am$ be irreducible affine varieties and
assume towards a contradiction that $X \times Y = U_1 \cup U_2$ is a
decomposition into closed proper subsets. Let $\pi_X, \pi_Y$ be the projections
onto $X, Y$ repectively. First note that
\begin{align*}
	U_1 \cup U_2
	&=
	X \times Y \\
	&=
	\pi_X(X) \times \pi_Y(Y) \\
	&=
	\pi_X(U_1 \cup U_2) \times \pi_Y(U_1 \cup U_2) \\
	&=
	(\pi_X(U_1) \cup \pi_X(U_2)) \times (\pi_Y(U_1) \cup \pi_Y(U_2)) \\
	&= 
	(\pi_X(U_1) \times \pi_Y(U_1))
	\cup 
	(\pi_X(U_1) \times \pi_Y(U_2)) \\
	&\,\, \cup 
	(\pi_X(U_2) \times \pi_Y(U_1))
	\cup 
	(\pi_X(U_2) \times \pi_Y(U_2)),
\end{align*}
from where it will follow that either $Y = \pi_Y(U_1) \cup \pi_Y(U_2)$, or $X =
\pi_X(U_1) \cup \pi_X(U_2)$ are non-trivial decompositions into subsets. We
will show that $\pi_X(U_i), \pi_Y(U_i)$ are closed, which will contradict $X,
Y$ being irreducible. \\

Let $f \in I(U_1) \subseteq \K[x_1, x_2, \ldots, x_n, y_1, y_2, \ldots, y_m]$
and fix the last $m$ indeterminates of $f$ to some point $y \in \pi_Y(U_1)$.
Call the new polynomial $g_{f, y}$. Then for any $x \in \pi_X(U_1)$ we have
$g_{f, y}(x) = f(x, y)$ and since $(x, y) \in U_1$ and $f \in I(U_1)$, we see
that $g_{f, y}$ vanishes on $\pi_X(U_1)$. We claim that
\[
	\pi_X(U_1) 
	= 
	\bigcap_{f \in I(U_1), y \in \pi_Y(U_1)}
	V
	\left( g_{f, y}\right).
\] 
We've already shown one inclusion, for the other direction, suppose that $x \in
X$ such that all $g_{f, y}$ vanish on $x$. Then $f(x, y) = 0$ for all $y \in Y,
f \in I(U_1)$, which in turn means that $x \times Y \subset V(I(U_1)) = U_1$ so
$x \in \pi_X(U_1)$. It follows that $\pi_X(U_1)$ is closed and we are done.

\subsection*{Ex 2.30} 
If $U_i$ is some strict descending chain of closed irreducible sets in $A$,
then by Ex 2.20 (a), we have that $\overline{U}_i$ is a strict descending chain
of closed sets in $X$, which are irreducible by Ex 2.20 (b).

\subsection*{Ex 2.33} 

First, let's identify said matrices with $\A^{6}$ according to
\[
\begin{pmatrix}
	a_1 & a_2 & a_3 \\
	a_4 & a_5 & a_6 \\
\end{pmatrix}.
\] 
The rank of a matrix is at most $1$ if and only all of its minors of 
order $\geq 2$ are zero. For our matrix above, this yields the following
set of equations
\begin{align*}
	a_1 a_5 &= a_2 a_4 \\
	a_1 a_6 &= a_3 a_4 \\
	a_2 a_6 &= a_3 a_5.
\end{align*}

If we let $X$ denote our set in $\A^{6}$ identified with the given set of
matrices, then it follows from the discussion above that 
\[
X = V(J = (x_1x_5 - x_2x_4, x_1x_6 - x_3x_4, x_2x_6 - x_3x_5)).
\] 
is a variety. To show that it's irreducible, note that we can surjectively
parameterize $X$ with $t_1, t_2, t_3, u$ according to 
\begin{align*}
	a_1 = t_1,
	a_2 &= t_2,
	a_3 = t_3, \\
	a_4 = ut_1 ,
	a_5 &= ut_2,
	a_6 = ut_3.
\end{align*}
Irreducibility follows from the following lemma. 
\begin{lemma}
	Let $X \subseteq \An$ be a variety that is parameterized by $a_i = f_i(y_1,
	y_2, \ldots, y_m)$. Then $I(X)$ is prime and $X$ is irreducible.
\end{lemma}
\begin{proof}
	$I(X)$ can be identified with the ideal of algebraic dependencies on the
	$f_i$. I.e the kernel of $\Kx \to \K[f_1, f_2, \ldots, f_n]$. This kernel
	is prime since the image of the map is a subring of $\K[y_1, y_2, \ldots,
	y_m]$ which is an integral domain.
\end{proof}


\subsection*{Ex 2.34} 

\subsubsection*{(a)} 

Let $n \in \N$ be such that $\dim X \geq n$. Then there exist some chain
\[
Y_0 \subsetneq Y_1 \subsetneq \ldots \subsetneq Y_n
\] 
of closed irreducible subsets. Since $Y_i$ is irreducible, any two non-empty
open sets in $Y_i$ must intersect. Let $V_i = Y_i \setminus Y_{i - 1}$. Then
each $V_i$ is open and non-empty in $Y_i$, since $Y_{i - 1}$ is closed in
$Y_i$. \\

Now, pick some $U$ which intersects $Y_0$. Since $U \cap Y_1$ is open and
non-empty in $Y_1$, it must intersect $V_1$, whence $U \cap Y_0 \subsetneq
U \cap Y_1$. Moving on, $U \cap Y_2$ is open in $Y_2$, hence it must
intersect $V_2$ and it follows that 
\[
	U \cap Y_0 \subsetneq U \cap Y_1 \subsetneq U \cap Y_2.
\] 
This procedure can be repeated to show that 
\[
	U \cap Y_0 \subsetneq U \cap Y_1 \subsetneq U \cap Y_2 \subsetneq
	\ldots \subsetneq U \cap Y_n \subset U
\] 
is an $n$-long chain of strict inclusions in $Y$. Moreover, each of the $U
\cap Y_i$ is irreducible in $U$ since a reduction of $Y_i \cap U = T_1 \cup
T_2$ yields a reduction
\[
	Y_i 
	= 
	(Y_i \setminus (U \setminus T_1))
	\cup
	(Y_i \setminus (U \setminus T_2)).
\]
Hence $\dim X \leq \sup_{i \in I}(\dim U_i)$. The reversed inequality
is immediate from Ex 2.30.

\subsubsection*{(b)} 

Let $X$ be an irreducible affine variety and 
\[
Y_0 \subsetneq Y_1 \subsetneq \ldots \subsetneq Y_n \subset X
\] 
a chain of irreducible subvarieties. Then if $U$ is non-empty and open in $X$,
we can translate the chain in affine space $T_{\bm{c}}(a_1, a_2, \ldots, a_n) =
(a_1 + c_1, \ldots, a_n + c_n)$ (this is a homeomorphism) such that
$T_{\bm{c}}(Y_0)$ intersects $U$. We can then repeat the argument of (a) to
show that 
\[
	T_{\bm{c}}(Y_0) \cap U \subsetneq T_{\bm{c}}(Y_1) \cap U \subsetneq \ldots
	\subsetneq T_{\bm{c}}(Y_n) \cap U \subset U
\] 
is a chain of irreducible closed subvarieties in $U$. \\

This is not the case in arbitrary topological spaces. Consider for example the
space $\N$ given the topology where the non-trivial closed subsets are of the
form $[1..n]$ and $[1..\infty]$. Then $\{0\}$ is open in $\N$ but has dimension
$0$. \\



\subsection*{Ex 2.35} 

First suppose that we have some chain
\[
	\{a\} \subset Y_0 \subsetneq Y_1 \subsetneq \ldots \subsetneq Y_n \subset X
\] 
of irreducible closed subsets $Y_i$. Then $Y_n = \bigcup_{i=1}^{r} \left(Y_n
\cap X_i\right)$, is a decomposition into closed sets, and since $Y_n$ is
irreducible, we need $Y_n \subseteq X_i$ for some $i$, whence $\max(\dim X_i :
a \in X_i) \geq \codim \{a\}$. \\

For the other direction, let $X_{i}$ be the irreducible variety containing $a$
with maximal dimension. Then by Prop 2.28 (b), we have $\codim_{X_i} \{a\} =
\dim X_i = \max(\dim X_j : a \in X_j)$.

\subsection*{Ex 2.36} 

\subsubsection*{(a)} 

$X$ being Noetherian is equivalent to 
Let $X$ be Noetherian and $U_i, i \in I$ be an open cover of $X$. Then $X$
admits a decomposition into irreducible subsets $X = X_1 \cup X_2 \cup \ldots
\cup X_r$. Let $U_i, i \in I_1$ be a subcover which covers $X_1$ where all
$U_i, i \in I_1$ intersect $X_1$, and no $U_i$ is contained in some other
$U_j$. Then the $U_i, i \in I_1$ must pairwise intersect each other. Use the
axiom of choice to pick a sequence of unique indices $i_1, i_2, \ldots$ in
$I_1$, and construct the chain
\[
	U_{i_1} \supseteq U_{i_1} \cap U_{i_2} (\supseteq U_{i_1} \cap U_{i_2})
	\cap U_{i_3} \supseteq \ldots,
\] 
Each intersection is non-empty, since it's an intersection non-empty as $X_1$
is irreducible. Moreover, since $X$ is Noetherian, this must either be a finite
chain so that $I_1$ is finite, or it must stabilize, so that 
\[
	U_{i_n} \subset \bigcap_{j = 1}^{n - 1} U_{i_j}
\] 
for some $n$. This contradicts how $I_1$ was constructed.

\subsubsection*{(b)} 

\textbf{Quick aside:} I revisited an old solution to this problem after finding
the reference to this problem in the start of Chapter 5. The old solution was
wrong, so I've re-solved this problem using tools from Chapters 1 through 4. I
also had trouble solving it the second time around as well, and after looking
at this link, https://math.stackexchange.com/a/119349/887520 , I decided to try
to do it using Noether Normalization. This was very difficult for me (took 2
days almost to show that the morphism $f$ below is surjective). So, this is all
just a heads up to say that this solution is a bit messy and uses a few
theorems and lemmas from all over the place. \\

Let $X$ be an irreducible subvariety in $\C[\x]$ of dimension $d \geq 1$. It
follows that the Krull dimension of $A(X)$ is $d$. The Noether normalization
lemma tells us that $A(X)$ is integral and finitely generated over some free
polynomial algebra $\C[y_1, y_2, \ldots, y_s]$. We will show that $d = s$.
First note that $d \geq s$, since we have a chain of length $s$ of prime ideals
$(0) \subsetneq (y_1) \subsetneq y_1, y_2) \subsetneq \ldots \subsetneq (y_1,
y_2, \ldots, y_s)$ which can be extended to a chain of the same length in
$A(X)$ by the Going Up Theorem. Now suppose that $\mathfrak{p}_0 \subsetneq
\mathfrak{p}_1 \subsetneq \ldots \subsetneq \mathfrak{p}_d$ is a chain of prime
ideals in $A(X)$ of length $d$. Then any ideal $\mathfrak{p}_i \cap \K[y_1,
y_2, \ldots, y_s]$ is prime, after which Corollary 5.9 in Atiyah-Macdonald
tells us that the chain contracts to a chain of strict inclusions of prime
ideals in $\C[y_1, y_2, \ldots, y_s]$. Exercise 11.7 in Atiyah-Macdonald tells
us that this ring has dimension $s$, so $s \geq d$, and we have $s = d$. \\

Now consider the inclusion $f^{*} : \C[y_1, y_2, \ldots, y_d] \to A(X)$ where
the $y_i$'s are algebraically independent polynomials in $A(X)$. We can
identify $\C[y_1, y_2, \ldots, y_d]$ with a free $\C$-algebra in $d$ variables,
and doing so induces a morphism of varieties. $f : X \to \A^{d}$. We claim that
$f$ is surjective. To show this, we will first construct $f$ explicitly. \\

By the proof of Corollary 4.8, we have that $f =
(\phi_1, \phi_2, \ldots, \phi_d)$ where $\phi_i = f^{*}(y_i)$, but $f^{*}$ is
just the inclusion, so $f = (y_1, y_2, \ldots, y_d)$. Now let $a \in \A^{d}$
and consider $J = (y_1 - a_1, y_2 - a_2, \ldots, y_d - a_d)$ as an ideal in
$A(X)$. We want to show that there is some $b \in V(J)$, since doing so
would imply that $y_i(b) - a_i = 0 \Leftrightarrow y_i(b) = a_i \Leftrightarrow
f(b) = a$. \\ 

First of, note that $A(X)$ is a finitely generated $\C[y_1, y_2, \ldots,
y_d]$-module by the Noether normalization lemma. We will consider $J' = (y_1 -
a_1, y_2 - a_2, \ldots, y_d - a_d)$ as an ideal in $\C[y_1, y_2, \ldots, y_d]$.
Note that $J \not = J'$, these two ideals have the same generators but
different parent rings! Now let's show that $V(J) \not = \emptyset$. \\

Suppose towards a contradiction that $V(J) = \emptyset$. Then $\sqrt{J} =
I(V(J)) = (1) \Rightarrow J = (1)$, so we can write 
\[
	1 = \sum_{i = 1}^{d} h_i(\x)y_i(\x)
\] 
for some set of polynomials $h_i \in A(X)$. It follows that $J'A(X) = A(X)$.
Now Nakayama's lemma grants us some $r \in \C[y_1, y_2, \ldots, y_d]$ such that
$rA(X) = 0$ and $r - 1 \in J'$. It follows that $r(a) = 1$ so $r \not = 0$. But
the $y_i$ are algebraically independent, so $y_i r = 0$ is a contradiction.
Hence $V(J) \not = \emptyset$ and we've shown that $f$ is surjective. \\

$f$ is a polynomial mapping, hence it's continuous in the classical topology.
So we have a surjective continuous map $f: X \to \A^{d} = \C^{d}$. We have that
$\C^{d}$ isn't compact (for $d \geq 1$), hence $X$ isn't either (since $X$
compact $\Rightarrow$ $f(X) = \C^{d}$ compact, a contradiction,
https://math.stackexchange.com/questions/26514/continuous-image-of-compact-sets-are-compact)


\subsection*{Ex 2.40} 

\subsubsection*{(a)} 

With $J = (x_1x_4 - x_2x_3)$, we recognize $V(J)$ as the set of $2$ by $2$
matrices with rank $\leq 1$. This variety is parameterizeable with 
\[
	a_1 = t_1, a_2 = t_2, a_3 = c t_1, a_4 = c t_2,
\] 
whence
\[
	R = \K[x_1, x_2, x_3, x_4]/J \cong \K[t_1, t_2, ct_1, ct_2] \subset \K[t_1, t_2, c]
\] 
so $R$ is an integral domain. Moreover, $R$ isn't a field, so $J$ is a prime
ideal which isn't maximal, whence it can at most have height $\dim(\K[x_1, x_2,
x_3, x_4]) - 1 = 3$, and $R$ can at most have dimension $3$. But in $\K[t_1,
t_2, ct_1, ct_2]$ we have the prime ideals 
\[
	(0) \subsetneq (t_1, t_2) \subsetneq (t_1, t_2, ct_1, ct_2)
\] 
So $R$ has dimension $3$. We quickly verify that all the ideals are prime.
$(0)$ is prime since we're in a domain. $(t_1, t_2, ct_1, ct_2)$ is prime since
we get a field when we quotient by it. $(t_1, t_2)$ is prime since any
polynomial in $\K[t_1, t_2, ct_1, ct_2]$ has terms where the sum of the degrees
of $t_1, t_2$ is greater than the degree of $c$, so $Q = \K[t_1, t_2, ct_1,
ct_2]/(t_1, t_2)$ consists of all polynomials in $\K[t_1, t_2, ct_1, ct_2]$
with terms where the degree of $c$ is equal to the sum of the degrees of $t_1,
t_2$. Hence $Q \cong \K[ct_1, ct_2]$ is an integral domain. Note that $(t_1)$
(or $(t_{2})$) isn't a prime ideal, since it contains $ct_1t_2$ but neither
$ct_1$ nor $t_2$ (or neither $ct_2$ nor $t_1$).

\subsubsection*{(b)} 

$x_1 | x_2x_3 = x_1x_4$, but $x_1 \not | x_2$, $x_1 \not | x_3$ (any element in
$J$ has $\deg \geq 2$, so $x_1$ won't divide any representatives of $x_2 + J$
or $x_3 + J$ since they all contain a term of $x_2$, or $x_3$ respectively).

\subsubsection*{(c)} 

Again, follows from $x_1 + J \not | x_2 + J$, and similar.

\subsubsection*{(d)} 

Under the isomorphic map $x_1 \mapsto t_1, x_2 \mapsto t_2, x_3 \to c t_1, x_4
\mapsto c t_2$, we see that $(x_1, x_2)$ is the contraction of the prime ideal
$(t_1, t_2)$, which has height $1$, so $(x_1, x_2)$ has height $1$ also. \\


\subsection*{Ex 3.12} 

\subsubsection*{(a)} 

Suppose that $A(X)$ is a UFD, $Y \subset X$ is an irreducible subvariety, and
that $\codim_X Y \geq 2$. We will begin by showing that there are two elements
$f_1, f_2 \in I(Y)$ which are prime in $A(X)$. \\

Since $Y$ is irreducible, $I(Y)$ is prime, and as $A(X)$ is a UFD, we must have
some element $f_1 \in I(Y)$ which is prime in $A(X)$. Moreover, $I(Y)$ has
height at least $2$, so it's strictly bigger than $(f_1)$ which has height $1$
by Prop 2.28 (c), hence $I(Y)$ must contain some element $m$ where $f_1 \not |
m$. But again, $I(Y)$ is prime, so it contains some prime factor $f_2$ of $m$.
\\

It follows that $(f_i) \subset I(Y) \Rightarrow V(f_i) \supset Y \Rightarrow
D(f_i) \subset U$. Hence we can write $\phi = h_1/f_1^{k_1}$ on $D(f_1)$, and
$\phi = h_2/f_2^{k_2}$ on $D(f_2)$ with $f_i \not | h_i$. On $D(f_1) \cap
D(f_2)$ we have 
\[
 h_1/f_1^{k_1}
 =
 h_2/f_2^{k_2}
 \Leftrightarrow
 h_1f_2^{k_2} = h_2 f_1^{k_1},
\] 
but $V(h_1f_2^{k_2} - h_2f_1^{k_1})$ is closed, so $h_1f_2^{k_2} = h_2
f_1^{k_1}$ must hold on $\overline{D(f_1) \cap D(f_2)}$. Our next claim is that
$\overline{D(f_1) \cap D(f_2)}$ is all of $A(X)$. To see this, note that that
$A(X)$ is a UFD, hence an integral domain, whence $X$ is irreducible and any
open subspace is dense. Thus $h_1f_2^{k_2} = h_2 f_1^{k_1}$ on all of $A(X)$,
and since $f_2^{k_1} | h_2f_1^{k_1}$, but $f_2$ divides neither $h_2$ nor
$f_1$, we have $k_1 = 0$, and $\phi = h_1 = h_2$ on $A(X)$. \\

For the other direction, we assume that $\codim_X(Y) = 1$, since if the
codimension is $0$ we'd have $Y = X$. Then Prop 2.38 $I(Y) = (f)$, with $f$ a
non-unit, so $Y = V(f) \Rightarrow U = D(f)$ and Corollary 3.10 tells us that
$\mathcal{O}_X(U) = \mathcal{O}_X(D(f)) \cong A_{f}(X)$ which isn't isomorphic
to $A(X)$ when $f$ isn't a unit.

\subsubsection*{(b)} 

Consider Exercise 2.40 and Example 3.3. We know that $(x_1, x_2)$ is a prime
ideal of height $1$ in $A(X)$, so $Y = V(x_1, x_2)$ has codimension $1$ in $X$,
but yet the example shows that $\mathcal{O}_X(X \setminus U)$ isn't $A(X)$.


\subsection*{Ex 3.20} 

We show that the corresponding localized coordinate rings are isomorphic, after
which Lemma 3.19 gives the desired isomorphisms. Let $\Kx_{I(a)} \to
A(X)_{I(a)}$ be given by 
\[
	\phi\left(
		g/f
	\right)
	=
	\frac{g + I(X)}{f + I(X)}.
\]
$\phi$ is well defined since $f \in \Kx \setminus I(a)$, and $a \in X
\Rightarrow I(a) \supset I(X)$, so $f + I(X)$ is non-zero. Suppose that
$\phi(g/f) = 0$. Then $(h + I(X))(g + I(X)) = hg + I(X) = 0$ for some $h + I(X)
\in A(X) \setminus I(a)$, which in turn implies that $hg \in I(X)$. Hence $hg/1
\in I(X)\Kx_{I(a)}$, but $h(a) \not = 0$, so $h$ is a unit in $\Kx_{I(a)}$, so
$hh^{-1}g/1 = g/1 \in I(X)\Kx_{I(a)}$, whence $\ker(\phi) \subseteq
I(X)\Kx_{I(a)}$. \\

Now suppose that $g \in I(X)$, then $\phi(g/1) = 0$, so all the generators of
$I(X)\Kx_{I(a)}$ lie in $\ker(\phi)$ and we have equality of the two ideals.
The desired result now follows from the first homomorphism theorem and Lemma
3.19.

\subsection*{Ex 3.21} 

\subsubsection*{(a)} 

Let $\epsilon \not = 0$ and define $f_{\epsilon} : \R \to \R$ as
follows,
\[
	f_{\epsilon}(x) = 
	\begin{cases}
		0 \text{ for } x \in (a - \epsilon, a + \epsilon), \\
		x - (a + \epsilon) \text{ for } x \in [a + \epsilon), \\
		x - (a - \epsilon) \text{ for } x \in (a - \epsilon].
	\end{cases}
\] 
Then $f_{\epsilon}$ is continuous, and agrees with the zero function on $(a -
\epsilon, a + \epsilon)$, so they reside in the same germ. Since we can pick
$\epsilon$ arbitrarily small, it follows that the functions in a given germ all
agree on $a$ only. Now let $I$ denote the ideal of all germs which are $0$ at
$a$. We claim that $\mathcal{F}_{a}/I$ is isomorphic to $\R$. To see this,
note that $f(a) = g(a)$ if and only if $f(a) - g(a) = 0$ and $f(a) - g(a) \in
I$. Hence every equivalence class of $\mathcal{F}_{a}/I$ may be identified
with the value which the stalks admit at $a$, and $\mathcal{F}_{a}/I \cong I$
is a field and $I$ is maximal. Moreover, it's the only maximal ideal since it
contains all non-units, indeed if $f(a) \not = 0$, then we can pick a small
enough neighbourhood of $a$ in which $f$ is invertible.

\subsubsection*{(b)} 
The open subsets of $\R$ are all infinite, so if two polynomials agree on an
open subset of $\R$ they must be equal. It follows that any stalk is isomorphic
to the polynomial ring which is not local.

\subsection*{Ex 3.22} 

\subsubsection*{(a)} 

Let $a \in U$. Then since $(U, \phi) \sim (U, \psi)$, we have by definition
some open $V_a \subset U$ containing $a$ such that $\restr{\phi}{V_a} =
\restr{\psi}{V_a}$. The $V_a : a \in U$ form an open cover of $U$, and $\phi =
\psi$ on $U$ by the gluing property.

\subsubsection*{(b)} 

The vanishing set of their difference is closed in $X$, and since their
difference vanishes on some open subset $V_a \subset X$, it vanishes on
$\overline{V_a} = X$.

\subsubsection*{(c)} 

Yes, consider for example the $f_{\epsilon}$ and the zero function from the
solution to Ex 3.21 (a) with $U = \R$.

\subsection*{Ex 3.23} 

Let $\phi : A(X)_{I(Y)} \to \mathcal{O}_{X, Y}$ be given by
\[
	\phi(g/f) = \overline{\left(D(f), g/f \right)}.
\] 
$\phi$ is well defined, for if $g/f = g'/f'$, then we have $h \in A(X)
\setminus I(Y)$ such that $h(gf' - g'f) = 0$. Hence $g/f$ and $g'/f'$ agree as
regular functions on the open set $U = D(f) \cap D(f') \cap D(h) = D(ff'h)$. We
have that $U$ is contained in both $D(f), D(f')$. Moreover, $U$ intersects $Y$
as $f, f', h \not \in I(Y)$ and $I(Y)$ is prime since $Y$ is irreducible so
$ff'h \not \in I(Y)$, so there must be some point $y \in Y$ where all $f, f',
h$ are non-zero and $y \in U \cap Y$. We can conclude that
$\overline{\left(D(f), g/f\right)} = \overline{\left(D(f'), g'/f'\right)}$. \\

Now suppose that $g/f \in \ker(\phi)$. Then we have that $0/1$ and $g/f$ agree
as functions on some neighbourhood $V$ which intersects $Y$. We can define
$g/f$ on all of $D(f)$. Vanishing sets of regular functions are closed, so
$g/f$ vanishes on the closure of $V$ in $D(f)$, but $Y$ is irreducible, so
$\overline{D(f)} = Y$ and $D(f)$ is irreducible as well, whence $g/f$ vanishes
on all of $D(f)$. It follows that $f(1 \cdot g - 0 \cdot f) = 0$ as polynomial
functions on $X$, hence $g/f = 0/1$ in $A(X)_{I(Y)}$. \\

Finally, to see that $\phi$ is surjective since, let $\overline{(U, \phi)} \in
\mathcal{O}_{X, Y}$, $y \in U$ and $V_y$ be an open subset of $U$ containing
$y$ such that $\phi$ is given by $g/f$ on $V_y$. Then $y \in D(f)$, so $D(f)
\cap Y \not = \emptyset$, and $\phi(g/f) = \overline{(D(f), g/f)}$ is a stalk
at $Y$ which is equal to $\overline{(U, \phi)}$ since they agree on $V_y$ (note
that $g/f$ is an element of the domain $A(X)_{I(Y)}$ since $f \not \in I(Y)$ as
$f(y) \not = 0$).

\subsection*{Ex 3.24} 
If $a \in V$, then $a \in U \cap V$, so any representative $(V, \phi)$ of a
germ in the original stalk can be restricted down to $U$. The other direction
is immediate since open subsets in $U$ are open in $X$ (by virtue of $U$ being
open). Passing up and down through the restriction doesn't change equivalence
class since any regular function defined on $V$ agrees with itself on $V \cap
U$. So the restriction of stalks is bijective. \\

Since the original restriction maps in the sheaves are homomorphisms, it follows
that if $(V, f), (V', f')$ are representatives of germs in $\mathcal{F}_a$,
that their product
\[
	(V \cap V', \restr{f}{V \cap V'}\restr{f'}{V \cap V'})
\]
restrict to the restrictions of their product,
\[
	(V \cap V' \cap U, \restr{\left(\restr{f}{V \cap V'}\restr{f'}{V \cap V'}\right)}{U})
	=
	(V \cap V' \cap U, \restr{f}{V \cap V' \cap U}\restr{f'}{V \cap V' \cap U}).
\]
The same is true for sums and any other algebraic operation, so we see that the
restriction of the stalk is an isomorphism.


\subsection*{Ex 4.12} 

A general affine conic be given by
\[
	f(x, y) 
	=
	a_5 x^{2}
	+
	a_4 xy
	+ 
	a_3 y^{2}
	+
	a_2 x
	+
	a_1 y
	+
	a_0,
\] 
where the $a_i$ are such that $(f)$ is prime. \\

We will show that $A = \K[x,y]/(f)$ is isomorphic exactly one of either $B_1 =
\K[x,y]/(g_1)$ or $B_2 = \K[x, y]/(g_2)$ where $g_1 = x^{2} - y$ and $g_2 = xy
- 1$. \\

First we show that $B_1 \not \cong B_2$. To see this, note that $B_1 \cong
\K[x]$ whilst $B_2 \cong \K[y, 1/y]$. We see that any element in $B_2$ is a sum
of units, whilst the only units in $B_1$ are scalars, so the two algebras can't
be isomorphic. \\

Now consider $f$ again. We will show that we can obtain either $g_1$ or $g_2$
from $f$ by a linear change of variables. \\ 

\textbf{case $a_5 = a_3 = 0$.} Since $f$ is quadric, we have $a_4 \not = 0$
in this case, so we can write
\[
	f(x, y) 
	=
	a_4 
	\left(
		xy
		+
		\frac{a_2}{a_4} x
		+
		\frac{a_1}{a_4} y
	\right)
	+
	a_0
	=
	a_4 
	\left(
		x
		+
		\frac{a_1}{a_4}
	\right)
	\left(
		y
		+
		\frac{a_2}{a_4}
	\right)
	-
	\frac{a_1a_2}{a_4}
	+
	a_0.
\] 
Since $f$ is irreducible, we can assume that $\frac{a_1a_2}{a_4} - a_0 \not =
0$, and after relabeling we have
\[
f = (ax + b)(y + c) - 1
\] 
where we were able to rescale $f$ so that $\frac{a_1a_2}{a_4} - a_0 = 1$ since
we only care about $V(f)$. We now have that $f(x, y) = g_2(ax + b, y + c)$, so
$\phi : \Kx \to \Kx$ where $\phi(x, y) = (ax + b, y + c)$ takes $I(g_2)$ to
$I(f)$, hence it induces a well defined homomorphism $\overline{\phi} :
A(V(g_2)) \to A(V(f))$, where $\overline{\phi}(h(x, y) + I(g_2)) = h(ax + b, y
+ c) + I(f)$, which is invertible with $\overline{\phi}^{-1}(h(x, y) + I(f)) =
h\left(\frac{x - b}{a}, y - c\right) + I(g_2)$. We've shown that $A(V(f)) \cong
A(V(g_2))$ whence $V(f) \cong V(g_2)$ by Corollary 4.8. \\

\textbf{case $a_5 \not = 0, a_4 \not = 0, a_3 = 0$}. By permuting variables,
this case covers when $a_3 \not = 0, a_4 \not = 0, a_5 = 0$ as well. We care
only about $V(f)$ so we can assume that $a_5 = 1$ and 
\begin{align*}
	f(x, y) 
	&= 
	x^{2} + a_4 xy + a_2 x + a_1 y + a_0. \\
	&= 
	\left(x + a_4y + a_2 - \frac{a_1}{a_4}\right)
	\left(x + \frac{a_1}{a_4}\right)
	+ a_0 - \frac{a_2a_1}{a_4} + \frac{a_1^{2}}{a_4^{2}},
\end{align*} 
and $a_0 - \frac{a_2a_1}{a_4} + \frac{a_1^{2}}{a_4^{2}} \not = 0$
since $f$ is reducible so after relabeling and division we get
\begin{align*}
	a f(x, y) 
	=
	\left(a x + b y + c\right)
	\left(x + d\right)
	- 1,
\end{align*} 
which just like in the previous case gives us an isomorphism between the $A(V(g_2))$ and $A(V(af)) = A(V(f))$
given by
\[
	\phi(h(x, y) + (g_2)) = h(ax + by + c, x + d) + (f)
\]
and 
\[
	\phi^{-1}(h(x, y) + (f)) = h\left(y - d, \frac{x}{b} + \frac{by}{a} - \frac{db}{a} - \frac{c}{b} \right) + (g_2).
\]


\textbf{case $a_5 \not = 0, a_4 = a_3 = 0$}. By permuting variables,
this case covers when $a_3 \not = 0, a_4 = a_5 = 0$ as well. We care
only about $V(f)$ so we can assume that $a_5 = 1$ and 
\begin{align*}
	f(x, y) 
	&= 
	x^{2} + a_2 x + a_1 y + a_0,
\end{align*} 
which gives us an isomorphism between the $A(V(g_1))$ and $A(V(f))$
given by
\[
	\phi(h(x, y) + (g_1)) = h(x, a_2 x + a_1 y + a_0) + (f)
\]
and 
\[
	\phi^{-1}(h(x, y) + (f)) = h\left(x, \frac{y}{a_1} - \frac{a_2}{a_1}x - \frac{a_0}{a_1}\right) + (g_1).
\]

\textbf{case $a_5 \not = 0, a_3 \not = 0$}. We care
only about $V(f)$ so we can assume that $a_5 = 1$ and 
\begin{align*}
	f(x, y) 
	&= 
	x^{2} + a_4 xy + a_3 y^{2} + a_2 x + a_1 y + a_0. \\
	&= 
	\left(
		x
		+
		Hy
		+
		P
	\right)
	\left(
		x
		+
		Ty
		+
		Q
	\right)
	+ 
	a_0 
	- 
	QP
\end{align*} 
where
\[
	H 
	= 
	\frac{a_4}{2} + \sqrt{\frac{a_4^{2}}{4} - a_3},\ 
	T 
	= 
	\frac{a_4}{2} - \sqrt{\frac{a_4^{2}}{4} - a_3},\ 
\] 
and
\[
	Q = \left(1 - \frac{H}{T}\right)^{-1}\left(a_2 - \frac{1}{T} a_1\right)
	P = \left(1 - \frac{T}{H}\right)^{-1}\left(a_2 - \frac{1}{H} a_1\right)
\] 
when $a_4^{2} \not = 4 a_3$. We get an isomorphism between $A(V(g_2))$ and $A(V(f))$ like in
previous cases but we skip the final calculations. This leaves one more case. \\

\textbf{case $a_5 \not = 0$, and $a_4^{2} = 4 a_3$.}
We can assume that $a_5 = 1$ and get
\begin{align*}
	f(x, y) 
	&= 
	a_4 x^{2} + a_4^{2} xy + a_3 y^{2} + a_2 x + a_1 y + a_0 \\
	&= 
	(x + \sqrt{a_3}y)^{2}
	+
	(a_4 - 2\sqrt{a_3}) y
	+
	a_2 x
	+ 
	a_1 y
	+ 
	a_0 \\
	&= 
	(x + \sqrt{a_3}y)^{2}
	+
	a_2 x
	+ 
	a_1 y
	+ 
	a_0,
\end{align*}
where we picked the square root of $a_3$ which yields cancelation as desired.
We see that $A(V(f))$ is isomorphic to $A(V(g_1))$ in this case, and also that
this case generalizes the $a_4 = a_3 = 0$ case.

\subsection*{Ex 4.13} 

Let $\catname{Aff}$ denote the category of affine varieties on, and
$\catname{Coord}$ denote the category of coordinate rings. 

\begin{lemma}
	Let $F : \catname{Aff} \to \catname{Coord}$ denote the map which assigns
	each affine variety it's coordinate ring. Then $F$ is an invertible
	contravariant functor, and $\catname{Aff}^{\op} \cong \catname{Coord}$ as
	categories.
\end{lemma}
\begin{proof}
	Remark 1.16 tells us that $F$ is bijective on the objects, and Corollary
	4.8 tells us that $f \mapsto f^{*}$ is bijective on the arrows. What
	remains to show is that $f \mapsto f^{*}$ preserves composition in a
	contravariant way. This is straightforward to verify. Let
	\[
		X \xrightarrow{f} Y \xrightarrow{g} Z
	\]
	be affine varieties and morphisms between them, and $h \in A(Z)$ be some
	regular function on $Z$. Then 
	\[
		(g \circ f)^{*}(h)
		=
		h \circ (g \circ f)
		=
		(h \circ g) \circ f
		=
		f^{*}(g^{*}(h))
		=
		(f^{*} \circ g^{*}) (h)
	\] 
\end{proof}

The previous lemma tells us that epimorphisms (right cancelative maps), are
mapped to monomorphisms (left cancelative) and vice versa. The problem though
is that epimorphisms/monomorphisms need not be surjective/injective
homomorphisms. The converse is true though, as we will soon see in the part
(a).

\subsubsection*{(a)}

Suppose that $f$ is surjective. Then $f$ is right cancelative in the category
of $\Set$, hence it's right cancelative and an epimorphism in the category of
$\catname{Aff}$. It follows that $f^{*}$ is a monomorphism in the category
$\catname{Coord}$. Now let $g \in \ker(f^{*})$, and consider the algebra
homomorphism $\phi : \K[x] \to A(Y)$ given by $\phi(x) = g$. Then $f^{*} \circ
\phi = f^{*} \circ 0$, which since $f^{*}$ is a monomorphism yields that $\phi$
is the zero map, so $g = 0$ in $A(Y)$ and $f^{*}$ is injective. \\

The converse is not true unfortunately, and this has to do with cases where the
image of $f$ is dense in $Y$, but not surjective. Consider for example $X = Y =
\A^{1}$, and $f : X \to Y$ given by $f(x) = x^{2}$. Then $f$ isn't surjective,
but $\im(f)$ is dense in $Y$, so any $g_1, g_2 \in A(Y)$ that agree on $\im(f)$
agree on all of $Y$. Hence $f^{*}(g_1) = f^{*}(g_2) \Rightarrow g_1 = g_2$ and
$f^{*}$ is injective.

\subsubsection*{(b)}

No, Example 4.9 exhibits a bijective morphism $f$ of varieties $f: \A^{1} \to
V(x_1^{2} - x_2^{3})$, but the corresponding algebra homomorphism $f^{*} :
A(V(x_1^{2} - x_2^{3})) \to \K[t]$ is not surjective.

\subsubsection*{(c)}

Yes, isomorphisms $\A^{1} \to \A^{1}$ correspond to polynomial maps which are
invertible by polynomial maps, and these are always linear (in the univariate
case).

\subsubsection*{(d)}

No! Consider for example $f(x, y) = (x, x^{2} + y)$. This map is regular and
bijective, and it's inverse is given by $f^{-1}(x, y) = (x, y - x^{2})$, which
is also regular and bijective.

\subsection*{Ex 4.19} 

\textbf{(a) $\cong$ (c)}. 
An isomorphism $f : \A^{1} \setminus \{1\} \to V(x_2 - x_1^{2}, x_3 - x_1^{3})
\setminus \{0\}$ is given by
\[
	f(x)
	=
	\left((x - 1), (x - 1)^{2}, (x - 1)^{3}\right),
\] 
where the inverse is
\[
	f^{-1}(x_1, x_2, x_3)
	=
	x_1 + 1.
\] 

\textbf{(d) $\not \cong$ (a)}. 
We have that $V(x_1x_2)$ is reducible so $A(V(x_1x_2))$ is not an integral
domain. Meanwhile, $\A^{1} \setminus \{1\} \cong D(x - 1)$ has coordinate ring
$\K[x]_{x - 1}$ which is an integral domain. \\

\textbf{(a) $\not \cong$ (e)}. 
We just saw that the coordinate ring of $\A^{1} \setminus \{1\}$
is local. This is not the case for $V(x_2^{2} - x_1^{3} - x_1^{2})$
since every point on the (infinite) variety induces a maximal ideal.

\textbf{(a) $\not \cong$ (f)}. 
We just saw that the coordinate ring of $\A^{1} \setminus \{1\}$ is local. This
is not the case for $V(x_2^{2} - x_1^{2} - 1)$ since every point on the
(infinite) variety induces a maximal ideal.

\textbf{(b) $\cong$ (d)}. 
We have $V(x_1^{2} + x_2^{2}) = V((x_1 - ix_2)(x_1 + ix_2))$, and if $a, b \in \A^{2}$
is such that $ab = 0$, then
\begin{align*}
	ab = 0
	\Leftrightarrow
	\left(\frac{a + b}{2} + i \frac{a - b}{2i}\right)
	\left(\frac{a + b}{2} - i \frac{a - b}{2i}\right) 
	= 0,
\end{align*}
so the isomorphism $f : \A^{2} \to \A^{2}$ $f(x_1, x_2) = \left(\frac{x_1 +
x_2}{2}, \frac{x_1 + x_2}{2i}\right)$ sends $V(x_1x_2)$ to $V((x_1 + ix_2)(x_1
- ix_2))$. \\

\textbf{(d) $\not \cong$ (e)}. 
We claim that $f = -x_2^{2} + x_1^{3} + x_1^{2}$ is irreducible. To see this,
note that any factors would need to be of the form
\begin{align*}
	(ax_1^{2} + bx_1 + cx_2 + d)
	(Bx_1 + Cx_2 + D)
	= \\
	aB x_1^{3} 
	+ 
	aC x_1^{2}x_2 
	+
	(aD + bB) x_1^{2}
	+
	(bC + Bc) x_1 x_2
	+
	cC x_2^{2}
	+
	(b + B) x_1
	+
	(c + C) x_2
	+ 
	dD
\end{align*} 
We get
\[
	aB = 1, cC = -1, aC = 0
\] 
which is impossible. It follows that $A(V(f))$ is an integral domain whilst
$A(V(xy))$ isn't so the two varieties are not isomorphic. \\

\textbf{(d) $\not \cong$ (f)}. 
We claim that $x_1^{2} - x_2^{2} - 1$ is irreducible. To see this,
note that any factors would need to be of the form
\[
	(ax_1 + bx_2 + c)
	(Ax_1 + Bx_2 + C)
	=
	aAx_1^{2} + bBx_2^{2} + cC
	+
	(aB + Ab)x_1x_2 + (aC + Ac)x_1 + (bC + bC) x_2.
\] 
so we need $a, b, c, A, B, C \in \C$ such that 
\begin{align*}
	aA &= 1 \\
	bB &= -1 \\
	cC &= -1 \\
	aB &= -Ab \\
	aC &= -Ac \\
	bC &= -Bc. \\
\end{align*}
If we fix $a = 1$ we get $A = 1$, then $B = -b$, $c = -C$, after which we get
$b = 1, B = -1$ or $b = -1, B = 1$. Assume the first case, then $bC = -Bc$
turns into $C = c$, contradicting our previous equation $c = -C$ since $c \not
= 0$ due to $cC = -1$. It follows that $A(V(x_1^{2} - x_2^{2} - 1))$ is an
integral domain which $A(V(x_1x_2))$ is not. \\


\textbf{(e) $\not \cong$ (f)}. 


Let $X = V(x_1^{2} - x_2^{2} - 1), Y = V(x_2^{2} - x_1^{3} - x_1^{2})$. When
graphing the projection of $Y$ onto $\R$, we see that the resulting curve
intersects itself at the origin. Meanwhile $X$ is isomorphic to the unit circle
by $(x_1, x_2) \mapsto (x_1, ix_2)$, which doesn't intersect itself. Thus we
guess that investigating the behaviour of $A(X)$ near the origin might lead to
a proof that the two varieties aren't isomorphic. We give it a try by looking
at the completion of $A(X)$ by the ideal $I((0, 0)) = (x, y)$. Proposition
10.13 in Atiyah MacDonald tells us that 
\[
	\widehat{A(Y)} 
	\cong 
	\K[[x_1, x_2]] \otimes_{\K[x_1,x_2]} A(Y)
	\cong 
	\K[[x_1,x_2]/I(Y).
\] 
We now claim that $f_Y$ factors in $\K[[x_1, x_2]]$, whence $\widehat{A(Y)}$
isn't a domain. We will use Hensel's Lemma to show this, more specifically, the
variant given in Theorem 7.3 in Eisenbud. We use the lemma to find a square root
of $x_1 + 1$ by searching for a solution of $g(z) = X^{2} - (x_1 + 1) = 0$
in $\widehat{A(Y)}[z]$. We have that $g'(X)^{2} = 4X^{2}$, and 
\[
	g(1) = 1 - x_1 - 1 = x_1 \equiv 0 \text{ mod } g'(1)(x_1, x_2) = (x_1, x_2),
\] 
whence applying the lemma gives us a root $b$ of $g$ in $\widehat{A(Y)}$, so
that $x_1 + 1 = b^{2}$ and $f_Y = x_2^{2} - x_1^{2}(x_1 + 1) = (x_2 - x_1b)
(x_2 + x_1b)$. \\

Since $X$ is isomorphic to the unit circle, we will just say that $X$ is the
unit circle from now on. We will now show that every completion of $A(X)$ is a
domain, after which the non-isomorphism will follow. Consider the completion of
$A(X)$ at the point $p$ on $X$ (I.e the maximal ideal $I(p)$). This is
isomorphic to the completion of $A(I(f_X(x_1 + p_1, x_2 + p_2)))$ at the
origin, where $f_X = x_1^{2} + x_2^{2} - 1$ (since we're just "moving" the
circle such that $p$ winds up at the origin). Denote the new polynomial by
$f_{X, p}$. Then
\begin{align*}
	f_{X, p} 
	&= 
	(x_1 + p_1)^{2} + (x_2 + p_2)^{2} - 1 \\
	&= 
	x_1^{2} + 2x_1p_2 + p_1^{2} + x_2^{2} + 2x_2p_2 + p_2^{2} - 1 \\
	&=
	x_1^{2} + 2x_1p_2 + x_2^{2} + 2x_2p_2
\end{align*} 
since $p$ is on the circle and $p_1^{2} + p_2^{2} = 1$. Furthermore,
\begin{align*}
	f_{X, p}
	&=
	x_1^{2} + 2x_1p_2 + x_2^{2} + 2x_2p_2
	&=
\end{align*}


We are done if we can show that $f_{X, p}$ is
irreducible in $\K[[x_1, x_2]]$ for all $p$. TODO Todo todo finnish

\textbf{Conclusion}:
We have the following equivalence classes
\begin{align*}
	(a) &\cong (c), \\
	(b) &\cong (d), \\
	&(e), \\
	&(f), \\
\end{align*}

\subsection*{Ex 5.7} 

\subsubsection*{(a)} 

I was stuck here for a long time, so we'll solve this exercise in a perhaps
overly detailed manner. \\

Let $X_1 = X_2 = \AA{1}$, and $U_i \subset X_i$ be
$D(x)$, I.e $\AA{1} \setminus 0$. Let $\PP{1}$ be $X_1$ glued with $X_2$ via
the isomorphism $\phi : U_1 \to U_2$, where $\phi : x \mapsto 1/x$. Let $i_1,
i_2$ be the injection maps from $X_1, X_2$ to $\PP{1}$. \\

Let $Y$ be an prevariety and $f : Y \to \PP{1}$ be a map. Let $Y_1 =
f^{-1}(i_1(X_1)), Y_2 = f^{-1}(i_2(X_2))$. We claim that $f$ is a morphism
precisely when both
\begin{align*}
	f_1 &: Y_1 \to \AA{1}, f_1 : y \mapsto i_1^{-1}(f(y)), \\ 
	f_2 &: Y_2 \to \AA{1}, f_2 : y \mapsto i_2^{-1}(f(y)).
\end{align*}
First of, we have that $f$ is a morphism if and only if both
\begin{align*}
	\hat{f}_1 &: Y_1 \to \PP{1}, \hat{f}_1 : y \mapsto f(y), \\ 
	\hat{f}_2 &: Y_2 \to \PP{1}, \hat{f}_2 : y \mapsto f(y), 
\end{align*}
are morphism by Remark 4.5 (b) and Lemma 4.6. Also, the injections $i_1, i_2$
are both isomorphisms onto their images (they are both their own inverse),
hence each $\hat{f}_k$ is a morphism exactly when $f_k = i_k^{-1} \circ
\hat{f}_k$ is a morphism. \\

Now, set $Y = \AA{1} \setminus 0 = D(x)$. Then $Y_1, Y_2$ are both open in
$\AA{1}$, and all such sets are distinguished. Thus we can suppose $Y_1 =
D(h)$ and we can write
\[
	f_1 = \frac{g}{h^{k}}.
\]
After canceling common factors and extracting all powers of $x$
we are left with an expression of the form
\[
	f = x^{m} \frac{G}{H}
\] 
where $m \in \Z$ and $(G, H) = 1, x \not | G$, $x \not | H$. But $i_1^{-1}$ is
the identity, so $f = f_1$ on $D(h_1)$ and $H f - x^{m} G = 0$ here. But
$D(h_1)$ is dense in $Y$, so $H f - x^{m} G = 0$ on all of $D(x)$ and 
\[
	f = x^{m} \frac{G}{H}
\] 
Dividing by $H$ might seem suspicious here, but this results in a well-defined
function since $(G, H) = 1$. Indeed, we can send any root $r$ of $H$ to $r
\mapsto i_2(0) = "\infty"$, after which chasing the definitions yields that $Y_1
= D(xH), Y_2 = D(xG)$ and $f_1 = f, f_2 = 1/f$ which are both well-defined
morphisms on their corresponding domains. \\ 

If $m = 0$, then $f(0) \in \AA{1} \setminus 0 = i_1(X_1)
\cap i_2(X_2)$ and we can trivially extend the domain of $f$ to all of
$\AA{1}$. If $m > 0$, then $f(0) = 0$ and we can extend $f_1$ by sending $f_1 :
0 \mapsto 0$, whilst we can leave $f_2$ untouched as $0 \not \in i_2(X_2)$ and
$0 \not \in Y_2$. If $m < 0$ we have $f(0) \not \in i_1(X_1)$, so we only need
to worry about $f_2$. Since $i_2^{-1} : x \mapsto 1/x$, we have that $f_2 =
x^{-m} \frac{H}{G}$ and we can simply extend $f_2$ by sending $f_2 : 0 \mapsto
0$ again.

\subsubsection*{(b)} 

In part (a) we had that every regular function from $D(x) \to \PP{1}$
could be written as $x^{m} \frac{G}{H}$ coprime, and as $G, H$ are 
univariate polynomials, they'll never be simultaneously zero. This is not
the case for coprime polynomial in two variables. \\

Consider the regular function $f : \AA{2} \setminus 0 = D(x) \cup D(y) \to
\PP{1}$ given by
\[
	f(x, y) = \frac{y}{x}
\] 
Then $f$ is regular because (using the notation of part (a) with $Y = D(x) \cup
D(y)$), $Y_1 = D(x)$, $Y_2 = D(y)$ and $f_1 = \frac{y}{x}, f_2 =
\frac{x}{y}$ are both quotient of polynomials with non-vanishing denominators
on their domains. \\

Now suppose towards a contradiction that $\hat{f}$ is an extension of $f$ to
the affine plane and let $\hat{f}_k = i_{k}^{-1} \circ
\restr{\hat{f}}{\hat{f}^{-1}(i_k(X_k))}$ as before. Then $(0, 0)$ must lie in
one or both of $\hat{f}^{-1}(i_1(X_1))$ or $\hat{f}^{-1}(i_2(X_2))$, and we
assume the index $1$ case. Now let $U_0$ be some open neighbourhood of $(0, 0)$
such that $\hat{f}_1$ is a rational function on $U_0$. We have $\hat{f}_1 = y/x$ on
$D(x) \cup D(y)$, hence on $U_0 \setminus (0, 0)$, and there is no rational
function on any open set $U$ containing $0$ which is $y/x$ on $U \setminus (0,
0)$, so we arrive in a contradiction.

\subsubsection*{(c)} 

Let $f : \PP{1} \to \AA{1}$ and $f_1 = f \circ i_1, f_2 = f \circ i_2$. Then
$f_1, f_2$ are both morphisms $\AA{1} \to \AA{1}$ by the glueing construction
of $\PP{1}$. On $\AA{1} \setminus \{0\}$ we have that $f_1(x) = f_2(1/x)$,
again by the construction of $\PP{1}$. Since $f_1, f_2$ are both in $A(\AA{1})
= \K[x]$, this is possible only if $f_1 = f_2 \in \K$.


\subsection*{Ex 5.8} 

We will solve these exercises using homogeneous coordinates. That is, we will
write points $i_1(x) = [x:1]$ and $i_2(x) = [1:x]$ with the equivalence
$[x_1:x_2] = \lambda [x_1:x_2]$ for all $\lambda \in \K \setminus 0$. Then
$i_1$ is injective since $i_1(x_1) = i_1(x_2)$ implies $[x_1 : 1] = [x_2 : 1]$
which is the case only when $x_1 = x_2$. Moreover if $x \not = 0$,
\[
	i_1(x) = [x:1] = 1/x [x:1] = [1:1/x] = i_2(1/x)
\] 
and we see that the equivalence classes $[x_1:x_2]$ correspond exactly to the
points on $\PP{1}$.



\subsubsection*{(a)} 

We begin by retracing some of the steps of Exercise 5.7 (a)
with this new notation to show the following lemma.

\begin{lemma}
	Let $f : \AA{1} \to \PP{1}$ be a morphism. Then $f$ must be of the form $f(x) =
	[p(x):q(x)]$, where $p, q \in \Kx$ are polynomials with no root in common. 
\end{lemma}
\begin{proof}
	Let $Y_1 = f^{-1}(i_1(\AA{1})), Y_2 = f^{-1}(i_2(\AA{1}))$ and $f_1 =
	i_1^{-1} \circ \restr{f}{Y_1},\ f_2 = i_2^{-1} \circ \restr{f}{Y_2}$. Then
	both $f_k$ are morphisms $\AA{1} \supseteq Y_k \to \AA{1}$, hence they can
	be written as 
	\[
		f_k = \frac{g_k}{h_k}
	\] 
	with $(g_k, h_k) = 1$. But then 
	\[
		\restr{f}{i_1(Y_1)} = i_1 \circ f_1 = [g_1(x)/h_1(x):1] = [g_1(x):h_1(x)],
	\] 
	and as $\PP{1}$ is irreducible and $Y_1$ is dense in $\PP{1}$, we have
	\[
		f = [g_1(x):h_1(x)]
	\] 
	on all of $\PP{1}$ and we are done. Note that it also follows that $g_1 =
	h_2, g_2 = h_1$.
\end{proof}

Now let $f : \PP{1} \to \PP{1}$ be an automorphism. From the lemma it follows
that we can write
\begin{align*}
	\restr{f}{i_1(\AA{1})}([x:1]) &= [p_1(x) : q_1(x)], \\
	\restr{f}{i_2(\AA{1})}([1:x]) &= [p_2(x) : q_2(x)],
\end{align*}
and as these expressions agree on the intersection $i_1(\AA{1}) \cap i_2(\AA{2})
= \{[x_1 : x_2] = [x_1/x_2 : 1] : x_1,x_2 \in \K \setminus 0 \}$, we have 
\[
	[p_1(x_1/x_2) : q_1(x_1/x_2)] = [p_2(x_1/x_2) : q_2(x_1/x_2)]
\] 
which in turn yields 
\[
	p_1(x)q_2(x) = p_2(x)q_1(x)
\] 
whenever $x \not \in V(q_1, q_2)$, but if $x \in V(q_1, q_2)$ then both sides are $0$
so we have 
\[
	p_1(x)q_2(x) = p_2(x)q_1(x)
\] 
everywhere and $(p_k, g_k) = 1$ yields $p_1 = c p_2, q_1 = c q_2$. If
we now relabel $p = p_1, q = q_1$, we get
\[
	f([x_1 : x_2]) = [p(x_1/x_2) : q(x_1/x_2)]
\] 
when $x_2 \not = 0$ and 
\[
	f([x_1 : x_2]) = [c p(x_2/x_1) : c q(x_2/x_1)] = [p(x_2/x_1) : q(x_2/x_1)]
\] 
when $x_1 \not = 0$. Moreover, since $f$ is injective, we have that $f$ is
injective on $i_1(\AA{1})$, hence $p, q$ must have degree $1$ and we are done.

\subsubsection*{(b)} 


\subsection*{Ex 5.9} 

For any ringed spaces $X, Y$, we always have that for any morphism $f : X \to
Y$ the pull-back $f^* : \mathcal{O}_Y(Y) \to \mathcal{O}_X(X)$ is an algebra
homomorphism. This is required by the definition of a morphism of a ringed
space. Thus we will only focus on the other direction for both subproblems.

\subsubsection*{(a)} 

Let $f^* : \mathcal{O}_Y(Y) \to \mathcal{O}_X(X)$ be an algebra homomorphism.
Let $y_1, y_2, \ldots, y_n$ be the coordinate functions on $Y$, and $\phi_i =
f^*(y_i)$. We claim that
\[
	f = (\phi_1, \phi_2, \ldots, \phi_n)
\] 
is a morphism $X \to Y$. \\ 

Let $U_i$ be a cover of $X$ by affine varieties. It follow from the definition
of a sheaf that $f^*$ followed by restriction to any $U_i$ is an algebra
homomorphism $\mathcal{O}_Y(Y) \to \mathcal{O}_{Y}(U_i)$ and we write this as
\[
	f_i^* = \restr{?}{U_i} \circ f^*.
\] 
From the proof of Corollary 4.8, we know that $f_i^*$ is the pullback of
$\restr{f}{U_i}$, hence $\restr{f}{U_i}$ is a morphism $U_i \to X$ and $f$ is a
morphism by Lemma 4.6.

\subsubsection*{(b)} 

No, for example Exercise 5.7 (c) tells us that $\mathcal{O}_{\PP{1}}(\PP{1})
\cong \K$, so there is only one non-zero $\K$-algebra homomorphism
$\mathcal{O}_{\PP{1}}(\PP{1}) \to \mathcal{O}_{\AA{1}}(\AA{1}) = \K[x]$ and
that is the injection $\K \to \K[x], k \mapsto k$. Meanwhile, from the
construction of $\PP{1}$ we already have two non-trivial homomorphism $i_1,
i_2$ for $\AA{1} \to \PP{1}$.


\subsection*{Ex 5.11} 

Let $Y \subseteq X$ be a closed subset and $U \subseteq X$ be an open affine
subset. Let $i : U \to \An$ be an embedding into affine space. Then $i(U \cap
Y)$ is closed in $\An$, hence a zero set of some polynomials $V(f_1, f_2,
\ldots, f_m)$. In other words, $U \cap Y$ is isomorphic to $V(f_1, f_2, \ldots,
f_m)$ via $i$ as ringed spaces. Hence $U \cap Y$ is an affine open set of $X$. \\

It follows that if $U_i$ is a finite cover of $X$ by affine open sets, then $Y
\cap U_i$ is a finite cover of $Y$ by affine open sets, and $Y$ is a
prevariety. \\

Moreover, the structure sheaf on $Y$ obtained from gluing the $U_i \cap Y$
together agrees with that of Construction 5.10 (b). Let $U \subset Y$ be an
open set, $\mathcal{O}_y$ be the structure sheaf from Construction 5.10 (b) and
$\mathcal{O}'_Y$ be the structure sheaf obtained from glueing $U_i \cap Y$. \\

We have $f \in \mathcal{O}'_Y(U)$ precisely when $f$ is regular on all $U_i
\cap Y$. Now let $a \in U$ and $i$ be such that $a \in U_i$. Then let $F$ be
any element in the preimage of $f$ under the quotient homomorphism $A(U_i) \to
A(U_i \cap Y) = A(U_i) + I(Y)$. Then $f$ and $F$ agree on $U_i \cap Y$, so $f
\in \mathcal{O}_Y(U)$ by Construction 5.10 (b). \\

Now let $f \in \mathcal{O}_Y(U)$. Then given a $U_i$ and $a \in U_i \cap Y$, we
have that there is some open $V_a \in X$ containing $a$ such that there is some
$g \in \mathcal{O}_X(V_a)$ and $\restr{f}{V_a \cap U} = \restr{g}{V_a \cap U}$.
But then $\restr{f}{V_a \cap U} \in \mathcal{O}_X(V_a \cap U)$, so given any $a
\in U_i \cap U$, we can find $V_a$ such that $\restr{f}{V_a \cap U}$ is
regular, hence $f$ is regular on $U_i \cap U$ and finally on $U$ by Lemma 4.6.



\subsection*{Ex 5.21} 

\begin{lemma}
	Let $V$ be a prevariety obtained by glueing the affine open sets $U_i$.
	Then $Z$ is closed in $V$ if and only if every $Z \cap U_i$ is closed. 
\end{lemma}
\begin{proof}
	We have that $Z$ is closed if and only if $V \setminus Z$ is open, which
	happens if and only if every $U_i \cap (V \setminus Z) = U_i \setminus Z$
	is open, and this is the case if and only if every $U_i \cap Z$ is closed.
\end{proof}

Let $i_1, i_2$ be the injections into $\PP{1}$ and define $X_1 = i_1(\AA{1}),
X_2 = i_2(\AA{1})$. Then $\PP{1} \times \PP{1}$ can be obtained by glueing
the patches $X_i \times X_j$, $i, j \in \{1, 2\}$. The diagonal $\Delta_{\PP{1}}$
intersects $X_1 \times X_1$ and $X_2 \times X_2$ as $\Delta_{\AA{1}}$ and is
thus closed there. It intersects $X_1 \times X_2, X_2 \times X_2$ as $V(xy -
1)$ and is closed there as well. Hence $\Delta_{\PP{1}}$ is closed in
$\PP{1} \times \PP{1}$ and $\PP{1}$ is separated.

\subsection*{Ex 5.22} 

\subsubsection*{(a)} 

Let $\pi_{xx} : (X \times Y) \times (X \times Y) \to X \times X$ be the
projection morphism onto the two $X$ coordinates. Then the inverse image of the
diagonal of $X$,
\[
	\pi_{xx}^{-1}(\Delta(X))
	=
	\{(x_1, y_1, x_2, y_2) \in (X \times Y) \times (X \times Y) : x_1 = x_2\}
\]
is closed in $(X \times Y) \times (X \times Y)$, and similarly, we have  
\[
	\pi_{yy}^{-1}(\Delta(Y))
	=
	\{(x_1, y_1, x_2, y_2) \in (X \times Y) \times (X \times Y) : y_1 = y_2\}
\]
closed as well. Intersecting the two yields $\Delta_{X \times Y}$ and we are
done.

\subsubsection*{(b)} 

Let $U_i, V_i$ be finite open affine coverings of $X, Y$ respectively. Then
$U_i \times V_j$ is a finite open covering of $X \times Y$. Every $U_i, V_j$ is
irreducible, since if $A \cup B = U_i$ are two closed non-trivial sets in
$U_i$, then $U_i \setminus A, U_i \setminus B$ are open in $U_i$ and in $X$,
whence $X \setminus (U_i \setminus A) = A \cup (X \setminus U_i)$ and $B \cup
(X \setminus U_i)$ are two closed non-trivial sets. Their union is 
\[
	(A \cup (X \setminus U_i))
	\cup
	(B \cup (X \setminus U_i))
	=
	(A \cup B)
	\cup
	(X \setminus U_i)
	=
	U_i
	\cup
	(X \setminus U_i)
	=
	X.
\] 
We now have that  every $U_i \times V_j$ is irreducible by Exercise 2.24. If we
can show that every $U_i \times V_j$ intersects every $U_r \times V_s$, then we
can apply Exercise 2.21 which yields $X \times Y$ irreducible. \\

Every $U_i, U_r$ and $V_j, V_s$ intersect since they are open sets in the
irreducible spaces $X, Y$, and it follows immediately that $U_i \times V_j$ and
$U_r \times V_s$ intersect. 

\subsection*{Ex 5.23} 

\subsubsection*{(a)} 

Let $U, V$ be affine open sets in the variety $X$, and $\pi_U : U \times V \to
X, \pi_V : U \times V \to X$ be the projections (followed by inclusions into
$X$). Then it follows from Proposition 5.20 (b) that $\{(u, v) \in U \times V:
\pi_U(u) = \pi_V(v)\} = \Delta_{U \cap V}$ is closed in $U \times V$. \\

Since both $U, V$ are affine, we have that $U \times V$ is affine as well. Let
$e : U \times V \xrightarrow{\sim} Z \subset \An$ be an embedding onto some
Zariski closed set in affine space. Then $e$ sends closed sets to closed sets.
Hence $e(\Delta_{U \cap V})$ is closed in $\An$, so $\Delta_{U \cap V}$ is an
affine variety. \\

Finally, $f : U \cap V \to \Delta_{U \cap V} , f : x \mapsto (x, x)$ is an
isomorphism with inverse $f^{-1} : (x, x) \mapsto x$, and $U \cap V$ is affine
as well.

\subsubsection*{(b)} 

Like above, we have that $X \cap Y \cong \Delta_{X \cap Y}$, so it's enough to
consider the diagonal. Also like above, the diagonal $\Delta_{X \cap Y}$ is
closed in $X \times Y$, and $X \times Y$ is affine so embeddable into some
Zariski closed set in affine space, hence $\Delta_{X \cap Y}$ is as well. From
now on we identify all of our varieties as embedded into affine space this way
(that is we may assume that $\Delta_{X \cap Y}, X \times Y \subset \AA{2n}$ are
Zariski closed). \\

Now, let $U$ be some irreducible component of $\Delta_{X \cap Y}$. We have $X
\times Y$ irreducible by Exercise 5.22, so we can apply Proposition 2.28 (b) to
get

\begin{align*}
	\dim \Delta_{X \cap Y} 
	&=
	\dim X \times Y
	- 
	\codim_{X \times Y}(U) \\
	&=
	\dim X 
	+ 
	\dim Y
	-
	\codim_{X \times Y}(U).
\end{align*} 

Hence we are done if we can show that $\codim_{X \times Y}(U) \leq n$. This
follows from Lemma XYZ which says that $\codim_{X \times Y}(U)$ is the same as
the height of $I(U)$ in $A(X \times Y)$. Since $U$ is an irreducible component
of $\Delta_{X \times Y}$, we have that $I(U)$ is a minimal prime ideal of
$I(\Delta_{X \times Y})$, hence $I(U)$ and $I(\Delta_{X \times Y})$ have the
same height ? 

but we claim that the height of $I(U)$

Now, $I(U)$ is a minimal ideal of
$I(\Delta_{X, Y})$


, and since $U \subseteq \Delta_{X \cap
Y}$, we have $I(U) \supseteq I(\Delta_{X \cap Y})$

$\cong A(X) \otimes A(Y)$.

TODO todo Todo : Finnish!

\subsection*{Ex 5.24}

We will look locally enough where $X$ is affine, and reduce this problem to the
affine case which we already solved in Exercise 2.34 (b). \\

Let $X$ be a variety, $U \subseteq X$ a dense open subset and 
\[
Y_0 \subsetneq Y_1 \subsetneq \ldots \subsetneq Y_n \subset X
\] 
a chain of irreducible closed subsets. Then let $Z$ be a affine open set which
intersects $Y_0$. Since $U$ is dense, we have that $U$ intersects $Z$ as well.
Let $e : Z \to \An$ be an embedding into affine space such that $e(Z)$ is the
zero locust of some set of polynomials. All our sets intersect $Z$, and we
write $Z' = e(Z), Y_i' = e(Z \cap Y_i), U' = e(Z \cap U)$. \\

We can use the construction from Exercise 2.34 (a) to show that $Y'_{i - 1}
\subsetneq Y_i'$ are strict inclusions. In brief, we set $V_i = Y_{i - 1}
\setminus Y_i$, which is non-empty and open in $Y_i$ and intersects $Z \cap
Y_i$ since $Y_i$ is irreducible, hence $\emptyset \not = e(Z \cap V_i) \subset Y'_i
\setminus Y'_{i - 1}$. \\

Now we claim that $U'$ is dense in $Z'$, which is equivalent to $U'$
intersecting every open set of $Z'$ (and this is the property we will need
anyhow). Let $V'$ be open in $Z'$ and $V = e^{-1}(V)$. Then $V$ is open in $Z$,
and by the definition of the subspace topology, we have some open $V''$ in $X$
such that $V'' \cap Z = V$. But $Z$ is open, so $V$ is open in $X$, and $V$
intersects $U$ since $U$ is dense in $X$. \\

To recap, we now have a strict chain of closed irreducible subsets 
in affine space 
\[
	Y'_0 \subsetneq Y'_1 \subsetneq \ldots \subsetneq Y'_n \subset Z'
\] 
and a dense open set $U' \subseteq Z'$, and we are free to use the translation
of Exercise 2.34 (b) to conclude that $\dim U' = \dim U \geq n$. \\

The other inequality is Exercise 2.30.

\subsection*{Ex 6.14}

Write $a = [a_0 : a_1 : \ldots : a_n]$ and $e_{i, j} = a_j x_i - a_i x_j$. We
claim that $I_p(\{a\}) = (e_{i, j} : 0 \leq i < j \leq n)$. First note that
$e_{i, j}(a) = a_j a_i - a_i a_j = 0$ for all $i, j$. Moreover, suppose that $b
= [b_0 : b_1 : \ldots : b_n]$ is such that $e_{i,j}(b) = 0$ for all $i, j$.
Then $a_i b_j = a_j b_i$ for all $i, j$, which is exactly the same thing as
saying all of the minors of order $2$ of the matrix 
\[
\begin{pmatrix}
	a_0 & a_1 & \ldots & a_n \\	
	b_0 & b_1 & \ldots & b_n \\	
\end{pmatrix}
\] 
vanish, which in turn happens if and only if the matrix has at
most rank $1$, I.e $a = b$ in $\Pn$. We've shown that $V((e_{i, j} : 0 \leq i <
j \leq n)) = \{a\}$ and we are done.

\subsection*{Ex 6.29}

In affine space, a line $L_1$ and a point $a$ not on this line, span a plane $P_1$,
and any line which intersect both $L_1, a$ lies on this plane. If we then pick
a line $L_2$ which doesn't intersect on this plane, we see that no line which
intersects both $L_1, a$, also intersects $L_2$. Any lines $L_2$ which doesn't
intersect $P_1$ is parallel to $L_1$, and since
parallel lines in general don't intersect in $\AA{3}$, we see that it's
impossible to find a third line $L$ which intersects $L_1, L_2, a$ 
when $L_1, L_2$ are parallel. \\

There is one more case where it's impossible to find such $L$. Let $q$ be the
point where $L_2$ meets $P_1$. If $q, a$ spans a line which is parallel to $L_1$,
then it's again impossible to find $L$ which intersects $L_1, L_2, a$, since such
a line would have to contain both $q, a$, whence it wont meet $L_1$. \\

This latter case correspond to when the plane $P_2$ containing $L_2$ and 
$a$ is parallel to $L_1$. \\

Projective space is more flexible as we have points at infinity, and parallel
lines intersect at infinity. Here if either $P_1, L_2$ or $P_2, L_1$ are
parallel, then there is a line $L$ through $a$ in each corresponding plane
$P_i$ which intersects the other line $L_j$ $i \not = j$ at infinity. \\

We solve the problem in $\AA{4}$. Here the lines $L_1, L_2$ corresponds to
planes through the origin spanned by the vectors $u_1, u_2$ and $v_1, v_2$
respectively. $a$ corresponds to a line through the origin spanned by a vector,
which we will denote by $a$ as well. The planes $P_1, P_2$ now correspond to
three dimensional hypersurfaces spanned by $a, u_1, u_2$ and $a, v_1, v_2$
respectively. Their intersection is a plane which contains $a$, spanned by say
$a, w$. Then the intersection plane meets the hypersurface corresponding to
$L_1$ in at least one point, parameterised by say $sa + tw = s'a + t'u_1 + r'
u_2$, and we have that $(s + s')a + tw = t'u_1 + r' u_2$, so the plane $\langle
a, w \rangle$ meets the plane of $\langle u_1, u_2 \rangle $. The same is of
course true of the plane $\langle v_1, v_2 \rangle $, hence the line $L$ in
$\PP{3}$ corresponding to $\langle a, w \rangle $ will meet all of $a, L_1,
L_2$. \\

This line is unique, since the plane $\langle a, w \rangle$ is unique. Indeed,
since $L_1 \not = L_2$, we have that the intersection $\langle a, u_1, u_2
\rangle \cap \langle a, v_1, v_2 \rangle$ is exactly a plane.

\subsection*{Ex 6.30}

\subsubsection*{(a)}
The only if part is immediate. \\

Let $R$ be a homogeneous ring such that for any homogeneous $f, g \in R$, we
have $fg = 0$ implies $f = 0$ or $g = 0$, and suppose that $p,q \in R$ are such
that $pq = 0$. Write $r = \deg p, s = \deg q$, and let $p_r, q_s$ be the
leading homogeneous components. Since a graded ring is the direct sum of it's
homogeneous components, every homogeneous component of $pq$ must be zero, in
particular the leading component $(pq)_{rs} = p_r q_s$. Our assumption on $R$
then tells us that either $p_r = 0$ or $q_s = 0$, and since these are the
leading components, $p = 0$ or $q = 0$, and $R$ is an integral domain.


\subsubsection*{(b)}

Suppose $f, g \in S(X)$ are such that $fg = 0$. Then $X = V(fg) = V(f) \cup
V(g)$ and if $X$ is irreducible, we must have either $V(f) = X$ or $V(g) = X$,
whence $f = 0$ or $g = 0$ and $S(X)$ must be an integral domain. \\

Similarly, if $X$ is reducible, let $X = U \cup V$ be a non-trivial
decomposition and $f \in I_p(U), g \in I_p(V)$ be non-zero polynomials. Then
$fg$ vanishes on both $U$ and $V$ hence on $X$ and $fg = 0$, so $S(X)$ isn't an
integral domain.

\subsection*{Ex 6.31}

\subsubsection*{(a)}

By Lemma 6.18, any strictly increasing chain of varieties in $\PP{n}$
corresponds to a chain of cones in $\AA{n+1}$, to which we can append a point
immediately after the empty variety
a la
\[
	X_0 = \emptyset \subsetneq \{a\} \subsetneq C(X_1) \subsetneq C(X_2)
	\subsetneq \ldots \subsetneq C(X_m),
\]
hence $\dim C(X) \geq 1 + \dim X$. \\

For the other direction, we take the opportunity to explore homogeneous 
coordinate ring and pass to the algebraic side. We provide a sequence
of lemmas to this end. 

\begin{lemma}
	Let $X$ be a projective variety. Then 
	\[
		S(X) = A(C(X)).
	\] 
\end{lemma}
\begin{proof}
	It follows from Remark 6.17 that $I_p(X) = I_a(C(X))$, after which the
	equality of the lemma follows from the definitions of homogeneous and affine
	coordinate rings.
\end{proof}

\begin{lemma}
	Let $X$ be a non-empty irreducible projective variety. Then $S(X)$ contains
	a homogeneous prime ideal $J$ of height $1$.
\end{lemma}
\begin{proof}
	Let $x_i$ be a non-zero coordinate on $C(X)$. Then $(x_i)$ is a homogeneous
	ideal in $A(C(X))$. Moreover, $(x_i)$ is prime, as it's the image of a
	prime ideal under the surjective ring homomorphism $\pi: \Kx \to A(C(X)),
	\pi: x \mapsto x + I_a(C(X))$. Finally, $(x_i)$ is its own minimal ideal,
	and therefore has height at least one by Krull's Principal Ideal theorem,
	and height exactly one since $X$ is irreducible, and $S(X) = A(C(X))$ is an
	integral domain so $(0)$ is prime there.
\end{proof}

\begin{corollary}
	If $X$ is a projective variety, then $1 + \dim X \leq \dim C(X)$.
\end{corollary}
\begin{proof}
	The respective coordinate rings are equal, and by the lemma, $A(C(X))$
	contains a maximal chain of prime ideals which are all homogeneous, 
	and since all of these but possibly the irrelevant ideals correspond
	to irreducible projective varieties, the corollary follows.
\end{proof}

\subsubsection*{(b)}
We can reduce to the case where $X, Y$ are pure dimensional by simply ignoring
irreducible components of non-maximal dimension. It follows from part (a) that
$\dim C(X) + \dim C(X) \geq n + 2$, whence Exercise 5.23 (a) yields
\[
	\dim C(X) \cap C(Y) \geq \dim C(X) + \dim C(Y) - (n + 1) \geq 1.
\]
Now, if $x \in C(X) \cap C(Y)$ then $\lambda x \in C(X)$ and $\lambda x \in
C(Y)$ for all $\lambda$, hence $x \in X \cap Y$ and $x \in C(X \cap Y)$. The
other inclusion follows similarly, hence $C(X \cap Y) = C(X) \cap C(Y)$ has
dimension at least $1$, so it's at least a line in affine space, and least a
point in projective space.

\subsection*{Ex 6.36}

In the real plane, our surface is given by the graph $x_2 = \pm
\sqrt{\frac{1}{x_1} + x_1^{2}}$ when $x_1 \in \R \setminus (-1, 0]$. For large
$x_1$, we have two asymptotes where $x_2 \to x_1$ and $x_2 \to -x_1$. Hence we
expect two points at infinity, namely $a = [0 : 1 : 1], b = [0 : -1 : 1]$. \\

When $x_1$ goes to $0$ from the positive side we see that $x_2 \to \pm
\frac{1}{\sqrt{x_1}} \to \pm \infty$. So we might also expect the points at
infinity $c = [0 : 0 : 1], d = [0 : 0 : -1]$. However, they are not in the
closure as we can see bellow. Perhaps because we don't have any points on the
curve with $x_1 = 0$? \\

We see that $a, b$ are the only points at infinity, since $\overline{X} =
V_p(x_1^{3} - x_1x_2^2 + x_0^{3})$, which when intersected with $V_p(x_0)$
becomes $V_p(x_1^{2} - x_2^{2}) = V_p((x_1 - x_2)(x_2 + x_1))$, and consists of
$a, b$ exactly. 


\subsection*{Ex 7.3}

\subsubsection*{(a)}

Let $\PP{1}$ be the projective line as introduced in Example 5.5 (a), and
$\PP{1'}$ be the projective line as given in Definition 6.1, using Notation
6.2, with structure sheaf as in Definition 7.2. We will show that the two are
isomorphic. \\

As a set, we have that $\PP{1} = X_0 \coprod X_1 / \sim$ where $X_0 = X_1 =
\AA{1}$ and $x_0 \sim x_1$ for $x_0 \in X_0, x_1 \in X_1$ whenever $x_0, x_1
\not = 0$ and $x_0 = 1/x_1$. \\ 

We claim that the function $f : \PP{1} \to \PP{1'}$ which sends $f(x_0) = [x_0 :
1], x_0 \in X_0$ and $f(x_1) = [1 : x_0], x_1 \in X_1$ is well-defined and
bijective. To show that it's well-defined, let $x_0 \sim x_1$. Then $x_0, x_1
\not = 0$ and
\[
	f(x_1) = [1 : x_1] = 1/x_1 [1 : x_1] = [1/x_1 : 1] = [x_0 : 1] = f(x_0).
\]
Now suppose that $f(x_0) = f(x_1)$. This is impossible whenever either $x_0 =
0$ or $x_1 = 0$, so we have $x_0, x_1 \not = 0$. Then
\[
	[1 : x_1] = [x_0 : 1] = [1 : 1/x_0]
\]
hence $x_1 = 1/x_0$ and $x_0 \sim x_1$. Finally, let $x = [x_0 : x_1] \in
\PP{1'}$. If $x_0 = 0$, then $x = f(0_0)$, otherwise $x = f(x_1/x_0)$ with
$x_1/x_0 \in X_1$. Hence $f$ is a bijective function. \\

Now suppose that $Z \subseteq \PP{1}$ is a closed set. Then $Z_0 = Z \cap X_0,
Z_1 = Z \cap X_1$ are both closed in $\AA{1}$. Let $J_0 = I(Z_0), J_1 =
I(Z_1)$. Then let $g \in J_1$ and $x_1 \in Z_1$. We have that $g^{h}([1 : x_1])
= g(x_1) = 0$, hence $g^{h}$ vanishes on $f(Z_1)$. Similarly, now let
$g^{h} \in J_1^{h}$ and $[x_0:x_1] \in V_p(J_0^{h})$. First of, every term but
the leading term of $g^{h}$ has a factor $x_0$, hence $g^{h}([0 : 1]) \not = 0$
and $x_0 \not = 0$. Thus we can assume $x_0 = 1$. Now, $(g^{h})^{i} = g$, so
$g(x_1) = g^{hi}(x_1) = g^{h}([1 : x_1]) = 0$. It follows that $f(Z_k) =
V_p(J_k^{h})$ for $k = 0,1$, and $f(Z) = V_p(J_0^{h}) \cup V_p(J_1)^{h}$
is closed. \\

Now let $Z \subseteq \PP{1'}$ be closed, and suppose $Z = V_p(J)$. Then let $g
\in J$, $[1 : x_1] \in Z \cap U_0$. Before we move on, we need some new
notation. Let $g^{i_0} = g(1, x_1)$ and $g^{i_1} = g(x_0, 1)$ be the
dehomogenizations onto each coordinate. Then $0 = g([1:x_1]) = g^{i_0}(x_1)$ so
$f^{-1}(Z \cap U_0) \subseteq V_a(J^{i})$. Now suppose that $x_1 \in
V_a(J^{i_0})$. Then again, $g([1:x_1]) = g^{i_0}(x_1) = 0$, so $f(x_1) \in Z
\cap U_0$ and $f^{-1}(Z \cap U_0) = V_a(J^{i_0})$. We get
\[
	V_a(J^{i_0}) \cup V_a(J^{i_1}) = f^{-1}(Z \cap U_0) \cup f^{-1}(Z \cap U_1) = f^{-1}(Z).
\]
Now, let $i_0, i_1$ be the injections $X_1, X_2 \to \PP{1}$. Then,
$i_1^{-1}(V_a(J^{i_0}))$ is closed in $X_1$, as it $i_1^{-1}$ leaves the $x_1$
coordinates unchanged and it's still the vanishing set of $J^{i_0}$. It follows
from the lemma bellow that $i_1^{-1}(V_a(J^{i_1}))$ is closed as well.

\begin{lemma}
	Let $X_0, X_1, i_0, i_1$ be as in the gluing construction of $\PP{1}$, and
	$Z \subseteq X_0$ be a closed set. Then $i_1^{-1}(i_0(X_0))$ is closed in
	$X_1$.
\end{lemma}
\begin{proof}
	Let $f \in V(X_0)$. Then define $\hat{f}(x) = x^{\deg(f)}f(1/x) \in \K[x]$.
	We have that $V(\hat{f}) \setminus \{0\} = V(\hat{f}(x)) \cap
	i_1^{-1}(i_0(X_0)) = i_1^{-1}(i_0(V(f)))$, and $f(0) \not = 0$ since $f$
	has a constant term which is equal to the leading coefficient of $f$. It
	follows that $V(\hat{f}) = i_1^{-1}(i_0(V(f)))$, and by picking generators
	of $X_0$, hatting them and intersecting their vanishing sets, we see that
	$i_1^{-1}(i_0(X_0))$ is closed in $X_1$.
\end{proof}

It follows that both $V_a(J^{i_k}), k=1,2$ are closed in both $X_k, k=1,2$, and
$f^{-1}(Z)$ is closed in $\PP{1}$. Hence $f : \PP{1} \to \PP{1'}$ is continuous. \\

It remains to show that $f, f^{-1}$ are morphisms of ringed spaces. So,
suppose that $g$ is regular on $\PP{1'}$. Then $g$ is locally the quotient
of two homogeneous polynomials of the same degree, and 
\[
	i_0^*(f^*(g)) = g(x_0, 1/x_0)
\]
is also locally a quotient of polynomials, hence regular on $X_0$. Similarly,
for $X_1$ as well. As $f^*(g)$ is regular on both $X_0$ and $X_1$, it's regular
on $\PP{1}$. \\

Now suppose that $g$ is regular on $\PP{1}$. This, by definition, is the case
exactly when both $g_0 = i_0^*(g), g_1 = i_1^*(g)$ are regular. Let $f_0 = f
\circ i_0, f_1 = f \circ i_1$. Then $\im(f_0) = U_1$, (where $U_1 = \{[x_0 : 1]
: x_0 \in \AA{1}\}$). \\

We have that
\[
	\restr{g \circ f^{-1}}{U_1} 
	=
	g \circ i_0 \circ i_0^{-1} \circ \restr{f^{-1}}{U_1} 
	= 
	g_0 \circ f_0^{-1},
\]
so by the gluing property of sheaves, we'll be done if we can show that $g_0
\circ f_0^{-1} : U_1 \to \AA{1}$ is regular. Locally, we can write $g_0 =
p_0/q_0$ as a quotient of polynomials, which in turn is the restriction to
$U_1$ of function which is locally a quotient of homogeneous polynomials
$\Kx[x_1,x_2]$ of the same degree via
\[
	g_0 \circ f_0^{-1}([x_0 : 1]) 
	= 
	g_0(x_0) 
	= 
	\frac{p_0(x_0)}{q_0(x_0)}
	= 
	\left(\frac{p_0^{h}}{q_0^{h}}x_1^{\deg q_0 - \deg p_0}\right)([x_0 : 1]),
\]
and we are done.

\subsubsection*{(b)}

Let $\mathcal{O}_{X}$ be the structure sheaf defined as a closed subvariety via
Construction 5.10 (b), and $\mathcal{O}_X'$ defined as in Definition 7.1. Let
$U \subseteq X$ be an open set and suppose $f \in \mathcal{O}_X(U)$. Then given
some $a \in U$, by Construction 5.10 (b), we have some $V$ open in $\Pn$
containing $a$, and $g \in \mathcal{O}_{\Pn}(V)$ such that $\restr{f}{V} =
\restr{g}{U}$, whence $f$ can be written as a quotient of two homogeneous
polynomials on $U \cap V \ni a$. Hence $f \in \mathcal{O}_X'(U)$. \\

Now suppose that $f \in \mathcal{O}_X'(U)$, and let $a \in U$. Then we have
some open $U_a \subseteq U$ containing $a$ such that $f$ can be written as a
quotient of two homogeneous polynomials of the same degree on $U_a$. Hence
$\restr{f}{U_a} \in \mathcal{O}_{\Pn}(U_a)$, and $f$ satisfies Construction
5.10 b with $V = U_a, \Psi = \restr{f}{U_a}$ for all $a \in U$.

\subsection*{Fleshing out the proof of Lemma 7.4}

First note that $f$ is continuous, since if $V(g_1, g_2, \ldots, g_m)$ is a
closed set in $\Pm$, then
\[
	f^{-1}(V(g_1, g_2, \ldots, g_m))
	=
	V(g_1 \circ f, g_2 \circ f, \ldots, g_m \circ f)
\] 
is closed in $X$, since compositions of quotients of homogeneous polynomials of
the same degree are again quotients of homogeneous polynomials of the same
degree. It follows that the $U_i$ are open in $\Pn$. \\

The next problem we deal with is how to apply Proposition 4.7 in the end of the
proof. For this we need $U_i$ affine, and I can't see why this is known.
Instead, define $V_i' \subset \Pn$ to be an affine cover of $\Pn$, just as we
defined $V_i$ to be an affine cover of $\Pm$. Then $U_i \cap V_j'$ is closed in
every $V_j'$, hence affine, and we can apply Proposition 4.7 here and lift back
up to $U_i$ with Lemma 4.6 whence $\restr{f}{U_i}$ is a morphism.

\subsection*{Ex 7.6}

By Example 7.5 (c), we have that $X = V_p(x_0^{2}) \subset \PP{2}$ is
isomorphic to $Y = \PP{1}$, meanwhile $S(Y) = \K[x] \not \cong \K[x_0,
x_1]/(x_0^{2}) = S(X)$ since $S(X)$ has a non-trivial element
$\overline{x_0}$ of order $2$, whilst $S(Y)$ doesn't.

\subsection*{Ex 7.8}

We solve Exercise 7.8 before 7.7, since we use 7.8 to solve 7.7. \\


Let $U_i$ be the standard affine cover of $\PP{n}$, let $V_{i, j}$ be an affine
cover of $f(U_i)$ for all $i$, and finally let $\hat{U}_{i, j} = f^{-1}(V_{i,
j})$. Then $\hat{U}_{i, j}$ is an open cover of the affine open set $U_i$,
hence all $\hat{U}_{i, j}$ are affine as well. Regular functions on
$\hat{U}_{i, j}$ are quotients of polynomials in the $x_k, k \not = i$, and it
follows from Proposition 4.XX that the restriction of $f$ to $\hat{U}_{i, j}$
is locally quotients such of polynomials in each coordinate. We can pass from
the affine embedding of $\hat{U}_{i, j}$ to the embedding in $\PP{n}$ via
tacking on the $x_k = 1$ coordinate to the tuple. When we do this, quotients of
polynomials in $x_k, k \not = i$ turn into their homogenised versions via
multiplying each term by a power of $x_k$ such that every term has the same
degree. Since the numerators are non-zero, and $\PP{n}$ is invariant to
multiplication, we can clear all denominators by multiplication with their
product, and what remains is a function which is locally a homogeneous
polynomial in each coordinate. \\

I.e we have that for every $a \in \PP{n}$, there is some open $V_a \ni a$ such
that we can write $\restr{f}{V_a} = (f_1^a, f_2^a, \ldots, f_m^a)$ with all
$f_i^a$ homogeneous. We have $\PP{n}$ compact, so we can pick a finite subcover
$V_a, a \in A$ with $|A|$ finite. Since only the zero polynomial is zero on any
open set, we have that $f$ admits the same representation as homogeneous
polynomials $f_i^a = f_i^b$ on $V_a, V_b$ whenever $V_a \cap V_b \not =
\emptyset$. But since $\PP{n}$ is connected and $|A|$ is finite, we can use
induction to conclude that $f_i^a = f_i^b$ for all $a, b \in A$, and we can
write $f = (f_1, f_2, \ldots, f_m)$ on all of $\PP{n}$.

\subsection*{Ex 7.7}
\subsubsection*{(a)}

The statement of the exercise is false. Suppose that $n = m = 2$, and $X =
V(x_0)$, and let $f : \PP{2} \to \PP{2}$ be given by $f([x_0 : x_1 : x_2]) = [1
: 1 :1]$. As $[1 : 1 : 1] \not \in X$, we have $f^{-1}(X) = \emptyset$. \\

We need the additional statement that $X \cap \im(f) \not = \emptyset$,
or equivalently that $f^{-1}(X) \not = \emptyset$. \\

We use Exercise 7.8 to solve this. \\

By Remark 6.33, we know that $I(X) = (g)$ for some homogeneous polynomial $g$.
Then
\[
	f^{-1}(X) 
	=  
	f^{-1}(V_p(g)) 
	=  
	f^{-1}(g^{-1}(0)) 
	= 
	(g \circ f)^{-1}(0)
	=
	V_p(g \circ f). 
\]
Let $h = g \circ f$. Then $h$ is a homogeneous polynomial since $f$ is (by
Exercise 7.8). Moreover, as $f^{-1}(X) = V_p(h) \not = \emptyset$, we see that
$h$ isn't a constant. \\

The irreducible components of $V_p(h)$ correspond to the principal ideals of
the prime factors of $h$. Any such ideal is prime, and Krull's Principal Ideal
Theorem tells us that their heights are at most one, hence the corresponding
varieties have dimension at least $n - 1$. \\

\subsubsection*{(b)}

This will follow from Exercise 7.8 after we prove the following lemma.

\begin{lemma}
	Let $n > m$ and $f_1, f_2, \ldots, f_m \in \KP{x}{n}$ be $m$ polynomials in
	$n$ variables. Then $V_a(f_1, f_2, \ldots, f_n)$ is either empty or infinite.
\end{lemma}
\begin{proof}
	Let $J = I_a(V_a(f_1, f_2, \ldots, f_m)) = (f_1, f_2, \ldots, f_m)$.
	Suppose that the $f_i$ share some root. Then $J$ is a proper ideal, and
	applying Krull's height theorem yields that every minimal ideal of $J$ has
	height at most $m$. Minimal prime ideals of $J$ correspond to irreducible
	components of $V(J)$, so Lemma 2.27 (b) tells us that the dimension of any
	irreducible component of $V_a(f_1, f_2, \ldots, f_m)$ is at least $n - m
	\geq 1$, whence $V_a(f_1, f_2, \ldots, f_m)$ is infinite as all finite
	affine sets have dimension $0$.
\end{proof}

Now, Exercise 7.8 tells us that any morphism $f : \PP{n} \to \PP{m}$ is of the
form $f = [f_1 : f_2 : \ldots : f_m ]$ where all $f_i$ have the same degree
$d$. Suppose towards a contradiction that such a morphism where $d \geq 1$
exist. Then since the $f_i$ are homogeneous, we have $f_i(0) = 0$ for all
$f_i$. Hence the lemma tells us that the $f_i$ will always have a common
non-zero root as well, which would yield an ill-defined morphism $f$ since $[0
: 0 : \ldots : 0]$ isn't a well defined projective point. Hence $\deg f_i = 0$
and $f$ must be constant. 

\subsubsection*{(c)}

We have non-constant automorphisms $\PP{n + m} \to \PP{n + m}$ by Example 7.5
(a). There are no non-constant morphisms $\PP{n + m} \to \PP{n}, \PP{n + m} \to
\PP{m}$ by Exercise 7.7 (b), hence no non-constant morphisms $\PP{n + m} \to
\PP{n} \times \PP{m}$.

\subsection*{Ex 7.14}

Let $a = [a_0 : a_1 : a_2], b = [b_0 : b_1 : b_2]$ be two points
on the cubic curve $X$. A line $L$ through $a, b$ can be parameterised 
as 
\[
	L = \{ [a_0 s + b_0 t : a_1 s + b_1 t : a_2 s + b_2 t] : [s : t] \in \PP{1} \}.
\] 
Let $h_{a, b} : \PP{1} \to L$ be the parameterisation above. We know that $X$
is the vanishing set of a homogeneous cubic in three variables. Name it $g$.
Then let $g_{a, b} = g \circ h_{a, b} \in \K[s, t]$. This is again a
homogeneous cubic, but in two variables. We already know that $g_{a, b}$
vanishes on $[0 : 1], [1 : 0]$, hence it must lie in the ideals $(s), (t)$ and
it follows that we can write $g_{a, b} = st(As + Bt)$ where not both $A$ and
$B$ are zero. We now see that the remaining root of $g_{a, b}$, namely $[B :
-A]$, is given by homogeneous polynomials of the same degree in $a, b$,
\begin{align*}
	[B:-A]
	=&
	[2B : -2A] \\
	=&
	[g_{a, b}(1, 1) - g_{a, b}(-1, 1) : -g_{a, b}(1, 1) + g_{a, b}(1, -1)\big] \\
	=&
	[g_{a, b}(1, 1) + g_{a, b}(1, -1) : -g_{a, b}(1, 1) + g_{a, b}(1, -1)\big] \\
	=& 
	[g(a_0 + b_0, a_1 + b_1, a_2 + b_2) + g(a_0 - b_0, a_1 - b_1, a_2 - b_2) \\
	 &: -g(a_0 + b_0, a_1 + b_1, a_2 + b_2) + g(a_0 - b_0, a_1 - b_1, a_2 - b_2)].
\end{align*} 
As $a \not = b$, we have $h_{a, b}$ bijective, and it follows that the
remaining root of $g$ is given by $f(a, b) = h_{a, b}([B:-A])$, which we can
write as
\[
	f(a, b) = [f_0(a, b) : f_1(a, b) : f_2(a, b)]
\]
where
\begin{align*}
	f_0([a_0 : a_1 : a_2], [b_0 : b_1 : b_2])
	=& 
		a_0 (g(a_0 + b_0, a_1 + b_1, a_2 + b_2) + g(a_0 - b_0, a_1 - b_1, a_2 - b_2)) \\
	 &+ b_0 (-g(a_0 + b_0, a_1 + b_1, a_2 + b_2) + g(a_0 - b_0, a_1 - b_1, a_2 - b_2)) \\
	=& 
		(a_0 - b_0) g(a_0 + b_0, a_1 + b_1, a_2 + b_2) \\
	 &+ (a_0 + b_0) g(a_0 - b_0, a_1 - b_1, a_2 - b_2),
\end{align*}
and similarly
\begin{align*}
	f_1([a_0 : a_1 : a_2], [b_0 : b_1 : b_2])
	=& 
		(a_1 - b_1) g(a_0 + b_0, a_1 + b_1, a_2 + b_2) \\
	 &+ (a_1 + b_1) g(a_0 - b_0, a_1 - b_1, a_2 - b_2), \\
	f_2([a_0 : a_1 : a_2], [b_0 : b_1 : b_2])
	=& 
		(a_2 - b_2) g(a_0 + b_0, a_1 + b_1, a_2 + b_2) \\
	 &+ (a_2 + b_2) g(a_0 - b_0, a_1 - b_1, a_2 - b_2).
\end{align*}
Now, the $f_i$ will only simultaneously vanish on $\Delta_X$, since we have
that at least one of $A, B$ non-zero, and $f_i = a_i B - b_i A$, so $f_0(a, b) =
f_1(a, b) = f_2(a, b) = 0$ would imply $B [a_0 : a_1 : a_2] = A [b_0 : b_1 :
b_2]$ hence $a = b$. It follows that $f$ is a well-defined function on $(X
\times X) \setminus \Delta_X$. It's a morphism by the following lemma.

\begin{lemma}
	Let $X \subseteq \PP{m} \times \PP{n}$ be a subvariety, and $f : X \to
	\PP{k}$ be a function $f = [f_1 : f_2 : \ldots : f_k]$ in which each $f_i$
	is a homogeneous polynomial in the coordinates of $[x_1 : x_2 : \ldots :
	x_m] \in \PP{m}$ and $[y_1 : y_2 : \ldots : y_n] \in \PP{n}$, $f_i(x_1,
	x_2, \ldots, x_m, y_1, y_2 \ldots y_n)$ such that all $f_i$ have degree
	$d$. Then $f$ is a morphism of varieties.
\end{lemma}
\begin{proof}
	Later TODO Todo todo
\end{proof}

Now consider the set $V_1 \subset (X \times X) \setminus \Delta_X$ given by
$(a, b)$ such that $f(a, b) = a$. Then $V_1$ is closed by Prop 5.20 (b), since
it's the subset where $\pi_1 = f$. Similarly, if $V_2$ is the set of $(a, b)$
where $f(a, b) = b$, it's closed as well. It now follows that $U$ is open since
\[
	U = (X \times X) \setminus (\Delta_X \cup V_1 \cup V_2).
\]

\subsection*{Ex 7.15}

We proceed as suggested in the hint. Let $X = \PP{1} \times \PP{1}$ and $Y =
\PP{1} \times \{[1 : 0]\} \subset X$ be a hypersurface. Let $f : X \to \PP{3},
f([x_0, x_1] : [y_0 : y_1]) = [x_0y_0 : x_0y_1 : x_1y_0 : x_1y_1]$ be the Segre
embedding of $X$. Then $Y$ is mapped to $f(Y) = \{[x_0 : 0 : x_1 : 0] : [x_0 :
x_1] \in \PP{1} \}$. In $\PP{3}$, we have that $f(X) = V_p(z_0z_3 - z_1z_2)$
and $f(Y) = (V_p)_{f(X)}(z_1, z_3)$. Moreover, $(I_p)_{f(X)}(f(Y)) =
(\overline{z}_1, \overline{z}_3)$ can't be generated by fewer polynomials.
Indeed, suppose towards a contradiction that we have some homogeneous
$\overline{g} \in S(f(X))$ such that $(I_p)_{f(X)}(f(Y)) = (\overline{g})$.
Then $\overline{g} | \overline{z}_1$ and $\overline{g} | \overline{z}_3$. But
both $\overline{z}_1, \overline{z}_3$ are irreducible, as they are degree $1$
elements in the graded ring $S(f(X))$, hence no such $\overline{g}$ exists. 


\subsection*{Ex 7.28}

\subsubsection*{(a)}

Let $f : \PP{n} \to X \subset \PP{N}$ be the degree $d$ embedding of $\PP{n}$,
and let $z_{k_0, k_1, \ldots, k_n} = z_{\bm{k}}$ be the coordinate of
$f(x_0^{k_0}x_1^{k_1} \ldots x_n^{k_n})$ where $\sum k_i = d$. Let $J$ be the
ideal generated of all polynomials of the form $z_{\bm{k}}z_{\bm{k}'} -
z_{\bm{r}}z_{\bm{r}'}$ where $\bm{k} + \bm{k}' = \bm{r} + \bm{r}'$, and write
$X' = V(J)$. Then $X \subseteq X'$ by definition of the Veronese embedding. \\ 


Now suppose that $x \in X'$. Then some coordinate of $x$ is non-zero, and we
can assume that $x_{d, 0, 0, \ldots, 0} = 1$. Let $\bm{e}_i$ be the $n +
1$-tuple with a $1$ at index $i$ and $0$ everywhere else, and consider $y \in
\PP{n}$ where $y_0 = 1$ and $y_i = x_{(d - 1)\bm{e}_0 + \bm{e}_i} = x_{d - 1,
0, \ldots, 0, 1, 0, \ldots 0}$. Then
\begin{align*}
	f(y)_{\bm{k}}
	&=
	y_0^{k_0}
	\prod_{i = 1}^{n}
	y_i^{k_i} \\
	&=
	x_{d \bm{e}_0}
	\prod_{i = 1}^{n}
	x_{(d - 1)\bm{e}_0 + \bm{e}_i}^{k_i} \\
	&=
	x_{d \bm{e}_0}
	\prod_{i = 1}^{n}
	x_{(d - k_i)\bm{e}_0 + k_i\bm{e}_i}
	x_{d\bm{e}_0}^{k_i - 1} \\
	&=
	x_{d \bm{e}_0}
	\prod_{i = 1}^{n}
	x_{(d - k_i)\bm{e}_0 + k_i\bm{e}_i} \\
	&=
	x_{\bm{k}}
	x_{d \bm{e}_0}^{n} \\
	&=
	x_{\bm{k}}
\end{align*}
hence $f(y) = x$ and $x \in \im(f) = X$, $X = X'$.

\subsubsection*{(b)}

Let $Y \subset \PP{n}$ be a projective variety, and $g_1, g_2, \ldots, g_m$ be
homogeneous polynomials of the same degree $2d$ which generate an ideal $J$
such that $V(J) = Y$. (such $f_i$ exist by Remark 6.XX). Then let $f : \PP{n} \to
X \subset \PP{N}$ be the degree $d$ Veronese embedding, and $Y' = f(Y)$. \\


Let $I_d$ denote $n + 1$-tuples whose sum is $d$. Since each $g_i$
is homogeneous of degree $2d$, it follows that we can write
\[
	g_i = \sum_{\bm{k}, \bm{r} \in I_d} a_{i, \bm{k}, \bm{r}} x^{\bm{k}}x^{\bm{r}}. 
\] 
Note that this description is not unique, indeed there are multiple tuples
$\bm{k}, \bm{r}$ which sum to the same tuple, which means that descriptions of
the form above are unique only up to the sum of all $a_{i, \bm{k}, \bm{r}}$ with
the same $\bm{k} + \bm{r}$. We shall not worry about this though, and simply
pick some set of $a_{i, \bm{k}, \bm{r}}$ such that the equation above holds. 
Now let
\[
	h_i = \sum_{\bm{k}, \bm{r} \in I_d} a_{i, \bm{k}, \bm{r}} z_{\bm{k}}z_{\bm{r}} \in S(X).
\]
Then since $h_i = g_i \circ f$, it follows that $y \in Y$ precisely when $f(y)
\in V_X(h_1, h_2, \ldots, h_m)$. We showed in part (a) that $X$ is the zero
locust of quadratic forms in $\PP{N}$, and since each $h_i$ is quadratic as
well, it now follows that $Y'$ also is the zero locust of quadratic forms in
$\PP{N}$.


\subsection*{Ex 7.30}

\subsubsection*{(a)}

Let $f = ax_0^{2} + bx_0x_1 + cx_0x_2 + dx_1^{2} + ex_1x_2 + fx_2^{2}$ be a
quadratic form on $\PP{2}$. Scaling the polynomial $f$ by a non-zero constant
does not change it's vanishing set, hence each quadric curve in $\PP{2}$
defined by a polynomial $f$ like above can be identified with the point
$[a:b:c:d:e:f] \in \PP{5}$. More over, this correspondence is bijective, since
if $V(f), V(g)$ are identified with the same point in $\PP{5}$, we have $f =
\lambda g$ hence $V(f) = V(g)$. \\

Under this correspondence, the set of conics in $\PP{2}$ are identified with
the subset of points $[a:b:c:d:e:f] \in \PP{5}$ such that $f = ax_0^{2} +
bx_0x_1 + cx_0x_2 + dx_1^{2} + ex_1x_2 + fx_2^{2}$ is an irreducible
polynomial. This is the $U$ described in the exercise. We will show that the
complement of $U$, $V = \PP{5} \setminus U$ is closed. \\

Suppose instead that $f = ax_0^{2} +
bx_0x_1 + cx_0x_2 + dx_1^{2} + ex_1x_2 + fx_2^{2}$
is reducible. Homogeneous polynomials factor into homogeneous polynomials,
and it follows that we can write
\[
	f 
	= 
	(Ax_0 + Bx_1 + Cx_2)(Dx_0 + Ex_1 + Fx_2)
\]
whence 
\begin{align*}
	a &= AD \\
	b &= AE + BD \\
	c &= AF + CD \\
	d &= BE \\
	e &= BF + CE \\
	f &= CF,
\end{align*}
which gives us a parameterisation of $V$ in terms of $\left([A:B:C], [D:E:F]\right) \in
\PP{2} \times \PP{2}$. \\

Now let $s : \PP{2} \times \PP{2} \to X \subset \PP{8}$ be the Segre embedding,
and $g : X \to \PP{5}$ be the map
\begin{align*}
	&g([z_{0,0}:z_{0,1}:z_{0,2}:z_{1,0}:z_{1,1}:z_{1,2}:z_{2,0}:z_{2,1}:z_{2,2}]) \\
	&=
	[z_{0,0}:z_{0,1}+z_{1,0}:z_{0,2}+z_{2,0}:z_{1,1}:z_{1,2}+z_{2,1}:z_{2,2}].
\end{align*}
$g$ is a linear automorphism followed by a projection, hence it's a closed
morphism by Proposition 7.17. Since $s$ is an isomorphism as well, we have that
$g \circ s$ is a closed morphism and $\im(g \circ s)$ is closed. But $\im(g
\circ s)$ is parameterised in exactly the same way as $V$, hence $V$ is closed
and $U$ is open.

\subsubsection*{(b)}

Let $A=[A_0:A_1:A_2]$ be a point in $\PP{2}$. It follows immediately from our
solution to (a), that the points $[a:b:c:d:e:f] \in U$ which correspond to
conics which pass through $A$ are precisely those points which satisfy the
linear equation $aA_0^{2} + bA_0A_1 + cA_0A_2 + dA_1^{2} + eA_1A_2 + fA_2^{2} =
0$. \\

\subsubsection*{(c)}

We introduce some new language to deal with this exercise. 

\begin{definition}
	A set of $k > n$ points $p_i \in \PP{n}$ are said to be in general position
	if no hyperplane of $\PP{n}$ contains a subset of $n + 1$ points $p_i$. I.e
	if no linear homogeneous polynomial $f \in S(\PP{n})$ annihilates all
	$p_i$.
\end{definition}

So, in our case, we have $5$ points $p_1,p_2,p_3,p_4,p_5 \in \PP{2}$
which are in general position. 

\begin{lemma}
	Let	$v_d : \PP{n} \to \PP{N}$ be the veronese embedding. Then any set of
	points in general position in $\PP{n}$ are mapped to a set of points in
	general position in $\PP{N}$ under $v_d$.
\end{lemma}
\begin{proof}
	Let $p_i, i \in [1..k] \in \PP{N}, k > N$ be a set of points, and suppose
	that $f \in S(\PP{N})_1$ is a linear homogeneous polynomial such that $p_i
	\in V_p(f)$ for all $i$. Then suppose that $p_i \in U_0$, then we have 
	\[
		(v^{d})^{-1}(p_i) 
		= 
		\left[
			1 : 
			\frac{x_1^{N}}{x_0 x_1^{N-1}} : 
			\frac{x_2^{N}}{x_0 x_2^{N-1}} :
			\ldots :
			\frac{x_n^{N}}{x_0 x_n^{N-1}}
		\right]
	\]
\end{proof}



\begin{lemma}
	Let $P_i \in \PP{n}$ be a set of $k > n$ points. Then
	the points are in general position if and only if their 
	corresponding vectors in $\AA{n + 1}$ are linearly independent.
\end{lemma}

Let $A_1,A_2,A_3,A_4,A_5 \in \PP{2}$ be $5$ points, such that no three of them
lie on the same line in $\PP{2}$, and let $f_1,f_2,f_3,f_4,f_5$ be their
corresponding linear functionals $\AA{6} \to \A$ as in (b). Then our first
claim is that the $f_i$ are linearly independent. We assume this for now, and
show how to use it to solve the exercise. \\

Suppose that $0 = \sum b_i f_i$ is a non-trivial linear independence. 
Let $F : \PP{2} \to \PP{5}$ be the map from (b) which sends a point $A_i=[A_{i,
0}:A_{i, 1}:A_{i, 2}]$ to $F(A_i) = [A_{i, 0}^{2} : A_{i, 0}A_{i, 1} : A_{i,
0}A_{i, 2} : A_{i, 1}^{2} : A_{i, 1}A_{i, 2} : A_{i, 2}^{2}]$. Note that $F$
coincides with the degree $2$ Veronese embedding. Now, as no combination
of more than $2$ $A_i$ lie on the same line

Suppose that $f_1 \in (f_2,f_3,f_4,f_5)$. Then since all $f_i$ are linear, it
follows that $f_1$ can be written as a $\K$-linear combination of the other
$f_i$. Hence $f_1,f_2,f_3,f_4,f_5$ linearly independent implies that $J =
(f_1,f_2,f_3,f_4,f_5)$ is a minimal generating set. Now inductively define
$S_i, J_i$ by $S_0 = S(\PP{5})$, $J_0 = (0)$ and $S_i = S_{i - 1}/J_i$, $J_i =
(f_i) + J_{i - 1}$. I.e each $S_i$ is obtained from $S_{i - 1}$ via taking the
quotient by $f_i$. Then $S_0 = \KP{x}{6}$, and if $S_i$ is isomorphic to a
polynomial ring in $6- i$ variables, then $S_{i + 1}$ is isomorphic to a
polynomial ring in $6 - (i + 1)$ variables since it's obtained from $S_i$ via a
non-trivial linear relation. It follows that each $J_i$ is prime, and that $J =
J_5 = (f_1,f_2,f_3,f_4,f_5)$ is a prime ideal of height $5$. Hence $V_p(J) =
\{a\}$ for some point $a \in \PP{5}$. Points $a \in V_p(J)$ correspond
bijectively with conics that pass through $A_1,A_2,A_3,A_4,A_5$ by part (b),
thus there is exactly one conic which passes through the given points so long
as the linear homogeneous polynomials which correspond to the point are
linearly independent. We will now show that this is the case whenever no three
$A_i$ lie on a line.
\\

TODO Todo todo

Suppose towards a contradiction that $\sum b_i f_i = 0, b_i \in \K$ is a
non-trivial linear dependence, and assume that $b_1 \not = 0$. Then we have
$f_1 = \sum_{i = 2}^{n}c_i f_i$, with $c_i = -b_i/b_1$. Write the point $A_i$
as $A_i = [A_{i, 0}:A_{i, 1}:A_{i, 2}]$. 
Then for all $i$ we have $A_{1, i} = \sum $



Then each point $A_i$ corresponds to a line
$L_i \in \AA{3}$ which passes through the origin, and no three lines lie in the
same plane.


Let $F : \PP{2} \to \PP{5}$ be the map from (b) which sends a point $A_i=[A_{i,
0}:A_{i, 1}:A_{i, 2}]$ to $F(A_i) = [A_{i, 0}^{2} : A_{i, 0}A_{i, 1} : A_{i,
0}A_{i, 2} : A_{i, 1}^{2} : A_{i, 1}A_{i, 2} : A_{i, 2}^{2}]$. Note that $F$
coincides with the degree $2$ Veronese embedding. A linear dependence on the
$f_i$ corresponds to a linear relation on $\im(F)$, 

but we showed in Exercise
7.28 that the Veronese embedding is given by
\[
	\im(F) 
	= 
	V(
		z_{0,1}z_{1,0} - z_{1, 1}z_{0,0},
		z_{0,2}z_{2,0} - z_{2, 2}z_{0,0},
		z_{1,2}z_{2,1} - z_{2, 2}z_{1,1},
	)
\]
and none of the polynomials inside t


Using our notation in from part (a), we can break down $g$
into a linear automorphism followed by a projection like this
\[
	g 
	= 
	\pi_{0,1,2,4,5,8} \circ 
	([z_{0,0}:z_{0,1}+z_{1, 0}:z_{0,2}+z_{2, 0}:z_{1,0}:
	z_{1,1}:z_{1,2}+z_{2,1}:z_{2,0}:z_{2,1}:z_{2,2}]),
\] 
and we see that $\dim(V) = 6$ 

We just need to find point $A, B, C \in \PP{2}$ that define linearly
independent function $f_A, f_B, f_C : U \to$. 


\subsection*{Ex 7.31}

The projection $f$ from $a = [0 : 0 : 1 : 0]$ to $L = V(y_2)$ is given by $f :
\PP{3} \setminus \{ a \} \to \PP{2}$ where $f([y_0, y_1, y_2, y_3]) = [y_0 :
y_1 : y_3]$. \\

\subsubsection*{(a)}

$f(X)$ may be parameterised as $f(X) = \{ [x_0^{3} : x_0^{2} x_1 : x_1^{3}] :
[x_0:x_1] \in \PP{1} \}$, so $f(X) = g(\PP{1})$ where $g([x_0:x_1]) = [x_0^{3}
: x_0^{2}x_1 : x_1^{3}]$. From our parameterisation of $f$, it follows that
\[
	f(X) \subseteq V(y_0^{2}y_2 - y_1^{3}).
\] 
Moreover, suppose $a = [a_0 : a_1 : a_2]\in V(y_0^{2}y_2 - y_1^{3})$. We will
show that $a$ lies in the image of $g$, hence in $f(X)$. First consider the
case $a_0 = 1$. Then $a_2 = a_1^{3}$ and we have
\[
	g([1 : a_1]) = [1 : a_1 : a_1^{3}]  = [a_0 : a_1 : a_2] = a.
\]
Now consider the case when $a_0 = 0$. Then $a_1 = 0$ as well, and $a_2$ must be
non-zero, so we can assume $a_2 = 1$. Then
\[
	g([0 : 1]) = [0 : 0 : 1] = [a_0 : a_1 : a_2] = a,
\]
hence $f(X) = V(y_0^{2}y_2 - y_1^{3})$. Note that for both cases, the element
in the pre-image is uniquely determined, hence $g$ is injective (thus bijective
onto its image). 

\subsubsection*{(b)}

Note that $g$ from part (a) is an isomorphism (the degree $3$ Veronese
embedding) composed with $f$. Hence $f$ is an isomorphism if and only if $g$
is. We claim that $g$ isn't an isomorphism, since if it were, it would be an
isomorphism on the affine subset $U_1 = \{ [x_0 : x_1] \in \PP{1} : x_1 \not =
0 \} \cong \AA{1}$, but on this set $g$ restricts to $g([x_0 : 1]) = [x_0^{3} :
x_0^{2} : 1]$, so we can identify $\restr{g}{U_1}$ with $\hat{g} : \AA{1} \to
\AA{2}$ where $\hat{g}(t) = (t^{3}, t^{2})$, and we saw in Example 4.9 that
this map isn't an isomorphism.



\subsection*{Ex 8.22}

To be extra clear for my own sake, and get things right, we will follow 
construction 8.12 step by step. Let $U \in G(2, 4)$ be a plane in 
$\K^4$. Then $U$ is spanned by two basis vectors, say 
\begin{align}
	v_1 &= a_1 e_1 + a_2 e_2 + a_3 e_3 + a_4 e_4, \\
	v_2 &= b_1 e_1 + b_2 e_2 + b_3 e_3 + b_4 e_4.
\end{align}
The correspondence of construction 8.12 first sends $U$ to 
\begin{align*}
	v_1 \wedge v_2
	=&
	(a_1 e_1 + a_2 e_2 + a_3 e_3 + a_4 e_4)
	\wedge
	(b_1 e_1 + b_2 e_2 + b_3 e_3 + b_4 e_4) \\
	=&
	(a_1 b_2 - a_2 b_1) e_1 \wedge e_2
	+
	(a_1 b_3 - a_3 b_1) e_1 \wedge e_3
	+
	(a_1 b_4 - a_4 b_1) e_1 \wedge e_4 \\
	 &+
	(a_2 b_3 - a_3 b_2) e_2 \wedge e_3
	+
	(a_2 b_4 - a_4 b_2) e_2 \wedge e_4
	+
	(a_3 b_4 - a_4 b_3) e_3 \wedge e_4,
\end{align*}
and then to 
\begin{align*}
	\left[
	a_1 b_2 - a_2 b_1
	:
	a_1 b_3 - a_3 b_1
	:
	a_1 b_4 - a_4 b_1
	:
	a_2 b_3 - a_3 b_2
	:
	a_2 b_4 - a_4 b_2
	:
	a_3 b_4 - a_4 b_3
	\right]
\end{align*}
in $\PP{5}$. Thus the correspondence sends the minor of columns $i, j$ in the
matrix $[v_1 v_2]^T$ to the coordinate $x_{i, j}$ in
\[
	[x_{1, 2} : x_{1, 3} : x_{1, 4} : x_{2, 3} : x_{2, 4} : x_{3, 4}] \in \PP{5}.
\]

\subsubsection*{(a)}

It's immediate from the correspondence above that $G(2, 4) \subseteq V(f =
x_{1, 2}x_{3, 4} - x_{1, 3}x_{2, 4} + x_{1, 4}x_{2, 3})$. To see that this
containment is an equality, note that $\dim G(2, 4) = 2(4 - 2) = 4$ and as $f$
is prime, $\dim V(f) = \dim \PP{5} - 1 = 4$ hence the varieties are the same as
they are both irreducible.

\subsubsection*{(b)}

A line $L'$ intersects $L$ in $\PP{3}$ exactly when the intersection of the
corresponding planes $U, U' \in \K^4$ intersect in a line. This happens if and
only if the basis vectors of the two spaces, $v_1, v_2, v_1', v_2'$, are
linearly dependent, and the determinant of the matrix $M(L, L') = \left[v_1\, v_2\,
v_1'\, v_2'\right]^T$ is $0$. The determinant expansion by minors of $\det(M(L, L'))$
along the top two rows shows that $\det(M(L, L')) = 0$ can be expressed as a
linear relation among the $2 \times 2$ minors contained in the bottom two rows,
with coefficients in the top to rows. Since the coordinates of the point
corresponding to $U'$ in $\PP{5}$ are given by it's $2 \times 2$ minors, this
is exactly what it means for the embedding of $G(2, 4)$ in $\PP{5}$ to be the
variety of a linear homogeneous polynomial. \\

It follows that the set of lines intersecting four general lines is given by
the vanishing set of four general linear polynomials in $\PP{5}$, which would
have dimension $1$, and hence should only consist of a single line in general.


\subsection*{Ex 8.23}
\subsubsection*{(a)}

Let $v_1, \ldots, v_k$ be an orthogonal basis for the $k$-dimensional subspace
$U \subseteq \K^{n}$. Then the distance $d_U(a)$ from $U$ to a point $a = (a_1,
\ldots, a_n) \in \K^n$ is given by $|a - a_{\|U}|$ where $a_{\|U}$ is the
projection of $a$ onto $U$ as
\[
	a_{\|U} = \sum_{i = 1}^{k} \frac{\langle a, v_i \rangle}{|v_i|^2}v_i.
\] 
After squaring and clearing denominators, we have that
\[
	f 
	=
	d_U(a)^2
	\prod_{i = 1}^{k} |v_i|^{4}
	=
	\sum_{i = 1}^{k} 
	(|v_i|^{2}a_i - \langle a, v_i \rangle)^{2}
\] 
is a polynomial in $a, U$ which is zero precisely when $d_U(a) = 0$, I.e when
$a \in L$. It follows that our variety is given by $V_{G(k, n), \K}(f)$,
however, we've thus far treated the embeddings in of the elements $U$ in
$\A^{n}$. It's clear though that $V(f)$ is a cone, since $a \in L \Rightarrow
ka \in kL, \forall k \in \K$ so the variety is a projective variety as well.

\subsubsection*{(b)}

This is really unintuitive to me so I'll begin with an example. Consider the
ambient space $\PP{3}$, and the two varieties $X = V_p(x_1^2 + x_2^2 - x_0^2,
x_3 - x_0)$ $Y = V_p(x_1^2 + x_2^2 - x_0^2, x_3 + x_0)$. Then on the affine
patch $U_0$ where $x_0 = 1$, we have that $X$ is the circle at height $1$, and
$Y$ is the circle at height $-1$. \\

Now consider the geometry of our setup in $\R^{3}$. We have two circles
parallel with the $x_1x_2$-plane at different heights $1, -1$, and take all the
lines joining them. Then if we fix some point $x \in X$, the join of $x$ with
$Y$ is a sort of hollow skew double sided cone, and when we rotate $x$ around
the circle $X$, the union of all skew cones trace a volume, i.e $3$-dimensional
space, which isn't all of $\R^3$, which is strange? There is one thing missing
from this picture though, we're not considering the entire complex circles.
Thus, taking all of the complex points into account on the circles as well, we
should expect that the resulting join is all of $\PP{3}$. Let's verify this on
the affine patch. \\

Let $p = [1:a:b:c] \in U_0$ be some fixed arbitrary point on our affine patch.
Then let $q_X = [1:X_1:X_2:1] \in X \cap U_0$ and $q_Y = [1:Y_1:Y_2:-1] \in Y
\cap U_0$ be affine points on our varieties. We will vary the points $q_X, q_Y$
such that the line through both points meets $p$ as well. The lines $L_X, L_Y$
through $p,q_X$ and $p, q_Y$ are given by 
\begin{align*}
	L_X
	&= 
	\{ [1 : ta + (1-t)X_1 : tb + (1-t)X_2 : tc + 1 - t] : t \in \C \}, \\
	L_Y
	&= 
	\{ [1 : sa + (1-s)Y_1 : sb + (1-s)Y_2 : sc - 1 + s] : s \in \C \}. 
\end{align*} 
We will now find $q_X, q_Y$ such that these lines coincide. First, we can use
the last coordinate to express $s$ in terms of $t$. We need $tc + 1 - t = sc -
1 + s$, hence $s = \frac{tc + 2 - t}{c + 1}$, and the two lines
can be jointly parameterised according to
\begin{align*}
	L_X
	&= 
	\{ [1 : ta + (1-t)X_1 : tb + (1-t)X_2 : tc + 1 - t] : t \in \C \}, \\
	L_Y
	&= 
	\left\{ \left[1 
		:
		\frac{tc + 2 - t}{c + 1}(a - Y_1) + Y_1 
		: 
		\frac{tc + 2 - t}{c + 1}(b - Y_2) + Y_2 
		: 
		tc + 1 - t
	\right] : t \in \C \right\}. 
\end{align*}
We see that these equations linearly determine each $Y_1, Y_2$ in terms of
$X_1, X_2$ respectively. Name this correspondence $Y_1 = f_1(X_1), Y_2 =
f_2(X_2)$. Then $X_1, X_2$ need to satisfy $X_1^{2} + X_1^{2} = 1$ and
$f_1(X_1)^2 + f_2(X_2)^2 = 1$. This system has a solution in the affine plane
since the circle is reducible into linear factors (over $\K$ algebraically
closed, but not over $\R$!), hence we can substitute one variable for the
other, and then solve the remaining polynomial. It follows that $p$ lies on the
join of $X, Y$ - hence we've verified that the join is all of $\PP{3}$ (or at
least that it contains $U_0$). \\

Now, let's solve the actual exercise. TODO Todo todo - maybe solve, spent a lot
of time on example.


\subsection*{Ex 9.9}

Let $f \in \K[x_0, x_1, \ldots, x_n]$ be some irreducible homogeneous
polynomial of degree $2$. Our first claim is that there exists a suitable
change of coordinates such that $f$ can be written as $f = x_0x_1 + x_2^2 +
x_3^2 + \ldots + x_m^2$ where $2 \leq m \leq n$. \\

To see this, first note that any bilinear form
\[
	f = \sum_{i \leq j} c_{i, j} x_i x_j
\]
can be written as 
\[
	f 
	=
	\begin{bmatrix}
		x_0 & x_1 & \ldots & x_n 
	\end{bmatrix}
	A
	\begin{bmatrix}
		x_0 \\
		x_1 \\
		\vdots \\
		x_n \\
	\end{bmatrix}
\] 
where $A$ is a symmetric matrix with $A_{i,j} = A_{j, i} = c_{i,j}/2$ for $i
\not = j$ and $A_{i, i} = c_{i, i}$. We can perform a change of coordinates
which diagonalises this matrix, and then by rescaling the coordinates, we can
do so in such a way that the first two elements on the diagonal are $1/2$, and
the remaining are either $1$ or $0$ ($f$ is irreducible so the first to
diagonal entries must be non-zero). After we've diagonalised it we can perform
a row swap to obtain a matrix of the form
\[
	\hat{A}
	=
	\begin{bmatrix}
		0 & 1/2 & 0 & 0 & \ldots & 0 \\
		1/2 & 0 & 0 & 0 &\ldots & 0 \\
		0 & 0 & 1 \text{ or } 0 & 0 & \ldots & 0 \\
		0 & 0 & 0 & 1 \text{ or } 0 & \ldots & 0 \\
		 &  &  &  & \ddots & \\
		0 & 0 & 0 & 0 & \ldots & 1 \text{ or } 0, \\
	\end{bmatrix}
\] 
which has a corresponding quadratic form with the desired shape. \\

So, suppose $f = x_0x_1 + x_2^2 + x_3^2 + \ldots + x_m^2$ is an irreducible
quadratic form, and consider $X = V_p(f)$. Then the projection $\pi : X \to
\AA{n - 1}$ given by $\pi([a_0:a_1:\ldots:a_n]) = (a_1,\ldots,a_n)$, and it's
invertible on $\AA{n - 1} \setminus 0$ with inverse
\[
	\pi^{-1}(a_1, \ldots, a_n) 
	=
	([-(a_2^{2} + a_3^{2} + \ldots + a_n^{2}) : a_1^{2} : a_1a_2 : \ldots : a_1 a_n]),
\]
hence $V_p(f)$ is birational to $\PP{n - 1}$. \\

An example of a irreducible quadric surface which isn't isomorphic to some
projective space, consider the Segre embedding of $\PP{1} \times \PP{1}$ in
$\PP{3}$. This is given by the variety $V_p(z_{0, 0}z_{1,1} - z_{0, 1}z_{1,
0})$, hence is an irreducible quadric hypersurface of $\PP{3}$, but $\PP{1}
\times \PP{1}$ isn't isomorphic to $\PP{2}$ by Exercise 7.7 (c).

\subsection*{Ex 9.18}

We follow along with Example 9.15. By Lemma 9.14, we have that the blow-up of
$\AA{3}$ along the $x_3$ axis satisfies 
\[
	\widetilde{\AA{3}}
	\subseteq 
	\{ 
		((a_1, a_2, a_3), [b_1 : b_2]) \in \AA{3} \times \PP{1}
		:
		b_1a_2 = b_2 a_1
	\}
	=:
	Y.
\] 

The affine patch $U_0 \subset Y$ where $b_1 = 1$, is given by
$V_{\AA{3} \times \PP{1}}(a_2 - a_1 b_2)$. Hence there is an isomorphism 
$f : \AA{3} \to U_0$ given by
\[
	f : 
	(a_1, b_2, a_3) 
	\mapsto
	((a_1, a_1 b_2, a_3), [1 : b_2]),
\]
and we see that $\AA{3}$ is birationally equivalent to $Y$, hence $\dim Y = 3$.
Moreover, $Y$ is covered by the $U_i$, which are all isomorphic to affine space
hence irreducible, and as they intersect, $Y$ is irreducible as well by
Exercise 2.21 (a). As $\widetilde{\AA{3}} \subseteq Y$ and both varieties have
dimension $3$, it follows that they are the same. \\

The blow-up $\pi : \widetilde{\AA{3}} \to \AA{3}$ maps $U = \AA{3} \setminus
V(x_1, x_2)$ to $U$, and the exceptional set is given by
\[
	\pi^{-1}(V(x_1, x_2))
	=
	\{ 
		((0, 0, a_3), [b_1 : b_2]) \in \AA{3} \times \PP{1}
	\}
	\cong
	\AA{1} \times \PP{1}.
\]

For two lines $L_1 = t(a_1, a_2, a_3), L_2 = t(b_1, b_2, b_3)$ through
$V(x_1,x_2)$ to intersect in the blow-up, we need them to intersect in
$\AA{3}$, and that their projections onto the first two coordinates coincide,
$t(a_1, a_2) = t(b_1, b_2)$, since these are the lines which get sent to the
projective axis. \\

The geometric interpretation of this is that the exceptional set parameterizes
the projected directions onto the $x_1,x_2$-plane. So in some sense, the
blow-up unravels all planes parallel $x_1, x_2$-plane where each plane is
unraveled in a way analogous to the blow-up of $\AA{2}$ at the origin. Note
however that the unraveling happens out into an extra fourth dimension, and has
nothing to do with the $x_3$ coordinate.

\subsection*{Ex 9.19}

Suppose that $f_1, \ldots, f_s$ is a generating set for $I(Y_1)$ and that $g_1,
\ldots, g_r$ is a generating set for $I(Y_2)$. The exceptional set of the
blow-up of $X$ is given by $\pi^{-1}(Y_1 \cap Y_2)$, and it follows
$\widetilde{Y}_i$ is the closure of the embedding of the $Y_i \setminus Y_j, j
\not = i$. The embeddings of these open sets are given by 
\begin{align*}
	\pi^{-1}(Y_1 \setminus Y_2)
	&=
	\{ 
		(a, [f_1(a):f_2(a):\ldots:f_s(a):g_1(a):g_2(a):\ldots:g_r(a)])
		:
		a \in Y_1 \setminus Y_2
	\} \\
	&=
	\{ 
		(a, [0:0:\ldots:0:g_1(a):g_2(a):\ldots:g_r(a)])
		:
		a \in Y_1 \setminus Y_2
	\}, 
\end{align*}
and 
\begin{align*}
	\pi^{-1}(Y_2 \setminus Y_1)
	&=
	\{ 
		(a, [f_1(a):f_2(a):\ldots:f_s(a):g_1(a):g_2(a):\ldots:g_r(a)])
		:
		a \in Y_2 \setminus Y_1
	\} \\
	&=
	\{ 
		(a, [f_1(a):f_2(a):\ldots:f_s(a):0:0:\ldots:0])
		:
		a \in Y_2 \setminus Y_1
	\}.
\end{align*}
We then see that the first $s$ coordinates of the projective part of
$\pi^{1}(Y_1 \setminus Y_2)$ are $0$, hence the same is true of the closure
$\widetilde{Y_1}$. Meanwhile, the last $r$ coordinates of the projective part
of $\pi^{1}(Y_2 \setminus Y_1)$ are $0$, hence the same is true of the closure
$\widetilde{Y_2}$. It follows that the two blow-ups are disjoint.


\subsection*{Ex 9.22}

\subsection*{(a)}

For a given polynomial $f \in \Kx$, let 
\[
	g_f(x_1, y_2, \ldots, y_n) 
	=
	f(x_1, x_1y_2, \ldots, x_1y_n)/x_1^{\text{min deg } f}.
\]
Then $g_f$ is always a polynomial in $\K[x_1, y_2, \ldots, y_n]$, as every term
of $f(x_1, x_1y_2, \ldots, x_1y_n)$ is divisible by $x_1^{\text{min deg } f}$.
Thus given some ideal $J \in \Kx$, we can define the ideal $J_g \in \K[x_1, y_2, \ldots, y_n]$
as $J_g = (g_f : f \in J)$. Moreover, $f(x_1, x_1y_2, \ldots, x_1 y_n) = 0$
implies that either $g_f(x_1, y_2, \ldots y_n) = 0$ or $x_1 = 0$. \\

Now, first consider the blow-up of an irreducible variety $X = V(J)$ at the
origin. Then $\widetilde{X}$ is irreducible as well. We've already seen that
the affine patch of $\widetilde{\AA{n}}$ given by $V_1 = \{(x, [y]) \subset
\widetilde{\AA{n}} : y_1 = 1\}$ can be parameterised as 
\[
	V_1 = \{ ((x_1, x_1y_2, \ldots, x_1 y_n), [1:y_2:\ldots:y_n]) \in \widetilde{\AA{n}}\},
\]
and if we consider $\widetilde{X} \cap V_1$ with this parameterisation,
we see that 
\begin{align*}
	\widetilde{X} \cap V_1
	&\subseteq
	\{
		((x_1, x_1y_2, \ldots, x_1 y_n), [1:y_2:\ldots:y_n]) 
		\in 
		\widetilde{\AA{n}}
		:
		f(x_1, x_1y_2, \ldots, x_1 y_n) = 0,
		f \in J
	\} \\
	&\subseteq
	\{
		((x_1, x_1y_2, \ldots, x_1 y_n), [1:y_2:\ldots:y_n]) 
		\in 
		\widetilde{\AA{n}}
		:
		x_1 g_f(x_1, y_2, \ldots, y_n) = 0, 
		f \in J
	\} \\
	&=
	V_{V_1}(x_1J_g) \\
	&=
	V_{V_1}(J_g) \cup V_{V_1}(x_1).
\end{align*}
But since $\widetilde{X}$ is irreducible, and $\widetilde{X} \cap V_1$ is open
in $\widetilde{X}$, we have $\widetilde{X} \cap V_1$ irreducible as well. This
means that either $\widetilde{X} \cap V_1 \subseteq V_{V_1}(J_g)$ or
$\widetilde{X} \cap V_1 \subseteq V_{V_1}(x_1)$. If $X \not \subset V(x_1)$
then clearly the first case must hold. If $X \subseteq V(x_1)$, then $x_1 \in
J$ and $1 \in J_g$ so $V_{V_1}(J_g) = \emptyset$, but we also have that $y_1 =
0$ for all $(x, [y]) \in \pi^{-1}(X_1 \setminus \{0\})$, hence $\widetilde{X}
\cap V_1 = \emptyset$ as well. In both cases $\widetilde{X} \cap V_1 \subseteq
V_{V_1}(J_g)$. \\

Meanwhile, we have an injective morphism 
\[
	\phi : V_{V_1}(J_g) \to X, \,
	((x_1, x_1y_2, \ldots, x_1 y_n), [1:y_2:\ldots:y_n]) 
	\mapsto 
	(x_1, y_2, \ldots, y_n),
\] 
hence $\dim V_{V_1}(J_g) \leq \dim X = \dim \widetilde{X} = \dim \widetilde{X}
\cap V_1$ (where the last equality follows from Exercise 5.24). To recap, we
have two irreducible varieties $\widetilde{X} \cap V_1$, $V_{V_1}(J_g)$, both
of which are closed in $V_1$, such that $\widetilde{X} \cap V_1 \subseteq
V_{V_1}(J_g)$ and $\dim V_{V_1}(J_g) = \dim \widetilde{X} \cap V_1$. It follows
that $\widetilde{X} \cap V = V_{V_1}(J_g)$ and we are done with the case when
$X$ is irreducible. \\

Now consider the case when $X = X_1 \cup \ldots \cup X_r = V(J_1) \cup \ldots
\cup V(J_r)$ is the irreducible decomposition of $X$. Then $\widetilde{X} =
\widetilde{X_1} \cup \ldots \cup \widetilde{X_r}$, hence 
\begin{align*}
	\widetilde{X} \cap V_1
	&=
	\bigcup_{i = 1}^{r} 
	(\widetilde{X}_r \cap V_1) \\
	&=
	\bigcup_{i = 1}^{r} 
	V_{V_1}((J_i)_g) \\
	&=
	V_{V_1}\left(
	\prod_{i = 1}^{r} 
	(J_i)_g
	\right),
\end{align*} 
so we are done if we can show that $(JJ')_g = J_gJ'_g$. But this is seen easily
since both ideals are generated by all products $g_{f} g_{f'}$ with $f \in J$
and $f' \in J'$. 

\subsection*{(b)}

First of, we have 
\begin{align*}
	g_f^{\ini}(x_1, y_2, \ldots, y_n)
	&=
	f^{\ini}(x_1, x_1y_2, \ldots, x_1 y_n) / x_1^{\deg(f^{\ini})} \\
	&=
	f^{\ini}(1, y_2, \ldots, y_n),
\end{align*}
so $f^{\ini}$ agrees with $g_f^{\ini}$ on $V_1$ if we evaluate $f^{\ini}$ on
the projective part of $V_1$ (i.e the $y$-coordinates). Moreover, on
$\pi^{-1}(0)$, $g_f = g_f^{\ini}$ since all terms but the initial term in $g_f$
is divisible by $x_1$. Combining these to facts shows that $f^{\ini} = g_f$ on
the projective part of $E_1$, where $E_1 = ({0} \times \PP{n - 1}) \cap V_1$ is
the first affine patch of the exceptional set of $\AA{n}$ blown up at the
origin. In other words, the first affine patch of the exceptional set of the
blow-up of $X$ is given by
\begin{align*}
	\pi^{-1}(0) \cap V_1
	&= 
	V_{E_1}(J_g) \\
	&= 
	0 \times V_{A_1}(J^{\ini})
\end{align*}
where $A_1$ is the first affine patch of $\PP{n - 1}$, and gluing together all
such patches yields
\begin{align*}
	\pi^{-1}(0)
	&= 
	0 \times V_p(J^{\ini}).
\end{align*}

\subsection*{(c)}

Let $hf \in (f)^{\ini}$. Then $hf^{\ini} = h^{\ini}f^{\ini} \in (f^{\ini})$ hence
$(f)^{\ini} = (f^{\ini})$. \\

For a counter example when $J$ isn't principal. Consider $J = (f_1 = x + y, f_2
= x - y)$ in lex order with $y < x$. Then $f_1^{\ini} = y, f_2^{\ini} = -y$, so
$(f_1^{\ini}, f_2^{\ini}) = (y)$. Meanwhile $x \in J$ so $x \in J^{\ini}$.


\subsection*{Ex 9.25}

\subsection*{(a)}

We already know what the blow-up of $\AA{2}$ at the ideal $(x_1, x_2)$ is from
Example 9.15. Denote this blow-up by $Y$. Our aim will be to show that the blow
up of the plane at $J = (f_1 = x_1^2, f_2 = x_1x_2, f_3 = x_2^2)$, which we
will denote by $\widetilde{\AA{2}}$, is isomorphic to $Y$. Let $X = V(J)$. Then
by Lemma 9.14, 
\begin{align*}
	\widetilde{\AA{2}}
	&\subseteq 
	\{
		((x_1, x_2), [y_1, y_2, y_3]) \in \AA{2} \times \PP{2} 
		:
		y_1 f_2(x_1,x_2) = f_1(x_1,x_2) y_2,
		y_1 f_3(x_1,x_2) = f_1(x_1,x_2) y_3,
		y_2 f_3(x_1,x_2) = f_2(x_1,x_2) y_3
	\} \\
	&=
	\{
		((x_1, x_2), [y_1, y_2, y_3]) \in \AA{2} \times \PP{2} 
		:
		y_1 x_1x_2 = x_1^2 y_2,
		y_1 x_2^2 = x_1^2 y_3,
		y_2 x_2^2 = x_1x_2 y_3
	\} \\
	&:= Y''.
\end{align*}
Moreover, on $\pi^{-1}(\AA{2} \setminus \{0\})$ we have 
$y_1y_3 = y^2$, hence this equation holds on its closure $\widetilde{\AA{2}}$
as well, and we have 
\[
	\widetilde{X} 
	\subseteq 
	V_{Y''}(y_1y_3 = y_2^2)
	:=
	Y'.
\] 
We will be done if we can show that $Y \cong Y'$, as from this it would follow
that $Y'$ is an irreducible variety of dimension $2$ containing
$\widetilde{\AA{2}}$, hence $\widetilde{\AA{2}} = Y' \cong Y$. \\

Consider the injective morphism 
\begin{align*}
	\phi &: Y \to Y', \\
	\phi &:
	(x, [y_1 : y_2])
	\mapsto
	(x, [y_1^2 : y_1y_2 : y_2^2]), 
\end{align*}
and let $U_i$ be the affine patch $y_i = 1$ of $Y'$. Then we can invert $\phi$
on these patches as 
\begin{align*}
	\phi^{-1} &: Y' \to Y, \\
	\phi^{-1} &:
	(x, [y_1=1 : y_2 : y_3]) 
	\mapsto
	(x, [y_1 : y_2]), \\
	\phi^{-1} &:
	(x, [y_1 : y_2=1 : y_3]) 
	\mapsto
	(x, [y_1 : y_2]), \\
	\phi^{-1} &:
	(x, [y_1 : y_2 : y_3=1]) 
	\mapsto
	(x, [y_2 : y_3]).
\end{align*}
Note that $\phi^{-1}$ is injective on all of $Y'$ as well, since $y_1y_3 =
y_2^2$ on $Y'$ hence any two $y_i, y_j$ determine the third $y_k$ uniquely up to
sign, but this signage can be ignored above as in the image of $\phi^{-1}$ we
have $[y_i : y_j] = [- y_i : - y_j]$. Hence $Y' \cong Y$ and we are done.

\subsection*{(b)}

Let's try the blow-up $\AA{2}$ at $(x_1^2, x_2)$. By Lemma 9.14, we have 
\begin{align*}
	\widetilde{\AA{2}}
	&\subseteq
	\{
		((x_1, x_2), [y_1, y_2]) \in \AA{2} \times \PP{1} 
		:
		x_1^2 y_2 = y_1 x_2.
	\} \\
	&=:
	Y
\end{align*}
On the affine patch $U_1$ where $y_1 \not = 0$ this becomes equation
$x_2 = y_2 x_1^2$, hence we have an isomorphism 
\[
	\phi_1 : \AA{2} \to U_1,
	(x_1, y_2)
	\mapsto 
	((x_1, x_1^2 y_2), [1 : y_2]).
\] 
Similarly, on the affine patch $U_2$ the equation becomes 
$x_1^2 = y_1 x_2$, and we have an isomorphism 
\[
	\phi_2 : V_{\AA{3}}(x_1^2 - x_2 x_3) \to U_2,
	(x_1, x_2, x_3)
	\mapsto 
	((x_1, x_2), [x_3 : 1]).
\] 
Hence $Y$ can be covered by irreducible patches of dimension $2$, and must
itself be an irreducible variety of dimension $2$, whence $\widetilde{\AA{2}} =
Y$. \\

Now, let's consider the tangent cone of $U_2 = V_{\AA{3}}(x_1^2 - x_2 x_3)$ at
the origin. As $x_1^2 - x_2x_3$ is homogeneous, this is simply $V(x_1^2 -
x_2x_3)$. \\

Meanwhile, let $\widehat{\AA{2}}$ be the blow-up of $\AA{2}$ at the origin.
As $\widehat{\AA{2}}$ can be covered by patches isomorphic to $\AA{2}$, it
follows that the tangent cone at any point is $C(\PP{1}) = \AA{2}$. \\

Now, as the coordinate ring of the tangent cone of $U_1$ is isomorphic to
$\Kx/(x_1^2 - x_2 x_3)$, hence not a UFD, it can't be isomorphic to any tangent
cone of $\widehat{\AA{2}}$ as they all have coordinate rings $\K[x_1,x_2]$, and
it now follows from the hint that $\widetilde{\AA{2}} \not \cong
\widehat{\AA{2}}$.



\subsection*{Ex 9.25}

\subsection*{(a)}

We will use the answer from here: https://math.stackexchange.com/questions/4001996/fundamental-points-of-cremona-plane-transformation, but it uses limits, and so we will motivate limits 
with a lemma first. 

\begin{lemma}
	Let $X, Y$ be varieties, $f : X \to Y$ a morphism, and $a_i \in X, i \in
	\N$ a sequence such that $\lim_{i \to \infty} a_i = a$ in the classical
	topology on $U$. Then
	\[
		\lim_{i \to \infty} f(a_i) = f(a).
	\]
\end{lemma}
\begin{proof}
	We can assume that $X, Y$ are affine by restricting $f$ to open affine sets
	containing $a, f(a)$. Then since Zariski closed subsets of affine varieties
	are closed in the classical topology, it follows that $f$ is continuous in
	the classical topology, after which our statement follows as well.
\end{proof}

Now assume that there exists an extension $\hat{f}$ of $f$ to all of $\PP{2}$.
Now consider two different sequences on $\PP{2}$, one on the
variety $V(x_1)$ given by $r_i = [1 : 0 : \lambda_i]$ and one on
$V(x_2)$ given by $s_i [1 : \lambda_i : 0]$ where $\lambda_i = 1/i \to 0$.
On the first sequence, $f$ is given by
\[
	f(r_i)
	=
	f([1 : 0 : \lambda_i]) 
	= 
	[0 : \lambda_i : 0]
	=
	[0 : 1 : 0],
\]
whilst, at the second sequence, $f(s_i) = [0 : 0 : \lambda_i] = [0 : 0 : 1]$.
We now arrive at the contradiction
\[
	[0 : 1 : 0]
	=
	\lim_{i \to \infty}
	\hat{f}(r_i)
	=
	\hat{f}([1 : 0 : 0])
	=
	\lim_{i \to \infty}
	\hat{f}(s_i)
	=
	[0 : 0 : 1].
\] 


\subsection*{(b)}

TODO Todo todo
LATER!

\subsection*{Ex 10.6}

Let $I_a, J_{f(a)}$ be the maximal ideals of $\mathcal{O}_{X, a},
\mathcal{O}_{Y, f(a)}$ respectively. By Corollary 10.5, it will suffice to show
that $f$ induces a linear map $g^* :  J_{f(a)} / J_{f(a)}^{2} \to I_a/I_a^{2}$
(note that the arrow goes in the opposite direction as it's a morphism in the
dual space of the tangent space). \\

We construct $g^*$ as 
\begin{align*}
	g^* :
	\left(U_{f(a)}, \phi\right) + J_{f(a)}^{2}
	&\mapsto
	(f^{-1}\left(U_{f(a)}\right), f^* \phi) + I_{a}^{2}.
\end{align*}
First of, $g$ is well defined, as $f^{-1}(U_{f(a)})$ is open in $X$ since $f$
is a morphism, $a \in f^{-1}(U_{f(a)})$, and $f^{*} \phi \in
\mathcal{O}_X(f^{-1}(U_{f(a)}))$. Linearity of $g^*$ follows by linearity of
$f^{*}$. \\

\subsection*{Ex 10.13}

\subsection*{(a)}

First suppose that $f = \bm{x}^{\bm{\alpha}}$ is a monomial.
If $\alpha_i \geq 1$, we have
\[
	\frac{\partial f}{\partial x_i} = \alpha_i \bm{x}^{(\alpha_1, \ldots, \alpha_i - 1, \ldots, \alpha_n)},
\]
hence 
\[
	x_i \frac{\partial f}{\partial x_i} = \alpha_i \bm{x}^{\bm{\alpha}} = \alpha_i f.
\]
If instead $\alpha_i = 0$ then we just have
\[
	\frac{\partial f}{\partial x_i} = 0.
\]
It now follows that  
\[
	\sum_{i = 1}^{n}
	x_i \frac{\partial f}{\partial x_i}
	=
	\sum_{i = 1}^{n}
	\alpha_i f	
	=
	f
	\sum_{i = 1}^{n}
	\alpha_i
	=
	d f.
\] 
The result now follow for general homogeneous polynomials since 
\[
	f \mapsto
	\sum_{i = 1}^{n}
	x_i \frac{\partial f}{\partial x_i}
\] 
is a linear operator.


\subsection*{(b)}

Assume WLOG that $a_0 = 1$ and consider the affine patch of $\AA{n}$ denoted
$U_0$. Let $f'_i(x_1, \ldots, x_n) = f_i(1, x_2, \ldots, x_n)$ denote the
dehomogenization of $f$. Then $X$ is smooth if and only if the affine Jacobi
criterion is holds for $X' = V(I') = V(f_1', \ldots f_n')$ at $a' = (a_1,
\ldots, a_n)$. We differentiate, and note that 
\[
	\frac{\partial f'_i}{\partial x_j}(a')
	=
	\frac{\partial f_i}{\partial x_j}(a),
\] 
for $j \not = 0$ hence $X$ is smooth at $a$ if and only if
$\left(\frac{\partial f_i}{x_j}\right)_{i, j \geq 1}$ has rank $n -
\codim_X\{a\}$. Note however that this is an $r \times n$ matrix, and does not
include the derivatives with respect to $x_0$. To deal with this, just note
that since $a_0 = 1$, it follows from part (a) that
\[
	\frac{\partial f_i}{\partial x_0}(a)
	=
	d f(a)
	-
	\sum_{i = 1}^{n}
	a_i \frac{\partial f}{\partial x_i}(a)
	=
	-
	\sum_{i = 1}^{n}
	a_i \frac{\partial f}{\partial x_i}(a),
\] 
hence the omitted $0$-th column is linearly dependent with the other columns
and does not change the rank.

\subsection*{Ex 10.17}

Let $\widetilde{X_k}$ be the blow-up of $X_k$ at the origin. The exceptional
set whenever $k > 0$ is given by $\{0\} \times V_p(x_2^2) = \{((0, 0), [1 :
0])\}$. Let's investigate $\widetilde{X_k}$ around this point, and 
consider the affine patch $U_1$ of $\widetilde{X_k}$. As in the Example 10.16,
Exercise 9.22 (a) tells us that this patch of the blow-up is given by
\[
	V\left(\frac{(x_1y_2)^2 - x_1^{2k + 1}}{x_1^{2}}\right)
	=
	V(y_2^{2} - x_1^{2(k - 1)})
	\cong
	X_{k - 1}.
\]
Hence, the $k$-th blow-up of $X_k$ at the origin is the first blow-up which is
smooth. Since isomorphic varieties have isomorphic blow-ups, it follows that
$X_k \not \cong X_l$ when $x \not = l$.

\subsection*{Ex 10.18}
Let $f : \PP{1} \to X \subset \PP{3}$ be the degree $3$ Veronese embedding
given by 
\[
	f 
	: 
	[x_0 : x_1] 
	\mapsto 
	[y_0 : y_1 : y_2 : y_3]
	=
	[x_0^{3} : x_0^{2}x_1 : x_0x_1^{2} : x_1^{3}].
\]
Then by Exercise 7.28, 
\[
	X \subseteq X' = V_p(f_1 = y_0y_3 - y_1y_2, f_2 = y_0 y_2^{3} - y_1^{3} y_3),
\] 
and as both $X, X'$ are irreducible varieties of codimension $2$ in $\PP{3}$,
we have $X = X'$ (we needed the dimensional argument to see that $X = X'$,
since we haven't motivated that $f_1,f_2$ generate the ideal of all the
polynomials of the form $z_{\bm{k}}z_{\bm{k}'} - z_{\bm{r}}z_{\bm{r}'}$ as
given in the solution of Exercise 7.28). \\

The Jacobian of $f_1, f_2$ is now given as 
\[
\begin{pmatrix}
	y_3 & -y_2 & -y_1 & y_1 \\	
	y_2^{3} & -3y_1^{2} y_3 & 3y_0y_2^{2} & -y_1^3 \\	
\end{pmatrix}
\] 
which when evaluated at any point $a \in X$, has rank at least $3 -
\codim_X(\{a\}) = 3 - 1 = 2$ since the upper row never vanishes, and the lower
row only vanishes when $a_1 = a_2 = 0$, which can't happen when $a \in X$.

\subsection*{Ex 10.22}

\subsection*{(a)}

Suppose that $X \subseteq \PP{N}$ with $\dim X = n$. The answer is trivial if
there is exist an automorphism $\phi$ on $\PP{N}$ such that at most $2n + 1$
coordinates on $\phi(X)$ are non-zero, so assume no such $\phi$ exist. I.e,
that for every $\phi \in \Aut(\PP{N})$, and $i \in [0..N]$, we have that the
affine patch $U_i$ of $\PP{N}$ intersects $X$. \\

We claim that $X$ must be smooth (?). \\

It now follows that


Then $S(X)$ contains a
chain of $n$ radical homogeneous ideals not equal to the projective ideal.
Let 


TODO Todo todo: Finnish!!!

\subsection*{(b)}

\subsection*{Ex 10.23}

\subsection*{(a)}

We prove the contrapositive. Suppose that $X = X_1 \cup \ldots \cup X_r \subset
\Pn$ is the irreducible decomposition of the hypersurface $X$, and that $r \geq
2$. Then by Exercise 6.31 (b), the two irreducible components $X_1, X_2$ meet
in some point $p \in X_1 \cap X_2$, which by Remark 10.10 (b) then must be a
singular point. \\

So, in general, any smooth variety $X \subset \Pn$ of pure dimension $\geq n/2$
is irreducible.

\subsection*{(b)}

Let $N = \PP{\binom{n + d}{n} - 1}$,
and $X \subseteq \PP{N}$ be the set of points corresponding
to homogeneous degree $d$ polynomials in $n + 1$ variables which carve out
smooth, hence irreducible, varieties of $\PP{n}$. We are asked to show that this
is a dense open set, but since projective space is irreducible, it will be
enough to show that it's an open non-empty set. \\

To see that $X$ is non-empty just consider some Fermat hypersurface
from Example 10.21. It remains only to show that $X$ is open. \\

We will first show that given some point $a \in \Pn$, the set of points in
$\PP{N}$ which correspond to degree $d$ forms vanishing at $a$ is closed. This
follows easily from the Jacobi criterion; A degree $d$ form $f \in \C[x_1,
\ldots, x_n]_d$ is singular at $a$ precisely if all partial derivatives of $f$
vanish simultaneously at $a$, and this is a closed condition on the
coefficients on $f$. Hence the subset $U_a \subset \PP{N}$ of points
corresponding to homogeneous degree $d$ polynomials which are regular at $a$ is
open. The $U_a$ cover $\PP{N}$ since no polynomial is singular everywhere. As
projective space is compact, there is a finite subcover of $U_a$, and their
intersection is also open, and equal to $X$.


\subsection*{Ex 10.24}

\subsection*{(a)}

It means that the gradient of $f$ is the same at $a$ and $b$. This means that
the tangent space of $f$ at $a$ is equal to the tangent space of $f$ at $b$.

\subsection*{(b)}

Since $f$ is a quadratic form, all partial derivatives of $f$ are linear forms.
Hence the parameterisation of the dual curve $F$ is a linear map. Moreover, as
these linear forms never simultaneously vanish at $X$ by assumption, the kernel
of this map interpreted in $\AA{3}$ intersects the cone $C(X)$ trivially. But
then the kernel of the map must be trivial, since $C(X)$ is a $3$-dimensional
surface in $\AA{3}$ (as $X$ is irreducible, it's not a line or a point in
$\PP{2}$), and can't be contained in any subspace of $\AA{3}$ of vector space
dimension less than $3$. I.e the linear map is an isomorphism, and the curve
parameterised by $F$ differs from the curve determined by $f$ by a change of
coordinates only, hence they must both be irreducible of degree $2$. \\

To do part (c), we'll need to verify that the dual of the dual is indeed the
original curve, and to do that, we will write a Macaulay2 script which
explicitly calculates the implicit form of $F$ for a generic conic $f$. We know
that $F$ is a conic, hence an implicit form for $F$ amounts to finding a linear
relation among products of order $2$ of the partial derivatives of $f$. I.e, if
we write 
\[
	f 
	= 
	a x_1^{2} + b x_1x_2 + c x_1x_3 + d x_2^2 + e x_2x_3 + r x_3^2,
\] 
and let $g_i = f'_{x_i}$, then we're looking for linear relations among 
\begin{align*}
	g_1^{2} &= 
	4a^2 x_1^2 + 4ab x_1 x_2 + 4ac x_1 x_3 + b^2 x_2^2 + 2bc x_2x_3 + c^2 x^3 \\
	g_1g_2 &= 
	2ab x_1^2 + (4ad + b^2) x_1 x_2 + (2ae + cb) x_1 x_3 + 2bd x_2^2 + (be + 2cd) x_2x_3 + ce x^3 \\
	g_1g_3 &= 
	2ac x_1^2 + (2ae + bc) x_1 x_2 + (4ar + c^2) x_1 x_3 + be x_2^2 + (2br + ce) x_2x_3 + 2cr x^3 \\
	g_2^{2} &= 
	b^2 x_1^2 + 4bd x_1 x_2 + 2be x_1 x_3 + 4 d^2 x_2^2 + 4de x_2x_3 + e^2 x^3 \\
	g_2g_3 &= 
	bc x_1^2 + (be + 2dc) x_1 x_2 + (2br + ec) x_1 x_3 + 2de x_2^2 + (4dr + e^2) x_2x_3 + 2er x^3 \\
	g_3^{2} &= 
	c^2 x_1^2 + 2ce x_1 x_2 + 4cr x_1 x_3 + e^2 x_2^2 + 4er x_2x_3 + 4r^2 x^3,
\end{align*}
Now, we're only concerned with $F(X)$, I.e we only need a quadratic form in the
$g_i$ which vanishes when $(x_1,x_2,x_3) \in X$. We can therefor reduce
everything modulo $f$. Here is a script which does this.

\begin{lstlisting}[language = Macaulay2]
clearAll

unorderedPairs = l -> 
	flatten ((0..<#l) / (i -> (toList (i..<#l) / (j -> (l_i, l_j)))))
degreeTwoMonomials = l -> unorderedPairs l / times

dualConic = f -> (
	R := ring f;
	baseR := baseRing R;
	Rq := R/ideal(f);

	-- extract cofficeints of quadratic forms of 
	-- partial derivatives % ideal(f) into a matrix
	g := flatten entries sub(jacobian f, Rq);
	gg := degreeTwoMonomials g;
	mons := degreeTwoMonomials gens R;
	monsQ := mons / (r -> r_Rq);
	ggCoeffs := gg / (x -> coefficients(x, Monomials=>monsQ)) / last;
	M := sub(fold((v1, v2) -> v1 | v2, ggCoeffs), baseR);

	-- Recast that matrix as a map from the free module baseR^6,
	-- to the quotient module baseR^6 modulo the relations 
	-- induced by f
	fCoeffsVec := last coefficients(f, Monomials=>mons);
	fCoeffsVecInBaseR := sub(fCoeffsVec, baseR);
	M = map(coker fCoeffsVecInBaseR, baseR^6, M);

	-- The kernel of M will be the baseR-linear dependecies 
	-- among the partial derivatives
	k := ker M;
	cs := (entries k_0) / (c -> sub(c, R));
	sum apply(mons, cs, times)
)

Rgeneric = QQ[a,b,c,d,e,r]
R = Rgeneric[x_0 .. x_2]

f = a * x_0^2 + b * x_0 * x_1 + c * x_0 * x_2 
	+ d * x_1^2 + e * x_1 * x_2 + r * x_2^2
F = dualConic f

<< "f: " << f << endl << endl;
<< "F: " << F << endl << endl;
<< "dualConic F: " << dualConic F << endl << endl;
<< "F(jacboian f) % f: " 
	<< F(toSequence flatten entries jacobian f) % f << endl;
<< "f(jacboian F) % F: " 
	<< f(toSequence flatten entries jacobian F) % F << endl;
\end{lstlisting}

The script produces the following output (with some manual wrapping added for
readability). \\
\begin{lstlisting}[language=Macaulay2output]
i0 : load "dual-conic.m2"


      2               2                        2
f: a*x  + b*x x  + d*x  + c*x x  + e*x x  + r*x
      0      0 1      1      0 2      1 2      2

       2         2                           2         2 
F: (- e  + 4d*r)x  + (2c*e - 4b*r)x x  + (- c  + 4a*r)x  + ...
                 0                 0 1                 1 

                                                     2         2
 ... + (- 4c*d + 2b*e)x x  + (2b*c - 4a*e)x x  + (- b  + 4a*d)x
                       0 2                 1 2                 2

                2               2                        2
dualConic F: a*x  + b*x x  + d*x  + c*x x  + e*x x  + r*x
                0      0 1      1      0 2      1 2      2

F(jacboian f) % f: 0
f(jacboian F) % F: 0
\end{lstlisting}

We see in this output that the dual curve of the dual curve is indeed the
original curve.

\subsection*{(c)}

A line $L_i$ in $\PP{2}$ is determined by a linear form like $L_i = V_p(X_i x_0
+ Y_i x_1 + Z_i x_2)$. Given $5$ lines $L_i$, let $p_i = [X_i : Y_i : Z_i]$.
The $L_i$ are in general position precisely when the $p_i$ are. From Exercise
7.30 (c) we know that any $5$ points of $\PP{2}$ in general position uniquely
determine a conic, so let $F$ be the conic determined by the $p_i$. Let $f$ be
the dual of $F$. We showed in part (b) that this implies that the dual of $f$
is $F$. This means that $F$ can be parameterised as $F = [f'_{x_0} : f'_{x_1} :
f'_{x_2}]$, and passes through the points $p_i$ for some inputs $a^i = [a^i_0 :
a^i_1 : a^i_2]$. In other words, 
\[
	[f'_{x_0}(a^i_0) : f'_{x_1}(a^i_1) : f'_{x_2}(a^i_2)]
	=
	p_i
	=
	[X_i : Y_i : Z_i],
\]
hence the tangent space of $f$ at $p_i$ is $L_i$ and we are done. \\

We end with a remark that that emphasises that the tangent space of $f$ is
indeed equal to $L_i$ at $a_i$, and not just parallel to it. Consider $f$ as a
function on affine space. Then since $f$ is a cone, the tangent plane at any
point will intersect the origin of $\AA{3}$. Similarly, if we interpret $L_i$
in affine space, it will be a plane intersecting the origin. Hence both are
spaces determined by an implicit equation of the form $Ax_0 + Bx_1 + C x_2 =
0$. I.e there is no non-zero "$m$" value in the equation as $Ax_0 + Bx_1 + C
x_2 = m$. Maybe this is obvious, I just needed to write it out for myself. 

\end{document}
