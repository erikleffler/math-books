\documentclass{article}
\usepackage[utf8]{inputenc}

\usepackage{mathtools}
\usepackage{algpseudocode}
\usepackage{amsfonts}
\usepackage{amsmath}
\usepackage{amssymb}
\usepackage{amsthm}
\usepackage{bm}
\usepackage{listings}
\usepackage{float}
\usepackage{fancyvrb}
\usepackage{xcolor}
\usepackage{tikz-cd}

\hbadness = 10000
\vbadness = 10000

\newcommand\restr[2]{{% we make the whole thing an ordinary symbol
  \left.\kern-\nulldelimiterspace % automatically resize the bar with \right
  #1 % the function
  \vphantom{\big|} % pretend it's a little taller at normal size
  \right|_{#2} % this is the delimiter
  }}

% Default fixed font does not support bold face
\DeclareFixedFont{\ttb}{T1}{txtt}{bx}{n}{12} % for bold
\DeclareFixedFont{\ttm}{T1}{txtt}{m}{n}{12}  % for normal
% Custom colors

\usepackage{color}
\definecolor{deepblue}{rgb}{0,0,0.5}
\definecolor{deepred}{rgb}{0.6,0,0}
\definecolor{deepgreen}{rgb}{0,0.5,0}

% Python style for highlighting
\newcommand\pythonstyle{\lstset{
language=Python,
basicstyle=\ttm,
morekeywords={self},              % Add keywords here
keywordstyle=\ttb\color{deepblue},
emph={MyClass,__init__},          % Custom highlighting
emphstyle=\ttb\color{deepred},    % Custom highlighting style
stringstyle=\color{deepgreen},
frame=tb,                         % Any extra options here
showstringspaces=false
}}

\definecolor{maccolor}{rgb}{0.3,0.3,0.8}
\newcommand\macoutput[1]{{\tt [Macaulay2 output o#1]}}%placeholder to compile w/o running M2
\lstdefinelanguage{Macaulay2}
{
basicstyle={\ttfamily},
keywordstyle={\color{maccolor!80!black}},
commentstyle={\color{gray}},
stringstyle={\color{red!40!black}},
rulecolor=\color{maccolor},
basewidth={1.2ex}, %workaround for prompts being same width as normal tt text
sensitive=false,
morecomment=[l]{--},
morecomment=[s]{-*}{*-},
morestring=[b]",
escapechar={`},
escapebegin={\rmfamily},
morekeywords={about,abs,AbstractToricVarieties,accumulate,Acknowledgement,acos,acosh,acot,addCancelTask,addDependencyTask,addEndFunction,addHook,AdditionalPaths,addStartFunction,addStartTask,Adjacent,adjoint,AdjointIdeal,AffineVariety,AfterEval,AfterNoPrint,AfterPrint,agm,AInfinity,alarm,AlgebraicSplines,Algorithm,Alignment,all,AllCodimensions,allowableThreads,ambient,analyticSpread,Analyzer,AnalyzeSheafOnP1,ancestor,ancestors,ANCHOR,and,andP,AngleBarList,ann,annihilator,antipode,any,append,applicationDirectory,applicationDirectorySuffix,apply,applyKeys,applyPairs,applyTable,applyValues,apropos,argument,Array,arXiv,Ascending,ascii,asin,asinh,ass,assert,associatedGradedRing,associatedPrimes,AssociativeAlgebras,AssociativeExpression,atan,atan2,atEndOfFile,Authors,autoload,AuxiliaryFiles,backtrace,Bag,Bareiss,baseFilename,BaseFunction,baseName,baseRing,baseRings,BaseRow,BasicList,basis,BasisElementLimit,Bayer,BeforePrint,beginDocumentation,BeginningMacaulay2,Benchmark,benchmark,Bertini,BesselJ,BesselY,betti,BettiCharacters,BettiTally,between,BGG,BIBasis,Binary,BinaryOperation,Binomial,binomial,BinomialEdgeIdeals,Binomials,BKZ,BlockMatrix,BLOCKQUOTE,BODY,Body,BoijSoederberg,BOLD,Book3264Examples,Boolean,BooleanGB,borel,Boxes,BR,break,Browse,Bruns,cache,CacheExampleOutput,CacheFunction,CacheTable,cacheValue,CallLimit,cancelTask,capture,catch,Caveat,CC,CDATA,ceiling,Center,centerString,Certification,ChainComplex,chainComplex,ChainComplexExtras,ChainComplexMap,ChainComplexOperations,ChangeMatrix,char,CharacteristicClasses,characters,charAnalyzer,check,CheckDocumentation,chi,Chordal,class,Classic,clean,clearAll,clearEcho,clearOutput,close,closeIn,closeOut,ClosestFit,CODE,code,codim,CodimensionLimit,coefficient,CoefficientRing,coefficientRing,coefficients,Cofactor,CohenEngine,CohenTopLevel,CoherentSheaf,CohomCalg,cohomology,coimage,CoincidentRootLoci,coker,cokernel,collectGarbage,columnAdd,columnate,columnMult,columnPermute,columnRankProfile,columnSwap,combine,Command,commandInterpreter,commandLine,COMMENT,commonest,commonRing,comodule,CompactMatrix,compactMatrixForm,CompiledFunction,CompiledFunctionBody,CompiledFunctionClosure,Complement,complement,complete,CompleteIntersection,CompleteIntersectionResolutions,Complexes,ComplexField,components,compose,compositions,compress,concatenate,conductor,ConductorElement,cone,Configuration,ConformalBlocks,conjugate,connectionCount,Consequences,Constant,Constants,constParser,content,continue,contract,Contributors,ConvexInterface,conwayPolynomial,ConwayPolynomials,copy,copyDirectory,copyFile,copyright,Core,CorrespondenceScrolls,cos,cosh,cot,CotangentSchubert,cotangentSheaf,coth,cover,coverMap,cpuTime,createTask,Cremona,csc,csch,current,currentColumnNumber,currentDirectory,currentFileDirectory,currentFileName,currentLayout,currentLineNumber,currentPackage,currentString,currentTime,Cyclotomic,Database,Date,DD,dd,deadParser,debug,debugError,DebuggingMode,debuggingMode,debugLevel,DecomposableSparseSystems,Decompose,decompose,deepSplice,Default,default,defaultPrecision,Degree,degree,degreeLength,DegreeLift,DegreeLimit,DegreeMap,DegreeOrder,DegreeRank,Degrees,degrees,degreesMonoid,degreesRing,delete,demark,denominator,Dense,Density,Depth,depth,Descending,Descent,Describe,describe,Description,det,determinant,DeterminantalRepresentations,DGAlgebras,diagonalMatrix,diameter,Dictionary,dictionary,dictionaryPath,diff,DiffAlg,difference,dim,directSum,disassemble,discriminant,dismiss,Dispatch,distinguished,DIV,Divide,divideByVariable,DivideConquer,DividedPowers,Divisor,DL,Dmodules,do,doc,docExample,docTemplate,document,DocumentTag,Down,drop,DT,dual,eagonNorthcott,EagonResolution,echoOff,echoOn,EdgeIdeals,edit,EigenSolver,eigenvalues,eigenvectors,eint,EisenbudHunekeVasconcelos,elapsedTime,elapsedTiming,elements,Eliminate,eliminate,Elimination,EliminationMatrices,EllipticCurves,EllipticIntegrals,else,EM,Email,End,end,endl,endPackage,Engine,engineDebugLevel,EngineRing,EngineTests,entries,EnumerationCurves,environment,Equation,EquivariantGB,erase,erf,erfc,error,errorDepth,euler,EulerConstant,eulers,even,EXAMPLE,ExampleFiles,ExampleItem,examples,ExampleSystems,Exclude,exec,exit,exp,expectedReesIdeal,expm1,exponents,export,exportFrom,exportMutable,Expression,expression,Ext,extend,ExteriorIdeals,ExteriorModules,exteriorPower,Factor,factor,false,Fano,FastMinors,FastNonminimal,FGLM,File,fileDictionaries,fileExecutable,fileExists,fileExitHooks,fileLength,fileMode,FileName,FilePosition,fileReadable,fileTime,fileWritable,fillMatrix,findFiles,findHeft,FindOne,findProgram,findSynonyms,FiniteFittingIdeals,First,first,firstkey,FirstPackage,fittingIdeal,flagLookup,FlatMonoid,flatten,flattenRing,Flexible,flip,floor,flush,fold,FollowLinks,for,forceGB,fork,FormalGroupLaws,Format,format,formation,FourierMotzkin,FourTiTwo,fpLLL,frac,fraction,FractionField,frames,FrobeniusThresholds,from,fromDividedPowers,fromDual,Function,FunctionApplication,FunctionBody,functionBody,FunctionClosure,FunctionFieldDesingularization,fusePairs,futureParser,GaloisField,Gamma,gb,GBDegrees,gbRemove,gbSnapshot,gbTrace,gcd,gcdCoefficients,gcdLLL,GCstats,genera,GeneralOrderedMonoid,GenerateAssertions,generateAssertions,generator,generators,Generic,GenericInitialIdeal,genericMatrix,genericSkewMatrix,genericSymmetricMatrix,gens,genus,get,getc,getChangeMatrix,getenv,getGlobalSymbol,getNetFile,getNonUnit,getPrimeWithRootOfUnity,getSymbol,getWWW,GF,gfanInterface,Givens,GKMVarieties,GLex,Global,global,globalAssign,globalAssignFunction,GlobalAssignHook,globalAssignment,globalAssignmentHooks,GlobalDictionary,GlobalHookStore,globalReleaseFunction,GlobalReleaseHook,Gorenstein,GradedLieAlgebras,GradedModule,gradedModule,GradedModuleMap,gradedModuleMap,gramm,GraphicalModels,GraphicalModelsMLE,Graphics,graphIdeal,graphRing,Graphs,Grassmannian,GRevLex,GroebnerBasis,groebnerBasis,GroebnerBasisOptions,GroebnerStrata,GroebnerWalk,groupID,GroupLex,GroupRevLex,GTZ,Hadamard,handleInterrupts,HardDegreeLimit,hash,HashTable,hashTable,HEAD,HEADER1,HEADER2,HEADER3,HEADER4,HEADER5,HEADER6,HeaderType,Heading,Headline,Heft,heft,Height,height,help,Hermite,hermite,Hermitian,HH,hh,HigherCIOperators,HighestWeights,Hilbert,hilbertFunction,hilbertPolynomial,hilbertSeries,HodgeIntegrals,hold,Holder,Hom,homeDirectory,HomePage,Homogeneous,Homogeneous2,homogenize,homology,homomorphism,HomotopyLieAlgebra,hooks,horizontalJoin,HorizontalSpace,HR,HREF,HTML,html,httpHeaders,Hybrid,HyperplaneArrangements,Hypertext,hypertext,HypertextContainer,HypertextParagraph,icFracP,icFractions,icMap,icPIdeal,id,Ideal,ideal,idealizer,identity,if,IgnoreExampleErrors,ii,image,imaginaryPart,IMG,ImmutableType,importFrom,in,incomparable,Increment,independentSets,indeterminate,IndeterminateNumber,Index,index,indexComponents,IndexedVariable,IndexedVariableTable,indices,inducedMap,inducesWellDefinedMap,InexactField,InexactFieldFamily,InexactNumber,InfiniteNumber,infinity,info,InfoDirSection,infoHelp,Inhomogeneous,input,Inputs,insert,installAssignmentMethod,installedPackages,installHilbertFunction,installMethod,installMinprimes,installPackage,InstallPrefix,instance,instances,IntegralClosure,integralClosure,integrate,IntermediateMarkUpType,interpreterDepth,intersect,intersectInP,Intersection,intersection,interval,InvariantRing,inverse,InverseMethod,inversePermutation,Inverses,inverseSystem,InverseSystems,Invertible,InvolutiveBases,irreducibleCharacteristicSeries,irreducibleDecomposition,isAffineRing,isANumber,isBorel,isCanceled,isCommutative,isConstant,isDirectory,isDirectSum,isEmpty,isField,isFinite,isFinitePrimeField,isFreeModule,isGlobalSymbol,isHomogeneous,isIdeal,isInfinite,isInjective,isInputFile,isIsomorphism,isLinearType,isListener,isLLL,isMember,isModule,isMonomialIdeal,isNormal,isOpen,isOutputFile,isPolynomialRing,isPrimary,isPrime,isPrimitive,isPseudoprime,isQuotientModule,isQuotientOf,isQuotientRing,isReady,isReal,isReduction,isRegularFile,isRing,isSkewCommutative,isSorted,isSquareFree,isStandardGradedPolynomialRing,isSubmodule,isSubquotient,isSubset,isSupportedInZeroLocus,isSurjective,isTable,isUnit,isWellDefined,isWeylAlgebra,ITALIC,Iterate,Jacobian,jacobian,jacobianDual,Jets,Join,join,Jupyter,K3Carpets,K3Surfaces,Keep,KeepFiles,KeepZeroes,ker,kernel,kernelLLL,kernelOfLocalization,Key,keys,Keyword,Keywords,kill,koszul,Kronecker,KustinMiller,LABEL,last,lastMatch,LATER,LatticePolytopes,Layout,lcm,leadCoefficient,leadComponent,leadMonomial,leadTerm,Left,left,length,LengthLimit,letterParser,Lex,LexIdeals,LI,Licenses,LieTypes,lift,liftable,Limit,limitFiles,limitProcesses,Linear,LinearAlgebra,LinearTruncations,lineNumber,lines,LINK,linkFile,List,list,listForm,listLocalSymbols,listSymbols,listUserSymbols,LITERAL,LLL,LLLBases,lngamma,load,loadDepth,LoadDocumentation,loadedFiles,loadedPackages,loadPackage,Local,local,localDictionaries,LocalDictionary,localize,LocalRings,locate,log,log1p,LongPolynomial,lookup,lookupCount,LowerBound,LUdecomposition,M0nbar,M2CODE,Macaulay2Doc,makeDirectory,MakeDocumentation,makeDocumentTag,MakeHTML,MakeInfo,MakeLinks,makePackageIndex,MakePDF,makeS2,Manipulator,map,MapExpression,MapleInterface,markedGB,Markov,MarkUpType,match,mathML,Matrix,matrix,MatrixExpression,Matroids,max,maxAllowableThreads,maxExponent,MaximalRank,maxPosition,MaxReductionCount,MCMApproximations,member,memoize,memoizeClear,memoizeValues,MENU,merge,mergePairs,META,method,MethodFunction,MethodFunctionBinary,MethodFunctionSingle,MethodFunctionWithOptions,methodOptions,methods,midpoint,min,minExponent,mingens,mingle,minimalBetti,MinimalGenerators,MinimalMatrix,minimalPresentation,minimalPresentationMap,minimalPresentationMapInv,MinimalPrimes,minimalPrimes,minimalReduction,Minimize,minimizeFilename,MinimumVersion,minors,minPosition,minPres,minprimes,Minus,minus,Miura,MixedMultiplicity,mkdir,mod,Module,module,ModuleDeformations,modulo,MonodromySolver,Monoid,monoid,MonoidElement,Monomial,MonomialAlgebras,monomialCurveIdeal,MonomialIdeal,monomialIdeal,MonomialIntegerPrograms,MonomialOrbits,MonomialOrder,Monomials,monomials,MonomialSize,monomialSubideal,moveFile,multidegree,multidoc,multigraded,MultigradedBettiTally,MultiGradedRationalMap,multiplicity,MultiplicitySequence,MultiplierIdeals,MultiplierIdealsDim2,MultiprojectiveVarieties,mutable,MutableHashTable,mutableIdentity,MutableList,MutableMatrix,mutableMatrix,NAGtypes,Name,nanosleep,Nauty,NautyGraphs,NCAlgebra,NCLex,needs,needsPackage,Net,net,NetFile,netList,new,newClass,newCoordinateSystem,NewFromMethod,newline,NewMethod,newNetFile,NewOfFromMethod,NewOfMethod,newPackage,newRing,nextkey,nextPrime,nil,NNParser,NoetherianOperators,NoetherNormalization,NonAssociativeProduct,NonminimalComplexes,nonspaceAnalyzer,NoPrint,norm,normalCone,Normaliz,NormalToricVarieties,not,Nothing,notify,notImplemented,NTL,null,nullaryMethods,nullhomotopy,nullParser,nullSpace,Number,number,NumberedVerticalList,numcols,numColumns,numerator,numeric,NumericalAlgebraicGeometry,NumericalCertification,NumericalImplicitization,NumericalLinearAlgebra,NumericalSchubertCalculus,numericInterval,NumericSolutions,numgens,numRows,numrows,odd,oeis,of,ofClass,OL,OldPolyhedra,OldToricVectorBundles,on,OneExpression,OnlineLookup,OO,oo,ooo,oooo,openDatabase,openDatabaseOut,openFiles,openIn,openInOut,openListener,OpenMath,openOut,openOutAppend,operatorAttributes,Option,OptionalComponentsPresent,optionalSignParser,Options,options,OptionTable,optP,or,Order,order,OrderedMonoid,orP,OutputDictionary,Outputs,override,pack,Package,package,PackageCitations,PackageDictionary,PackageExports,PackageImports,PackageTemplate,packageTemplate,pad,pager,PairLimit,pairs,PairsRemaining,PARA,Parametrization,parent,Parenthesize,Parser,Parsing,part,Partition,partition,partitions,parts,path,pdim,peek,PencilsOfQuadrics,Permanents,permanents,permutations,pfaffians,PHCpack,PhylogeneticTrees,pi,PieriMaps,pivots,PlaneCurveSingularities,plus,poincare,poincareN,Points,polarize,poly,Polyhedra,Polymake,PolynomialRing,Posets,Position,position,positions,PositivityToricBundles,POSIX,Postfix,Power,power,powermod,PRE,Precision,precision,Prefix,prefixDirectory,prefixPath,preimage,prepend,presentation,pretty,primaryComponent,PrimaryDecomposition,primaryDecomposition,PrimaryTag,PrimitiveElement,Print,print,printerr,printingAccuracy,printingLeadLimit,printingPrecision,printingSeparator,printingTimeLimit,printingTrailLimit,printString,printWidth,processID,Product,product,ProductOrder,profile,profileSummary,Program,programPaths,ProgramRun,Proj,Projective,ProjectiveHilbertPolynomial,projectiveHilbertPolynomial,ProjectiveVariety,promote,protect,Prune,prune,PruneComplex,pruningMap,Pseudocode,pseudocode,pseudoRemainder,Pullback,PushForward,pushForward,Python,QQ,QQParser,QRDecomposition,QthPower,Quasidegrees,QuaternaryQuartics,QuillenSuslin,quit,Quotient,quotient,quotientRemainder,QuotientRing,Radical,radical,RadicalCodim1,radicalContainment,RaiseError,random,RandomCanonicalCurves,RandomComplexes,RandomCurves,RandomCurvesOverVerySmallFiniteFields,RandomGenus14Curves,RandomIdeals,randomKRationalPoint,RandomMonomialIdeals,randomMutableMatrix,RandomObjects,RandomPlaneCurves,RandomPoints,RandomSpaceCurves,Range,rank,RationalMaps,RationalPoints,RationalPoints2,ReactionNetworks,read,readDirectory,readlink,readPackage,RealField,RealFP,realPart,realpath,RealQP,RealQP1,RealRoots,RealRR,RealXD,recursionDepth,recursionLimit,Reduce,reducedRowEchelonForm,reduceHilbert,reductionNumber,ReesAlgebra,reesAlgebra,reesAlgebraIdeal,reesIdeal,References,ReflexivePolytopesDB,regex,regexQuote,registerFinalizer,regSeqInIdeal,Regularity,regularity,relations,RelativeCanonicalResolution,relativizeFilename,Reload,remainder,RemakeAllDocumentation,remove,removeDirectory,removeFile,removeLowestDimension,reorganize,replace,RerunExamples,res,reshape,ResidualIntersections,ResLengthThree,Resolution,resolution,ResolutionsOfStanleyReisnerRings,restart,Result,resultant,Resultants,return,returnCode,Reverse,reverse,RevLex,Right,right,Ring,ring,RingElement,RingFamily,ringFromFractions,RingMap,rootPath,roots,rootURI,rotate,round,rowAdd,RowExpression,rowMult,rowPermute,rowRankProfile,rowSwap,RR,RRi,rsort,run,RunDirectory,RunExamples,RunExternalM2,runHooks,runLengthEncode,runProgram,same,saturate,Saturation,scan,scanKeys,scanLines,scanPairs,scanValues,schedule,schreyerOrder,Schubert,Schubert2,SchurComplexes,SchurFunctors,SchurRings,SCRIPT,scriptCommandLine,ScriptedFunctor,SCSCP,searchPath,sec,sech,SectionRing,SeeAlso,seeParsing,SegreClasses,select,selectInSubring,selectVariables,SelfInitializingType,SemidefiniteProgramming,Seminormalization,separate,SeparateExec,separateRegexp,Sequence,sequence,Serialization,serialNumber,Set,set,setEcho,setGroupID,setIOExclusive,setIOSynchronized,setIOUnSynchronized,setRandomSeed,setup,setupEmacs,sheaf,SheafExpression,sheafExt,sheafHom,SheafOfRings,shield,ShimoyamaYokoyama,short,show,showClassStructure,showHtml,showStructure,showTex,showUserStructure,SimpleDoc,simpleDocFrob,SimplicialComplexes,SimplicialDecomposability,SimplicialPosets,SimplifyFractions,sin,singularLocus,sinh,size,size2,SizeLimit,SkewCommutative,SlackIdeals,sleep,SLnEquivariantMatrices,SLPexpressions,SMALL,smithNormalForm,solve,someTerms,Sort,sort,sortColumns,SortStrategy,source,SourceCode,SourceRing,SPACE,SpaceCurves,SPAN,span,SparseMonomialVectorExpression,SparseResultants,SparseVectorExpression,Spec,SpechtModule,SpecialFanoFourfolds,specialFiber,specialFiberIdeal,SpectralSequences,splice,splitWWW,sqrt,SRdeformations,stack,stacksProject,Standard,standardForm,standardPairs,StartWithOneMinor,stashValue,StatePolytope,StatGraphs,status,stderr,stdio,step,StopBeforeComputation,stopIfError,StopWithMinimalGenerators,Strategy,String,STRONG,StronglyStableIdeals,STYLE,Style,style,SUB,sub,SubalgebraBases,sublists,submatrix,submatrixByDegrees,Subnodes,subquotient,SubringLimit,Subscript,subscript,SUBSECTION,subsets,substitute,substring,subtable,Sugarless,Sum,sum,SumOfTwists,SumsOfSquares,SUP,super,SuperLinearAlgebra,Superscript,superscript,support,SVD,SVDComplexes,switch,SwitchingFields,sylvesterMatrix,Symbol,symbol,SymbolBody,symbolBody,SymbolicPowers,symlinkDirectory,symlinkFile,symmetricAlgebra,symmetricAlgebraIdeal,symmetricKernel,SymmetricPolynomials,symmetricPower,synonym,SYNOPSIS,syz,Syzygies,SyzygyLimit,SyzygyMatrix,SyzygyRows,syzygyScheme,TABLE,Table,table,take,Tally,tally,tan,TangentCone,tangentCone,tangentSheaf,tanh,target,Task,taskResult,TateOnProducts,TD,temporaryFileName,tensor,tensorAssociativity,TensorComplexes,terminalParser,terms,TEST,Test,testExample,testHunekeQuestion,TestIdeals,TestInput,tests,TEX,tex,TeXmacs,texMath,Text,TH,then,Thing,ThinSincereQuivers,ThreadedGB,threadVariable,Threshold,throw,Time,time,times,timing,TITLE,TO,to,TO2,toAbsolutePath,toCC,toDividedPowers,toDual,toExternalString,toField,TOH,toList,toLower,top,top,topCoefficients,Topcom,topComponents,topLevelMode,Tor,TorAlgebra,Toric,ToricInvariants,ToricTopology,ToricVectorBundles,toRR,toRRi,toSequence,toString,TotalPairs,toUpper,TR,trace,transpose,TriangularSets,Tries,Trim,trim,Triplets,Tropical,true,Truncate,truncate,truncateOutput,Truncations,try,TSpreadIdeals,TT,tutorial,Type,TypicalValue,typicalValues,UL,ultimate,unbag,uncurry,Undo,undocumented,uniform,uninstallAllPackages,uninstallPackage,Unique,unique,Units,Unmixed,unsequence,unstack,Up,UpdateOnly,UpperTriangular,URL,urlEncode,Usage,use,UseCachedExampleOutput,UseHilbertFunction,UserMode,userSymbols,UseSyzygies,utf8,utf8check,validate,value,values,Variable,VariableBaseName,Variables,Variety,variety,vars,Vasconcelos,Vector,vector,VectorExpression,VectorFields,VectorGraphics,Verbose,Verbosity,Verify,VersalDeformations,versalEmbedding,Version,version,VerticalList,VerticalSpace,viewHelp,VirtualResolutions,VirtualTally,VisibleList,Visualize,wait,WebApp,wedgeProduct,weightRange,Weights,WeylAlgebra,WeylGroups,when,whichGm,while,width,wikipedia,Wrap,wrap,WrapperType,XML,xor,youngest,zero,ZeroExpression,zeta,ZZ,ZZParser}
}
\lstalias{Macaulay2output}{Macaulay2}


\lstnewenvironment{python}[1][]
{
\pythonstyle
\lstset{#1}
}
{}

\theoremstyle{definition}

\newtheorem{theorem}{Theorem}[section]
\newtheorem{definition}[theorem]{Definition}
\newtheorem{corollary}[theorem]{Corollary}
\newtheorem{lemma}[theorem]{Lemma}

\newcommand{\Z}{\mathbb{Z}}
\newcommand{\Q}{\mathbb{Q}}
\newcommand{\R}{\mathbb{R}}
\newcommand{\C}{\mathbb{C}}
\newcommand{\K}{\mathbb{K}}
\renewcommand{\P}{\mathbb{P}}
\newcommand{\F}{\mathbb{F}}
\newcommand{\N}{\mathbb{N}}
\newcommand{\A}{\mathbb{A}}

\newcommand{\x}{\bm{x}}
\newcommand{\Kx}{\K[\bm{x}]}
\newcommand{\KP}[2]{\K[#1_1, #1_2, \ldots, #1_{#2}]}

\renewcommand{\AA}[1]{\A^{#1}}
\newcommand{\An}{\A^n}
\newcommand{\Am}{\A^m}

\newcommand{\PP}[1]{\P^{#1}}
\newcommand{\Pn}{\P^n}
\newcommand{\Pm}{\P^m}

\newcommand{\Hom}{\text{Hom}}
\newcommand{\Aut}{\text{Aut}}
\newcommand{\End}{\text{End}}
\newcommand{\Iso}{\text{Iso}}

\newcommand{\HF}{\text{HF}}
\newcommand{\HS}{\text{HS}}
\newcommand{\HP}{\text{HP}}


\newcommand{\lm}{\text{lm}}
\newcommand{\nr}{\text{nilrad}}
\newcommand{\spec}{\text{spec}}
\newcommand{\codim}{\text{codim}}
\newcommand{\ann}{\text{ann}}
\newcommand{\im}{\text{im}}
\newcommand{\id}{\text{id}}
\newcommand{\height}{\text{height}}

\newcommand{\coker}{\text{coker}}

\newcommand{\lt}{\text{lt}}
\newcommand{\Lt}{\text{Lt}}
\newcommand{\Lm}{\text{Lm}}

\newcommand{\catname}[1]{{\normalfont\textbf{#1}}}
\newcommand{\Set}{\catname{Set}}
\newcommand{\CRing}{\catname{CRing}}
\newcommand{\Top}{\catname{Top}}
\newcommand{\op}{\catname{op}}

\setlength{\parindent}{0pt}




\begin{document}

\section*{Ch 1}

\subsection*{Ex 1.3.9}

\subsubsection*{(a)}

We have $V(P) = V(P_1) \cup V(P_2)$, and since the variety of a prime ideal is
irreducible, we must have some $V(P_i) = V(P)$, hence $P = \sqrt{P} = I(V(P)) =
I(V(P_i)) = \sqrt{P_i} = P_i$.

\subsubsection*{(b)}

Let $g \in (I_1 \cap I_2) : f$. Then $gf \in I_1 \cap I_2$, hence $gf \in I_1$
and $gf \in I_2$, so $g \in (I_1 : f) \cap (I_2 : f)$. \\

Now suppose $g \in (I_1 : f) \cap (I_2 : f)$. Then $gf \in I_1$ and $gf \in
I_2$, hence $gf \in I_1 \cap I_2$, so $g \in (I_1 \cap I_2 : f)$.

\subsubsection*{(c)}

Let $f \in \sqrt{I_1} \cap \sqrt{I_2}$. Then we have $n_1, n_2$ such that
$f^{n_i} \in I_i$. Suppose that $n_1 \geq n_2$. Then $f^{n_1} \in I_1 \cap I_2$
and $f \in \sqrt{I_1 \cap I_2}$. \\

Now let $f \in \sqrt{I_1 \cap I_2}$. Then we have $n$ such that $f^{n} \in I_i$
and $f \in \sqrt{I_1} \cap \sqrt{I_2}$. 

\subsection*{Ex 1.3.11}

Since $R$ is Noetherian we must have some maximal $J = \ann(m) \in S$. We claim
that $J$ must be prime. To see this, let $fg \in J$ and suppose that $g \not
\in J$. Then $fg m = 0$ hence $f \in \ann(gm) \supseteq \ann(m) = J$. But
$\ann(gm) \in S$ hence $\ann(gm) = \ann(m)$ and $f \in J$.

\subsection*{Ex 1.3.12}

\subsubsection*{(a)}

Let $I = \bigcap Q_i$ be a irredundant primary decomposition. Then $\sqrt{I} =
\bigcap \sqrt{Q_i} = \bigcap P_i = \bigcap_{P_i : P_i \text{is a minimal ideal
of } I}$.

\subsubsection*{(b)}

Let
\begin{align*}
	I 
	&= 
	(y(y + z), x(x - z), (x + z)(x - z)) \\
	&= 
	(y(y + z), x(x - z), z(x - z)).
\end{align*}
We will try to find a primary decomposition for $I$ by first finding the set of
primes among $\sqrt{I : f}$ for $f \in R$. We see that 
\begin{align*}
	I : (x - z)
	&=
	(x, z, y(y + z))
	=
	(x, z, y^{2}) \\
	I : x 
	&=
	(x - z, y(y + z), z(x - z))
	=
	(x - z, y(y + z)) \\
	I : z 
	&=
	(x - z, y(y + z), z(x - z))
	=
	(x - z, y(y + z)) \\
	I : (y + z)
	&=
	(y, x(x - z), z(x - z)) \\
	I : y
	&=
	(y + z, x(x - z), z(x - z)),
\end{align*}
and 
\begin{align*}
	\sqrt{I : (x - z)}
	&=
	(x, z, y) \\
	\sqrt{I : x}
	&=
	(x - z, y(y + z)) \\
	\sqrt{I : z}
	&=
	(x - z, y(y + z)) \\
	\sqrt{I : (y + z)}
	&=
	(y, x(x - z), z(x - z))
	=
	(y, x - z) \\
	\sqrt{I : y}
	&=
	(y + z, x(x - z), z(x - z))
	=
	(y + z, (x - z)),
\end{align*}
out of which the primes are $(x, z, y), (y, x - z)$ and $(y + z, x - z)$. 
We can now guess that 
\[
	I' (y + z, x - z) \cap (y, x - z) \cap (xz, x^2, z^2, y(y + z))
\] 
is a primary decomposition of $I$. The first two ideals of the decomposition
are prime, hence primary. To see that $J = (xz, x^2, z^2, y(y + z))$ is
primary, note that $x, y, z$ generate $\K[x, y, z] / J$ and are all nilpotent
($y$ is nilpotent as $y^2 = -yz$ and $y^4 = (-yz)^2 = z^2(-y)^2 = 0$). Hence
every element, and in particular every zero divisor in $\K[x, y, z]/J$ is
nilpotent and $J$ is primary. To see that $I = I'$, note that
\begin{align*}
	I'
	&=
	(y + z, x - z) \cap (y, x - z) \cap (xz, x^2, z^2, y(y + z)) \\
	&=
	(y(y + z), x - z) \cap (xz, x^2, z^2, y^2, y(y + z)) \\
	&=
	(y(y + z), x - z) \cap (xz, x(x - z), z(x - z), y(y + z)) \\
	&=
	(y(y + z), x - z) \cap \left(
		(xz, y(y + z))
		\cup 
		(x(x - z), y(y + z))
		\cup
		(z(x - z), y(y + z))
	\right) \\
	&=
	(y(y + z), xz(x - z))
	\cup
	(y(y + z), x(x - z)) 
	\cup
	(y(y + z), z(x - z))  \\
	&=
	(y(y + z), x(x - z), z(x - z)) \\
	&=
	I
\end{align*} 

\subsubsection*{(c)}

Let's calculate $J = (xz - y^2, xw - yz) : (x, y)$.
We have
\begin{align*}
	J
	&=
	(xz - y^2, xw - yz) : (x, y) \\
	&= 
	(xz - y^2, xw - yz) : x 
	\cap
	(xz - y^2, xw - yz) : y \\
	&=
	\left(
		(xz - y^2, xw - yz) 
		\cup
		(z^2 - wy)
	\right)
	\cap
	\left(
		(xz - y^2, xw - yz) 
		\cup
		(wy - z^2)
	\right) \\
	&=
	(xz - y^2, xw - yz, z^2 - wy).
\end{align*}
As both $J$ and $(x, y)$ are prime, 

\subsection*{Ex 1.4.2}

Let $U = \An \setminus Z$ be Zariski open. Then $Z$ is the vanishing set of an
ideal and since $\Kx$ is Noetherian such ideals are finitely generated. We get
\begin{align*}
	U 
	&= 
	\An \setminus Z \\
	&=
	\An
	\setminus
	V(f_1, f_2, \ldots, f_m) \\
	&=
	\An
	\setminus
	\left(
		\bigcap_{i \in [m]} V(f_i)
	\right)\\
	&=
	\bigcup_{i \in [m]}
	\left(
		\An
		\setminus
		V(f_i)
	\right) \\
	&=
	\bigcup_{i \in [m]}
	U_{f_i},
\end{align*}
so not only can every open set be written as union of distinguished open sets,
but even more as a finite union of distinguished open sets.

\subsection*{Ex 1.4.4}

We use the fact that the ring of regular functions on all of $X$ is equal to
$A(X)$. This is shown in the text immediately after the Nullstellensatz. \\

Let $f : X \to Y$ be an isomorphism of affine varieties where $f_i$ are the
components of $f$, and define $f^* : A(Y) \to A(X)$ as $f^*(g) = g \circ f$. It
follows from the rules of composition that $f^*$ is a $\K$-algebra
homomorphism. Now suppose that $g \in \ker(f^*)$. Then $g$ vanishes on all of
$\im(f)$, but $f$ is an isomorphism so $\im(f) = Y$ hence $g = 0$ and $f^*$ is
injective. A similar argument applied to $f^{-1}$ shows that $(f^{-1})^*$ is an
injective homomorphism $A(X) \to A(Y)$, and with two injective homomorphisms in
opposing directions, it follows that $A(X) \cong A(Y)$. \\

We tackle the other direction, and let $f^*: A(Y) \to A(X)$ be a $\K$-algebra
homomorphism. Then let $f_i = f^*(y_i)$ where $y_i$ is the $i$-th coordinate
function on $Y$, and $f = (f_1, f_2, \ldots, f_m)$. Then $f : X \to \Am$, and
we can construct $F^* : \Kx \to A(X)$ via $F^*(g) = g \circ f$. Now let 
$\pi_Y : \Kx \to A(Y)$ be the projection onto the quotient. Then we have $(F^*
- (f^* \circ \pi_Y)) : \Kx \to A(X)$ and given some coordinate function $x_i
\in \Kx$, we have 
\[
	(F^* - (f^* \circ \pi_Y))(x_i)
	=
	F^*(x_i) - f^*(x_i + I(Y))
	=
	(x_i \circ f) - f^*(y_i)
	=
	f_i - f_i
	=
	0,
\] 
and since these $x_i$ generate $\Kx$, we have $F^* = f^* \circ \pi_Y$. Hence
$\ker(F^*) = I(Y)$ and it follows that $\im(f) = Y$ since the elements in
$\ker(F^*)$ are precisely those which vanish on $\im(f)$. I.e we can improve
and write $f : X \to Y$, and we claim that $f^*(g) = g \circ f$. To see this,
we just do a similar calculation as above,
\[
	f^*(y_i) - (y_i \circ f)
	=
	f_i - f_i
	= 
	0.
\] 
Now, $f$ must be surjective, as $\ker(f^*)$ is the set of polynomial function
in $A(Y)$ which vanish on $\im(f)$, whence $\im(f) = Y$ since $\ker(f^*) =
\emptyset$. We can construct a surjective function in the other direction in a
similar manner, and it follows that $X \cong Y$.

\subsection*{Ex 2.2.1}

Note that the exercise starts counting from $x_0$. We will start counting at
$x_1$, and thus expect a results with $n - 1$ substituted wherever we find $n$.
Now, $\dim(\Kx_i)$ is given by the amount of monomials of degree $i$ in $n$
variables. I.e the amount of ordered $n$-integer partitions of $i$, which is
$\binom{n - 1 + i}{i}$ (amount of ways to place $n - 1$ separators among $i$
elements).

\subsection*{Ex 2.2.5}

Let $c_i(n)$ be the coefficient of $t^i$ in the expansion of $(1 + t + t^2 +
\ldots)^n$. Then $c_i(n)$ is given by the number of ordered ways we can choose
$n$ non-negative powers of $t$ such that the sum of their powers is $i$. This
is again the amount of ordered $n$-integer partitions of $i$, hence $\binom{n -
1 + i}{i}$ as above, and we see that $\HF(\Kx, i) = c_i(n)$. Hence $\HS(\Kx, t)
= (\HS(\K[x], t))^n$ and we are done.

\subsection*{Ex 2.2.6}

Let $R = \K[x, y, z]/(x^2, y^3, z^4)$. Note that the generators are nilpotent
in $R$, and therefore so is all of $R$, and $R$ is Artinian. In particular, all
terms of degree greater or equal to $2 + 3 + 4 = 9$ vanish by the pigeonhole
principle. Instead of computing the dimension of each homogeneous component of
$R$ by hand, we generalize and inductively compute the Hilbert series of $R_n =
\Kx / I_n$ where $I_n = (x_1^{2}, x_2^{3}, \ldots, x_n^{n + 1})$. First of, we
have

\begin{align*}
	(R_1)_0 &= \K \\
	(R_1)_1 &= \K x \\
	(R_1)_2 &= 0 \\
	(R_1)_3 &= 0 \\
			&\vdots
\end{align*}

Now, note that $R_k$ is $R_{k - 1}[x_k + (x_k^{k + 1})]$. It
follows that 

\[
	(R_k)_i 
	= 
	(R_{k - 1})_i
	\oplus
	x_k (R_{k - 1})_{i - 1}
	\oplus
	x_k^2 (R_{k - 1})_{i - 2}
	\oplus
	\ldots
	\oplus
	x_k^{k - 1} (R_{k - 1})_{i - (k - 1)},
	\oplus
	x_k^{k} (R_{k - 1})_{i - k},
\] 
whence 
\[
	\HF(R_k, i)
	=
	\sum_{j = 0}^{\min(k, i)} \HF(R_{k - 1}, i - j).
\] 
I haven't figured out how to find a closed form expression for $\HF(R_k, i)$. \\

If there are $k_i \in \N_{>0}$ for each $i \in [n]$ such that $x_i^{k_i} \in
I$, since then every generator in the quotient is nilpotent, and it follows
that the whole quotient ring is as well. For the other direction, suppose $I$
is an ideal such that $\Kx/I$ is Artinian. Then the strictly descending
sequence $(x_i + I) \supsetneq (x_i^2 + I) \supsetneq (x_i^3 + I)$ must
stabilize after some $k_i$ and $x_i^{k_i} + I = x_i^{k_i + 1} + I$ implies
$x_i^{k_i}(x_i - 1) \in I$ so $x_i^{k_i} \in I$ as it's a monomial ideal, hence
homogeneous ideal, and such ideals contain all terms of the polynomials they
contain.

\subsection*{Ex 2.3.1}

\subsubsection*{(a)}

We begin by calculating $\ker(\phi)$ and $\im(\phi)$.
By looking at the matrix, we see that $[1, 1, 1] \subseteq \ker(\phi)$
and that $[1, -1, 0] = \phi([1, 0, 0])$, $[-1, 0, 1] = \phi([0, 0, 1])$,
so by a dimensionality argument it follows that 
\[
	\ker(\phi)
	=
	\{[k, k, k] : k \in \K\},\, 
	\im(\phi)
	=
	\{[k_1 - k_2, -k_1, k_2] : k_1, k_2 \in \K\}.
\] 
Thus
\[
	H_1(V) = \ker(\phi) = \K[1, 1, 1]
\]
and
\[
	H_2(V) 
	= 
	V_0 / \im(\phi)
	=
	[\K, \K, \K] / (\K[1, -1, 0] \oplus \K[-1, 0, 1]).
\]

\subsubsection*{(b)}

Using the Rank-Nullity Theorem yields

\begin{align*}
	\sum_{i = 0}^{n}
	(-1)^{i} 
	\dim(H_i(V))
	&=
	\sum_{i = 0}^{n}
	(-1)^{i} 
	\left(
		\dim(\ker(\phi_i))
		-
		\dim(\im(\phi_{i + 1}))
	\right) \\
	&=
	\sum_{i = 0}^{n}
	(-1)^{i} 
	\left(
		\dim(V_i) 
		-
		\dim(\im(\phi_i))
		- 
		\dim(\im(\phi_i + 1))
	\right) \\
	&=
	\sum_{i = 0}^{n}
	(-1)^{i} 
	\dim(V_i),
\end{align*}
where we assume that the sequence is padded with zero spaces like $0 \to \ldots
\to 0$ whence $\dim(\im(\phi_0)) = \dim(\im(\phi_{n+1}) = 0$.

\subsection*{Ex 2.3.4}

For the base case, assume that $\Delta P(i) = c$ is constant. Then $P(i) = c
P(i - 1)$, and it follows by induction that $P(n) = P(0)c^n$. \\

For the inductive step, suppose that $\Delta P(i)$ has degree $s$ and is of the
form $a_s i^s + \ldots a_0$. Then we follow the hint and let $h(i) = a_s s!
\binom{i}{s + 1}$. We then have 
\begin{align*}
	\Delta h
	&=
	a_s s! \frac{i!}{(i - s - 1)!(s + 1)!}
	-
	a_s s! \frac{(i - 1)!}{(i - s - 2)!(s + 1)!} \\
	&=
	a_s s! \left(
		\frac{i!}{(i - s - 1)!(s + 1)!}
		-
		\frac{(i - s - 1)(i - 1)!}{(i - s - 1)!(s + 1)!}
	\right) \\
	&=
	a_s s! 
	\frac{i! - (i - s - 1)(i - 1)!}{(i - s - 1)!(s + 1)!} \\
	&=
	a_s s! 
	\frac{(s + 1)(i - 1)!}{(i - s - 1)!(s + 1)!} \\
	&=
	a_s
	\frac{(i - 1)!}{(i - s - 1)!} \\
	&=
	a_s
	\prod_{k = 1}^{s + 1} (i - k),
\end{align*}
hence $\lt (\Delta h(i)) = a_s i^{s}$, and $\Delta P(i) - \Delta(h(i))$ has degree $s -
1$. Applying the inductive assumption, it follows that $P(i) - h(i)$ is a
polynomial with rational coefficients, and since $h(i)$ is constructed as such,
$P(i)$ is as well, and we are done.

\subsection*{Ex 2.3.5}
\subsubsection*{(a)}

We can assume that $p_n \not = 0$ since at least one coordinate must be
non-zero. Now consider the graded and linear change of coordinates $f_p : \Kx
\to \Kx$ where $f_p : x_i \mapsto x_i + \frac{p_i x_n}{p_n}$ which satisfies
$f_p(0 : 0 : \ldots : 0 : 1) = p$. Precomposition with $f_p$ induces a graded
isomorphism $f_p^* : R/I(p) \cong R/I(0 : 0 : \ldots : 0 : 1)$, hence we may
just as well assume that $p = (0 : 0 : \ldots : 0 : 1)$ from the beginning. \\

It now follows that $R/I(p) = \Kx/(x_0 : x_1 : \dots : x_{n - 1}) \cong \Kx$
hence $\HF(R/I(p), i) = 1$ and $\HP(R/I(p), i) = 1$.

\subsubsection*{(b)}

Let $J = I(p_1) \cap I(p_2)$. Then $\ker(\phi) = J \oplus J$, and we see that
the sequence is exact when the function $I(p_1) \cap I(p_1) \to I(p_1) \oplus
I(p_2)$ is given by $i \mapsto i \oplus i$.

\subsubsection*{(c)}

We have that $I(p_1) + I(p_2) = I(p_1 \cap p_2) = I(\emptyset) = (1) = R$,
hence $R/(I(p_1) + I(p_2)) = 0$. All homomorphisms from (b) are graded, and
since the Hilbert polynomial is additive on graded exact sequences (since the
rank of a vector space is by the Rank-Nullity theorem), we have 
\[
	\HP(I(p_1) \cap I(p_2))
	=
	\HP(I(p_1) \oplus I(p_2))
	-
	\HP(I(p_1) + I(p_2))
	=
	\HP(I(p_1) \oplus I(p_2))
	-
	\HP(R).
\] 
From (a), it follows that $\HP(I(p_1)) = \HP(R) - \HP(R/I(p_1)) = \HP(R) - 1$,
hence $\HP(I(p_1) \oplus I(p_2)) = 2\HP(R) - 2$ and
t follows that
\begin{align*}
	\HP(R/(I(p_1 \cup p_2))
	&=
	\HP(R)
	-
	\HP(I(p_1) \cap I(p_2)) \\
	&=
	\HP(R)
	-
	\left(
		\HP(I(p_1) \oplus I(p_2))
		-
		\HP(R)
	\right) \\
	&=
	\HP(R)
	-
	\left(
		2\HP(R)
		-
		2
		-
		\HP(R)
	\right) \\
	&=
	2.
\end{align*}
Repeating this argument inductively, suppose that $p_1, \ldots, p_m \in \P^{n}$
are $m$ projective points, and that $\HP(R/I(p_1, p_2 \ldots, p_{m - 1})) = m -
1$. Then let $J = I(p_1, p_2, \ldots, p_{m-1})$ and consider the exact sequence 
\[
\begin{tikzcd}
  0 
  \arrow[r] & 
  J \cap I(p_n) 
  \arrow[r, "\psi"] & 
  J \oplus I(p_n) 
  \arrow[r, "\phi"] & 
  J + I(p_n) 
  \ar[r] & 
  0
\end{tikzcd}
\] 
where $\psi$ is the diagonal injection as above and $\phi(f, g) = f - g$.
Then like before, $\{ p_1, p_2, \ldots, p_{n-1} \} \cap \{ p_n \} = \emptyset$,
hence $J + I(p_n) = (1) = R$. By the inductive assumption, 
$\HP(R/J) = n - 1$ so
\[
	\HP(J \oplus I(p_n)) 
	= 
	\HP(R) - (n - 1) 
	+ 
	(\HP(R) - 1)
	=
	2\HP(R) - n,
\]
and by the additivity of Hilbert series, 
\[
	\HP(R/(J \cap I(p_n)))
	=
	\HP(R)
	-
	\HP(J \oplus I(p_n)) 
	=
	\HP(R)
	-
	\left(2\HP(R) - n - \HP(R)\right)
	= 
	n
\] 

\subsection*{Ex 2.3.11}
A homogeneous degree two polynomial in $\K[x, y, z]$ is of the form 
\[
	f = a_0 x^2 + a_1 xy + a_2 xz + a_3 y^2 + a_4 yz + a_5 z^2,
\]
with at least one $a_i$ non-zero, and it's zero set (a conic curve) is
invariant to multiplication by $\K^*$. Hence we can identify such a conic curve
with the point $(a_0 : a_1 : a_2 : a_3 : a_4 : a_5) \in \PP{5}$. \\

The Jacobian of such a conic is given by
\[
Jf
=
\begin{pmatrix}
	2a_0 x + a_1y + a_2z \\
	a_1 x + 2a_3 y + a_4 z \\
	a_2 x + a_4 y + 2 a_5 z
\end{pmatrix}^T
\] 
Hence the conic is smooth whenever
\[
	Y_f = 
	V\left(
		f,
		2a_0 x + a_1y + a_2z, 
		a_1 x + 2a_3 y + a_4 z, 
		a_2 x + a_4 y + 2 a_5 z
	\right)
\]
is empty. However, if we look closely, we 
see that $f \in (2a_0 x + a_1y + a_2z, a_1 x + 2a_3 y + a_4 z, a_2 x + a_4 y + 2 a_5 z)$
as
\begin{align*} 
		&x(2a_0 x + a_1y + a_2z) +
		y(a_1 x + 2a_3 y + a_4 z) +
		z(a_2 x + a_4 y + 2 a_5 z) \\
		=
		&2a_0 x^2 + a_1 xy + a_2 xz +
		a_1 xy + 2a_3 y^2 + a_4y z +
		a_2 xz + a_4 yz + 2a_5 z^2 \\
		=
		&2a_0 x^2 + 2a_1 xy + 2 a_2 xz +
		2a_3 y^2 + 2 a_4y z +
		2a_5 z^2 \\
		=
		&2 f,
\end{align*}
and it follows that
\[
	Y_f = 
	V\left(
		2a_0 x + a_1y + a_2z, 
		a_1 x + 2a_3 y + a_4 z, 
		a_2 x + a_4 y + 2 a_5 z
	\right).
\] 
But then $Y_f$ is exactly $\ker A$ where $A$
is the matrix
\[
A
=
\begin{pmatrix}
	2a_0 & a_1 & a_2 \\
	a_1 & 2a_3 & a_4 \\
	a_2 & a_4 & 2 a_5
\end{pmatrix}^T
\]
The characteristic polynomial of $A$ is given by
\begin{align*}
	p_A = \det(A - \lambda I) 
	=
	&-
	x^{3} \\
	&+
	\left(2a_0+2a_3+2a_5\right)x^{2} \\
	&+
	\left(a_1^{2}+a_2^{2}-4a_0a_3+a_4^{2}-4a_0a_5-4a_3a_5\right)x \\
	&-
	2a_2^{2}a_3+2a_1a_2a_4-2a_0a_4^{2}-2a_{ 2}^{2}a_5+8a_0a_3a_5,
\end{align*} 
and since a symmetric matrix has the same rank as number of non-zero
eigenvalues, we see that $\ker A = Y_f$ contains homogeneous points precisely
when
\[
	0 
	= 
	p_A(0)
	=
	2a_2^{2}a_3+2a_1a_2a_4-2a_0a_4^{2}-2a_{ 2}^{2}a_5+8a_0a_3a_5.
\] 
This is a closed condition on the points in $\PP{5}$, and the open subset
formed by the complement is the set of points which correspond to smooth
conics.

\subsection*{Ex 3.1.3}

Let $(R, \mathfrak{m})$ be a local ring, and $P$ a f.g projective $R$-module.
Let $f_1, \ldots, f_m$ be a generating set for $P$ which is minimal with
respect to cardinality, and let $\phi : R^{m} \to P, \phi : e_i \mapsto f_i$ be
a surjective homomorphism onto $P$. Suppose $(a_1, \ldots, a_m) \in \ker \phi$.
Then $\sum a_i f_i = 0$, and it follows that no $a_i$ is a unit, as otherwise
we could remove the corresponding $f_i$ from our generating set and end up with
a smaller set. But as $R$ is local, any non-unit lies in $\mathfrak{m}$, hence
$\ker(\phi) \subseteq \mathfrak{m}R^m$. As $P$ is projective, we have $R^m = P
\oplus \ker(\phi)$, and $\mathfrak{m}R^m = \mathfrak{m}P \oplus
\mathfrak{m}\ker(\phi)$, so $\ker(\phi) \subseteq \mathfrak{m}P \oplus
\mathfrak{m}\ker(\phi)$, and $\ker(\phi) \cap P = \emptyset$ implies that
$\ker(\phi) \subseteq \mathfrak{m}\ker(\phi)$. Hence $\ker(\phi) = 0$ by
Nakayama, and $P = R^m$.

\subsection*{Ex 3.2.1}

Consider the free resolution 
\[
\begin{tikzcd}
  0 
  \arrow[r] & 
  R(-d - e)
  \arrow[r, "\phi"] & 
  R(-d) \oplus R(-e)
  \arrow[r, "\psi"] & 
  R
  \arrow[r] & 
  R/(f, g)
  \ar[r] & 
  0,
\end{tikzcd}
\] 
where we will discuss the components of the sequence in the following
paragraph. \\

Let $e_1 = (1, 0), e_2 = (0, 1)$ be generators for $R(-e) \oplus R(-d)$. Then
$\deg e_1 = d$ and $\deg e_2 = e$. We define the map $\psi$ as given by
$\psi(h_1, h_2) = fh_1 + gh_2$. This is a degree $0$ map. Suppose $(h_1, h_2)
\in \ker(\phi)$. Then $fh_1 = -gh_2$, and after dividing by the greatest common
denominator of $h_1, h_2$, we can suppose that $h_1, h_2$ are coprime as well.
As $R$ is a UFD, and $h_1, h_2$ and $f, g$ are pairwise coprime, it must be the
case that $h_1 = f$ and $h_2 = -g$. It's now easy to see that $\ker(\phi) =
((-g, f))$, and as $\ker(\phi)$ is a principal ideal generated by a degree $d +
e$ element, it's isomorphic to $R(-d - e)$ as a graded $R$-module. Hence the
sequence is a graded exact sequence, and it follows that 
\begin{align*}
	\HP(R/(f, g), i)
	&=
	\HP(R(-d-e), i)
	-
	\HP(R(-d), i)
	-
	\HP(R(-e), i)
	+
	\HP(R, i) \\
	&=
	\HP(R, i - d - e)
	-
	\HP(R, i - d)
	-
	\HP(R, i - e)
	+
	\HP(R, i) \\
	&=
	\binom{i - d - e + 2}{2}
	-
	\binom{i - d + 2}{2}
	-
	\binom{i - e + 2}{2}
	+
	\binom{i + 2}{2} \\
	&=
	\frac{1}{2}
	\big(
	(i - d - e + 2)(i - d - e + 1) \\
	&-
	(i - d + 2)(i - d + 1) \\
	&-
	(i - e + 2)(i - e + 1) \\
	&+
	(i + 2)(i + 1)
	\big) \\
	&=
	\frac{1}{2}
	\big(
	(i^2 - 2id - 2ie + 3i d^2 + 2de - 3d + e^2 - 3e + 2) \\
	&-
	(i^2 - 2id + 3i d^2 - 3d + 2) \\
	&-
	(i^2 - 2ie + 3i e^2 - 3e + 2) \\
	&+
	(i^2 + 3i + 2)
	\big) \\
	&=
	\frac{1}{2}
	\left(
		2de
	\right) \\
	&=
	d e.
\end{align*}
Now, since $f, g$ share no non-trivial divisor, it follows that the
intersection of their vanishing sets is finite, and Bezout's Theorem now
follows from our result above combined with Exercise 2.3.5 (c).

\subsection*{Ex 3.3.1}

Consider the sequence
\[
\begin{tikzcd}
  0 
  \arrow[r] & 
  R(-d) / (I : f)
  \arrow[r, "\phi"] & 
  R / I
  \arrow[r, "\psi"] & 
  R / (I, f)
  \ar[r] & 
  0,
\end{tikzcd}
\] 
where $\psi$ is the canonical quotient map and $\phi : g + (I : f) \mapsto fg +
I$. First note that $g$ is well defined, since if $g_1 + (I : f) = g_2 + (I :
f)$, then $(g_1 - g_2)f \in I$, hence $fg_1 + I = fg_2 + I$. Moreover, this
sequence is exact. It's immediate from the definitions that $\im(\phi)
\subseteq \ker(\psi)$, whilst the other inclusion follows from the fact that
$\ker(\psi)$ consists of $g + I \in R/I$ such that $g \in (I, f)$, I.e $g + I =
hf + I = \phi(h + (I : f))$ for some $h \in R$. The sequence is clearly graded
as well, and we are done.

\subsection*{Resolution of a Regular Sequence}

From now on, elements in product/sum sets will be denoted with separating $;$
as opposed to $,$, so $(x;y) \in X \times Y$ isn't confused with the ideal $(x,
y)$ generated by $x, y$. \\

The text in Ch 3.3 asks us to compute the resolution of the ideal 
generated by some regular sequence, so let's do that. \\

Let $R = \K[x,y,z,w]$. Then $f_1 = x - w$ is a non-zero divisor on $R$. $f_2 =
z^3 - w^3$ is a non-zero divisor on $R/(f_1)$. Finally, $f_3 = y^2 + yx + x^2$ is a
non-zero divisor on $R/(f_1, f_2 ,f_3)$, and we pick this to be our module $M$.
\\

The first step of our sequence will be
\[
\begin{tikzcd}
  R(-1) \oplus R(-3) \oplus R(-2)
  \arrow[r, "\psi"] & 
  R / (f_1, f_2, f_3)
  \ar[r] & 
  0,
\end{tikzcd}
\] 
where $\psi : e_i \mapsto f_i$. Now suppose $g = (g_1; g_2; g_3) \in
\ker(\phi)$. Then our hypothesis that $f_1,f_2,f_3$ is regular implies that
$g_3 \in (f_1, f_2)$, so we can write $g_3 = A_1 f_1 + A_2 f_2$. But then 
\begin{align*}
	0 
	&= 
	(A_1 f_1 + A_2 f_2)f_3 + g_2 f_2 + g_1 f_1 \\
	&= 
	(A_2 f_3 + g_2) f_2 + (A_1 f_3 + g_1) f_1,
\end{align*} 
and again our hypothesis implies that $f_1, f_2$ are coprime so $f_1 | (A_2 f_3
+ g_2)$ and $f_2 | (A_1 f_3 + g_1)$. This gives $g_2 \in (f_1, f_3)$ and $g_1
\in (f_2, f_3)$. We claim that $\ker(\psi)$ is given by
\[
	I 
	= 
	\left(
		(f_2; -f_1; 0), 
		(f_3; 0; -f_1), 
		(0; f_3; -f_2)
	\right).
\]
It's easy to see that $I \subseteq \ker(\psi)$. For the other inclusion,
we will argue via the rank of each graded piece. By the Rank-Nullity
Theorem, we expect 
\[
	\dim (R(-1) \oplus R(-3) \oplus R(-2))_i
	=
	\dim (\ker(\phi))_i
	+
	\dim (f_1, f_2, f_3)_i,
\] 
and after simplifying
\begin{align*}
	\dim_{\K} (\ker(\phi))_i
	&=
	\binom{3 + i - 1}{i}
	+
	\binom{3 + i - 3}{i}
	+
	\binom{3 + i - 2}{i}
	-
	\dim_{\K} (f_1, f_2, f_3)_i \\
	&=
	\binom{i + 2}{i}
	+
	\binom{i + 1}{i}
	+
	\binom{i}{i}
	-
	\dim_{\K} (f_1, f_2, f_3)_i.
\end{align*} 
Thus the hard part lies in computing $\dim_{\K}(f_1,f_2,f_3)_i$.

we already showed that $(g_1; g_2; g_3) \in \ker(\psi)$ means that
we can write 
\begin{align*}
	g_1 &= A_2 f_2 + A_3 f_3 \\
	g_2 &= B_1 f_1 + B_3 f_3 \\
	g_3 &= C_1 f_1 + C_2 f_2,
\end{align*}
hence 
\begin{align*}
	(g_1; g_2; g_3)
	&=
	(A_2 f_2 + A_3 f_3; B_1 f_1 + B_3 f_3; C_1 f_1 + C_2 f_2) \\
	&=
	(A_2 f_2; B_1 f_1; 0) 
	+
	(A_3 f_3; 0; C_1 f_1) 
	+
	(0; B_3 f_3; C_2 f_2),
\end{align*}
and the fact that $(g_1;g_2;g_3) \in \ker(\psi)$ means that 
\begin{align*}
	0
	&=
	A_2 f_2 f_1 + B_1 f_1 f_2
	+
	A_3 f_3 f_1 + C_1 f_1 f_3
	+
	B_3 f_3 f_2 + C_2 f_2 f_3 \\
	&=
	(A_2 + B_1) f_2 f_1
	+
	(A_3 + C_1) f_3 f_1
	+
	(B_3 + C_2) f_3 f_2.
\end{align*} 
If we now consider points on $V(f_1)$, the above equation turns into $(B_3 +
C_2) f_3 f_2 = 0$, hence $B_3 + C_2 \in (f_1)$ since our hypothesis implies
that $f_1$ is coprime to both $f_2$ and $f_3$.
TODO Todo todo!!! maybe finnish

\subsection*{Ex 3.3.1}

We will solve the exercise in a slightly more abstract setting. Suppose we have
a graded free resolution of $M$ as
\[
\begin{tikzcd}
	G:
	0
	\arrow[r] & 
	G_m
	\arrow[r, "\xi_m"] & 
	\ldots
	\arrow[r, "\xi_2"] & 
	G_1
	\arrow[r, "\xi_1"] & 
	G_0
	\arrow[r, "\xi_0"] & 
	M
	\ar[r] & 
	0,
\end{tikzcd}
\]
and graded module monomorphisms $f : N = M(-d) \to M$, $\phi_i : F_i = G_i(-d)
\to G_i$ of degree $d$ such that the following diagram commutes 
\[
\begin{tikzcd}
	& & & & & 0 \ar[d] & \\
	F:
	0
	\arrow[r] & 
	F_m
	\arrow[d, "\phi_m"]
	\arrow[r, "\psi_m"] & 
	\ldots
	\arrow[r, "\psi_2"] & 
	F_1
	\arrow[d, "\phi_1"]
	\arrow[r, "\psi_1"] & 
	F_0
	\arrow[d, "\phi_0"]
	\arrow[r, "\psi_0"] & 
	N
	\arrow[d, "f"]
	\ar[r] & 
	0 \\
	G:
	0
	\arrow[r] & 
	G_m
	\arrow[r, "\xi_m"] & 
	\ldots
	\arrow[r, "\xi_2"] & 
	G_1
	\arrow[r, "\xi_1"] & 
	G_0
	\arrow[r, "\xi_0"] & 
	M
	\arrow[d, "g"]
	\ar[r] & 
	0 \\
	& & & & & \coker(f) \ar[d] & \\
	& & & & & 0, &
\end{tikzcd}
\] 
where the $\psi_i$ are the same as the maps $\xi_i$ but between the regraded
modules $F_i$. We will find a graded free resolution to $\coker(f)$. \\ 

First note that both $g, \xi_0$ are surjective, hence their composition is as
well. Thus we let $H_0 = G_0$ and $\partial_0 : H_0 \to \coker(f)$ be given by
$\partial_0 = g \circ \xi_0$. We want to define the next module in the sequence
as $H_1 = F_0 \oplus G_1$. Before we determine the function $\partial_1 : H_1
\to H_0$, let's investigate $\ker(\partial_0)$. First note that $\im(\xi_1)
\subseteq \ker(\partial_0)$ by exactness of $G$. Moreover, we have that
\[
	\ker(\partial_0)
	=
	\xi_0^{-1}(\ker(g))
	=
	\xi_0^{-1}(\im(f)),
\] 
and as $\psi_0$ is surjective, we have $\im(f \circ \psi_0) = \im(f)$,
now commutativity of the diagram yields
\[
	\ker(\partial_0)
	=
	\xi_0^{-1}(\im(f \circ \psi_0))
	=
	\xi_0^{-1}(\im(\xi_0 \circ \phi_0)).
\]
Thus $a \in \ker(\partial_0)$ if and only if $a - \phi_0(b) \in \ker(\xi_0) =
\im(\xi_1)$ for some $b \in F_0$. But
\[
	\partial_0 \circ \phi_0
	=
	g \circ \xi_0 \circ \phi_0
	=
	g \circ f \circ \psi_0
	=
	0,
\]
hence $\phi_0(b) \in \ker(\xi_0)$ for all $b \in F_0$, so we need $a \in
\ker(\xi_0) = \im(\xi_1)$ as well. Using all our results thus far, we get
\begin{align*}
	\ker(\partial_0)
	&= 
	\{ \phi(a') + \xi_1(b') : (a', b') \in F_0 \oplus G_1 \} \\
	&= 
	\im([\phi_0, \xi_1]),
\end{align*}
and we define $\partial_1 = [\phi_0, \xi_1]$. \\

Now we define $H_2 = F_1 \oplus G_2$ and try to determine the map $\partial_2 :
H_2 \to H_1$ such that $\im(\partial_2) = \ker(\partial_1)$. To do so, we start
with an element $c \in F_1$. By commutativity of our diagram, $(a = \psi_1(c),
b = -\phi_1(c))$ is a pair in $\ker(\partial_1)$. As $\phi_0$ is injective, it
follows that there only is one $a \in F_0$ for every $b \in G_1$ such that $(a,
b) \in \ker(\partial_1)$, namely $a = -\phi_0^{-1}(\xi_1(b))$. Fixing $a$
instead, we see that any $b$ which lies in $\xi_1^{-1}(-\phi_0(a))$ works. I.e,
if $a = \psi_1(c)$, then any $b$ such that $b - \phi_1(c) \in \ker(\xi_1) =
\im(\xi_2)$. Hence
\[
	\ker(\partial_1) 
	=
	\{ (\psi_1(c), -\phi_1(c) + \xi_2(d)) : (c, d) \in (F_1 \oplus G_2) \} 
	=
	\im\left(
		\begin{bmatrix}
			\psi_1 & 0 \\	
			-\phi_1 & \xi_2 \\	
		\end{bmatrix}
	\right),
\]
and we can set
\[
	\partial_2
	=
	\begin{bmatrix}
		\psi_1 & 0 \\	
		-\phi_1 & \xi_2 \\	
	\end{bmatrix}.
\] 

The remaining steps follow by similar arguments.
TODO Todo todo: Maybe finish/solve the actual exercise instead

\subsection*{Ex 4.1.10}
\subsubsection*{(a)}

Let $G' = G = \{ f_1 = x^2 + y, f_2 = xy + x \} $.
Then 
\begin{align*}
	S(f_1, f_2)
	&=
	yf_1 - xf_2 \\
	&=
	yx^2 + y^2 - yx^2 - x^2 \\
	&=
	- x^2 + y^2, \\
\end{align*}
and $-x^2 + y^2$ reduces to $y^2 + y$ modulo $G'$
so we add $G' := G' \cup \{ f_3 = y^2 + y\}$.
We now have
\begin{align*}
	S(f_1, f_3)
	&=
	y^2f_1 - x^2f_3 \\
	&=
	y^2x^2 + y^3 - x^2y^2 - x^2y \\
	&=
	-x^2y + y^3,
\end{align*}
which reduces to $0$ modulo $G'$ according to
\begin{align*}
	(-x^2y + y^3) - (-yf_1 + yf_3)
	&=
	(-x^2y + y^3) - (x^2y - y^2 + y^3 + y^2) \\
	&=
	0.
\end{align*}
We now compute
\begin{align*}
	S(f_2, f_3)
	&=
	yf_2 - xf_3 \\
	&=
	xy^2 + xy - xy^2 - xy \\	
	&=
	0,
\end{align*}
and we are done, with a SAGBI basis for $I$ given by $G' = \{ x^2 + y, xy + x,
y^2 + y \} $

\subsubsection*{(b)}

Let $G' = G = \{ f_1 = x + y + z, f_2 = xy + xz + yz, f_3 = xyz \}$.
Then 
\begin{align*}
	S(f_1, f_2)
	&=
	yf_1 - f_2 \\
	&=
	xy + y^2 + yz - xy - xz - yz \\
	&=
	y^2 - xz,
\end{align*}
which has a leading term indivisible by any of the leading terms of $G'$, hence
we add $G' := G' \cup \{ f_4 = y^2 - xz \}$. Since the leading term of $f_1$
divides that of $f_2$, and $S(f_1, f_2)$ reduces to $0$ by our new set $G'$
containing $f_4$, we can conclude that $f_2$ is redundant and update $G' := G'
\setminus \{ f_2 \}$.
We now compute
\begin{align*}
	S(f_1, f_3)
	&=
	yzf_1 - f_2 \\
	&=
	xyz + y^2z + yz^2 - xyz \\
	&=
	y^2z + yz^2,
\end{align*}
which reduces to $-z^3$ according to
\begin{align*}
	(y^2z + yz^2) - (zf_4 + z^2f_1)
	&=
	(y^2z + yz^2) - (y^2z - xz^2 + xz^2 + yz^2 + z^3) \\
	&=
	-z^3,
\end{align*}
hence we add $G' := G' \cup \{ f_5 = z^3 \}$. Again $\lm(f_1) | \lm(f_3)$,
hence $f_3$ is redundant after adding $f_5$, and  we update $G' := G' \setminus
\{ f_3 \}$. We now compute
\begin{align*}
	S(f_1, f_4)
	&=
	y^2f_1 - xf_3 \\
	&=
	xy^2 + y^3 + y^2z - xy^2 + x^2z	\\
	&=
	y^3 + x^2z + y^2z
\end{align*}
which reduces to $0$ according to
\begin{align*}
	(y^3 + x^2z + y^2z) - (yf_4 + xzf_1 + zf_4)
	&=
	(y^3 + x^2z + y^2z) - 
	(y^3 - xyz + x^2z + xyz + xz^2 + y^2z - xz^2) \\
	&=
	0.
\end{align*}
We now compute
\begin{align*}
	S(f_1, f_5)
	&=
	z^3f_1 - xf_5 \\
	&=
	xz^3 + yz^3 + z^4 - xz^3 \\
	&=
	yz^3 + z^4
\end{align*}
which reduces to $0$ according to
\begin{align*}
	(yz^3 + z^4) - (yf_5 + zf_5)
	&=
	(yz^3 + z^4) - (yz^3 + z^4) \\
	&=
	0.
\end{align*}
We now compute
\begin{align*}
	S(f_4, f_5)
	&=
	z^3f_4 - y^2f_5 \\
	&=
	y^2z^3 + xz^4 - y^2z^3 \\
	&=
	xz^4
\end{align*}
which reduces to $0$ according to
\begin{align*}
	(xz^4) - (xzf_5)
	&=
	0,
\end{align*}
and we are done with the Groebner basis $G = (x + y + z, y^2 - xz, z^3)$.

\subsection*{Ex 4.1.11}

Such a change of variables may be obtained as $L : \Q^n \to \Q^n$ where $L(x_n)
= a_1 x_1 + a_2 x_2 + \ldots + a_n x_n$ and $L(x_i) = x_i$ for $i \not = n$.
Then $L$ is invertible, and as $\Q$ is infinite, we can pick $a_1, a_2, \ldots,
a_n$ such that each $L(q_n)$ is unique for all $q \in X$. \\

For the other part of the question, let $p_i, i \not = n \in \K[x_n]$ be the
Lagrange polynomial of nodes $(q_n, q_i)$ for $q \in X$. Then $p_i(q) = q_i$,
so $x_i - p_i(x_n)$ will vanish on all of $X$. Let $p_n = \prod_{q \in X} (x_n
- q_n)$. It's clear that every $g_i$ lies in $I(X)$, where $g_i = x_i -
p_i(x_n), i \not = n$ and $g_n = p_n$. Moreover, no polynomial in $I(X)$ has a
leading monomial smaller than $x_n^d$. To see this, note that if the leading
monomial of some $f \in I(X)$ is a power $x_n^k$, then since we are using lex
order, $f$ contains no variables $x_1, \ldots, x_{n-1}$. It follows that $f$
must be of degree at least $d$ since it's a univariate polynomial in $x_n$
vanishing at $d$ different points. It's now easy to see that the $g_i$ form a
Groebner basis since $\lm(g_i) = x_i$ when $i \not = n$.

\subsection*{Ex 4.2.6}
\subsubsection*{(a)}

$I$ generated by a subset of the variables $\Rightarrow$ $R/I$ is a free
polynomial ring in the remaining variables, hence a domain $\Rightarrow$ $I$ is
prime. \\

$I$ is prime $\Rightarrow$ $I$ is irreducible and radical $\Rightarrow$ $I$ is
generated by pure powers of variables as it's irreducible, but then the
exponents of those powers are all $1$ since $I$ is radical.

\subsubsection*{(b)}

The radical of an ideal $I$ is the intersection of all prime ideals containing
$I$. Since prime ideals are generated by pure powers, it follows that the
radical of $I$ is generated by squarefree monomials. So, $I$ radical
$\Rightarrow$ $I$ generated by squarefree monomials. \\

Now suppose that $I$ is generated by squarefree monomials. Then $I$ is an
intersection of prime ideals, hence it's own radical.

\subsubsection*{(c)}

This is just the definition of a primary ideal applied to the minimal
generators of $I$ (note that $xm$ a minimal generator $\Rightarrow x \not \in
I$), so the $\Rightarrow$ direction is immediate. \\

Suppose that $I$ has the described property and that $ab \in I$. Then every
variable in every monomial of $a$ exists to some power in $I$. We can then pick
$N$ large enough so that every term of the expansion of $a^N$ contains a power
of a variable which lies in $I$, hence $a^N$ lies in $I$. \\

\subsection*{Ex 4.4.4}

The Lex order used in Exercise 4.1.11 implies that $p_n$ (using notation of the
exercise) is a Groebner basis for the discrete variety projected onto the $x_n$
axis. Of course, $\K[x_n]$ is a PID, so the Groebner basis is just single
polynomial $p_n$, and furthermore this polynomial is just a product of the
linear forms vanishing on the $x_n$ coordinates of the variety. \\

For the exercise, we needed the $q_n$ to be different as otherwise we couldn't
construct the Lagrange polynomials. It's unclear to me how to interpret this
geometrically. If the $q_n$ coordinates of the points in $X$ weren't all
different, then $p_n$ wouldn't have degree $d$ since the projection onto $x_n$
would consist of $< d$ points - but I'm not sure what this means. 

\subsection*{Ex 4.4.6}

The map is invertible by projection onto the first two coordinates, $\phi^{-1}
= \pi : (a,b,c,d,e) \mapsto (a, b)$. Hence the two varieties are isomorphic,
whence we see that $X = \phi(\AA{2})$ is an irreducible variety of dimension
$2$. \\

Thus we're expecting $I(\Gamma_{\phi})_5 = I(X)$ to be generated by $3$
irreducible polynomials. A good guess seems to be $I(X) = (t_0^2 - t_2, t_1^2 -
t_4, t_0t_1 - t_3)$. Moreover, if we choose the Lex ordering $t_4 > t_3 > t_2 >
t_1 > t_0$, then this might have a good chance of being a Groebner basis as
well, since no element in $I$ can involve only $t_0$ and $t_1$ (Since $p, q$
are algebraically independent). \\

We compute a Groebner basis with Macaulay2 and see that our guesses are indeed
correct.

\subsection*{Ex 5.1.4}

We assume that $\partial$ is defined to be linear.
We then have
\begin{align*}
	\partial \partial \{ v_1, \ldots, v_n \}
	&=
	\partial 
	\sum_{i = 1}^{n} 
	(-1)^{i + 1} 
	\{ v_1, \ldots, \hat{v_i}, \ldots, v_n \} \\
	&=
	\sum_{i = 1}^{n} 
	(-1)^{i + 1} 
	\partial 
	\{ v_1, \ldots, \hat{v_i}, \ldots, v_n \} \\
	&=
	\sum_{i = 1}^{n} 
	(-1)^{i + 1} 
	\left(
		\sum_{j < i}
		(-1)^{j + 1}
		\{ v_1, \ldots, \hat{v_j}, \ldots, \hat{v_i} \ldots, v_n \}
		+
		\sum_{j > i}
		(-1)^{j}
		\{ v_1, \ldots, \hat{v_j}, \ldots, \hat{v_i} \ldots, v_n \}
	\right).
\end{align*}

In this sum, a term $\{ v_1, \ldots, \hat{v_r}, \ldots, \hat{v_s} \ldots, v_n
\}$ appears twice, one time when $i = r, j = s$ with sign $(-1)^{r +
1}(-1)^{s}$, and one time when $i = s, j = r$ with sign $(-1)^{s + 1}(-1)^{r +
1}$, and these two terms cancel each other. Hence the whole sum is just $0$.

\subsection*{Ex 5.1.15}

First, we define what a connected component of a simplicial complex $\Delta$ of
$V$ is. By considering the geometric realisation $|\Delta|$, it seems sensible
to say that two faces $\tau, \sigma \in \Delta$ are incident whenever $\tau
\cap \sigma \not = \emptyset$, and that two faces $\tau, \sigma \in \Delta$ lie
on the same connected component if there is a sequence of incident faces
\[
	\tau = \tau_0, \ldots, \tau_k = \sigma \in \Delta,\, 
	\tau_i \cap \tau_{i + 1} \not = \emptyset.
\] 
This is clearly an equivalence relation, and we will now show that it coincides
with the hint. \\

Suppose that $v_i, v_j$ lie in the same connected component. Then
there is some sequence of incident faces
\[
	\{v_i\}  = \tau_0, \ldots, \tau_k = \{v_j\}.
\] 
Let $\sigma_i = \tau_i \cap \tau_{i + 1}$. Then $\sigma_i \in \Delta$ by the
definition of a simplicial complex. Let $w_i \in \sigma_i$ be a representative
element drawn from every $\sigma_i$. Then for $0 < i < k$, both $w_i$ and $w_{i
- 1}$ must lie in $\tau_i$, since both $\sigma_i, \sigma_{i - 1}$ are subsets
of $\tau_i$. Now let $c_i = \{ \sigma_i, \sigma_{i - 1} \} $. Since $c_i
\subset \tau_i \in \Delta$ we have $c_i \in \Delta$ as well, and the $c_i$ form
(one of) the sequence of incident edges 
\[
	\{v_i\}  = \tau_0, c_1 \ldots, c_{k - 1}, \tau_k = \{v_j\}
\] 
which we wanted to find. \\

Now, the rank of $H_0(\Delta)$ is given as $\ker(\partial_0)/\im(\partial_1)$
where $\partial_i : C_i \to C_{i - 1}$, so let's try to understand these two
modules first. We begin with $\im(\partial_1)$, which is a submodule of $C_0$
and is given as $\{ \partial m : m \in C_1 \}$. Let $\Delta_i \subset \Delta$
be the subset of oriented $i$-simplices. Then by definition, $m \in C_1$
precisely when
\[
	m 
	=
	\sum_{\tau_i \in C_1}
	r_i \tau_i.
\]
so $C_1$ is the set of formal $R$-linear combinations of edges (two element
sets) in $\Delta$. Meanwhile, $C_0$ is the set of formal $R$-linear
combinations of vertices. As $C_{-1}$ is defined to be $0$, $\ker(\partial_0)$
is all of $C_0$. Hence $H_0$ is the cokernel of $\partial_1$. \\

Our claim is now that $v + \im(\partial_1) = u + \im(\partial_1)$ for $v, u \in
V$ if and only if $v$ and $u$ are connected. Let
\[
	C_1 \ni m 
	=
	\sum_{v_i, v_j : (i,j) \in I_m} r_{i, j} \{v_i,v_j\}.
\]
Then 
\[
	\partial(m)
	=
	\sum_{v_i, v_j : (i,j) \in I_m} r_{i, j} \partial(\{v_i,v_j\})
	=
	\sum_{(i,j) \in I_m} r_{i, j} (v_i - v_j).
\] 
Now suppose that $v + \im(\partial_1) = u + \im(\partial_1)$. Then $v - u \in
\im(\partial_1)$, and we have some $m \in C_1$ such that $v - u = \sum_{(i,j)
\in I_m} r_{i, j} (v_i - v_j)$. In the sum of $\partial(m)$, only incident
edges can contribute to cancelation of terms in the image. Also, we can never
cancel all but one terms form a single connected component, since cancelation
among two edges in the boundary either cancels all four vertices, or just two
of them. Hence that $v, u$ are the only vertices which remain after cancelation
of the sum $\sum_{(i,j) \in I_m} r_{i, j} (v_i - v_j)$, implies that they must
belong to the same connected component. Similarly, if $v$ is connected to $u$
via som sequence of edges $\{v_i, v_{i + 1}\}$, the the boundary of this sum of
edges telsecopes and is reduced down to $v - u$, hence $v + \im(\partial_1) = u
+ \im(\partial_1)$. \\

It follows that the dimension of $H_0$ is the number of connected components 
of $\Delta$.

\subsection*{Ex 5.1.17}

First of all, $\partial_2$ has full rank, hence $H_0 = \ker(\partial_2) =
0$. \\

Similarly, we can quickly see that $\partial_1$ has atleast rank $3$ since
it's a "staircase with three steps", and since $\im(\partial_2)$ has dimension
$2$ and lies in $\ker(\partial_1)$ which has dimension at most $5 - 3 = 2$,
they must be equal, thus $H_1 = 0$. Moreover, this proves that
$\ker(\partial_2)$ is $2$ dimensional, a result we will need later. \\

Finally, $H_0$ has rank $1$ by Exercise 5.1.15. \\

If we remove $C_2$ from the sequence, then $H_1 = \ker(\partial_1)$ which we've
already showed is two dimensional.

\subsection*{Ex 5.1.17}

\subsubsection*{(a)}

First note that the orientation of phases doesn't really matter to us when
computing the homologies. We just have to be carefull to be consistent with
what orderings we use for a given basis element, but that order can be whatever
we wan't. We use the following Macaulay2 code to compute the Betti numbers

\begin{lstlisting}[language=Macaulay2]
clearAll

-- The abstract simplicial complex over `V` generated by
-- `faces`. That is, we expect `faces` to be a subset
-- of the powerset of `V`, and return the smallest simplicial
-- complex containing `faces` according to Definition 5.1.3

simplicialComplex = (V, faces) -> (
	orderedFaces := faces / sort;
	maxSize := max (orderedFaces / length);
	singletons := V / (s -> {s});
	C := {singletons};
	subFaces := unique flatten (orderedFaces / subsets);
	for i in 2..maxSize do(
		C = append(C, select(subFaces, f -> #f == i));
	);
	C
)

indexOf = (l, e) -> (
	i := position(l, x -> x === e);
	if instance(i, Nothing) then -1 else i
);

boundary = (v, C, R) -> (
	img := C_(#v - 2);
	boundaryVec := vector{#img : 0_R};
	sign := 1_R;
	for i in 0..<#v do (
		bv := select(v, e -> (e =!= v_i));
		k := indexOf(img, bv);
		addVec := vector toList join((k):0_R, 1:(sign), (#img - (k + 1)):0_R);
		boundaryVec = boundaryVec + addVec;
		sign = -1_R * sign;
	);
	boundaryVec
)

simplicalComplexMaps = (C, R) -> (
	maps := {};
	for i in 1..<#C do (
		vecs := C_i / (face -> boundary(face, C, R));
		maps = append(maps, matrix(vecs));
	);
	maps
)


V = toList (v_0..v_9)

torus = {
	{v_0,v_5,v_3},
	{v_0,v_1,v_5},
	{v_1,v_5,v_2},
	{v_2,v_6,v_5},
	{v_2,v_0,v_6},
	{v_0,v_6,v_3},
	{v_4,v_3,v_5},
	{v_4,v_5,v_8},
	{v_7,v_5,v_8},
	{v_7,v_5,v_6},
	{v_7,v_9,v_6},
	{v_7,v_8,v_9},
	{v_6,v_9,v_3},
	{v_9,v_4,v_3},
	{v_4,v_0,v_8},
	{v_1,v_0,v_8},
	{v_9,v_1,v_8},
	{v_9,v_1,v_2},
	{v_9,v_0,v_2},
	{v_9,v_0,v_4}
}

torusC = simplicialComplex(V, torus)
torusMapsQQ = simplicalComplexMaps(torusC, QQ)
torusMapsZZ2 = simplicalComplexMaps(torusC, ZZ/2)
ccTorusQQ = chainComplex torusMapsQQ
ccTorusZZ2 = chainComplex torusMapsZZ2
homTorusQQ = prune HH ccTorusQQ
homTorusZZ2 = prune HH ccTorusZZ2

<< "Torus over QQ, rank H_0: " << rank homTorusQQ_0 << endl;
<< "Torus over QQ, rank H_1: " << rank homTorusQQ_1 << endl;
<< "Torus over QQ, rank H_2: " << rank homTorusQQ_2 << endl;

<< "Torus over ZZ/2, rank H_0: " << rank homTorusZZ2_0 << endl;
<< "Torus over ZZ/2, rank H_1: " << rank homTorusZZ2_1 << endl;
<< "Torus over ZZ/2, rank H_2: " << rank homTorusZZ2_2 << endl;


klein = {
	{v_0,v_5,v_3},
	{v_0,v_1,v_5},
	{v_1,v_5,v_2},
	{v_2,v_6,v_5},
	{v_2,v_0,v_6},
	{v_0,v_6,v_4},
	{v_4,v_3,v_5},
	{v_4,v_5,v_8},
	{v_7,v_5,v_8},
	{v_7,v_5,v_6},
	{v_7,v_9,v_6},
	{v_7,v_8,v_9},
	{v_6,v_9,v_4},
	{v_9,v_4,v_3},
	{v_4,v_0,v_8},
	{v_1,v_0,v_8},
	{v_9,v_1,v_8},
	{v_9,v_1,v_2},
	{v_9,v_0,v_2},
	{v_9,v_0,v_3}
}

kleinC = simplicialComplex(V, klein)
kleinMapsQQ = simplicalComplexMaps(kleinC, QQ)
kleinMapsZZ2 = simplicalComplexMaps(kleinC, ZZ/2)
ccKleinQQ = chainComplex kleinMapsQQ
ccKleinZZ2 = chainComplex kleinMapsZZ2
homKleinQQ = prune HH ccKleinQQ
homKleinZZ2 = prune HH ccKleinZZ2

<< "Klein over QQ, rank H_0: " << rank homKleinQQ_0 << endl;
<< "Klein over QQ, rank H_1: " << rank homKleinQQ_1 << endl;
<< "Klein over QQ, rank H_2: " << rank homKleinQQ_2 << endl;

<< "Klein over ZZ/2, rank H_0: " << rank homKleinZZ2_0 << endl;
<< "Klein over ZZ/2, rank H_1: " << rank homKleinZZ2_1 << endl;
<< "Klein over ZZ/2, rank H_2: " << rank homKleinZZ2_2 << endl;
\end{lstlisting} 

\vspace*{20px}
It produces the following output.
\vspace*{20px}

\begin{lstlisting}[language=Macaulay2output]
ii9 : load "simplicial-complex.m2"
Torus over QQ, rank H_0: 1
Torus over QQ, rank H_1: 2
Torus over QQ, rank H_2: 1
Torus over ZZ/2, rank H_0: 1
Torus over ZZ/2, rank H_1: 2
Torus over ZZ/2, rank H_2: 1
Klein over QQ, rank H_0: 1
Klein over QQ, rank H_1: 1
Klein over QQ, rank H_2: 0
Klein over ZZ/2, rank H_0: 1
Klein over ZZ/2, rank H_1: 2
Klein over ZZ/2, rank H_2: 1	
\end{lstlisting}


\subsection*{Ex 5.2.2}

The definition of a simplicial complex requires that if $\tau \in \Delta$, then
$\sigma \in \Delta$ whenever $\sigma \subset \tau$.

\subsection*{Ex 5.2.3}

\subsubsection*{(a)}

We informally show how to turn a simplicial $3$-polytope $P$ into a
planar graph. Pick any triangular face, and call it $U$. Call the remaining
part of the polytope $U'$. Then $U$ and $U'$ glue together to form $P$ at the
edges of the triangular face we picked. Both $U, U'$ form surfaces that are
isomorphic to a disc, I.e neither surfaces has any kinds of holes. Now
stretch the triangle $U$ so that the remaining nodes of the polytope all lie
inside it. That $U'$ is two dimensional without holes implies that the graph
inside the vertices of $U$ is planar (after stretching), hence the whole graph
is planar. \\

The result is now just a reformulation of Euler's characteristic for planar
graphs. For planar graphs stemming from simplicial polytopes, every face is a
triangle. Hence the planar graph is a collection of triangles with glued together
at some of their edges. For one triangle, we have $3$ vertices, $3$, edges, and
$2$ faces (inner and outer), hence $f_0 - f_1 + f_2 = 3$. If we inductively add
a node to a triangular planar graph, we always add $1$ edge, $1$ face, and $2$
edges, which maintains Euler's characteristic. We can also add an edge between 
two existing nodes. This adds a face and an edge, which again maintains the 
euler characteristic.

\subsubsection*{(b)}

Since the boundary of a $3$-polyope is isomorphic to the sphere, every edge on
the boundary graph of the polyope must be incident to two faces, and as our
polytope is simplicial, every face is incident to three edges. It follows 

We do the same induction as in part (a). For a triangular graph of $3$ nodes,
we have $3$ edges and $2$ faces (inner and outer). Hence the given identity
holds. If we add a node to a triangular graph with $f_1 = m, f_2 = n, 2m = 3n$,
we add $1$ face and $2$ edges, and $2m + 1 = 3n + 2$

\subsection*{Ex 6.1.4}

If $sM = 0$ for some $s \in S$, then for $m/s' \in M_{S}$ we have $m/s' =
sm/ss' = 0/ss'$, hence $M_S = 0$. \\

Now suppose $M_S = 0$. Then for every $m/s' \in M_S$, there is some $s \in S$
such that $sm = 0$. Let $m_1, \ldots, m_k$ be generators of $M$, and let $s_i
\in S$ be such that $s_i m_i = 0$. Then let $s = \prod s_i$. It follows that
$sm_i = 0$ for all $i$, whence $sM = 0$.

\subsection*{Ex 6.1.5}

By the previous exercise, $P$ is in the support of $M$ if and only if $s \in S
= R \setminus P$ and $\ann(M)$ have empty intersection, I.e iff $\ann(M)
\subseteq P$.

\subsection*{Ex 6.1.7}

Let $\lambda_i, i \in [1..r]$ be a set of linear forms, and $f$ their product. 
Then any point $p \in V_P(f)$ 


\end{document}
