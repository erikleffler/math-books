\documentclass{article}
\usepackage[utf8]{inputenc}

\usepackage{mathtools}
\usepackage{algpseudocode}
\usepackage{amsfonts}
\usepackage{amsmath}
\usepackage{amssymb}
\usepackage{amsthm}
\usepackage{bm}
\usepackage{listings}
\usepackage{float}
\usepackage{fancyvrb}
\usepackage{xcolor}
\usepackage{tikz-cd}

\hbadness = 10000
\vbadness = 10000

\newcommand\restr[2]{{% we make the whole thing an ordinary symbol
  \left.\kern-\nulldelimiterspace % automatically resize the bar with \right
  #1 % the function
  \vphantom{\big|} % pretend it's a little taller at normal size
  \right|_{#2} % this is the delimiter
  }}

% Default fixed font does not support bold face
\DeclareFixedFont{\ttb}{T1}{txtt}{bx}{n}{12} % for bold
\DeclareFixedFont{\ttm}{T1}{txtt}{m}{n}{12}  % for normal
% Custom colors

\usepackage{color}
\definecolor{deepblue}{rgb}{0,0,0.5}
\definecolor{deepred}{rgb}{0.6,0,0}
\definecolor{deepgreen}{rgb}{0,0.5,0}

% Python style for highlighting
\newcommand\pythonstyle{\lstset{
language=Python,
basicstyle=\ttm,
morekeywords={self},              % Add keywords here
keywordstyle=\ttb\color{deepblue},
emph={MyClass,__init__},          % Custom highlighting
emphstyle=\ttb\color{deepred},    % Custom highlighting style
stringstyle=\color{deepgreen},
frame=tb,                         % Any extra options here
showstringspaces=false
}}

\lstnewenvironment{python}[1][]
{
\pythonstyle
\lstset{#1}
}
{}

\theoremstyle{definition}

\newtheorem{theorem}{Theorem}[section]
\newtheorem{definition}[theorem]{Definition}
\newtheorem{corollary}[theorem]{Corollary}
\newtheorem{lemma}[theorem]{Lemma}

\newcommand{\Z}{\mathbb{Z}}
\newcommand{\Q}{\mathbb{Q}}
\newcommand{\R}{\mathbb{R}}
\newcommand{\C}{\mathbb{C}}
\newcommand{\K}{\mathbb{K}}
\newcommand{\F}{\mathbb{F}}
\newcommand{\N}{\mathbb{N}}
\newcommand{\A}{\mathbb{A}}

\newcommand{\x}{\bm{x}}
\newcommand{\Kx}{\K[\bm{x}]}
\newcommand{\Kxy}{\K[x, y]}
\newcommand{\An}{\A^n}
\newcommand{\Am}{\A^m}

\newcommand{\Hom}{\text{Hom}}
\newcommand{\Aut}{\text{Aut}}
\newcommand{\End}{\text{End}}
\newcommand{\Iso}{\text{Iso}}


\newcommand{\lm}{\text{lm}}
\newcommand{\nr}{\text{nilrad}}
\newcommand{\spec}{\text{spec}}
\newcommand{\codim}{\text{codim}}
\newcommand{\ann}{\text{ann}}
\newcommand{\im}{\text{im}}
\newcommand{\id}{\text{id}}
\newcommand{\lcm}{\text{lcm}}

\newcommand{\catname}[1]{{\normalfont\textbf{#1}}}
\newcommand{\Set}{\catname{Set}}
\newcommand{\CRing}{\catname{CRing}}
\newcommand{\Top}{\catname{Top}}
\newcommand{\op}{\catname{op}}

\setlength{\parindent}{0pt}


\begin{document}

\section*{Ch 1}

\subsection*{Ex 1.8}
Suppose towards a contradiction that $F = y^{2} + x - x^{3}$ was reducible via
$F = GH$ with $\deg(G), \deg(H) > 0$. Then $\deg(F) = \deg(G) + \deg(H)$ and as
neither of these are constant a unit, we can assume $\deg(G) = 2, \deg(H) = 1$.
Thus we can write
\[
	G = x^{2} + a_1 xy + a_2 y^{2} + a_3 x + a_4 y + a_5,\ 
	H = x + b_1 y + b_2,
\] 
and by multiplying the two together and comparing to the coefficients of $F$ we get
\begin{align*}
	b_1 + a_1 &= 0 \\
	a_1 b_1 + a_2 &= 0, \\
	a_2 b_1 &= 0,
\end{align*}
so either $a_2$ or $b_1$ is zero. If it's $a_2$ we need $a_1b_1 = 0$ by the
second equation, whence $b_1 = a_1 = 0$ by the first equation. If it's $b_1$,
we get $a_1=0$ by the first equation and $a_2 = 0$ by the second equation.
Hence $a_1 = a_2 = b_1 = 0$ and
\[
	G = x^{2} + a_3 x + a_4 y + a_5,\ 
	H = x + b_2.
\] 
But then $GH$ can't possibly be $F$, since $GH$ doesn't contain the term
$y^{2}$, a contradiction.

\subsection*{Ex 1.16}
\subsubsection*{(a)}

We have that $L = y - tx - t$ and $F$ intersect at the points where
$0 = (tx + t)^{2} + x^{2} - 1$
which after moving things around gives
\[
	x^{2} + \frac{2t^{2}}{t^{2} + 1}x + \frac{t^{2} - 1}{t^{2} + 1} = 0
\] 
This quadratic has the solutions 
\[
	x_1 
	= 
	\frac{-t^{2}}{t^{2} + 1} + \sqrt{\frac{t^{4}}{(t^{2} + 1)^{2}} + \frac{1 - t^{2}}{t^{2} + 1}} 
	= 
	\frac{-t^{2}}{t^{2} + 1} + \sqrt{\frac{1}{(t^{2} + 1)^{2}}}
	= 
	\frac{1 - t^{2}}{t^{2} + 1}
\] 
and
\[
	x_2 
	= 
	-\frac{1 + t^{2}}{t^{2} + 1}
	=
	-1
\] 
Solving for $y = tx + t$ gives
\[
	y_1 
	= 
	\frac{t - t^{3}}{t^{2} + 1} + t
	= 
	\frac{t - t^{3} + t^{3} + t}{t^{2} + 1}
	=
	\frac{2t}{t^{2} + 1},
\] 
and $y_2 = 0$. As $t$ goes from $-\infty$ to $\infty$, it sweeps the circle,
and we see that all points on the circle lie in the set
\[
	V(F) 
	= 
	\{0, 1\} 
	\cup 
	\left\{
		\left(
			\frac{1 - t^{2}}{t^{2} + 1}, 
			\frac{2t}{t^{2} + 1},
		\right)
		:
		t \in K, 1 + t^{2} \not = 0
	\right\}
\]

\subsubsection*{(b)}
$(a, b, c)$ is a Pythagorean triple exactly when $(a/c, b/c) \in F$ where $K = \Q$. We
can write $t = u/v$ with $u, v \in \Z$ whence
\begin{align*}
	V(F) 
	&= 
	\{0, 1\} 
	\cup 
	\left\{
		\left(
			\frac{1 - (u/v)^{2}}{(u/v)^{2} + 1}, 
			\frac{2(u/v)}{(u/v)^{2} + 1},
		\right)
		:
		u, v \in \Z
	\right\} \\
	&=
	\{0, 1\}  
	\cup 
	\left\{
		\left(
			\frac{u^{2} - u^{2}}{u^{2} + v^{2}}, 
			\frac{2uv}{u^{2} + v^{2}},
		\right)
		:
		u, v \in \Z
	\right\}
\end{align*}
and the statement follows.


\subsection*{Ex 2.6}

\subsubsection*{(a)}

By Prop 1.12 (a), we have some $\hat{p}(x) \in (F, G)$, which since both $F, G$
vanish at the origin, we can write $\hat{p}(x) = x^{n_x} p(x)$ for some $n_x \geq
1$ and $p(0) \not = 0$. Then in $\mathcal{O}_{0}$,
\[
	x^{n_x} = \frac{x^{n_x}p(x)}{p(x)} = \frac{\hat{p}(x)}{p(x)} \in (F, G)\mathcal{O}_0,
\] 
and it follows that $x^{n_x} = 0 \in \mathcal{O}_{0}/ (F, G)$. Picking $n =
\lcm(n_x, n_y)$ yields $x^{n} = y^{n} = 0$.

\subsubsection*{(b)}

Let 
\[
	\frac{1}{\hat{g}} 
	\in 
	\mathcal{O}_0(F, G),
\] 
and write $g = 1 - \frac{\hat{g}}{\hat{g}(0)}$. Note that $\hat{g}(0) \not = 0$
by the definition of our local ring. Then $g$ doesn't have a constant term, and
therefore, $g^{2n} = 0$, since all terms in $g^{2n}$ has degree at least $2n$,
and must contain either $x^{n}$ or $y^{n}$ which are equal to $0$ by part (a)
Let $k \in \N$ be the smallest natural number such that $g^{k + 1} = 0$. Then
\[
\frac{1}{1 - \hat{g}(0)g} \sum_{i = 0}^{k} \left(\hat{g}(0)g\right)^{k} = 1,
\] 
and
\[
	\left(\frac{1}{\hat{g}}\right)^{-1} 
	= 
	\left(\frac{1}{1 - \hat{g}(0)g}\right) 
	= 
	\sum_{i=0}^{k} \left(\hat{g}(0)g\right)^{k}
\]
is a polynomial representative.

\subsubsection*{(c)}

By (a) and (b), every element in $\mathcal{O}_{(0, 0)}/(F, G)$ is a linear
combination of terms $x^{i}y^{j}$ with $i, j \leq n$. This is a finite set and
it follows that $\mu_{0}(F, G) \in \N$.

\subsection*{Ex 2.7}

\subsubsection*{(a)}

Suppose towards a contradiction that the powers $F^{i}$ are linearly dependent
(over $\K$) in $\mathcal{O}_0 / (G)$. Then let $\pi : \mathcal{O}_0/(G) \to
\mathcal{O}_0/(F, G)$ be the canonical projection. Then $\ker(\pi) =
(F)\mathcal{O}_0/(G)$ is generated by the powers $F^{i}$ as a $\K$-vector
space, hence finite dimensional by hypothesis. The Nullity-Rank Theorem now
yields 
\[
	\dim \left(\mathcal{O}_0 / (G)\right) 
	= 
	\dim \left((F)\mathcal{O}_0/(G)\right) 
	+ 
	\dim \left(\mathcal{O}_0/(F, G)\right),
\] 
but the two terms on the RHS are finite, whilst $\mathcal{O}_0 / (G)$ is
infinite dimensional by the following lemma. 


\begin{lemma}
	Let $F$ be a curve. Then $\dim \left(\mathcal{O}_0 / (F)\right) = \infty$
\end{lemma} 
\begin{proof}

	We have either $(F) \cap \K[x] = \emptyset$ or $(F) \cap \K[y] = \emptyset$
	since $\K \in \K[x] \cap \K[y]$ and $\K \cap (F) = \emptyset$. Assume $(F)
	\cap \K[y] = \emptyset$. Then the powers of $y$ are linearly independent in
	$\Kxy / (G)$. Indeed, a linear combination of powers in $y$ over $\K$ is
	the same thing as a polynomial $p(y) \in \K[y]$, and no such polynomial
	lies in $(G)$. Moreover, if $a(x, y) \in \Kxy$ is such that $a(x, y)p(y)
	\in (G)$, then $G | ap$, but $G$ and $p$ are coprime, so $G | a$ and $a(x,
	y) = 0$ in $\Kxy/(G)$. Hence $p$ gets sent to a non-zero element when
	localizing at $0$ by bullet point 2) in the text after Prop 3.1 in
	Atiyah-Macdonald, and the powers $y$ remain linearly independent in
	$\mathcal{O}_0/(G)$. 

\end{proof}



\subsubsection*{(b)}

Let $H$ be the common component. Then $(F, G) \subseteq (H)$, so
\[
	\dim \left(\mathcal{O}_0/(H)\right) \leq \dim \left(\mathcal{O}_0/(F, G)\right),
\] 
and we showed in part (a) that $\mathcal{O}_0/(H)$ is infinite-dimensional for
any curve $H$.

\end{document}
