\documentclass{article}
\usepackage[utf8]{inputenc}

\usepackage{mathtools}
\usepackage{algpseudocode}
\usepackage{amsfonts}
\usepackage{amsmath}
\usepackage{amssymb}
\usepackage{amsthm}
\usepackage{bm}
\usepackage{listings}
\usepackage{float}
\usepackage{fancyvrb}
\usepackage{xcolor}
\usepackage{tikz-cd}

\hbadness = 10000
\vbadness = 10000

\newcommand\restr[2]{{% we make the whole thing an ordinary symbol
  \left.\kern-\nulldelimiterspace % automatically resize the bar with \right
  #1 % the function
  \vphantom{\big|} % pretend it's a little taller at normal size
  \right|_{#2} % this is the delimiter
  }}

% Default fixed font does not support bold face
\DeclareFixedFont{\ttb}{T1}{txtt}{bx}{n}{12} % for bold
\DeclareFixedFont{\ttm}{T1}{txtt}{m}{n}{12}  % for normal
% Custom colors

\usepackage{color}
\definecolor{deepblue}{rgb}{0,0,0.5}
\definecolor{deepred}{rgb}{0.6,0,0}
\definecolor{deepgreen}{rgb}{0,0.5,0}

% Python style for highlighting
\newcommand\pythonstyle{\lstset{
language=Python,
basicstyle=\ttm,
morekeywords={self},              % Add keywords here
keywordstyle=\ttb\color{deepblue},
emph={MyClass,__init__},          % Custom highlighting
emphstyle=\ttb\color{deepred},    % Custom highlighting style
stringstyle=\color{deepgreen},
frame=tb,                         % Any extra options here
showstringspaces=false
}}

\lstnewenvironment{python}[1][]
{
\pythonstyle
\lstset{#1}
}
{}

\theoremstyle{definition}

\newtheorem{theorem}{Theorem}[section]
\newtheorem{definition}[theorem]{Definition}
\newtheorem{corollary}[theorem]{Corollary}
\newtheorem{lemma}[theorem]{Lemma}

\newcommand{\Z}{\mathbb{Z}}
\newcommand{\Q}{\mathbb{Q}}
\newcommand{\R}{\mathbb{R}}
\newcommand{\C}{\mathbb{C}}
\newcommand{\K}{\mathbb{K}}
\newcommand{\F}{\mathbb{F}}
\newcommand{\N}{\mathbb{N}}
\newcommand{\A}{\mathbb{A}}

\newcommand{\x}{\bm{x}}
\newcommand{\Kx}{\K[\bm{x}]}
\newcommand{\An}{\A^n}
\newcommand{\Am}{\A^m}

\newcommand{\Hom}{\text{Hom}}
\newcommand{\Aut}{\text{Aut}}
\newcommand{\End}{\text{End}}
\newcommand{\Iso}{\text{Iso}}


\newcommand{\lm}{\text{lm}}
\newcommand{\nr}{\text{nilrad}}
\newcommand{\spec}{\text{spec}}
\newcommand{\codim}{\text{codim}}
\newcommand{\ann}{\text{ann}}
\newcommand{\im}{\text{im}}
\newcommand{\id}{\text{id}}

\newcommand{\catname}[1]{{\normalfont\textbf{#1}}}
\newcommand{\Set}{\catname{Set}}
\newcommand{\CRing}{\catname{CRing}}
\newcommand{\Top}{\catname{Top}}
\newcommand{\op}{\catname{op}}

\setlength{\parindent}{0pt}




\begin{document}

\section*{Ch 1}

\subsection*{Ex 1.1}
We have that 
\[
	p + \mathbf{v}_{pq} + \mathbf{v}_{qr} = q + \mathbf{v}_{qr} = r = p + \mathbf{v}_{pr},
\]
so by A3, we see that $\mathbf{v}_{pq} + \mathbf{v}_{qr} = \mathbf{v}_{pr}$.


\subsection*{Ex 1.2}

We have that 
\begin{align*}
	\phi \circ \phi^{-1} (u, v)
	&=
	\phi\left(
		\frac{2u}{u^{2} + v^{2} + 1},
		\frac{2v}{u^{2} + v^{2} + 1},
		\frac{u^{2} + v^{2} - 1}{u^{2} + v^{2} + 1}
	\right) \\
	&=
	\frac{u^{2} + v^{2} + 1}{u^{2} + v^{2} + 1 -u^{2} - v^{2} + 1}
	\left(
		\frac{2u}{u^{2} + v^{2} + 1},
		\frac{2v}{u^{2} + v^{2} + 1}
	\right) \\
	&=
	\frac{1}{2}
	\left(
		2u,
		2v
	\right)  \\
	&=
	(u, v),
\end{align*}
and since $\phi$ is bijective, we have that $\phi^{-1}$ is its two sided
inverse.

\subsection*{Ex 1.3}

$x \mapsto x^{3}$ is bijective on $\mathbb{R}$, and it follows that it induces
a chart on $\mathbb{R}$.

\subsection*{Ex 1.4}
$\phi_{x+}$ is injective since $x$ is determined by $y$ up to sign, after which
it's fully determined by the requirement $x > 0$ on the codomain. Moreover, its
image is open since it's $(-1, 1) \times \R$. The other set and function pairs
are charts by analogous arguments, and it's easy to see that their union is
$C$.

\subsection*{Ex 1.5}
A1 was shown in Ex 1.4. For A2, note that $U_{x+} \cap U_{x-} = \emptyset$, and
that $\phi_{x+}(U_{x+} \cap U_{y+}) = (0, 1) \times \R$ and similarly for the
other combinations. For A3 we have
\[
	\phi_{x+} \circ \phi_{y+}^{-1} (x, z)
	=
	\phi_{x+} \left(x, \sqrt{1 - x^{2}}, z\right)
	=
	\left(\sqrt{1 - x^{2}}, z\right)
\]
which is infinitely differentiable in both variables on the domain $(0, 1)
\times \R$. Similar arguments shows smoothness for the other combinations.

\subsection*{Ex 1.7}

Suppose that $\mathcal{A}_{1}, \mathcal{A}_{2}, \mathcal{A}_{3}$ are atlases.
Then $\mathcal{A}_i$ is compatible with itself, since any new chart transition
maps are identities. It's easy to see that the relation of being compatible is
symmetric, since the order of listing has no effect. Moreover, suppose that
$\mathcal{A}_1, \mathcal{A}_2$ and $\mathcal{A}_2, \mathcal{A}_3$ are
compatible, and consider $U_i \in \mathcal{A}_1, V_j \in \mathcal{A}_3$. Then
for any $W_k \in \mathcal{A}_3$ we have that $\phi_i \circ \phi_k^{-1}, \phi_k
\circ \phi_j^{-1}$ are both infinitely smooth on $\phi_k(W_k \cap U_i),
\phi_j(V_j \cap W_k)$ respectively. We can restrict these domains further to
obtain infinite smoothness on $\phi_k(V_j \cap W_k \cap U_i), \phi_j(V_j \cap
W_k \cap U_i)$, and since $\phi_k \circ \phi_j^{-1}(\phi_j(V_j \cap W_k \cap
U_i)) = \phi_k(V_j \cap W_k \cap U_i)$, we see that the composition of the two
functions $\phi_i \circ \phi_j^{-1}$ is infinitely smooth at $\phi_{j}(V_j \cap
W_k \cap U_i)$ as well. Since $\mathcal{A}_2$ covers $M$, we see that $\phi_i
\circ \phi_j^{-1}$ is infinitely smooth on all of $\phi_J(V_j \cap U_i)$ and we
are done.


\end{document}
