\documentclass{article}
\usepackage[utf8]{inputenc}

\usepackage{mathtools}
\usepackage{algpseudocode}
\usepackage{amsfonts}
\usepackage{amsmath}
\usepackage{amssymb}
\usepackage{mathrsfs}
\usepackage{amsthm}
\usepackage{bm}
\usepackage{listings}
\usepackage{float}
\usepackage{fancyvrb}
\usepackage{xcolor}
\usepackage{tikz-cd}

\definecolor{maccolor}{rgb}{0.3,0.3,0.8}
\newcommand\macoutput[1]{{\tt [Macaulay2 output o#1]}}%placeholder to compile w/o running M2
\lstdefinelanguage{Macaulay2}
{
basicstyle={\ttfamily},
keywordstyle={\color{maccolor!80!black}},
commentstyle={\color{gray}},
stringstyle={\color{red!40!black}},
rulecolor=\color{maccolor},
basewidth={1.2ex}, %workaround for prompts being same width as normal tt text
sensitive=false,
morecomment=[l]{--},
morecomment=[s]{-*}{*-},
morestring=[b]",
escapechar={`},
escapebegin={\rmfamily},
morekeywords={about,abs,AbstractToricVarieties,accumulate,Acknowledgement,acos,acosh,acot,addCancelTask,addDependencyTask,addEndFunction,addHook,AdditionalPaths,addStartFunction,addStartTask,Adjacent,adjoint,AdjointIdeal,AffineVariety,AfterEval,AfterNoPrint,AfterPrint,agm,AInfinity,alarm,AlgebraicSplines,Algorithm,Alignment,all,AllCodimensions,allowableThreads,ambient,analyticSpread,Analyzer,AnalyzeSheafOnP1,ancestor,ancestors,ANCHOR,and,andP,AngleBarList,ann,annihilator,antipode,any,append,applicationDirectory,applicationDirectorySuffix,apply,applyKeys,applyPairs,applyTable,applyValues,apropos,argument,Array,arXiv,Ascending,ascii,asin,asinh,ass,assert,associatedGradedRing,associatedPrimes,AssociativeAlgebras,AssociativeExpression,atan,atan2,atEndOfFile,Authors,autoload,AuxiliaryFiles,backtrace,Bag,Bareiss,baseFilename,BaseFunction,baseName,baseRing,baseRings,BaseRow,BasicList,basis,BasisElementLimit,Bayer,BeforePrint,beginDocumentation,BeginningMacaulay2,Benchmark,benchmark,Bertini,BesselJ,BesselY,betti,BettiCharacters,BettiTally,between,BGG,BIBasis,Binary,BinaryOperation,Binomial,binomial,BinomialEdgeIdeals,Binomials,BKZ,BlockMatrix,BLOCKQUOTE,BODY,Body,BoijSoederberg,BOLD,Book3264Examples,Boolean,BooleanGB,borel,Boxes,BR,break,Browse,Bruns,cache,CacheExampleOutput,CacheFunction,CacheTable,cacheValue,CallLimit,cancelTask,capture,catch,Caveat,CC,CDATA,ceiling,Center,centerString,Certification,ChainComplex,chainComplex,ChainComplexExtras,ChainComplexMap,ChainComplexOperations,ChangeMatrix,char,CharacteristicClasses,characters,charAnalyzer,check,CheckDocumentation,chi,Chordal,class,Classic,clean,clearAll,clearEcho,clearOutput,close,closeIn,closeOut,ClosestFit,CODE,code,codim,CodimensionLimit,coefficient,CoefficientRing,coefficientRing,coefficients,Cofactor,CohenEngine,CohenTopLevel,CoherentSheaf,CohomCalg,cohomology,coimage,CoincidentRootLoci,coker,cokernel,collectGarbage,columnAdd,columnate,columnMult,columnPermute,columnRankProfile,columnSwap,combine,Command,commandInterpreter,commandLine,COMMENT,commonest,commonRing,comodule,CompactMatrix,compactMatrixForm,CompiledFunction,CompiledFunctionBody,CompiledFunctionClosure,Complement,complement,complete,CompleteIntersection,CompleteIntersectionResolutions,Complexes,ComplexField,components,compose,compositions,compress,concatenate,conductor,ConductorElement,cone,Configuration,ConformalBlocks,conjugate,connectionCount,Consequences,Constant,Constants,constParser,content,continue,contract,Contributors,ConvexInterface,conwayPolynomial,ConwayPolynomials,copy,copyDirectory,copyFile,copyright,Core,CorrespondenceScrolls,cos,cosh,cot,CotangentSchubert,cotangentSheaf,coth,cover,coverMap,cpuTime,createTask,Cremona,csc,csch,current,currentColumnNumber,currentDirectory,currentFileDirectory,currentFileName,currentLayout,currentLineNumber,currentPackage,currentString,currentTime,Cyclotomic,Database,Date,DD,dd,deadParser,debug,debugError,DebuggingMode,debuggingMode,debugLevel,DecomposableSparseSystems,Decompose,decompose,deepSplice,Default,default,defaultPrecision,Degree,degree,degreeLength,DegreeLift,DegreeLimit,DegreeMap,DegreeOrder,DegreeRank,Degrees,degrees,degreesMonoid,degreesRing,delete,demark,denominator,Dense,Density,Depth,depth,Descending,Descent,Describe,describe,Description,det,determinant,DeterminantalRepresentations,DGAlgebras,diagonalMatrix,diameter,Dictionary,dictionary,dictionaryPath,diff,DiffAlg,difference,dim,directSum,disassemble,discriminant,dismiss,Dispatch,distinguished,DIV,Divide,divideByVariable,DivideConquer,DividedPowers,Divisor,DL,Dmodules,do,doc,docExample,docTemplate,document,DocumentTag,Down,drop,DT,dual,eagonNorthcott,EagonResolution,echoOff,echoOn,EdgeIdeals,edit,EigenSolver,eigenvalues,eigenvectors,eint,EisenbudHunekeVasconcelos,elapsedTime,elapsedTiming,elements,Eliminate,eliminate,Elimination,EliminationMatrices,EllipticCurves,EllipticIntegrals,else,EM,Email,End,end,endl,endPackage,Engine,engineDebugLevel,EngineRing,EngineTests,entries,EnumerationCurves,environment,Equation,EquivariantGB,erase,erf,erfc,error,errorDepth,euler,EulerConstant,eulers,even,EXAMPLE,ExampleFiles,ExampleItem,examples,ExampleSystems,Exclude,exec,exit,exp,expectedReesIdeal,expm1,exponents,export,exportFrom,exportMutable,Expression,expression,Ext,extend,ExteriorIdeals,ExteriorModules,exteriorPower,Factor,factor,false,Fano,FastMinors,FastNonminimal,FGLM,File,fileDictionaries,fileExecutable,fileExists,fileExitHooks,fileLength,fileMode,FileName,FilePosition,fileReadable,fileTime,fileWritable,fillMatrix,findFiles,findHeft,FindOne,findProgram,findSynonyms,FiniteFittingIdeals,First,first,firstkey,FirstPackage,fittingIdeal,flagLookup,FlatMonoid,flatten,flattenRing,Flexible,flip,floor,flush,fold,FollowLinks,for,forceGB,fork,FormalGroupLaws,Format,format,formation,FourierMotzkin,FourTiTwo,fpLLL,frac,fraction,FractionField,frames,FrobeniusThresholds,from,fromDividedPowers,fromDual,Function,FunctionApplication,FunctionBody,functionBody,FunctionClosure,FunctionFieldDesingularization,fusePairs,futureParser,GaloisField,Gamma,gb,GBDegrees,gbRemove,gbSnapshot,gbTrace,gcd,gcdCoefficients,gcdLLL,GCstats,genera,GeneralOrderedMonoid,GenerateAssertions,generateAssertions,generator,generators,Generic,GenericInitialIdeal,genericMatrix,genericSkewMatrix,genericSymmetricMatrix,gens,genus,get,getc,getChangeMatrix,getenv,getGlobalSymbol,getNetFile,getNonUnit,getPrimeWithRootOfUnity,getSymbol,getWWW,GF,gfanInterface,Givens,GKMVarieties,GLex,Global,global,globalAssign,globalAssignFunction,GlobalAssignHook,globalAssignment,globalAssignmentHooks,GlobalDictionary,GlobalHookStore,globalReleaseFunction,GlobalReleaseHook,Gorenstein,GradedLieAlgebras,GradedModule,gradedModule,GradedModuleMap,gradedModuleMap,gramm,GraphicalModels,GraphicalModelsMLE,Graphics,graphIdeal,graphRing,Graphs,Grassmannian,GRevLex,GroebnerBasis,groebnerBasis,GroebnerBasisOptions,GroebnerStrata,GroebnerWalk,groupID,GroupLex,GroupRevLex,GTZ,Hadamard,handleInterrupts,HardDegreeLimit,hash,HashTable,hashTable,HEAD,HEADER1,HEADER2,HEADER3,HEADER4,HEADER5,HEADER6,HeaderType,Heading,Headline,Heft,heft,Height,height,help,Hermite,hermite,Hermitian,HH,hh,HigherCIOperators,HighestWeights,Hilbert,hilbertFunction,hilbertPolynomial,hilbertSeries,HodgeIntegrals,hold,Holder,Hom,homeDirectory,HomePage,Homogeneous,Homogeneous2,homogenize,homology,homomorphism,HomotopyLieAlgebra,hooks,horizontalJoin,HorizontalSpace,HR,HREF,HTML,html,httpHeaders,Hybrid,HyperplaneArrangements,Hypertext,hypertext,HypertextContainer,HypertextParagraph,icFracP,icFractions,icMap,icPIdeal,id,Ideal,ideal,idealizer,identity,if,IgnoreExampleErrors,ii,image,imaginaryPart,IMG,ImmutableType,importFrom,in,incomparable,Increment,independentSets,indeterminate,IndeterminateNumber,Index,index,indexComponents,IndexedVariable,IndexedVariableTable,indices,inducedMap,inducesWellDefinedMap,InexactField,InexactFieldFamily,InexactNumber,InfiniteNumber,infinity,info,InfoDirSection,infoHelp,Inhomogeneous,input,Inputs,insert,installAssignmentMethod,installedPackages,installHilbertFunction,installMethod,installMinprimes,installPackage,InstallPrefix,instance,instances,IntegralClosure,integralClosure,integrate,IntermediateMarkUpType,interpreterDepth,intersect,intersectInP,Intersection,intersection,interval,InvariantRing,inverse,InverseMethod,inversePermutation,Inverses,inverseSystem,InverseSystems,Invertible,InvolutiveBases,irreducibleCharacteristicSeries,irreducibleDecomposition,isAffineRing,isANumber,isBorel,isCanceled,isCommutative,isConstant,isDirectory,isDirectSum,isEmpty,isField,isFinite,isFinitePrimeField,isFreeModule,isGlobalSymbol,isHomogeneous,isIdeal,isInfinite,isInjective,isInputFile,isIsomorphism,isLinearType,isListener,isLLL,isMember,isModule,isMonomialIdeal,isNormal,isOpen,isOutputFile,isPolynomialRing,isPrimary,isPrime,isPrimitive,isPseudoprime,isQuotientModule,isQuotientOf,isQuotientRing,isReady,isReal,isReduction,isRegularFile,isRing,isSkewCommutative,isSorted,isSquareFree,isStandardGradedPolynomialRing,isSubmodule,isSubquotient,isSubset,isSupportedInZeroLocus,isSurjective,isTable,isUnit,isWellDefined,isWeylAlgebra,ITALIC,Iterate,Jacobian,jacobian,jacobianDual,Jets,Join,join,Jupyter,K3Carpets,K3Surfaces,Keep,KeepFiles,KeepZeroes,ker,kernel,kernelLLL,kernelOfLocalization,Key,keys,Keyword,Keywords,kill,koszul,Kronecker,KustinMiller,LABEL,last,lastMatch,LATER,LatticePolytopes,Layout,lcm,leadCoefficient,leadComponent,leadMonomial,leadTerm,Left,left,length,LengthLimit,letterParser,Lex,LexIdeals,LI,Licenses,LieTypes,lift,liftable,Limit,limitFiles,limitProcesses,Linear,LinearAlgebra,LinearTruncations,lineNumber,lines,LINK,linkFile,List,list,listForm,listLocalSymbols,listSymbols,listUserSymbols,LITERAL,LLL,LLLBases,lngamma,load,loadDepth,LoadDocumentation,loadedFiles,loadedPackages,loadPackage,Local,local,localDictionaries,LocalDictionary,localize,LocalRings,locate,log,log1p,LongPolynomial,lookup,lookupCount,LowerBound,LUdecomposition,M0nbar,M2CODE,Macaulay2Doc,makeDirectory,MakeDocumentation,makeDocumentTag,MakeHTML,MakeInfo,MakeLinks,makePackageIndex,MakePDF,makeS2,Manipulator,map,MapExpression,MapleInterface,markedGB,Markov,MarkUpType,match,mathML,Matrix,matrix,MatrixExpression,Matroids,max,maxAllowableThreads,maxExponent,MaximalRank,maxPosition,MaxReductionCount,MCMApproximations,member,memoize,memoizeClear,memoizeValues,MENU,merge,mergePairs,META,method,MethodFunction,MethodFunctionBinary,MethodFunctionSingle,MethodFunctionWithOptions,methodOptions,methods,midpoint,min,minExponent,mingens,mingle,minimalBetti,MinimalGenerators,MinimalMatrix,minimalPresentation,minimalPresentationMap,minimalPresentationMapInv,MinimalPrimes,minimalPrimes,minimalReduction,Minimize,minimizeFilename,MinimumVersion,minors,minPosition,minPres,minprimes,Minus,minus,Miura,MixedMultiplicity,mkdir,mod,Module,module,ModuleDeformations,modulo,MonodromySolver,Monoid,monoid,MonoidElement,Monomial,MonomialAlgebras,monomialCurveIdeal,MonomialIdeal,monomialIdeal,MonomialIntegerPrograms,MonomialOrbits,MonomialOrder,Monomials,monomials,MonomialSize,monomialSubideal,moveFile,multidegree,multidoc,multigraded,MultigradedBettiTally,MultiGradedRationalMap,multiplicity,MultiplicitySequence,MultiplierIdeals,MultiplierIdealsDim2,MultiprojectiveVarieties,mutable,MutableHashTable,mutableIdentity,MutableList,MutableMatrix,mutableMatrix,NAGtypes,Name,nanosleep,Nauty,NautyGraphs,NCAlgebra,NCLex,needs,needsPackage,Net,net,NetFile,netList,new,newClass,newCoordinateSystem,NewFromMethod,newline,NewMethod,newNetFile,NewOfFromMethod,NewOfMethod,newPackage,newRing,nextkey,nextPrime,nil,NNParser,NoetherianOperators,NoetherNormalization,NonAssociativeProduct,NonminimalComplexes,nonspaceAnalyzer,NoPrint,norm,normalCone,Normaliz,NormalToricVarieties,not,Nothing,notify,notImplemented,NTL,null,nullaryMethods,nullhomotopy,nullParser,nullSpace,Number,number,NumberedVerticalList,numcols,numColumns,numerator,numeric,NumericalAlgebraicGeometry,NumericalCertification,NumericalImplicitization,NumericalLinearAlgebra,NumericalSchubertCalculus,numericInterval,NumericSolutions,numgens,numRows,numrows,odd,oeis,of,ofClass,OL,OldPolyhedra,OldToricVectorBundles,on,OneExpression,OnlineLookup,OO,oo,ooo,oooo,openDatabase,openDatabaseOut,openFiles,openIn,openInOut,openListener,OpenMath,openOut,openOutAppend,operatorAttributes,Option,OptionalComponentsPresent,optionalSignParser,Options,options,OptionTable,optP,or,Order,order,OrderedMonoid,orP,OutputDictionary,Outputs,override,pack,Package,package,PackageCitations,PackageDictionary,PackageExports,PackageImports,PackageTemplate,packageTemplate,pad,pager,PairLimit,pairs,PairsRemaining,PARA,Parametrization,parent,Parenthesize,Parser,Parsing,part,Partition,partition,partitions,parts,path,pdim,peek,PencilsOfQuadrics,Permanents,permanents,permutations,pfaffians,PHCpack,PhylogeneticTrees,pi,PieriMaps,pivots,PlaneCurveSingularities,plus,poincare,poincareN,Points,polarize,poly,Polyhedra,Polymake,PolynomialRing,Posets,Position,position,positions,PositivityToricBundles,POSIX,Postfix,Power,power,powermod,PRE,Precision,precision,Prefix,prefixDirectory,prefixPath,preimage,prepend,presentation,pretty,primaryComponent,PrimaryDecomposition,primaryDecomposition,PrimaryTag,PrimitiveElement,Print,print,printerr,printingAccuracy,printingLeadLimit,printingPrecision,printingSeparator,printingTimeLimit,printingTrailLimit,printString,printWidth,processID,Product,product,ProductOrder,profile,profileSummary,Program,programPaths,ProgramRun,Proj,Projective,ProjectiveHilbertPolynomial,projectiveHilbertPolynomial,ProjectiveVariety,promote,protect,Prune,prune,PruneComplex,pruningMap,Pseudocode,pseudocode,pseudoRemainder,Pullback,PushForward,pushForward,Python,QQ,QQParser,QRDecomposition,QthPower,Quasidegrees,QuaternaryQuartics,QuillenSuslin,quit,Quotient,quotient,quotientRemainder,QuotientRing,Radical,radical,RadicalCodim1,radicalContainment,RaiseError,random,RandomCanonicalCurves,RandomComplexes,RandomCurves,RandomCurvesOverVerySmallFiniteFields,RandomGenus14Curves,RandomIdeals,randomKRationalPoint,RandomMonomialIdeals,randomMutableMatrix,RandomObjects,RandomPlaneCurves,RandomPoints,RandomSpaceCurves,Range,rank,RationalMaps,RationalPoints,RationalPoints2,ReactionNetworks,read,readDirectory,readlink,readPackage,RealField,RealFP,realPart,realpath,RealQP,RealQP1,RealRoots,RealRR,RealXD,recursionDepth,recursionLimit,Reduce,reducedRowEchelonForm,reduceHilbert,reductionNumber,ReesAlgebra,reesAlgebra,reesAlgebraIdeal,reesIdeal,References,ReflexivePolytopesDB,regex,regexQuote,registerFinalizer,regSeqInIdeal,Regularity,regularity,relations,RelativeCanonicalResolution,relativizeFilename,Reload,remainder,RemakeAllDocumentation,remove,removeDirectory,removeFile,removeLowestDimension,reorganize,replace,RerunExamples,res,reshape,ResidualIntersections,ResLengthThree,Resolution,resolution,ResolutionsOfStanleyReisnerRings,restart,Result,resultant,Resultants,return,returnCode,Reverse,reverse,RevLex,Right,right,Ring,ring,RingElement,RingFamily,ringFromFractions,RingMap,rootPath,roots,rootURI,rotate,round,rowAdd,RowExpression,rowMult,rowPermute,rowRankProfile,rowSwap,RR,RRi,rsort,run,RunDirectory,RunExamples,RunExternalM2,runHooks,runLengthEncode,runProgram,same,saturate,Saturation,scan,scanKeys,scanLines,scanPairs,scanValues,schedule,schreyerOrder,Schubert,Schubert2,SchurComplexes,SchurFunctors,SchurRings,SCRIPT,scriptCommandLine,ScriptedFunctor,SCSCP,searchPath,sec,sech,SectionRing,SeeAlso,seeParsing,SegreClasses,select,selectInSubring,selectVariables,SelfInitializingType,SemidefiniteProgramming,Seminormalization,separate,SeparateExec,separateRegexp,Sequence,sequence,Serialization,serialNumber,Set,set,setEcho,setGroupID,setIOExclusive,setIOSynchronized,setIOUnSynchronized,setRandomSeed,setup,setupEmacs,sheaf,SheafExpression,sheafExt,sheafHom,SheafOfRings,shield,ShimoyamaYokoyama,short,show,showClassStructure,showHtml,showStructure,showTex,showUserStructure,SimpleDoc,simpleDocFrob,SimplicialComplexes,SimplicialDecomposability,SimplicialPosets,SimplifyFractions,sin,singularLocus,sinh,size,size2,SizeLimit,SkewCommutative,SlackIdeals,sleep,SLnEquivariantMatrices,SLPexpressions,SMALL,smithNormalForm,solve,someTerms,Sort,sort,sortColumns,SortStrategy,source,SourceCode,SourceRing,SPACE,SpaceCurves,SPAN,span,SparseMonomialVectorExpression,SparseResultants,SparseVectorExpression,Spec,SpechtModule,SpecialFanoFourfolds,specialFiber,specialFiberIdeal,SpectralSequences,splice,splitWWW,sqrt,SRdeformations,stack,stacksProject,Standard,standardForm,standardPairs,StartWithOneMinor,stashValue,StatePolytope,StatGraphs,status,stderr,stdio,step,StopBeforeComputation,stopIfError,StopWithMinimalGenerators,Strategy,String,STRONG,StronglyStableIdeals,STYLE,Style,style,SUB,sub,SubalgebraBases,sublists,submatrix,submatrixByDegrees,Subnodes,subquotient,SubringLimit,Subscript,subscript,SUBSECTION,subsets,substitute,substring,subtable,Sugarless,Sum,sum,SumOfTwists,SumsOfSquares,SUP,super,SuperLinearAlgebra,Superscript,superscript,support,SVD,SVDComplexes,switch,SwitchingFields,sylvesterMatrix,Symbol,symbol,SymbolBody,symbolBody,SymbolicPowers,symlinkDirectory,symlinkFile,symmetricAlgebra,symmetricAlgebraIdeal,symmetricKernel,SymmetricPolynomials,symmetricPower,synonym,SYNOPSIS,syz,Syzygies,SyzygyLimit,SyzygyMatrix,SyzygyRows,syzygyScheme,TABLE,Table,table,take,Tally,tally,tan,TangentCone,tangentCone,tangentSheaf,tanh,target,Task,taskResult,TateOnProducts,TD,temporaryFileName,tensor,tensorAssociativity,TensorComplexes,terminalParser,terms,TEST,Test,testExample,testHunekeQuestion,TestIdeals,TestInput,tests,TEX,tex,TeXmacs,texMath,Text,TH,then,Thing,ThinSincereQuivers,ThreadedGB,threadVariable,Threshold,throw,Time,time,times,timing,TITLE,TO,to,TO2,toAbsolutePath,toCC,toDividedPowers,toDual,toExternalString,toField,TOH,toList,toLower,top,top,topCoefficients,Topcom,topComponents,topLevelMode,Tor,TorAlgebra,Toric,ToricInvariants,ToricTopology,ToricVectorBundles,toRR,toRRi,toSequence,toString,TotalPairs,toUpper,TR,trace,transpose,TriangularSets,Tries,Trim,trim,Triplets,Tropical,true,Truncate,truncate,truncateOutput,Truncations,try,TSpreadIdeals,TT,tutorial,Type,TypicalValue,typicalValues,UL,ultimate,unbag,uncurry,Undo,undocumented,uniform,uninstallAllPackages,uninstallPackage,Unique,unique,Units,Unmixed,unsequence,unstack,Up,UpdateOnly,UpperTriangular,URL,urlEncode,Usage,use,UseCachedExampleOutput,UseHilbertFunction,UserMode,userSymbols,UseSyzygies,utf8,utf8check,validate,value,values,Variable,VariableBaseName,Variables,Variety,variety,vars,Vasconcelos,Vector,vector,VectorExpression,VectorFields,VectorGraphics,Verbose,Verbosity,Verify,VersalDeformations,versalEmbedding,Version,version,VerticalList,VerticalSpace,viewHelp,VirtualResolutions,VirtualTally,VisibleList,Visualize,wait,WebApp,wedgeProduct,weightRange,Weights,WeylAlgebra,WeylGroups,when,whichGm,while,width,wikipedia,Wrap,wrap,WrapperType,XML,xor,youngest,zero,ZeroExpression,zeta,ZZ,ZZParser}
}
\lstalias{Macaulay2output}{Macaulay2}


\hbadness = 10000
\vbadness = 10000

\newcommand\restr[2]{{% we make the whole thing an ordinary symbol
  \left.\kern-\nulldelimiterspace % automatically resize the bar with \right
  #1 % the function
  \vphantom{\big|} % pretend it's a little taller at normal size
  \right|_{#2} % this is the delimiter
  }}

% Default fixed font does not support bold face
\DeclareFixedFont{\ttb}{T1}{txtt}{bx}{n}{12} % for bold
\DeclareFixedFont{\ttm}{T1}{txtt}{m}{n}{12}  % for normal
% Custom colors

\usepackage{color}
\definecolor{deepblue}{rgb}{0,0,0.5}
\definecolor{deepred}{rgb}{0.6,0,0}
\definecolor{deepgreen}{rgb}{0,0.5,0}

% Python style for highlighting
\newcommand\pythonstyle{\lstset{
language=Python,
basicstyle=\ttm,
morekeywords={self},              % Add keywords here
keywordstyle=\ttb\color{deepblue},
emph={MyClass,__init__},          % Custom highlighting
emphstyle=\ttb\color{deepred},    % Custom highlighting style
stringstyle=\color{deepgreen},
frame=tb,                         % Any extra options here
showstringspaces=false
}}

\lstnewenvironment{python}[1][]
{
\pythonstyle
\lstset{#1}
}
{}

\theoremstyle{definition}

\newtheorem{theorem}{Theorem}[section]
\newtheorem{definition}[theorem]{Definition}
\newtheorem{corollary}[theorem]{Corollary}
\newtheorem{lemma}[theorem]{Lemma}

\newcommand{\Z}{\mathbb{Z}}
\newcommand{\Q}{\mathbb{Q}}
\newcommand{\R}{\mathbb{R}}
\newcommand{\C}{\mathbb{C}}
\newcommand{\K}{\mathbb{K}}
\renewcommand{\P}{\mathbb{P}}
\newcommand{\F}{\mathbb{F}}
\newcommand{\N}{\mathbb{N}}
\newcommand{\A}{\mathbb{A}}

\newcommand{\x}{\bm{x}}
\newcommand{\Kx}{\K[\bm{x}]}
\newcommand{\KP}[2]{\K[#1_1, #1_2, \ldots, #1_{#2}]}

\renewcommand{\AA}[1]{\A^{#1}}
\newcommand{\An}{\A^n}
\newcommand{\Am}{\A^m}

\newcommand{\PP}[1]{\P^{#1}}
\newcommand{\Pn}{\P^n}
\newcommand{\Pm}{\P^m}

\newcommand{\Hom}{\text{Hom}}
\newcommand{\Aut}{\text{Aut}}
\newcommand{\End}{\text{End}}
\newcommand{\Iso}{\text{Iso}}

\newcommand{\Spec}{\text{Spec}}


\newcommand{\lm}{\text{lm}}
\newcommand{\nr}{\text{nilrad}}
\newcommand{\nilrad}{\text{nilrad}}
\newcommand{\spec}{\text{spec}}
\newcommand{\spn}{\text{span}}
\newcommand{\codim}{\text{codim}}
\newcommand{\ann}{\text{ann}}
\newcommand{\im}{\text{im}}
\newcommand{\id}{\text{id}}
\newcommand{\height}{\text{height}}
\newcommand{\ini}{\text{in}}
\newcommand{\Frac}{\text{Frac}}

\newcommand{\catname}[1]{{\normalfont\textbf{#1}}}
\newcommand{\Set}{\catname{Set}}
\newcommand{\CRing}{\catname{CRing}}
\newcommand{\Top}{\catname{Top}}
\newcommand{\op}{\catname{op}}

\setlength{\parindent}{0pt}




\begin{document}

\subsection*{Ex I-1}

We have that 
\begin{itemize}
	\item $\Spec(\Z) = \{0\} \cup \{(p) : p \in \Z \text{ a prime number}\}$.
	\item $\Spec(\Z/(3)) = \{(0)\}$.
	\item $\Spec(\Z/(6)) = \{(2), (3)\}$.
	\item $\Spec(\Z_{(3)}) = \{(0), (3)\}$.
	\item $\Spec(\C[x]) = \{0\} \cup \{(x - a) : a \in \C \}$.
	\item $\Spec(\C[x]/x^2) = \{(x)\}$.
\end{itemize}

\subsection*{Ex I-2}
We have that 
\[
	15 
	\mapsto 
	15/1 + (7) 
	=  
	6/1 + (7)
	\in \kappa(7),
\]
and  
\[
	15 
	\mapsto 
	15/1 + (5) 
	=  
	0 + (5)
	\in \kappa(15).
\]

\subsection*{Ex I-3}
\subsubsection*{(a)}

First of, $(x - a)$ is prime (maximal even) since $\C[x]/(x-a) \cong \K$ is a
field. The quotient morphism $\C[x] \to \C[x]/(x - a)$ sends $x$ to $a$, hence
the morphism is given by evaluation at $a$. Composing this with the canonical
morphism to the localisation at $(x - a)$, we see that $p(x) \mapsto p(a)/1 +
(x - a) \in \kappa((x-a))$.

\subsubsection*{(b)}
Follows by the same reasoning as above, since $I(a) = (x_1 - a_1, \ldots, x_n - a_n)$
and the quotient morphism is given by evaluation at $a$.

\subsection*{Ex I-4}
\subsubsection*{(a)}

The point $(x - a)$ are all closed as they form maximal ideals. The closure of
$(0)$ is all of $\Spec(R)$, since any prime ideal contains $0$. None of the
points in $\Spec(R)$ are open, since all distinguished basis elements $X_f
\subseteq \Spec(R)$ are infinite, as $X_f$ contains all $(x - a)$ such that $x
- a$ isn't a factor of $f$, I.e $f(a) \not = 0$.

\subsubsection*{(b)}

First of, $R$ consists of all fractions $f/g$ where $g(0) \not = 0$. We know
from commutative algebra that the prime ideals of $R$ are the prime ideals of
$\K[x]$ which are contained in $(x)$. As $\K$ is a field, $\K[x]$ is a PID, but
then any non-zero ideal $(f) \subsetneq (x)$ is generated by a polynomial $f$
which is divisible by $x$, say $f = x^k g$ where $x \not | g$. Then in the
localisation, $(x^k g) = (x^k)$ is not prime. Hence the only primes in $R$ are
$(0)$ and $(x)$, so $\Spec(R) = \{(0), (x)\}$.

\subsection*{Ex I-8}
\subsubsection*{(a)}

The elements of $\overline{\mathscr{F}}$ are given by all germs $s_x \in
\mathscr{F}_x$ for points $x \in X$. Given an open set $U \subseteq X$, the
preimage $\pi^{-1}(U)$ is given by all germs over points $x \in U$. Hence
$\pi^{-1}(U) = \overline{\restr{\mathscr{F}}{U}}$. This is open in $\overline{\mathscr{F}}$
since it's the union over all open sets $V \subset U$ given by
\[
	\pi^{-1}(U) 
	= 
	\overline{\restr{\mathscr{F}}{U}}
	=
	\bigcup_{V \subseteq U}
	\bigcup_{s \subseteq \mathscr{F}(V)}
	\mathscr{V}(V, s).
\]

Meanwhile, given some basis element $\mathscr{V}(U, s') \subseteq
\overline{\mathscr{F}}$, the preimage $\sigma^{-1}(\mathscr{V}(U, s'))$ is
given by the largest (possibly empty) open subset $V \subseteq U$ where $s$ and
$s'$ agree. To see this, first suppose that we have some $x \in V$. Then $s_x =
s'_x$ so $\sigma(x) = s_x = s'_x \in \mathscr{V}(U, s')$. If $x \not \in V$
however, then $s_x \not = s'_x$, since otherwise we would have some open subset
$V'_x$ containing $x$ where $s, s'$ agrees, whence $(U \cap V'_x) \cup V
\supsetneq V$ would be an open set where $s$ and $s'$ agrees, contradicting the
maximality of $V$. It follows that $\sigma(x) \not \in \mathscr{V}(U, s')$ in
this case. As $V$ is open, $\sigma$ is continuous. Moreover, 
\[
	\pi(\sigma(x))
	=
	\pi(s_x)
	=
	x
\] 
so $\sigma$ is a section of $\pi$.

\subsubsection*{(b)}

Now let $\sigma$ be a continuous section of $\pi$ on $U$. Then consider $x \in
U$, and a basic open set $\mathscr{V}(V_i, s^x), V_x \subseteq U$ in
$\overline{\mathscr{F}}$ such that $\sigma(x) \in \mathscr{V}(V, s^x)$. Then $x
\in V_x$ as $x = \pi(\sigma(x)) \subseteq \pi(\mathscr{V}(V_x, s^x)) = V_x$.
Moreover, $(x, \sigma(x)) = (x, s^x_x)$ so $\sigma(x) = s^x_x$, and this holds
for all 

\[
	x 
	\in 
	\sigma^{-1}(\mathscr{V}(V_x, s^x)) 
	= 
	\pi(\mathscr{V}(V_x, s^x))
	= 
	V_x.
\]

So if we let $V_x, x \in U$ be a cover for $U$, then $\sigma$ coincides with $y
\to s^x_y$ for all $y \in V_x$ and $x \in U$. It follows that $s^{x'}_y =
s^x_y$ whenever $y \in V_x \cap V_x'$, hence $s^{x'} = s^x$ on $V_x \cap
V_{x'}$, and by applying the sheaf property, we obtain a unique $s \in
\mathscr{F}(U)$ such that $\restr{s}{V_x} = s^x$ for all $x \in U$. Hence
$\sigma(x) = s_x$.


\subsection*{Ex I-9}

We prove both implications in contrapositive form. \\

Suppose that $\phi(U)(s) = \phi(U)(s')$ for $s \not = s' \in \mathscr{F}(U)$.
Then $\phi_x(s) = \phi_x(s')$ for all $x \in U$, even though $s_x \not = s'_x$
for all $x \in U$ (as stalks determine sections), hence $\phi_x$ is not
injective for all $x \in U$. \\

Now suppose that $s \not = s' \in \mathscr{F}(U)$ and that $\phi_x(s) =
\phi_x(s')$ for all $x \in U$. Then $\phi(U)(s) = \phi(U)(s')$, again as stalks
determine sections, hence $\phi(U)$ isn't injective.


\subsection*{Ex I-10}
\subsubsection*{(a)}

$\phi$ clearly commutes with restriction, hence induces a endomorphism of
sheaves. Now, $\phi$ is not surjective, as the function $x \mapsto x$ has no
continuous square root in $\C \setminus \{0\}$ (indeed, we can only define it
on a slit plane or Riemann surface etc...). Meanwhile, $\phi_a$ is surjective
for every $a \in \C \setminus \{0\}$, since if $f$ is continuous and non-zero,
then we can make a branch cut of $\C$ which doesn't intersect some
neighbourhood $U_a$ of $a$, and define $g : U_a \to \C \setminus \{0\}$ as the
square root of $f$ with respect to this branch cut, and then extending $g$ to
all of $\C \setminus \{0\}$ in some continuous way.

\subsubsection*{(b)}

\subsection*{Ex I-11}

First of, the given property is indeed universal as sections are determined by
germs. Moreover, we will show both constructions satisfy this universal
property, hence they are equivalent. \\

First some notation that will aid us with the first construction of
$\mathscr{F}'$. Given an open set $U \subseteq X$, let $\mathscr{U}$ be the set
of all open covers of $U$ in $X$. Given an open cover $\mathcal{U} \in
\mathscr{U}$, and a set of sections $S$, consisting of $s_i$ for every $U_i \in
\mathcal{U}$ we say that $S$ is a consistent set of sections if the $s_i$ agree
on the intersections of the $U_i$. We then construct the sheaf $\mathscr{F}'$
as
\[
	\mathscr{F}'(U) 
	= 
	\{
		g(S) 
		: 
		S \text{ is a consistent set of sections for some cover } \mathcal{U} \in \mathscr{U} 
	\},
\]
where $g(S)$ is the "formal" gluing of the sections of $S$. I.e $g(S)$ is such
that it restricts to $s_i$ on every $U_i$. $\mathscr{F}'$ is a pre sheaf since
every cover of $U$ can be restricted to a cover of an open subset of $U$.
Moreover, given to consistent sets $S$, $S'$ on covers $\mathcal{U},
\mathcal{U}'$ of $U$, we consider $g(S)$ and $g(S')$ to be equivalent if they
restrict to the same section on any open subset $V \subseteq U$. Then
$\mathscr{F}$ is a sheaf by construction, and we have a morphism of presheaves
$\phi \mathscr{F} \to \mathscr{F}'$, since if $U \subseteq X$ is an open set,
then $U$ is a cover of $U$ so any section $s \in \mathscr{F}(U)$ can be
identified with a section $\phi(s) \in \mathscr{F}'(U)$, and if $s, s' \in
\mathscr{F}(U)$ are two different sections $s \not = s'$, then they don't agree
on the cover $\{U\}$ of $U$, hence $\phi(s) \not = \phi(s')$. Finally, $\phi_x$
is an isomorphism for every $x \in X$ since the sections of $\mathscr{F}'$ are
locally sections of $\mathscr{F}$. \\

Now let $\mathscr{F}''$ be the second construction, I.e the sheaf of continuous
sections of $\pi : \overline{\mathscr{F}} \to X$. We will show that this is
indeed a sheaf, by demonstrating a more general class of sheaves that originate
from surjective continuous maps $\psi : A \to X$ of topological spaces. \\

Let $\psi : A \to X$ be a continuous surjective map of topological spaces, and
assign to every open set $U \subseteq X$, the set of continuous sections of
$\psi$ as
\[
	\mathscr{S}(U)
	=
	\{
		\sigma \in C(U, A)
		:
		\psi \circ \sigma = \id_U
	\}.
\]
Then $\mathscr{S}$ defines a sheaf on $X$. Indeed if $U \subseteq X$ is an open
subset, and $\sigma : U \to A$ is a continuous section of $\psi$, then it also
a continuous section on any subset $V \subseteq U$. Similarly, if $s_i : U_i
\to X$ are continuous sections of $\psi$ which agree on the intersections of
the $U_i$, then they induce a unique continuous function $s : U \to A$ in the
sheaf of continuous functions $X \to A$ and this function $s$ is also a section
of $\psi$. Hence $\mathscr{S}$ is sheaf. \\

Now, since $\pi : \overline{\mathscr{F}} \to X$ is a continuous surjective map,
$\mathscr{F}''$ is a sheaf. It remains to show that the two sheaves $\mathscr{F}'$
and $\mathscr{F}''$ are isomorphic. \\

Let $U \subseteq X$ be an open subset with a point $x \in U$, and $(x, s_x) =
\sigma(x)$. We showed in Exercise I-8 that $\sigma : U \to
\overline{\mathscr{F}}$ given by $\sigma : x \mapsto (x, s_x)$ is a section to
$\pi$, hence $\sigma \in \mathscr{F}''(U)$. Now, we have that $\mathscr{V}(U,
s)$ is an open set containing $(x, s_x)$, and we can construct the open set
$V_s \subseteq U$ as $\sigma^{-1}(\mathscr{V}(U, s))$. Then $\pi$ and $\sigma$
are inverse homeomorphisms when restricted to $V$ and $\sigma(V)$. I.e we've
shown that $\pi$ is locally a homeomorphism. \\

If $U' \subseteq X$ is some other subset which contain $x$, and $\sigma' \in
\mathscr{F}''(U')$ is a section such that $\sigma'(x) = \sigma(x)$. Then since
$\pi$ is locally a homeomorphism, and $\sigma, \sigma'$, are both locally
inverse to $\pi$ with intersecting images, $x \in \im(\sigma) \cap
\im(\sigma')$, it must be that $\sigma$ and $\sigma'$ are locally the same. I.e
there exist some neighbourhood of $x$ where $\sigma$ and $\sigma'$ agree. Thus
$\sigma$ and $\sigma'$ induce the same stalk at $x$, and it follows that any
germ $\sigma_x \in \mathscr{F}''_x$ is fully determined by it's value at $x$.
\\

Moreover, any point $(x, s_x)$ corresponds to a germ $\sigma_x \in \mathscr{F}_x$
since we showed above that any $\sigma : x \to (x, s_x)$ is a section. \\

We've shown that the germs of $\mathscr{F}''$ correspond to points in
$\overline{\mathscr{F}}$, just like the germs of $\mathscr{F}'$. As a sheaf is
fully determined by its germs, we see that $\mathscr{F}''$ is equivalent to
$\mathscr{F}'$.

\subsection*{Ex I-13}

Notation: When $U \subseteq X$, we write  $\mathscr{B}(U) = \{B \in
\mathscr{B} : B \subseteq U\}$ \\

We begin by verifying that $\mathscr{F}$ is indeed a sheaf. Let $U' \subseteq
U$ be an open set. Then the inverse system of basis elements $\mathscr{B}(U')$
is contained in the system $\mathscr{B}(U)$, hence there is a natural
projection 
\[
	\pi^{U}_{U'}
	:
	\varprojlim_{V \in \mathscr{B}(U)} \mathscr{F}(V)
	\to
	\varprojlim_{V \in \mathscr{B}(U')} \mathscr{F}(V)
\] 
given by
\[
	\pi^{U}_{U'}
	:
	f
	\mapsto
	\prod_{V \in \mathscr{B}(U')}
	\pi_{V}(f)
\]
where $\pi_V : \varprojlim_{W \in \mathscr{B}(U)} \mathscr{F}(W)
\to V$ is the projection map to the component corresponding to $V$. We define
the restriction of $\mathscr{U}$ to $\mathscr{U'}$ be $\pi'$. It's easy to see
that this satisfies the required axioms for a presheaf. \\

Now suppose that $U_i, i \in I$ is an open cover of $U$, and that $f_i \in
\mathscr{F}(U_i)$ are sections that agree on the intersections of the $U_i$.
Then each $U_i$ can be written as a union of basis elements in $U_i$,
\[
	U_i = \bigcup_{B \in \mathscr{B}(U_i)} B.
\] 
The union of all $\mathscr{B}(U_i)$ is a cover of $U$, and on each $B_{ij} \in
\mathscr{B}_i$, we have $f_{ij} = \restr{f_i}{B_{ij}}$ and they all agree on
the intersections. Any basis element $V \in \mathscr{B}(U)$ is covered by basis
elements $B_{ij}$, and we can uniquely lift the $f_{ij}$ to $f_V \in
\mathscr{F}(V)$ since $\mathscr{F}$ is a $\mathscr{B}$-sheaf. We define $f \in
\mathscr{F}(U)$ to be the direct product of all of these lifts, and we see that
the sheaf axiom is satisfied. \\

Moreover, if $U \in \mathscr{B}$, then any element $f$ in the inverse limit of
$\mathscr{F}(V)$ with $V \in \mathscr{B}(U)$ is uniquely defined by it's
projection onto $U$, hence the inverse limit is isomorphic to $\mathscr{U}$ in
this case. In more generality; when an inverse system has a greatest element
$x$, then the inverse limit is isomorphic to the component at $x$, since it
uniquely defines all other components. \\

Finally, there is only one sheaf (up to isomorphism) on $X$ that agrees with
$\mathscr{F}$ over $\mathscr{B}$ since sections are defined by stalks, and
$\mathscr{B}$ uniquely defines all stalks. Hence the extension of a
$\mathscr{B}$-sheaf to a sheaf on $X$ is unique.


\subsection*{Ex I-15}

If $U' \subseteq U \subseteq Y$ are open sets in $Y$, and $f \in
\mathscr{F}(\alpha^{-1}(U))$, then we can restrict $f$ in $\mathscr{F}$ to
$\alpha^{-1}(U')$. Hence restrictions are well defined in $\alpha_*
\mathscr{F}$. \\

Suppose we have an open cover $U_i, i \in I$ of $U \subseteq Y$. Then
$\alpha^{-1}(U_i), i \in I$ is an open cover of $\alpha^{-1}(U)$, hence gluing
can be delegated to the preimages as well.

\subsection*{Ex I-17}

Noetherian rings have Noetherian spectra, where a topological space is
Noetherian if all descending chains of closed subsets stabilize, or
equivalently, all ascending chains of open subsets stabilize. Indeed, suppose
that 
\[
	\Spec(R) = Z_0 \supseteq Z_1 \supseteq Z_2 \supseteq \ldots
\] 
is an infinite ascending chain of closed subsets. Then as every closed subset
is of the form $Z_i = \{\mathfrak{p} \in \Spec(R) : J_i \subseteq
\mathfrak{p}\}$ for some ideal $J_i$, the chain above induces a reversed
infinite ascending chain 
\[
	(0) = J_0 \subseteq J_1 \subseteq \ldots.
\] 
Since $R$ is Noetherian, the chain of ideals must stabilize. But then the chain
of closed subsets stabilizes as well, hence $\Spec(R)$ is Noetherian. \\

Subspaces of Noetherian spaces are again Noetherian, so any subset of
$\Spec(R)$ is Noetherian. Now, if we have an open cover $U_i, i \in I$ of a
Noetherian space, then we can form a chain of open subsets
\[
	U_{i_0} 
	\subseteq 
	U_{i_0} \cup U_{i_1} 
	\subseteq 
	U_{i_0} \cup U_{i_1} \cup U_{i_2} 
	\subseteq 
	\ldots,
\] 
and this chain must eventually stabilize. Hence the cover admits a finite
subcover.

\subsection*{Ex I-20}

\subsubsection*{(a)}

The spectrum is given by 
\[
	X_1 
	= 
	\{
		(x)
	\},
\] 
with the only open sets being the whole space $X_1 = \{(x)\}$ and the empty set
$\emptyset$. The structure sheaf is trivial, as
\[
	\mathcal{O}(X_1) = R_1,\,  
	\mathcal{O}(\emptyset) = 0. 
\] 

\subsubsection*{(b)}

The spectrum is given by 
\[
	X_2 
	= 
	\{
		(x),
		(x - 1)
	\},
\] 
with the open basis 
\[
	D_{x - 1} = \{(x)\},
	D_{x} = \{(x - 1)\},
	D_{0} = X_2.
\] 
We give the structure sheaf on the distinguished open sets. First of 
\[
	\mathcal{O}(D_{x}) 
	= 
	(R_2)_{(x)}  
	=
	(\C[x]/(x(x-1)))_{(x)},
\] 
and as $x = x(x - 1)/(x - 1) = 0/(x - 1) = 0 \in (\C[x]/(x(x-1)))_{(x)}$, it follows that 
$\mathcal{O}(D_{x}) = \C$. \\

Similarly, 
\[
	\mathcal{O}(D_{x - 1}) 
	= 
	(R_2)_{(x - 1)}  
	=
	(\C[x]/(x(x-1)))_{(x - 1)}.
\] 
and as $x - 1 = x(x - 1)/x = 0/x = 0 \in (\C[x]/(x(x-1)))_{(x)}$, it follows that 
$x = 1 \in \mathcal{O}(D_{x})$ and $\mathcal{O}(D_{x}) = \C$. 

\subsubsection*{(c)}

The spectrum is given by 
\[
	X_3 
	= 
	\{
		(x),
		(x - 1)
	\},
\] 
with the open basis 
\[
	D_{x - 1} = \{(x)\},
	D_{x} = \{(x - 1)\},
	D_{0} = X_2.
\] 

We give the structure sheaf on the distinguished open sets. First of 
\[
	\mathcal{O}(D_{x}) 
	= 
	(R_2)_{(x)}  
	=
	(\C[x]/(x^2(x-1)))_{(x)}.
\] 
Now $x^2 = x^2(x - 1)/(x - 1) = 0/(x - 1) = 0 \in (\C[x]/(x(x-1)))_{(x)}$ hence
$x$ is not a unit in $\mathcal{O}(D_{x})$. However, $x$ is not $0$ either,
since whenever we take the ring of fractions with respect to some
multiplicative set $R \to S^{-1}R$, an element $a$ get's sent to zero if and
only if $sa = 0$ for some $s \in S$. Now $xa = 0$ in $\mathcal{O}(D_{x})$
implies that $x(x-1) | a$, whence $a \in (x)$, hence $x \not = 0$ in
$\mathcal{O}(D_{x})$. As $x$ is neither $0$ nor a unit, it's an algebraic
element over the field, and we already know it satisfies $x^2 = 0$, hence this
must be its minimal polynomial, so
\[
	\mathcal{O}(D_{x}) = C[x]/(x^2).
\]

Similarly to part (b), 
\[
	\mathcal{O}(D_{x - 1}) 
	= 
	(R_2)_{(x - 1)}  
	=
	(\C[x]/(x^2(x-1)))_{(x - 1)},
\] 
and as $x - 1 = x^2(x - 1)/x^2 = 0/x^2 = 0 \in (\C[x]/(x(x-1)))_{(x)}$, it follows that 
$x = 1 \in \mathcal{O}(D_{x})$ and $\mathcal{O}(D_{x}) = \C$. 

\subsubsection*{(d)}

$(f = x^2 + 1)$ is a maximal ideal in $\R[x]$ since $f$ is irreducible and
$\R[x]$ is a PID, hence $\R[x]/(x^2 + 1)$ is a field. I.e the spectrum is
trivial. Even more, we recognize that it is isomorphic to $\C$, since the
minimal polynomial of $i$ over $\R$ is $x^2 + 1$. \\


\subsection*{Ex I-21}

We develop a fair bit of theory before we treat the two separate cases. \\

First note that $X_f \cap X_g = X_{fg}$ for any $f, g \in R$. Indeed if a prime
ideal doesn't contain $fg$, then it doesn't contain $f$ or $g$. Similarly, if a
prime ideal doesn't contain $f$ or $g$, it can't contain $fg$ by definition. So
finite intersections of distinguished basis elements are again distinguished
basis elements for any ring $R$, and we only need to consider the case of $U =
\bigcup X_f$ open sets that are unions of distinguished basis elements. \\

Now we give some lemmas regarding closures and dense sets in $\Spec(R)$. 

\begin{lemma}
	Let $Y \subseteq \Spec(R)$, and $S \subseteq R$. Then $Y \subseteq V(S)$ if
	and only if
	\[
		S
		\subseteq
		\bigcap_{\mathfrak{p} \in Y} \mathfrak{p}.
	\] 
\end{lemma}
\begin{proof}
	If $S$ is contained in every prime ideal in $Y$, then the set of all prime
	ideals containing $S$ will contain as a subset $Y$. \\

	Similarly, if $Y$ is a subset of all prime ideals containing $S$, then all
	prime ideals in $Y$ will contain $S$.
\end{proof}

\begin{lemma}
	Suppose that $Y \subseteq \Spec(R)$. Then
	$\overline{Y} = \Spec(R)$ if and only if 
	\[
		\bigcap_{\mathfrak{p} \in Y} \mathfrak{p}
		=
		\nilrad(R),
	\] 
	if and only if $Y$ contains all minimal primes of $R$.
\end{lemma}
\begin{proof}
	First note that all minimal primes of $R$ intersect to the nilradical, and
	any proper subset of the minimal primes intersect to an ideal greater than
	the nilradical. This follows from prime avoidance (Prop 1.11.ii in Atiyah
	Macdonald). Hence the second statement is equivalent to the third. \\

	If $\bigcap_{\mathfrak{p} \in Y} \mathfrak{p} = \nilrad(R)$, then any $S
	\subseteq \Spec(R)$ such that $Y \subseteq V(S)$ will have to lie in the
	nilradical $S \subseteq \nilrad(R)$ by the previous lemma. But every prime
	ideal contains the nilradical, so $V(S) = \Spec(R)$ hence $\overline{Y} =
	\Spec(R)$. \\

	Now suppose that $\bigcap_{\mathfrak{p} \in Y} \mathfrak{p} = I \supsetneq
	\nilrad(R)$. Then $V(I)$ doesn't contain some minimal prime of $R$, since
	the intersection of the minimal primes is the nilradical, and $\overline{Y}
	\subseteq V(I) \subsetneq \Spec(R)$.
\end{proof}

\begin{lemma}
	Let $X_f \subseteq \Spec(R)$ be a distinguished open subset. Then $X_f$ is
	dense if and only if $f$ is not contained in any minimal prime of $R$.
\end{lemma}
\begin{proof}
	$X_f$ is dense if and only all minimal primes are in $X_f$ if and only 
	if no minimal prime contains $f$.
\end{proof}

Now let's investigate the structure of open sets $U \subseteq \Spec(R)$.

\begin{lemma}
	Let $U \subseteq \Spec(R)$ be an open set. If $\mathfrak{p} \in U$, and
	$\mathfrak{q}$ is a prime ideal such that $\mathfrak{q} \subseteq
	\mathfrak{p}$, then $\mathfrak{q} \in U$.
\end{lemma}
\begin{proof}
	We can write 
	\[
		U = \bigcup_{f \in F} X_f
	\] 
	where $F$ might be infinite. Then $\mathfrak{p} \in U$ precisely when
	$\mathfrak{p} \in X_f$ for some $f \in F$ and this in turn happens if and
	only if $f \not \in \mathfrak{p}$. But if $f \not \in \mathfrak{p}$, then
	$f \not \in \mathfrak{q}$, hence $\mathfrak{q} \in U$ in this case.
\end{proof}

\begin{lemma}
	Suppose that $U \subseteq \Spec(R)$ is an open subset. Then $U$ is dense if
	and only if it contains some dense open distinguished set $X_f$.
\end{lemma}
\begin{proof}
	If $X_f$ is dense and $X_f \subseteq U$, then $\overline{U}$ is clearly all
	of $\Spec(R)$. We move on to the other direction and let $U$ be a dense
	open set of $\Spec(R)$. \\

	Let $V(I) = \Spec(R) \setminus U$. Then for any $f \in I$,
	$X_f \subseteq U$, since 
	\begin{align*}
		\mathfrak{p} \in X_f
		&\Leftrightarrow
		f \not \in \mathfrak{p} \\
		& \Rightarrow
		I \not \subset \mathfrak{p} \\
		&\Leftrightarrow
		\mathfrak{p} \not \in V(I) \\
		&\Leftrightarrow
		\mathfrak{p} \in U.
	\end{align*} 
	As $U$ is dense, $U$ contains all minimal primes of $R$, whence $V(I)$
	doesn't contain any. I.e no minimal prime of $R$ contains $I$. It follows
	then by prime avoidance (Prop 1.11.i in Atiyah Macdonald) that $I$ isn't
	contained in the union of all minimal primes of $R$. Hence there is some $f
	\in I$ which isn't contained in any minimal prime. This means that $X_f$ is
	dense, and we've already shown that $X_f \subseteq U$ so we are done.
\end{proof}

It follows that it's enough to consider dense open distinguished sets
$\mathscr{U}'$ for the exercise. I.e that 
\[
	\varinjlim_{U \in \mathscr{U}}
	\mathcal{O}_X(U)
	=
	\varinjlim_{U \in \mathscr{U'}}
	\mathcal{O}_X(U).
\] 
Indeed, given any $U \in \mathscr{U}$ and $\sigma \in \mathscr{O}_X(U)$, pick a
dense open set $X_f \subseteq U$, and let $\restr{\sigma}{X_f}$ be the
restriction of $\sigma$ to $X_f$. Then $\sigma \sim \restr{\sigma}{X_f}$, and
we see that every section over a dense open set is equivalent to a section over
a dense open distinguished set. \\

Let $F \subset R$ denote the (multiplicative) set of elements which aren't
contained in any minimal prime of $R$. We've just shown that
\[
	\varinjlim_{U \in \mathscr{U}}
	\mathcal{O}_X(U)
	=
	\varinjlim_{f \in F}
	R_f.
\] 
The elements of $\varinjlim_{f \in F} R_f$ consists of quotients $g/h$ where $h
\in F$ modulo the relation that $g/h \sim g'/h'$ whenever $g/h = g'/h'$ in some
dense subset $X_r \subseteq X_h \cap X_{h'} = X_{hh'}$. I.e $g/h \sim g'/h'$
whenever there is some $r \in F$ such that $r = r'hh'$ and $rgh' = rg'h$, which
becomes $r'hh'gh' = r'hh'g'h$. But this is the same as saying that there is
some arbitrary $r' \in F$ such that $r'gh' = r'g'h$ since we can just multiply
both sides by $hh'$. This is exactly the same equivalence relation as that of
$F^{-1}R$, hence
\[
	\varinjlim_{U \in \mathscr{U}}
	\mathcal{O}_X(U)
	=
	F^{-1}R.
\] 

\subsubsection*{(a)}

When $R$ is an integral domain, then $(0)$ is a prime ideal, hence the only
minimal prime, and so the set of elements not in any minimal prime is $F =
R^*$. I.e
\[
	\varinjlim_{U \in \mathscr{U}}
	\mathcal{O}_X(U)
	=
	F^{-1}R
	=
	\Frac(R).
\] 

\subsubsection*{(b)}

Now consider the case when $R$ is Noetherian. Note that $F^{-1}R$ has Krull
dimension $0$ for any ring $R$, since $\Spec(F^{-1}R)$ contains only the
minimal primes of $R$. Hence when $R$ is Noetherian, $F^{-1}R$ is Artinian
(Theorem 8.5 in Atiyah Macdonald). It then follows that the set $m \subset
\Spec(R)$ of minimal prime ideals in $R$ is finite (Theorem 8.3 in Atiyah
Macdonald). By the Structure Theorem of Artin rings, we now have that $F^{-1}R$
can be written as 
\[
	F^{-1}R
	=
	\prod_{m_i \in m}
	\frac{F^{-1}R}{m_i^{k}}
\] 
where $k \in \N$ is the smallest integer such that the $\nilrad(F^{-1}R)^{k} =
0$. I don't know how to proceed from here though.

\subsection*{Ex I-23}

$\K[x]$ is a PID, and whenever a ring $R$ is a PID, the proper open sets $U
\subsetneq \Spec(R)$ are always distinguished $U = X_f$ for some $f \in R$.
Indeed, the complement $V(I) = \Spec(R) \setminus U$ is always induced by a
principle ideal $I = (f)$ and we see that $U = X_f$. \\

The structure sheaf is now completely described by $\mathcal{O}_X(X_f) =
\K[x]_f$ and $\mathcal{O}_X(\K[x]) = \K[x]$.

\end{document}
