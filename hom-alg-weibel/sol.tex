\documentclass{article}
\usepackage[utf8]{inputenc}

\usepackage{mathtools}
\usepackage{algpseudocode}
\usepackage{amsfonts}
\usepackage{amsmath}
\usepackage{amssymb}
\usepackage{amsthm}
\usepackage{bm}
\usepackage{listings}
\usepackage{float}
\usepackage{fancyvrb}
\usepackage{xcolor}
\usepackage{tikz-cd}

\hbadness = 10000
\vbadness = 10000

\newcommand\restr[2]{{% we make the whole thing an ordinary symbol
  \vphantom{\big|} % pretend it's a little taller at normal size
  \right|_{#2} % this is the delimiter
  }}

% Default fixed font does not support bold face
\DeclareFixedFont{\ttb}{T1}{txtt}{bx}{n}{12} % for bold
\DeclareFixedFont{\ttm}{T1}{txtt}{m}{n}{12}  % for normal
% Custom colors

\usepackage{color}
\definecolor{deepblue}{rgb}{0,0,0.5}
\definecolor{deepred}{rgb}{0.6,0,0}
\definecolor{deepgreen}{rgb}{0,0.5,0}

% Python style for highlighting
\newcommand\pythonstyle{\lstset{
language=Python,
basicstyle=\ttm,
morekeywords={self},              % Add keywords here
keywordstyle=\ttb\color{deepblue},
emph={MyClass,__init__},          % Custom highlighting
emphstyle=\ttb\color{deepred},    % Custom highlighting style
stringstyle=\color{deepgreen},
frame=tb,                         % Any extra options here
showstringspaces=false
}}

\lstnewenvironment{python}[1][]
{
\pythonstyle
\lstset{#1}
}
{}

\theoremstyle{definition}

\newtheorem{theorem}{Theorem}[section]
\newtheorem{definition}[theorem]{Definition}
\newtheorem{corollary}[theorem]{Corollary}
\newtheorem{lemma}[theorem]{Lemma}

\newcommand{\Z}{\mathbb{Z}}
\newcommand{\Q}{\mathbb{Q}}
\newcommand{\R}{\mathbb{R}}
\newcommand{\C}{\mathbb{C}}
\newcommand{\K}{\mathbb{K}}
\renewcommand{\P}{\mathbb{P}}
\newcommand{\F}{\mathbb{F}}
\newcommand{\N}{\mathbb{N}}
\newcommand{\A}{\mathbb{A}}

\DeclareMathOperator{\cone}{cone}

\newcommand{\x}{\bm{x}}
\newcommand{\Kx}{\K[\bm{x}]}
\newcommand{\KP}[2]{\K[#1_1, #1_2, \ldots, #1_{#2}]}

\renewcommand{\AA}[1]{\A^{#1}}
\newcommand{\An}{\A^n}
\newcommand{\Am}{\A^m}

\newcommand{\PP}[1]{\P^{#1}}
\newcommand{\Pn}{\P^n}
\newcommand{\Pm}{\P^m}

\newcommand{\Hom}{\text{Hom}}
\newcommand{\Aut}{\text{Aut}}
\newcommand{\End}{\text{End}}
\newcommand{\Iso}{\text{Iso}}
\newcommand{\Mor}{\text{Mor}}

\newcommand{\lm}{\text{lm}}
\newcommand{\nr}{\text{nilrad}}
\newcommand{\Spec}{\text{Spec}}
\newcommand{\spec}{\Spec}
\newcommand{\codim}{\text{codim}}
\newcommand{\ann}{\text{ann}}
\newcommand{\im}{\text{im}}
\newcommand{\id}{\text{id}}
\newcommand{\height}{\text{height}}

\newcommand{\catname}[1]{{\normalfont\textbf{#1}}}
\newcommand{\Set}{\catname{Set}}
\newcommand{\CRing}{\catname{CRing}}
\newcommand{\Top}{\catname{Top}}
\newcommand{\op}{\catname{op}}
\newcommand{\mo}{\catname{mod}}
\newcommand{\Ch}{\catname{Ch}}
\newcommand{\Ab}{\catname{Ab}}
\newcommand{\CK}{\catname{K}}
\newcommand{\ob}{\catname{ob}}

\setlength{\parindent}{0pt}




\begin{document}

\subsection*{Exercise 1.1.1}

We have $\im(d_{n + 1}) = (4)$ and $\ker(d_n) = (2)$. $(4) \subset (2)$ so the
maps induce a complex, and the homology at all positive degrees is given by the
$\Z$-module quotient $(4)/(2) = \Z/(2)$.

\subsection*{Exercise 1.1.2}

Let $d, d'$ be the differential maps in $C$ and $D$ respectively. Let $x \in
\ker(d_n)$ be a cycle in $C$. Then $0 = ud_n(x) = d'_n u(x)$, hence $u(x) \in
\ker(d'_n)$ and $u(Z_n(C)) \subseteq Z_n(D)$. \\

Now let $x \in \im(d_{n + 1})$, and $x' \in d_{n + 1}^{-1}(x)$. Then $u(x) = u
d_{n+1}(x') = d'_{n + 1}u(x) \in \im(d'_{n+1})$ and $u(B_n(C)) \subseteq
B_n(D)$. \\

To see that $H_n$ is a functor from $\Ch(\mo-R) \to \mo-R$, Note that
\begin{itemize}
	\item $H_n$ send complexes to modules (trivial),
	\item we just showed that $H_n$ send morphism of complexes to morphisms of modules,
	\item $H_n$ sends the identity morphisms to identity morpshisms (trivial),
	\item and $H_n$ preserves composition by
		\[
			(u'u)(x + B_n(C))
			=
			u'(u(x + B_n(C))).
		\] 
\end{itemize}

\subsection*{Exercise 1.1.5}

(1) and (2) are just reformulations of each other, we will show that (2) and
(3) are equivalent. \\

(2) $\Rightarrow$ (3):
The zero map $0 \to C$ is well-defined for any complex $C$, and in our
particular case, it is a quasi-isomorphism as $H_n = 0$ for all $n$. \\

(3) $\Rightarrow$ (2):
Only the $0$-module is isomorphic to $0$, hence $H_n = 0$
for all $n$

\subsection*{Exercise 1.2.1}

For the direct product, note that an element is in the boundary of the product
$a \in B_n\left(\prod A_{\alpha}\right)$ if and only if it's in the boundary of
every component $a \in \prod B_n\left(A_{\alpha}\right)$. The analagous argument
applies to cycles as well, and it follows that the homologies are isomorphic.
\\

Note here that the "if" direction only holds since an element in the direct
proudct may have non-zero entries for infinitely many indices. Hence the
argument can't be immediately carried over for the direct sum. \\

For the direct sum, an element is in the boundary of the product $a \in
B_n\left(\bigoplus A_{\alpha}\right)$ if and only if it's in the boundary of
every component $a \in \prod B_n\left(A_{\alpha}\right)$, and $a_{\alpha} = 0$
for all but finitely many indicies, I.e $a \in \bigoplus
B_n\left(A_{\alpha}\right)$. Now all arguments generalize analogously.

\subsection*{Exercise 1.2.2}

We first show that the two concepts for "kernel of a map", conicide. \\

Let $f : A \to B$ be an $R$-linear map, and $g : C \to A$ be such that $gf =
0$. Then $\im(g) \subseteq \ker(f)$, hence we can factor $g$ so that $g = i g'$
with $g' : G \to \ker(f), i : \ker(f) \to A$. Moreover, $g'$ is uniquely
determined by $\ker(f), i$, hence $\ker(f), i$ satisfies the universal properties
of kernels in additive categories. Since universal categories are unique up to
isomorphism, it follows that the two definitions of the kernel of a map
coincide. \\

Now, if $i : A \to B$ is a monic map, and $f$ is the kernel of $i$, then $fi =
0$ hence $f = 0$, so $\ker(i) = 0$ by the universal property and $i$ is a
monomorphism. \\

Similarly, if $\ker(i) = 0$, let $f$ be the kernel of $i$. Then if $g$ is a map
such that $gi = 0$, we have that $g$ factors through $f$, via say $g = g'f$.
But then $g = g'f = g'0 = 0$ and $i$ is mono. \\

The dual statement with cokernels, epis and epimorphisms is analagous.

\subsection*{Exercise 1.2.3}

The argument above about kernels coinciding can be applied to every degree
simultaneously. All that remains to show is that the functions $g', i$ are
morphisms of chain complexes, i.e commute with differentials, but this is easy
to see as $g'$ is just $g$ with a narrower codomain, and $i$ is just the
identity map with wider codomain.

\subsection*{Exercise 1.2.4}

In Exercise 1.2.3, we showed that kernel and cokernels may be applied 
component-wise to modules of a chain complex in $\Ch(R-\text{mod})$.
We first show that we can do this in any abelian category. 

\begin{lemma}
	Let $\mathcal{A}$ be an abelian category, $(B, d), (C, d') \in
	\Ch(\mathcal{A})$ be two chain complexes and $f : B \to C$ a morphism of
	complexes. Let $A_i = \ker(f_i)$, and $j_i : \ker(f_i) \to A_i$ be the
	injections. Then, as $f_{i-1} d_i j_i = d'_i f_i j_i = 0$, we have a unique
	map $d''_i : A_i \to A_{i - 1}$ given by the universal property
	of the kernel of $f_{i - 1}$ as in the following commutative diagram 
	\[
	\begin{tikzcd}
		A_i \arrow[r, dashed, "d''_i"] \arrow[d, "j_i"] & A_{i-1} \arrow[d, "j_{i-1}"]\\
		B_i \arrow[d, "f_i"] \arrow[r, "d_i"] & B_{i-1} \arrow[d, "f_{i-1}"] \\
		C_i \arrow[r, "d'_i"] & C_{i-1}. \\
	\end{tikzcd}
	\] 

	Then $(A, d'')$ is a kernel of $f$.
\end{lemma}
\begin{proof}
	Suppose that $(A', d'''), g : A' \to B$ are such that $gf = 0$. Then $g_i
	f_i = 0$ for all $i$, and by the universal property of $A_i = \ker(f_i)$,
	we have unique $h_i$ such that $g_i = j_i h_i$. Then $h : A' \to A$ is a
	map of chain complexes since 
	\[
		j_{i-1} h_{i-1} d'''_i
		=
		g_{i-1} d'''_i
		=
		d_i g_i
		=
		d_i j_i h_i
		=
		j_{i-1} d''_i h_i,
	\] 
	which since $j_{i-1}$ is mono implies
	\[
		h_{i-1} d'''_i
		=
		d''_i h_i.
	\]
	Moreover, $h$ is unique with this property since the $h_i$ are unique.
\end{proof}

The statement and proofs for epimorphisms is dual to the one above. \\

The statement of the exercise now follows since if 
\[
	0 \rightarrow A \xrightarrow{f} B \xrightarrow{g} C \rightarrow 0
\] 
is a sequence in $\Ch(\mathcal{A})$, then kernels and cokernels (hence also
images) may be taken component-wise, so $\ker(g) = \im(f)$ if and only if
$\ker(g_i) = \im(f_i)$ for all $i$.

\subsection*{Exercise 1.2.5}

Let $x \in \ker(d_{2n})$, with 
\[
	x = (x_{n + i, n -i})_{i \in \N}.
\] 
with each $x_{r, s} \in C_{r, s}$. Since $C$ is bounded, we have upper and
lower bounds $N_u, N_l$ such that we can write
\[
	x = (x_{n + i, n -i})_{i = N_l}^{N_u}.
\] 
After relabeling we can assume that $N_l = 0$, and write $N_u = N$. \\

So, suppose that $x = \sum_{i = 0}^{N} x_{n + i, n - i}$ and that $d(x) = 0$.
Then 
\begin{align*}
	d^h(x_{n, n}) &= 0, \\
	d^h(x_{n + i + 1, n + i - 1}) + d^v(x_{n + i, n - i}) &= 0,
	\ \forall i \in [N - 1], \\ 
	d^v{x_{n + N, n - N}} &= 0.
\end{align*} 
Since the rows are exact, the first condition tells us that there is $y_{n + 1,
n} \in C_{n + 1, n}$ such that $d^h(y_{n + 1, n}) = x_{n, n}$. Then 
\begin{align*}
	0
	&=
	d^h(x_{n + 1, n - 1}) + d^v(x_{n, n}) \\
	&=
	d^h(x_{n + 1, n - 1}) + d^v(d^h(y_{n + 1, n})) \\
	&=
	d^h(x_{n + 1, n - 1}) - d^h(d^v(y_{n + 1, n})) \\
	&=
	d^h(x_{n + 1, n - 1} - d^v(y_{n + 1, n})),
\end{align*} 
hence we have $y_{n + 2, n - 1} \in C_{n + 2, n - 1}$
such that $d^h(y_{n + 2, n - 1}) = x_{n + 1, n - 1} - d^v(y_{n + 1, n}))$.
Continuing on this way, we can write $y = \sum_{i = 0}^{N+1} y_{n + 1 + i, n - i}$,
and get
\begin{align*}
	d(y) 
	&=
	\sum_{i = 0}^{N + 1} d(y_{n + 1 + i, n - i}) \\
	&=
	\sum_{i = 0}^{N + 1} 
	d^h(y_{n + 1 + i, n - i})
	+
	d^v(y_{n + 1 + i, n - i}) \\
	&=
	\sum_{i = 0}^{N + 1} 
	x_{n + i, n - i} - d^v(y_{n + i, n - i + 1}) 
	+
	d^v(y_{n + 1 + i, n - i}) \\
	&=
	\sum_{i = 0}^{N + 1} 
	x_{n + i, n - i}  \\
	&= 
	x.
\end{align*}

\subsection*{Exercise 1.2.7}

\subsubsection*{(1)}

Each component of 
\[
	0 
	\rightarrow 
	Z(C) 
	\rightarrow 
	C 
	\xrightarrow{d} 
	B(C)[-1] 
	\rightarrow 
	0
\] 
is given by
\[
	0 
	\rightarrow 
	\ker(d_i) 
	\rightarrow 
	C_{i} 
	\xrightarrow{d_i}
	\im(d_i)
	\rightarrow 0,
\] 
and is clearly exact. That these component maps commute with differentials 
to form morphisms of complexes is easy to see.

\subsubsection*{(2)}

Each component of 
\[
	0 
	\rightarrow 
	H(C) 
	\rightarrow 
	C/B(C) 
	\xrightarrow{d} 
	Z(C)[-1] 
	\rightarrow
	H(C)[-1] 
	\rightarrow 
	0
\] 
is given by
\[
	0 
	\rightarrow 
	\ker(d_i) / \im(d_{i + 1})
	\rightarrow 
	C_i / \im(d_{i + 1})
	\xrightarrow{d_i} 
	\ker(d_{i - 1}) / \im(d_i \circ d_{ i + 1})
	\cong
	\ker(d_{i - 1}) 
	\rightarrow
	\ker(d_{i - 1})/\im(d_{i})
	\rightarrow 
	0
\] 
and is clearly exact. That these component maps commute with differentials to
form morphisms of complexes is easy to see.

\subsection*{Exercise 1.2.8}

I don't know why we need $B[-1]$ and not $B$ in the row $q = 1$ of 
$D$. I'll assume this is a mistake, and define $D$ as 
\[
\begin{tikzcd}
	& \ldots \arrow[d] 
	& \ldots \arrow[d]
	& \ldots \arrow[d] 
	& \ldots \arrow[d]
	& \ldots \arrow[d] 
	\\
	\ldots
	& 0 \arrow[l] \arrow[d] 
	& 0 \arrow[l] \arrow[d] 
	& 0 \arrow[l] \arrow[d] 
	& 0 \arrow[l] \arrow[d] 
	& 0 \arrow[l] \arrow[d] 
	& \ldots \arrow[l]
	\\
	\ldots
	& B_{-2} \arrow[l, "d_B"] \arrow[d, "f"]
	& B_{-1} \arrow[l, "d_B"] \arrow[d, "-f"]
	& B_{0} \arrow[l, "d_B"] \arrow[d, "f"]
	& B_{1} \arrow[l, "d_B"] \arrow[d, "-f"]
	& B_{2} \arrow[l, "d_B"] \arrow[d, "f"]
	& \ldots \arrow[l]
	\\
	\ldots
	& C_{-2} \arrow[l, "d_C"] \arrow[d] 
	& C_{-1} \arrow[l, "d_C"] \arrow[d] 
	& C_{0} \arrow[l, "d_C"] \arrow[d] 
	& C_{1} \arrow[l, "d_C"] \arrow[d] 
	& C_{2} \arrow[l, "d_C"] \arrow[d] 
	& \ldots \arrow[l]
	\\
	\ldots
	& 0 \arrow[l] \arrow[d] 
	& 0 \arrow[l] \arrow[d] 
	& 0 \arrow[l] \arrow[d] 
	& 0 \arrow[l] \arrow[d] 
	& 0 \arrow[l] \arrow[d] 
	& \ldots \arrow[l]
	\\
	& \ldots
	& \ldots
	& \ldots
	& \ldots
	& \ldots,
	\\
\end{tikzcd}
\] 
where $C$ is in row $q = 0$. We can then define $C'$
as
\[
\begin{tikzcd}
	& \ldots \arrow[d] 
	& \ldots \arrow[d]
	& \ldots \arrow[d] 
	& \ldots \arrow[d]
	& \ldots \arrow[d] 
	\\
	\ldots
	& 0 \arrow[l] \arrow[d] 
	& 0 \arrow[l] \arrow[d] 
	& 0 \arrow[l] \arrow[d] 
	& 0 \arrow[l] \arrow[d] 
	& 0 \arrow[l] \arrow[d] 
	& \ldots \arrow[l]
	\\
	\ldots
	& C_{-2} \arrow[l, "d_C"] \arrow[d] 
	& C_{-1} \arrow[l, "d_C"] \arrow[d] 
	& C_{0} \arrow[l, "d_C"] \arrow[d] 
	& C_{1} \arrow[l, "d_C"] \arrow[d] 
	& C_{2} \arrow[l, "d_C"] \arrow[d] 
	& \ldots \arrow[l]
	\\
	\ldots
	& 0 \arrow[l] \arrow[d] 
	& 0 \arrow[l] \arrow[d] 
	& 0 \arrow[l] \arrow[d] 
	& 0 \arrow[l] \arrow[d] 
	& 0 \arrow[l] \arrow[d] 
	& \ldots \arrow[l]
	\\
	& \ldots
	& \ldots
	& \ldots
	& \ldots
	& \ldots,
	\\
\end{tikzcd}
\]
where $C$ is in row $q = 0$, and $B$ as 
\[
\begin{tikzcd}
	& \ldots \arrow[d] 
	& \ldots \arrow[d]
	& \ldots \arrow[d] 
	& \ldots \arrow[d]
	& \ldots \arrow[d] 
	\\
	\ldots
	& 0 \arrow[l] \arrow[d] 
	& 0 \arrow[l] \arrow[d] 
	& 0 \arrow[l] \arrow[d] 
	& 0 \arrow[l] \arrow[d] 
	& 0 \arrow[l] \arrow[d] 
	& \ldots \arrow[l]
	\\
	\ldots
	& B_{-2} \arrow[l, "d_B"] \arrow[d] 
	& B_{-1} \arrow[l, "d_B"] \arrow[d] 
	& B_{0} \arrow[l, "d_B"] \arrow[d] 
	& B_{1} \arrow[l, "d_B"] \arrow[d] 
	& B_{2} \arrow[l, "d_B"] \arrow[d] 
	& \ldots \arrow[l]
	\\
	\ldots
	& 0 \arrow[l] \arrow[d] 
	& 0 \arrow[l] \arrow[d] 
	& 0 \arrow[l] \arrow[d] 
	& 0 \arrow[l] \arrow[d] 
	& 0 \arrow[l] \arrow[d] 
	& \ldots \arrow[l]
	\\
	& \ldots
	& \ldots
	& \ldots
	& \ldots
	& \ldots,
	\\
\end{tikzcd}
\]
where $B$ is in row $q = 0$.

\subsection*{Exercise 1.4.5}
\subsubsection*{(a)}

First note that any map is chain homotopic to itself via $s = 0$. Now, let $f,
g, h : C \to D$ be chain maps, and $s, s'$ be chain homotopies $f$ to $g$ and
$g$ to $h$ respectively. Then $g - f = -sd - ds = -sd + d(-s)$ so $-s$ is a
chain homotopy $g$ to $f$. Moreover,
\begin{align*}
	f - h
	&=
	(f - g) + (g - h) \\
	&=
	sd + ds + s'd + ds' \\
	&=
	(s + s')d + d(s + s')
\end{align*}
so $s + s'$ is a chain homotopy from $f$ to $h$, and we've verified that chain
homotopy indeed induces an equivalence relation on chain maps. \\

The category of modules is abelian, so $\Hom(C, D)$ is an abelian group, hence
we only need to show that the inherited addition is well-defined in
$\Hom_{\CK}(C, D)$. Let $f, g : C \to D$ be chain homotopic maps via $s$ and $h : C \to D$
be another map. Then 
\begin{align*}
	f + h
	&=
	g + h + sd + ds,
\end{align*} 
hence $s$ is a chain homotopy $f + h$ to $g + h$ and addition is well-defined.

\subsubsection*{(b)}

We have 
\begin{align*}
	vfu
	&=
	v(g + sd + ds)u \\
	&=
	vgu + vsdu + vdsu \\
	&=
	vgu + vsud + dvsu,
\end{align*}
hence $vfu$ is chain homotopic to $vgu$ via $vsu : B \to E$. \\

It follows that composition is well-defined in $\CK$, hence $\CK$ is a
category. Moreover, addition of morphisms distributes over composition since 
these operations are inherited from $\Ch$, hence $\CK$ is an $\Ab$-category.

\subsubsection*{(c)}

We have
\begin{align*}
	f_1 + f_2
	&=
	g_1 + g_2 + s_1d + ds_1 + s_2d + ds_2 \\
	&=
	g_1 + g_2 + (s_1 + s_2)d + d(s_1 + s_2),
\end{align*}
hence the congruence relation of homotopy equivalence is additive. \\

We now show that additive congruence relations $\sim$ on additive categories
$C$ induce additive quotient categories $C / \sim$ and functors $C \to C /
\sim$. The hom-sets are abelian group as quotienting by $\sim$ is the same on
the hom-sets as taking the quotient by the subgroup of elements that are $\sim
0$. Moreover, addition distributes over composition on the hom-sets as these
operations are inherited from $C$. Hence $C / \sim$ is an $\Ab$-category \\

The zero object $z \in \ob(C)$ is also a zero object in $C / \sim$.
Similarly, if $A, B \in C/ \sim$, then $A, B \in C$ hence we have a product
object $A \times_C B \in C$. We will show that this also is a product object in
$C / \sim$. Let $f_1, f_2$ be morphisms $Y \to A, B$ in $C / \sim$, and let
$f_1', f_2'$ be representatives of these morphisms in $C$. Then there exists a
unique $f' : Y \to A \times_C B$ such that $f_i' = \pi_i f'$, thus there is a
unique map $f : Y \to A$ in the quotient category satisfying the same
properties. \\

The functor is additive induces homomorphisms $\Hom \to \Hom_{\CK}$ as $\sim$
is additive. \\

\subsubsection*{(d)}

No, and to show this we give a counterexample. Let
\[
	C = \ldots \rightarrow 0 \rightarrow \Z \rightarrow 0 \rightarrow \ldots,
\]
and
\[
	D = \ldots \rightarrow 0 \rightarrow \Z/p\Z \rightarrow 0 \rightarrow \ldots,
\]
and $f : C \to D$ be the chain map induced by the projection $\Z \to \Z/p\Z$.
Then the kernel of $f$ in $\Ch$ is given by
\[
	\ker(f) = \ldots \rightarrow 0 \rightarrow \Z \rightarrow 0 \rightarrow \ldots,
\]
with the injection $i$ induced by the map $\Z \to \Z, n \mapsto pn$. In $\CK$
however, $i$ isn't monic. Indeed, let TODO FINNISH LATER

\subsection*{Exercise 1.5.1}

We have 
\[
	d(b, c)
	=
	(-d(b), d(c) - b),
\] 
and
\[
	dsd(b, c)
	=
	ds(-d(b), d(c) - b)
	=
	d(b - d(c), 0)
	=
	(d(d(c)) - d(b), d(c) - b)
	=
	(-d(b), d(c) - b),
\] 
hence $s$ is a splitting. Moreover, $\cone(C)$ is split exact as
\[
	(ds + sd)(b, c)
	=
	d(-c, 0) + s(-d(b), d(c) - b)
	=
	(d(c), c) + (b - d(c), 0)
	=
	(b, c).
\] 

\subsection*{Exercise 2.2.1}


Let $P$ be a projective object in $\Ch$, and consider the surjective
morphism $\pi : \cone(P) \to P[-1]$. Surjective morphisms onto projective
objects split, hence $P[-1]$ is a direct summand of the complex $\cone(P)$,
which is split exact by Exercise 1.5.1. Now, $H_n$ preserves direct sums by
Exercise 1.2.1, so $P[-1]$ must be exact as well. \\

Moreover, as $\pi$ sends $(p_{n - 1}, p_n) \to -p_{n - 1}$, we have that the
splitting $i : P[-1] \to \cone(P)$ of $\pi$ is of the form $i : p_{n - 1}
\mapsto (-p_{n - 1}, \phi(p_{n - 1}))$ for some map $\phi : P[-1] \to P$. \\

Now, let $d, d'$ be the differentials on $P, \cone(P)$. Then $-d$ is the differential 
on $P[-1]$, and as $i$ is a chain map, it commutes with $-d$ and $d'$. Thus
\begin{align*}
	0
	&=
	-d \circ -d \\
	&=
	i \circ -d \circ -d \\
	&=
	d' \circ i \circ -d \\
	&=
	d' \circ (d, \phi \circ -d) \\
	&=
	(-d \circ d, (d \circ \phi \circ - d) - d) \\
	&=
	(0, d \circ -\phi \circ d - d),
\end{align*}
hence $d \circ -\phi \circ d = d$, and $-\phi$ is a splitting, so $P[-1]$ is split exact. \\

Finally, to see that each $P_n$ is projective, let $g : B \to C$ be a
surjective module morphism, and $f : P_n \to C$. Then Let $B., C.$ be the
complexes with the $n$-th entry as $B, C$ and the rest $0$.

We skip the other direction.

\subsection*{Exercise 2.2.2}

Suppose that $\mathcal{A}$ has enough projectives, and let $(B, d) \in
\Ch(\mathcal{A})$, and that $\pi_i : P_i \to B_i$ are surjections onto the
components of $B$ from projective objects $P_i$. Then as $\pi_{i - 1}$ is a
surjection, we have that $d_i \circ \pi_i$ factors through $\pi_{i - 1}$. I.e
there exists $h_i : P_i \to P_{i - 1}$ such that the following diagram commutes
\[
	\begin{tikzcd}
		P_i \arrow[d, "\pi_i"] \arrow[r, "h_i"] & P_{i-1} \arrow[d, "\pi_{i-1}"] \\
		B_i \arrow[r, "d_i"] & B_{i-1}. \\
	\end{tikzcd}
\]
Now, we don't know that whether or not $h \circ h$ is $0$ or not, so we can't
form a chain complex out of $P, h$ as is. But let's pretend that $P, h$ is a
chain complex, and form $\cone(P)$ as in Exercise 1.5.1. Then $\cone(P)$ is a
chain complex, even if $P$ might not be, because 
\[
	\begin{bmatrix}
		-h & 0 \\
		-\id & h
	\end{bmatrix}^2
	=
	\begin{bmatrix}
		-h \circ - h & 0 \\
		(-h \circ -\id) + (h \circ - \id) & h \circ h
	\end{bmatrix}
	=
	\begin{bmatrix}
		0 & 0 \\
		0 & 0
	\end{bmatrix}.
\] 
LAST EQUALITY IS WRONG SINCE WE DON'T KNOW THAT $hh$ is $0$. Moreover,
$\cone(P)$ is split exact for the same reason as in Exercise 1.5.1. Indeed, let
$s : \cone(P)_i \to \cone(P)_{i + 1}$ be given by 
\[
	s(b, c) = (-c, 0),
\] 
and $d'$ be the differential on $\cone(P)$. Then 
\begin{align*}
	(sd' + d's)(b, c)
	&=
	s(-h(b), h(c) - b) + d'(-c, 0) \\
	&=
	(b - h(c), 0) + (h(c), c) \\
	&=
	(b, c), 
\end{align*} 
hence $\cone(P)$ is split exact. As the $\pi_i$ commute with $h_i$, the map
$\cone(P) \to B$ given by $(b_i, c_i) \to (-\pi_i, 0)$ is a chain map, which is
also surjective by choice of $\pi_i$.

\subsection*{Exercise 2.3.1}

Any ideal $J \subset \Z/(m)$ is of the form $(d)$ where $dk = m$. So, let $f :
J \to \Z/(m)$ be a $\Z/(m)$-module morphism, where $f(d) = a$. Then, as $0 =
f(dk) = kf(d)$, we must have that $k'm = kf(d)$, hence $dk' = f(d)$. Thus we
can extend $f$ to $R \to R$ by sending $1 \mapsto k'$, and $R$ is an injective
$R$-module by Baer's Criterion. \\

Now, consider the $R$-module $M = \Z/d$ with $dk = m$, and some prime $p$
divides $d$ and $m/d$. Then we have an injective $R$-linear map $M \to R$ given
by $1 \mapsto k$, which if $M$ was injective, would have a splitting. But the
hypothesis assures that this is not the case, hence $M$ is not injective.

\subsection*{Exercise 2.4.2}

Let $A \in \mathcal{A}$, and $P. \to A.$ be a projective resolution. Then 
\begin{align*}
	U(L_i(F))(A)
	&=
	U(H_i(F(P.))).
\end{align*}
Now, given any chain complex over an abelian category
\[
	C. : 
	\ldots \rightarrow 
	C_{n + 1} \xrightarrow{d_{n + 1}}
	C_{n} \xrightarrow{d_{n}}
	C_{n - 1} \rightarrow
	\ldots,
\] 
we can form short exact sequences  
\[
	0 \rightarrow
	\ker
\] 
but exact functors commute with taking homologies, and to see this, just split any
chain complex into short exact sequences. 

\end{document}
