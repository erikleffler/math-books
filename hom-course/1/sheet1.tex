\documentclass{article}
\usepackage[utf8]{inputenc}

\usepackage{rotating}
\usepackage{mathtools}
\usepackage{algpseudocode}
\usepackage{amsfonts}
\usepackage{amsmath}
\usepackage{amssymb}
\usepackage{amsthm}
\usepackage{bm}
\usepackage{listings}
\usepackage{float}
\usepackage{fancyvrb}
\usepackage{xcolor}
\usepackage{tikz-cd}

\hbadness = 10000
\vbadness = 10000

\newcommand\restr[2]{{% we make the whole thing an ordinary symbol
  \left.\kern-\nulldelimiterspace % automatically resize the bar with \right
  #1 % the function
  \vphantom{\big|} % pretend it's a little taller at normal size
  \right|_{#2} % this is the delimiter
  }}

% Default fixed font does not support bold face
\DeclareFixedFont{\ttb}{T1}{txtt}{bx}{n}{12} % for bold
\DeclareFixedFont{\ttm}{T1}{txtt}{m}{n}{12}  % for normal
% Custom colors

\usepackage{color}
\definecolor{deepblue}{rgb}{0,0,0.5}
\definecolor{deepred}{rgb}{0.6,0,0}
\definecolor{deepgreen}{rgb}{0,0.5,0}

% Python style for highlighting
\newcommand\pythonstyle{\lstset{
language=Python,
basicstyle=\ttm,
morekeywords={self},              % Add keywords here
keywordstyle=\ttb\color{deepblue},
emph={MyClass,__init__},          % Custom highlighting
emphstyle=\ttb\color{deepred},    % Custom highlighting style
stringstyle=\color{deepgreen},
frame=tb,                         % Any extra options here
showstringspaces=false
}}

\lstnewenvironment{python}[1][]
{
\pythonstyle
\lstset{#1}
}
{}

\theoremstyle{definition}

\newtheorem{theorem}{Theorem}[subsection]
\newtheorem{definition}[theorem]{Definition}
\newtheorem{corollary}[theorem]{Corollary}
\newtheorem{lemma}[theorem]{Lemma}

\newcommand{\Z}{\mathbb{Z}}
\newcommand{\Q}{\mathbb{Q}}
\newcommand{\R}{\mathbb{R}}
\newcommand{\C}{\mathbb{C}}
\newcommand{\K}{\mathbb{K}}
\renewcommand{\P}{\mathbb{P}}
\newcommand{\F}{\mathbb{F}}
\newcommand{\N}{\mathbb{N}}
\newcommand{\A}{\mathbb{A}}


\newcommand{\x}{\bm{x}}
\newcommand{\Kx}{\K[\bm{x}]}
\newcommand{\KP}[2]{\K[#1_1, #1_2, \ldots, #1_{#2}]}

\newcommand{\oo}{\mathcal{O}}
\newcommand{\osp}[1]{\oo_{\Spec\left(#1\right)}}
\newcommand{\rospu}[2]{\restr{\osp{#1}}{#2}}
\newcommand{\oop}[2]{\oo_{\P^{#1}_{#2}}}
\newcommand{\ox}{\mathcal{O}_X}

\renewcommand{\AA}[1]{\A^{#1}}
\newcommand{\An}{\A^n}
\newcommand{\Am}{\A^m}

\newcommand{\PP}[1]{\P^{#1}}
\newcommand{\Pn}{\P^n}
\newcommand{\Pm}{\P^m}

\newcommand{\Hom}{\text{Hom}}
\newcommand{\Aut}{\text{Aut}}
\newcommand{\End}{\text{End}}
\newcommand{\Iso}{\text{Iso}}
\newcommand{\Mor}{\text{Mor}}

\newcommand{\lm}{\text{lm}}
\newcommand{\nr}{\text{nilrad}}
\newcommand{\Spec}{\text{Spec}}
\newcommand{\Proj}{\text{Proj}}
\newcommand{\proj}{\Proj}
\newcommand{\spec}{\Spec}
\newcommand{\codim}{\text{codim}}
\newcommand{\coker}{\text{coker}}
\newcommand{\colim}{\text{colim}}
\newcommand{\ann}{\text{ann}}
\newcommand{\im}{\text{im}}
\newcommand{\id}{\text{id}}
\newcommand{\height}{\text{height}}

\newcommand{\Tor}{\text{Tor}}
\newcommand{\Ext}{\text{Ext}}
\newcommand{\HK}{\mathbf{K}}

\newcommand{\catname}[1]{{\normalfont\textbf{#1}}}
\newcommand{\Set}{\catname{Set}}
\newcommand{\CRing}{\catname{CRing}}
\newcommand{\Ch}{\catname{Ch}}
\newcommand{\Top}{\catname{Top}}
\newcommand{\Rep}{\catname{Rep}}
\newcommand{\rep}{\catname{rep}}
\newcommand{\Mod}{\catname{Mod}}
\newcommand{\RMod}{\,_{R}\catname{Mod}}
\newcommand{\ModR}{\catname{Mod}_{R}}
\newcommand{\op}{\catname{op}}
\newcommand{\Ab}{\catname{Ab}}

\setlength{\parindent}{0pt}




\begin{document}

\section*{List 1}

\subsection*{Exercise 1}

\subsubsection*{(c)}

Let $V, T$ be a $k$-vector space. Then we can consider $V$ as a $k[t]$ module
by letting $t$ act on $V$ via $t^k v \mapsto T^{\circ k}(v)$ for $v \in V$,
where $T^{\circ k}$ denotes $k$ repeated applications. \\

Now let $M$ be a $k[t]$ module. Then we can consider $M$ as a $k$-space, and
multiplication by $t$, $\cdot t : m \mapsto tm$ as a $k$-linear map. \\

These two procedures are clearly inverse each other, and so $k[t]$ modules are
the same as pairs of $V, T$ of a $k$-space and an endomorphism on it.

\subsubsection*{(d)}

Let $M$ be a $k[G]$-module. Then for each $g$ in $G$, define $\rho_g \in
\Aut_{k}(M)$ by $\rho_{g} : m \mapsto gm$. Note that $\rho_g$ has an inverse
$\rho_{g^{-1}}$, and so really is an automorphism. Then the map $g \mapsto
\rho_g$ is a group homomorphism, as $\rho_e = I$ and
\[
	\rho_{gf}(m)
	=
	gfm
	=
	g\rho_{f}(m)
	=
	\rho_{g}\rho_f(m).
\]

Now let $\rho : G \to \Aut_k(V)$ be a $G$-representation over some $k$-space
$V$. Then we can turn $V$ into a $k(G)$-module by letting $g \in G$ act on $V$
via $gv \mapsto \rho(g)(v)$, and extending algebraically. This is well defined
as if $g, f \in M$, then
\[
	\rho(gf)(v)
	=
	gfv
	=
	\rho(g) \circ \rho(f)(v),
\]
and as each $\rho(g)$ lives in $\Aut_k(V)$, everything distributes in the right
way.

These two procedures (functors) are inverse each other and so yada yada... \\


\subsection*{Exercise 2}

\subsubsection*{(a)}

A $k[x]$-module is the same as a vector space $V$ together with an
endomorphism $T$. A $k[x]/(x^n)$-module should then such a pair $V, T$ where
$T^{\circ n} = 0$. 

\subsubsection*{(b)}

First, of $k[x, x^{-1}] \cong k[x, y]/(xy)$, so a $k[x, x^{-1}]$-module ought
to be a vector space $V$ together with two mutually inverse automorphisms $T,
T^{-1}$.


\subsection*{Exercise 6}

\subsubsection*{(c)}

Let $\phi : V \to W$ be a $k[t]$-linear map. Then for $v \in V$ we have 
\[
	\phi(T(v))
	=
	\phi(t v)
	=
	t \phi(v)
	=
	W(\phi(v)).
\] 

Now let $\phi : V \to W$ be a $k$-linear map which commutes with $T, W$. Then 
\[
	\phi(t v)
	=
	\phi(T(v))
	=
	W(\phi(v))
	=
	t \phi(v),
\]
and $\phi$ is a $k[t]$-linear map as well.

\subsection*{Exercise 9}

\begin{align*}
	(\phi + \psi)(am + n)
	&=
	\phi(a m + n) + \psi(am + n) \\
	&= 
	a\phi(m) + \phi(n) + a\psi(m) + \psi(n) \\
	&=
	a(\phi + \psi)(m) + (\phi + \psi)(n)
\end{align*}

\subsection*{Exercise 10}

Any $\Z$-map $\phi$ out of $\Z$ is determined by where it sends $1$, since
$\phi(n) = n\phi(1)$. \\ 

\subsubsection*{(a)}

We can send $1$ anywhere, as $\phi_k : n \mapsto k n$
is an $\Z$-map for all $k \in \Z$. Indeed, 
\[
	\phi_k(am + n)
	=
	kam + kn
	=
	a\phi_k(m) + \phi_k(n).
\] 
Moreover, $(\phi_k + \phi_{l})(m) = km + lm = (k + l)m = \phi_{k + l}(m)$, and
$\phi_k + \phi_l = 0 \Leftrightarrow k + l = 0$ so $\Hom_{\Z}(\Z, \Z) \cong \Z$.

\subsubsection*{(b)}

We can send $1$ anywhere in $Z/(m)$, as $\phi_k : n \mapsto k n + (m)$ is an
$\Z$-map for all $k \in \Z$. Indeed, 
\[
	\phi_k(ax + n)
	=
	kax + kn + (m)
	=
	a\phi_k(x) + \phi_k(n).
\] 
Moreover, $(\phi_k + \phi_{l})(x) = kx + lx + (m) = (k + l)x + (m) = \phi_{k +
l}(x)$, and $\phi_k + \phi_l = 0 \Leftrightarrow k + l \in (m)$, and so
$\Hom_{\Z}(\Z, \Z/(m)) \cong \Z/(m)$.

\subsubsection*{(c)}

Let $\phi : A \to \Hom_{\Z}(\Z, A)$ be given by $\phi(a) : k \mapsto ka$. Then
$\phi(a)$ is a $\Z$-map since $\phi(a)(lm + n) = lma + na = l\phi(a)(m) +
\phi(a)(n)$ and $\phi$ is group homomorphism as $\phi(a + b)(k) = ka + kb =
\phi(a)(k) + \phi(b)(k)$ and $\phi(0)(k) = 0k = 0$ is the zero-map. Moreover,
$\phi$ is injective, since $\phi(a) = 0$ gives that $0 = \phi(a)(1) = 1a = a$.
Finally, $\phi$ is surjective since if $\psi : \Z \to A$, then $\psi(n) = n
\psi(1) = n \phi(\psi(1))$. We've shown that $A \cong \Hom_{\Z}(\Z, A)$.

\subsubsection*{(d)}

Let $\phi : \Z/(m) \to \Z$ be a $\Z$-map. Then $m\phi(1) = \phi(m) = 0$, hence $\phi(1) = 0$
since $\Z$ is a domain, and $\phi = 0$.

\subsubsection*{(f)}

Let $\phi : \Q \to \Z$ be a $\Z$-map. Then $b\phi(a/b) = \phi(a)$, and so $b |
\phi(a)$. But $a/b = ca/cb$ for any $c$, and so $c | \phi(a)$ for all $c
\in \Z$. It follows that $\phi(a) = 0$ and so $\phi = 0$.

\subsection*{Exercise 18}

\subsubsection*{(b)}

Let $M$ be an $R/I$ module, and let $R$ act on $M$ by 
\[
	rm = (r + I)m.
\] 
Then any $i \in I$ annihilates all of $M$ since $im = (0 + I)m = 0$. \\

Now let $M$ be an $R$ module which is annihilated by $I$. Then let 
$R/I$ act on $M$ by 
\[
	(r + I)m = rm.
\] 
This is well defined, since if $r + I = r' + I$, we have $r - r' \in I$ and so 
\[
	(r + I)m
	=
	rm
	=
	rm - rm + r'm
	=
	r'm
	=
	(r' + I)m.
\] 

\subsubsection*{(c)}

Any $m + IM \in M/IM$ is annihilated by $I$. Hence $M/IM$ is an $R/I$-module by
the part (b).

\subsubsection*{(d)}

Suppose $m_1, \ldots m_n$ generate $M$, and let $m + IM \in M/IM$. Then we can
write $m$ as an $R$-linear combination of the $m_i$,
\[
	m = \sum_{i = 1}^{n} r_i m_i.
\] 
Moreover, 
\[
	m + IM
	=
	\sum_{i = 1}^{n} r_i (m_i + IM)
\] 
and since $r_i$ acts as $r_i + I$ on $M/IM$,
\[
	m + IM
	=
	\sum_{i = 1}^{n} r_i (m_i + IM)
	=
	\sum_{i = 1}^{n} (r_i + I) (m_i + IM)
\]
and so the $m_i + IM$ generates $M/IM$ as a $R/I$-module. \\

To see that the size of the genarating set may decrease, note that we may have
some $m_i \in IM$. For example if $R = k[x]$, $M = R/(x^2 - 1)$, $I = (x)$.
Then $1$ and $x$ generate $M$, but $M/IM = k$ is generated by $1$ as a $R/I =
k$ module.

\subsubsection*{(e)}

Suppose that $M = \bigoplus_{i = 1}^{n} Re_i$ is a free module with basis $e_1,
\ldots, e_n$. Then we showed in the part (d) that $M/IM$ is generated by $e_i +
IM$ as an $R/I$ module. We now show that the $e_i + IM$ doesn't satisfy any
non-trivial $R/I$-linear relation. We have
\[
	0
	=
	\sum_{i = 1}^{n} (r_i + I)(e_i + IM)
\] 
if and only if 
\[
	im
	=
	\sum_{j = 1}^{n} r_j e_j
\]
for some $i \in I, m \in M$. But then we can write 
\[
	m = \sum_{j = 1}^{n} r_j' e_j
\] 
and as the $e_i$ are a basis, it follows that $i r_j' - r_j = 0$ for all $j$.
Hence $r_j \in I$, and $r_j + I = 0$.

\subsection*{Exercise 19}

It follows from Exercise 18.(e) that $M/IM$ has a basis of cardinality $|J|$
whenever $J$ is a basis for $M$. Furthermore as $R/I$ is a vector space, every
basis of $M/IM$ has the same cardinality, and so every basis of $M$ has the
same cardinality.

\subsection*{Exercise 20}

\subsubsection*{(a)}

First of, as $R$ is a PID we in particular have that every ideal is finitely
generated so $R$ is Noetherian. It follows that if we let $S$ be the set of
ideals of the form $f(N)$ for $R$-linear maps $f : M \to R$, then $S$ has a
maximal element $u(N)$.

\subsubsection*{(b)}

Let $u(N) = (a_1)$. If $a_1 = 0$, we have that $N$ is in the kernel of every
morphism $M \to R$. As $M$ is free, we can suppose $M = \bigoplus_{i \in I} R
E_i$, and we have projections $\pi_i : M \to R$ onto the $i$-th cordinate for
each $i \in I$. Since $N$ is in the kernel of every $\pi_i$, it follows that no
element of $N$ has any non-zero coordinate. Hence $N = 0$.

\subsubsection*{(c)}

Let $e'_1 = r_1x_1 + \ldots + r_n x_n$. Then let $b_i = u(x_i)$. We then have 
\[
	a_1
	=
	u(e'_1)
	=
	u(r_1x_1 + \ldots + r_n x_n)
	=
	r_1 b_1 + \ldots + r_n b_n.
\] 
Let $r$ be the generator of $(r_1, \ldots, r_n)$ and $b_i'$ be such that $r =
r_1 b_1' + \ldots r_n b_n'$. Then define the $R$-map $u' : x_i \mapsto b_i'$.
Then $u'(e'_1) = r$, hence $u'(N) \supseteq (r) \supseteq (a_1) = u(N)$, whence
maximality of $u(N)$ yields $r = a_1$. The desired result now follows as
$\pi_i(e'_1) = r_i \in (a_1)$.

\subsubsection*{(d)}

Follows immediately from our solution above.

\subsubsection*{(e)}

We have $a_1 u(e_1) = u(a_1e_1) = u(e_1') = a_1$, hence $u(e_1) = 1$ as $R$ is
a domain. Let $M' = \ker(u)$, and define $\phi : M \to M' \oplus R$ by $\phi :
x \mapsto (x - u(x)e_1, u(x))$. Then indeed $u(x - u(x)e_1) = u(x) - u(x) = 0$
so $\phi$ is well-defined. Now let $x \in \ker(\phi)$. Then by looking at the
second coordinate of $\phi$, we see that $x \in \ker(u)$, whence we get that
$\phi(x) = (x, 0)$ so $x = 0$, and $\phi$ is injective. Now let $x, r \in M'
\oplus R$. Then $\phi(x + re_1) = (x + re_1 - u(x + re_1), u(x + re_1)) = (x,
r)$ and so $\phi$ is surjective as well, hence an isomorphism.

\subsubsection*{(f)}

Suppose first that the rank of $M$ is $1$. We claim that $e_1$ generates $M$.
To see this, note that $e_1' = r_1x_1$ and from part (c) it follows that $r =
r_1 = a_1$ in this case so $e_1 = x_1$ and $e_1$ is a basis for $M$. Moreover,
$e_1' = a_1e_1$ is a basis for $N$ since \\

Now suppose that $M$ is of rank $n$, and the statement holds for all modules of
rank $< n$. Then let $e_1, a_1, u$ be as above. As $M \cong \ker(u) \oplus R$,
there exists a basis $e_2, e_3, \ldots, e_n$ for $\ker(u)$ such that $e_2,
\ldots, e_m$ is a basis for $N \cap \ker(u)$.


\section*{List 2}

\subsection*{Exercise 2}

\subsubsection*{(a), (b), (c)}

Let $m \in \ker(f)$. Then $f(m) = 0$ and so by injectivity of $f''$ and
commutativity, we get $m'' = 0$. It follows by exactness that there is $m' \in
M'$ in the preimage of $m$. As $f(m) = 0$, again by commutativity and the fact
that $M' \to N' \to N$ are all injective, we have that $m' = 0$. It follows now
that $m$ is $0$ by injectivity. \\

Here we used injectivity of $f'$ and $f''$. \\

Now let $n \in N$. Then let $m'' = (f'')^{-1}(n'')$ and $m_0$ be some element
in the preimage of $m''$. Then $f(m_0) - n$ is mapped to $0$ along $N \to N''$,
and so there is some element $n_1' \in N'$ in the preimage of $f(m_0) - n$
by exactness. As $f'$ is surjective, we have $m_1' \in (f')^{-1}(n_1')$. 
Now have that $f(m_1) = f(m_0) - n$ by commutativity, and so $f(m_1 - m_0) = n$. \\

Here we used surjectivity of $f'$ and $f''$

\subsubsection*{(d)}

Let $n'' \in N''$. Then by exactness, we have $n \in N$ which maps to $n''$,
and by surjectivity of $f$ we have $m \in M$ which maps to $n$. It follows that 
$f''(m'') = n''$ by commutativity. \\

Here we only used surjectivity of $f$.

\subsubsection*{(e)}

We give a counter example
\[
\begin{tikzcd}
    0 
	\arrow{r} 
	& 
	\mathbb{\Z} 
	\arrow{r}{6}
	\arrow{d}{3} 
	& 
	\Z
	\arrow{r}
	\arrow{d}{1} 
	& 
	\Z/6\Z
	\arrow{r}
	\arrow{d} 
	& 
	0 \\
    0
	\arrow{r} 
	& 
	\Z
	\arrow{r}{2} 
	& 
	\Z
	\arrow{r}
	& 
	\Z/2\Z
	\arrow{r} 
	& 
	0
\end{tikzcd}
\] 

\subsubsection*{(f)}

Let $m' \in \ker(f')$. Then $f(m) = 0$, and since $m' \mapsto m \mapsto f(m)$
are all injective, we have $m' = 0$. \\

Here we used injectivity of $f$ only.

\subsubsection*{(g)}

We give a counter example.
\[
\begin{tikzcd}
    0 
	\arrow{r} 
	& 
	\mathbb{\Z} 
	\arrow{r}{6}
	\arrow{d}{3} 
	& 
	\Z
	\arrow{r}
	\arrow{d}{1} 
	& 
	\Z/6\Z
	\arrow{r}
	\arrow{d} 
	& 
	0 \\
    0
	\arrow{r} 
	& 
	\Z
	\arrow{r}{2} 
	& 
	\Z
	\arrow{r}
	& 
	\Z/2\Z
	\arrow{r} 
	& 
	0
\end{tikzcd}
\]

\subsection*{Exercise 3}
\subsubsection*{(a)}

We give a counter example.
\[
\begin{tikzcd}
	\Z/8\Z
	\arrow{r}
	\arrow{d} 
	& 
	\Z/2\Z
	\arrow{r}
	\arrow{d}
	& 
	0
	\arrow{r}
	\arrow{d} 
	& 
	0 \\
	\Z/4\Z
	\arrow{r}
	&
	\Z/2\Z
	\arrow{r}
	& 
	0
	\arrow{r} 
	& 
	0
\end{tikzcd}
\]

\subsubsection*{(b)}

This is a stronger statement than Exercise 2 (g), which we gave a counter example for. I.e
it is false.

\subsection*{Exercise 10}

\subsubsection*{(a)}
Both statements are equivalent to $f_2 = 0$.

\subsubsection*{(b)}

If $f_1$ is surjective, $\ker(f_2) = M_2$ and so $f_2 = 0$. If $f_4$ is
injective, $\im(f_3) = 0$, and so $f_3 = 0$. Then $M_3 = \ker(f_3) = \im(f_2) =
0$.

\subsection*{Exercise 13}

\subsubsection*{(a)}

Let $\hat{G} = \{\hat{g}_1, \ldots, \hat{g}_m\}$ be a generating set for $M''$,
and $G$ be some set of choices $g_i \in \pi^{-1}(\hat{g}_i)$ for each
$\hat{g}?i$. Also let $F = \{f_1, \ldots, f_n\}$ be the set of generators for
$M'$ injected into $M$. Then let $m \in M$. Let $m''$ be the image of $m$ in
$M''$. Then we can write $m''$ as a linear combination of the $\hat{g}_i$, and
pulling this back to $M$ we get a linear combination of $g_i$ which maps to the
same element as $m$. It follows that $m - \sum a_i g_i$ is in the kernel, and
thus can be written as a linear combination of $f_i$, after which we can see
that $m$ is a linear combination of elements from $F$ and $G$. \\

\subsubsection*{(b)}

$M''$ is generated by the image of the generators of $M$.

\subsubsection*{(c)}

We have that $M''$ is always finitely generated whenever $M$ is. Hence the
statement is equivalent to saying that all submodules of any finitely generated
module $M$ are finitely generated. This is true over Noetherian rings, hence we
need to consider some non-Noetherian $R$ to construct a counterexample. Moreover,
If $R$ is non-Noetherian, then it will have some infinitely generated ideal, which 
we can take as our submodule. A counterexample is given by

\[
	\begin{tikzcd}
		0
		\arrow{r}
		&
		(x_1, x_2, \ldots)
		\arrow{r}
		&
		k[x_1, x_2, \ldots]
		\arrow{r}
		&
		k
		\arrow{r}
		&
		0
	\end{tikzcd}.
\] 


\subsection*{Exercise 19}

Being projective is the same thing as being a direct summand of a free module,
and over PID's, submodules of free modules are free.

\subsection*{Exercise 21}

Let 

\subsection*{Exercise 22}

Let $P$ be a projective $k[t]$-module. Then $P$ is a direct summand of some
free $k[t]$-module $F = \bigoplus_{i \in I} k[t]$, and $P$ is naturally graded.
Now, $t$ acts as a degree $1$ map on $F$, and therefore on $P$ as well. Our
result now follows from the fact that no finite dimensional subspace of $F$ can
have a degree $1$ endomorphism.

\section*{List 3}

\subsection*{Exercise 1}
\subsubsection*{(a)}
Let 
\[
\begin{tikzcd}
	& N'
	\arrow{r}{f} 
	& N
	\arrow{r}{g}
	& N''
	\arrow{r} 
	& 0
\end{tikzcd}
\] 
be an exact sequence. We will show that 
\[
\begin{tikzcd}
	& 0
	\arrow{r}
	& \Hom(N'', M)
	\arrow{r}{g^{*}} 
	& \Hom(N, M)
	\arrow{r}{f^{*}}
	& \Hom(N', M)
\end{tikzcd}
\]
is exact. First, as $g$ is surjective, it follows that $hg = 0 \Rightarrow h =
0$ and so $g^{*}$ is injective. Moreover, $f^{*}g^{*} = (gf)^{*} = 0$ since the
original sequence is exact. Finally, if $h \in \ker(f^{*})$, then $hf = 0$, and
as $g$ is the cokernel of $f$, it follows that we have a lift $h'' \in
\Hom(N'', M)$ such that $h''g = g^{*}(h'') = h$. Hence $\im(g^{*}) =
\ker(f^{*})$ and the sequence is exact.

\subsubsection*{(b)}

$\Hom(\_, M)$ being right exact is exactly what it means for $M$ to be injective, so for a counterexample,
we need to pick some non-injective module $M$, and we'll chose the $\Z$-module $\Z$. Consider the exact sequence
\[
\begin{tikzcd}
	& 0
	\arrow{r}
	& \Z
	\arrow{r}{2} 
	& \Z
	\arrow{r}
	& \Z/2\Z
	\arrow{r}
	& 0
\end{tikzcd}
\]
and the identity morphism $\id \in \Hom(\Z, \Z)$. There is no other morphism $f
\in \Hom(\Z, \Z)$ such that $2 f = \id$.

\subsection*{Exercise 3}
\subsubsection*{(a)}

This follows from the fact that left-adjoint functors commute with colimits,
\begin{align*}
	\left(
		\bigoplus_{i \in I} R e_i
	\right)
	\otimes_{R}
	\left(
		\bigoplus_{j \in J} R e_j
	\right)
	&=
	\bigoplus_{i \in I} 
	\left(
		R e_i \otimes_{R} \bigoplus_{j \in J} R e_j
	\right) \\
	&=
	\bigoplus_{i \in I} 
	\bigoplus_{j \in J} 
	R e_i \otimes_{R} R e_j \\
	&=
	\bigoplus_{(i, j) \in I \times J} 
	R (e_i \otimes_{R} e_j).
\end{align*}

\subsubsection*{(b)}

Suppose that $E \oplus M = F$ and $E' \oplus M' = F'$ with $F, F'$ free. Then
\begin{align*}
	F \otimes F'
	&=
	(E \oplus M) 
	\otimes
	(E' \oplus M') \\
	&=
	E \otimes E'
	\oplus
	E \otimes M'
	\oplus
	M \otimes E'
	\oplus
	M \otimes M'
\end{align*} 
and so $E \otimes E'$ is a direct summand of the free module $F \otimes F'$, hence projective.

\subsection*{Exercise 5}

By the structure theorem, $\Z/6 \cong \Z/3 \oplus \Z/2$ and we see that $\Z/3$
is a direct summand of the free $\Z/6$-module $\Z/6$.

\subsection*{Exercise 9}
\subsubsection*{(a)}

We name the functions in the diagram,
\[
\begin{tikzcd}
	& 0
	\arrow{r}
	& K
	\arrow{r}{f}
	& P
	\arrow{r}{g}
	\arrow[dashed]{d}{\exists ! p}
	& M
	\arrow{r}
	\arrow[equal]{d}
	& 0 \\
	& 0
	\arrow{r}
	& K'
	\arrow{r}{f'}
	& P'
	\arrow{r}{g'}
	& M'
	\arrow{r}
	& 0.
\end{tikzcd}
\] 
We have existence of $p$ by the fact that $P$ is projective and $g'$
surjective. Now $g'pf = gf = 0$ by commutativity and exactness of the top row,
and so $\im(pf) \subset \im(f')$ by exactness of the bottom row. As $f'$ is
injective, it has an inverse on its image $(f')^{-1} : \im(f) \to K'$, giving
us a map $k : K \to K'$ by $k = (f')^{-1} pf$ and we have a commutative diagram
\[
\begin{tikzcd}
	& 0
	\arrow{r}
	& K
	\arrow{r}{f}
	\arrow{d}{k}
	& P
	\arrow{r}{g}
	\arrow{d}{p}
	& M
	\arrow{r}
	\arrow[equal]{d}
	& 0 \\
	& 0
	\arrow{r}
	& K'
	\arrow{r}{f'}
	& P'
	\arrow{r}{g'}
	& M'
	\arrow{r}
	& 0.
\end{tikzcd}
\]

\subsubsection*{(b)}

We will show that the following sequence is exact,
\[
	\begin{tikzcd}[ampersand replacement=\&]
		\& 0
		\arrow{r}
		\& K
		\arrow{r}{\begin{bsmallmatrix} f \\ k \end{bsmallmatrix}}
		\& P \oplus K'
		\arrow{r}{\begin{bsmallmatrix} p & -f' \end{bsmallmatrix}}
		\& P'
		\arrow{r}
		\& 0
	\end{tikzcd}.
\] 
The composition of the two middle maps is given by
\[
	\begin{bsmallmatrix} p & -f' \end{bsmallmatrix}
	\begin{bsmallmatrix} f \\ k \end{bsmallmatrix}
	=
	pf - f'k
\] 
which is $0$ by commutativity of the diagram from part (a). Moreover,
$\begin{bsmallmatrix} f \\ k \end{bsmallmatrix}$ is injective as $f$ is. \\

Let $(a, b) \in \ker \begin{bsmallmatrix} p & -f' \end{bsmallmatrix}$. Then
$p(a) = f'(b)$. By injectivity of $f$, we have some $c \in K$ such that $f(c) =
a$. Then $f'(k(c)) = f'(b)$ by commutativity, and $b = k(c)$ by injectivity of
$f'$. It follows that $(a, b) = (f(c), k(c)) \in \im(\begin{bsmallmatrix} f \\
k \end{bsmallmatrix})$ and we have exactness at $P \oplus K'$. \\

Finally, to see that $\begin{bsmallmatrix} p & -f' \end{bsmallmatrix}$ is
surjective, let $a' \in P'$. Then let $a \in g^{-1}g'(a')$. By commutativity,
$g'p(a) = g'(a')$, hence $p(a) - a' \in \ker(g') = \im(f')$ and we have some $b
\in K'$ such that $f'(b) = p(a) - a'$. It follows that
\[
	\begin{bsmallmatrix} p & -f' \end{bsmallmatrix}
	\begin{bsmallmatrix} a \\ b \end{bsmallmatrix}
	=
	p(a) - f'(b)
	=
	p(a) - p(a) + a' 
	= 
	a'
\]
and we see that the map is surjective.

\subsubsection*{(c)}

As $P'$ is projective, the sequence from (b) splits and $K \oplus P' \cong K' \oplus P$.

\subsection*{Exercise 10}

Let $F'' \to F' \to M \to 0$ be a finite presentation of $M$. Then, let $K' =
\im(F'' \to F')$. By Exercise 9, we have the following commutative diagram.
\[
\begin{tikzcd}
	& 0
	\arrow{r}
	& K
	\arrow{r}{f}
	\arrow{d}{k}
	& F
	\arrow{r}{g}
	\arrow{d}{p}
	& M
	\arrow{r}
	\arrow[equal]{d}
	& 0 \\
	& 0
	\arrow{r}
	& K'
	\arrow{r}{f'}
	& F'
	\arrow{r}{g'}
	& M'
	\arrow{r}
	& 0,
\end{tikzcd}
\]
and $F \oplus K' \cong F' \oplus K$. As $F, F'$ and $K'$ all are finitely
generate, it follows that $K$ must be finitely generated. \\

We use this to give a module which is not finitely presented. Let $R = k[x_1, x_2, \ldots]$
and $M = k$. Then we have the exact sequence 
\[
\begin{tikzcd}
	& 0
	\arrow{r}
	& (x_1, x_2, \ldots)
	\arrow{r}
	& k[x_1, x_2, \ldots]
	\arrow{r}
	& k
	\arrow{r}
	& 0.
\end{tikzcd}
\] 
and as $k[x_1, x_2, \ldots]$ is free and of finite rank $1$, whilst $(x_1, x_2,
\ldots)$ is not finitely generated, it follows that $k$ cannot be finitely
presented.

\section*{List 4}

\subsection*{Exercise 1}

If $\id, \id' : A \to A$ are to identity morphisms then $\id = \id \circ \id' =
\id'$.

\subsection*{Exercise 2}
Suppose both $g, g' : B \to A$ are two-sided inverses of $f : A \to B$.
Then
\[
	g 
	= 
	g \circ \id 
	= 
	g \circ f \circ g'
	= 
	\id \circ g'
	= 
	g'
\]
and the two morphisms are equal. To see that one-sided inverses need not be
unique, consider $f : \Z \to \Z^{2}$ with $f : a \to (a, 0)$ and $g, g' : \Z^{2}
\to \Z$ with $g : (a, b) \to a + b$ and $g' : (a, b) \to a$.

\subsection*{Exercise 3}

As natural transformations compose, composition is well-defined in
$\text{Fun}(\mathcal{C}, \mathcal{D})$. Moreover, the identity on a functor $F$
is given by $(\id_F)_X = \id_{F(X)}$. \\

Now suppose that $\eta : F \to G$ is an isomorphism of the functors $F, G :
\mathcal{C} \to \mathcal{D}$ with inverse $\xi : G \to F$. Then in particular,
\[
	\xi_X \circ \eta_{X} = (\id_{F})_X = \id_{F(X)},
\]
and 
\[
	\eta_X \circ \xi_X 
	=
	(\id_{G})_X
	=
	\id_{G(X)},
\]
so $\eta_X : F(X) \to G(X)$ and $\xi_X : G(X) \to F(X)$ are mutually inverse
each other, hence isomorphisms. \\

Suppose instead that $\eta : F \to G$ is a natural isomorphism. Then let $\xi$
be a family of maps for each object $X \in \mathcal{C}$ such that $\xi_X : G(X)
\to F(X)$ is the two-sided inverse of $\eta_X : F(X) \to G(X)$. Then for any
morphism $f : X \to Y$, consider the diagram below.
\[
\begin{tikzcd}
	& F(X)
	\arrow{r}{\eta_X}
	\arrow{d}{F(X)}
	& G(X)
	\arrow{r}{\xi_X}
	\arrow{d}{G(X)}
	& F(X) 
	\arrow{d}{F(X)}
	\\
	& F(Y)
	\arrow{r}{\eta_Y}
	& G(Y)
	\arrow{r}{\xi_Y}
	& F(Y) \\
\end{tikzcd}
\] 
The left square commutes and the composition of the horizontal maps are
identity maps, hence the outermost rectangle defined by these composition
commute as well. It follows that
\[
	F(X) \circ \xi_X \circ \eta_X
	=
	\xi_Y \circ G(X) \circ \eta_X,
\]
and as $\eta_X$ is an isomorphism, the right square commutes and $\xi$ defines
a natural transformation inverse to $\eta$.


\subsection*{Exercise 4}

Define $\eta_X : \Hom_{R}(M \otimes N, X) \to \Hom_S(N, \Hom_{R}(M, X))$ by
\[
	\eta_X(\phi) : n \mapsto (m \mapsto \phi(m \otimes n)).
\] 
Then $\eta_X(\phi)(n)$ is an $R$-module morphism as 
\begin{align*}
	\eta_X(\phi)(n)(rm + m')
	&=
	\phi((rm + m') \otimes n) \\
	&=
	\phi(r(m \otimes n) + m' \otimes n)) \\
	&=
	r\phi(m \otimes n) + \phi(m' \otimes n') \\
	&=
	r\eta_X(\phi)(n)(m)
	+
	\eta_X(\phi)(n)(m'),
\end{align*}
and $\eta_X(\phi)$ is an $S$-module morphism as
\begin{align*}
	\eta_X(\phi)(sn + n')(m)
	&=
	\phi(m \otimes (sn + n')) \\
	&=
	\phi(m \otimes sn + m \otimes n') \\
	&=
	\phi(ms \otimes n) + \phi(m \otimes n') \\
	&=
	\eta_X(\phi)(n)(ms)
	+
	\eta_X(\phi)(n')(m) \\
	&=
	(s\eta_X(\phi)(n))(m)
	+
	\eta_X(\phi)(n')(m).
\end{align*}

Naturality of $\eta_X$ follows by the fact that both $F(X)$ and $G(X)$
ultimately send elements into $X$, and postcomposing with and $X \xrightarrow{f}
Y$ can be done before or after $\eta_X$ and the result will be the same. \\

I won't do the rest.

\subsection*{Exercise 5}

Denote the natural isomorphism by $\eta$ and let $f_A = \eta(\id_A) : B \to A$
and $f_B = \eta^{-1}(\id_B) : A \to B$. Then we have the following commutative
diagram by naturality along $f_B : A \to B$.
\[
\begin{tikzcd}
	& \Hom(A, A)
	\arrow{r}
	\arrow{d}{(f_B)_{*}}
	& \Hom(B, A) 
	\arrow{d}{(f_B)_{*}} \\
	& \Hom(A, B)
	\arrow{r}
	& \Hom(B, B) 
\end{tikzcd}
\] 
By looking at where $\id_A$ is sent, we see that $f_B \circ \eta(\id_A) = f_B
\circ f_A$ is equal to $\eta \circ f_B \circ \id_A = \eta \circ f_B = \id_B$,
I.e that $f_B f_A = \id_B$. If we now follow the same procedure in the
naturality diagram along $f_A : B \to A$, we get that $f_A f_B = \id_A$, hence
the two maps are isomorphisms between $A$ and $B$.

\subsection*{Exercise 8}

Suppose that $L$ is another $R$-module such that 
\[
\begin{tikzcd}
	& L
	\ar{r}{\phi}
	\ar{d} 
	& M
	\ar{d}{f}
	\\
	& 0
	\ar{r}
	& N
\end{tikzcd}.
\] 
Then $f \circ \phi = 0$ by commutativity, and so $\phi$ factors through the kernel of $f$,
and we have
\[
\begin{tikzcd}
	& L
	\ar[dashed]{rd}{\exists ! h} 
	\ar[bend left=30]{drr}{\phi}
	\ar[bend right=30]{ddr}
	\\
	& & \ker(f)
	\ar{r}
	\ar{d} 
	& M
	\ar{d}{f}
	\\
	& & 0
	\ar{r}
	& N
\end{tikzcd}
\] 
whence we have verified that $\ker(f)$ is the colimit of the cospan.

\section*{List 5}

\subsection*{Exercise 1}

Let $X \xrightarrow{f} Y$, $Y \xrightarrow{g_1} Z$ and $Y \xrightarrow{g_2} Z$
be maps of sets where $f$ is surjective and $g_1f = g_2f$. Then let $y \in Y$.
As $f$ is surjective, there is $x \in X$ such that $f(x) = y$. We then have
$g_1(y) = g_1(f(x)) = g_2(f(x)) = g_2(y)$, and as $y$ was arbitrary, $g_1 =
g_2$ and $f$ is epic. \\

For the other direction, suppose instead that $f$ is epic, and let $y \in Y$.
Suppose towards a contradiction that $y \not \in \im(f)$. Then we may construct
two maps $g_1, g_2 : Y \to \{1, 2\}$ which agree on all elements on $Y$, except
for that $g_i(y) = i$. Then $g_1 \not = g_2$ but $g_1f = g_2f$, contradicting
the fact that $f$ is an epimorphism.

\subsection*{Exercise 3}

The category of free $\Z/4\Z$-modules is an $\Ab$-category since it is
immediate from the definition of an $\Ab$-category that any full subcategory of
an $\Ab$-category remains an $\Ab$-category. \\

It is additive since finite direct sums of free modules remain free. \\

Finally, it is not abeliean since for example, the kernel of a morphism of free
$\Z/4\Z$-modules need not be free. Indeed, consider the map 
\[
	\Z/4\Z \overset{2}\to \Z/4\Z.
\] 
In the category of ordinary (not necessarily free) $\Z/4\Z$-modules, this map
has kernel $2\Z/4\Z \cong \Z/2\Z$ which is not a free $\Z/4\Z$-module, and we
will now show that it has no kernel in the category of free $\Z/4\Z$-modules. 
Suppose towards a contradiction that $f : M \to \Z/4\Z$ is the kernel of our map.
Then $2f = 0$ and $\im(f) \subset \{0, 2\}$. We will show that any kernel 
is a monomorphism, and that monomorphisms in our category are injective,
whence it follows that $|M| \leq 2$ hence $M = 0$. This is a contradiction
since the sequence 
\[
	\Z/4\Z \overset{2}\to \Z/4\Z \overset{2}\to \Z/4\Z
\] 
is exact and doesn't factor through $0$. Let's prove the results we need.

\begin{lemma}
	Equalizers are monic in any category.
\end{lemma}
\begin{proof}
	Let $e : E \to X$ be the equalizer of $f, f' : X \to Y$. Then suppose that
	$g, g' : Z \to E$ are such that $eg = eg'$. As $feg = f'eg = f'eg'$, there
	exist a unique map $Z \to E$ which which factorize $eg = eg'$ through $e$.
	Thus uniqueness forces $g = g'$ and $e$ is monic.
\end{proof}

\begin{corollary}
	Kernels are always monic
\end{corollary}
\begin{proof}
	The kernel of any map $f : X \to Y$ is the equalizer of $f$ and $0$.	
\end{proof}

\begin{lemma}
	Monic morphisms in the category of free $\Z/4\Z$-modules are injective.
\end{lemma}
\begin{proof}
	Let $f : M \to N$ be a monic morphism of free $\Z/4\Z$-modules	where $M =
	\bigoplus_{i \in I} \Z/4\Z e_i$ and $N = \bigoplus_{j \in J} \Z/4\Z u_j$.
	Suppose that $f(x) = f(y)$. Then let $g_1, g_2 : \Z/4\Z \to M$ be the
	morphisms which sends $g_1(1) = x$ and $g_2(1) = y$. Then $f(g_1(1)) =
	f(g_2(1))$ and as morphisms of free modules are determined by where they
	send generators, $fg_1 = fg_2$. As $f$ is monic, it follows that $g_1 =
	g_2$ hence $x = y$ and $f$ is injective.
\end{proof}

\subsection*{Exercise 5}

\subsubsection*{(1)}

Suppose that $f : X \to Y$ is a monomorphism. Then $f \circ \ker(f) = 0 = f
\circ 0$ so $\ker(f) = 0$ as $f$ is mono.

\subsubsection*{(2)}

Suppose that $f : X \to Y$ is both a mono- and epimorphism. As we're in an
abelian, and in particular additive category, it follows from part (1) that the
kernel and cokernel of $f$ is $0$. Moreover, as we're in an abelian category,
$f$ is the kernel of it's cokernel $Y \to 0$, and the following sequence is
exact,
\[
\begin{tikzcd}
	& 0
	\ar{r}
	& X
	\ar{r}{f}
	& Y
	\ar{r}
	& 0.
\end{tikzcd}
\] 
As $0 \circ \id_Y = 0$, $\id_Y$ factors through $\ker 0 = f$ via some $g : Y
\to X$ and we have a right-inverse $fg = \id_Y$. Moreover, this is also a
left-inverse since $fgf = (fg)f = f$ and as $f$ is monic, $gf = \id_X$.

\subsection*{Exercise 6}

We prove this our own way. First we'll state and prove some lemmas that help
us in being explicit.

\begin{lemma}
	Let $F, G : \mathcal{C} \to \mathcal{D}$ be functors, $\eta : F \Rightarrow
	G$ be a natural transformation, and $H : \mathcal{B} \to \mathcal{C}$ be a
	functor. Then we have a natural transformation
	\[
		\xi : F \circ H \Rightarrow
		G \circ H
	\]
	where
	\[
		\xi_{X} = \eta_{H(X)}.
	\]
	Similarly, if $L : \mathcal{D} \to \mathcal{E}$ is a functor, then we have
	a natural transformation
	\[
		\chi : L \circ F \Rightarrow L \circ G
	\]
	where
	\[
		\chi_{X} = L(\eta_X).
	\]
\end{lemma}
\begin{proof}
	For the first statement, let $f : X \to Y$ in $\mathcal{B}$. Then $H(f) : H(X) \to H(Y)$ in
	$\mathcal{C}$ and we have $\eta_{H(X)}, \eta_{H(Y)}$ such that the following square
	commutes,
	\[
	\begin{tikzcd}
		& F \circ H (X)
		\arrow{r}{\eta_{H(X)}}
		\arrow{d}{F \circ H (f)}
		& G \circ H (X)
		\arrow{d}{G \circ H (f)} \\
		& F \circ H (Y)
		\arrow{r}{\eta_{H(Y)}}
		& G \circ H(Y)
	\end{tikzcd}.
	\] 
	For the second statement, let $f : X \to Y$ be in $\mathcal{C}$. Then we have morphisms 
	$\eta_X, \eta_Y$ making the following square commute
	\[
	\begin{tikzcd}
		& F(X)
		\arrow{r}{\eta_{X}}
		\arrow{d}{F(f)}
		& G(X)
		\arrow{d}{G(f)} \\
		& F(Y)
		\arrow{r}{\eta_{Y}}
		& G(Y),
	\end{tikzcd}
	\]
	and as functors preserve commutative diagrams, we get
	\[
	\begin{tikzcd}
		& L \circ F(X)
		\arrow{r}{L(\eta_{X})}
		\arrow{d}{L \circ F(f)}
		& L \circ G(X)
		\arrow{d}{L \circ G(f)} \\
		& L \circ F(Y)
		\arrow{r}{L(\eta_{Y})}
		& L \circ G(Y).
	\end{tikzcd}
	\]
\end{proof}

\begin{lemma}
	Let $X \in \mathcal{C}$ and $F : \mathcal{C} \to \mathcal{D}$ be a functor.
	Then
	\[
		F \circ \Delta_{X} = \Delta_{F(X)}.
	\]
\end{lemma}
\begin{proof}
	$\Delta_X$ sends all objects to $X$ and all morphisms to $\id_X$. $F \circ
	\Delta_X$ sends all objects to $F(X)$ and all morphisms to $\id_{F(X)}$.
	The same is true of $\Delta_{F(X)}$.
\end{proof}

\begin{lemma}
	Let $I, \mathcal{C}$ and $\mathcal{D}$ be categories.
	Furthermore, let
	\[
		F : I \to \mathcal{C}, \,
		G : I \to \mathcal{D}
	\]
	be functors and 
	\[
		L : \mathcal{C} \to \mathcal{D}, \,
		R : \mathcal{D} \to \mathcal{C}, \,
	\] 
	be left and right adjoint functors via the adjugant $\Phi$. Suppose there
	is a natural transformation
	\[
		\eta : L \circ F \to G.
	\] 
	Then there is a natural transformation 
	\[
		\xi : F \to R \circ G
	\] 
	given by
	\[
		\xi_X = \Phi_{F(X), G(X)}(\eta_X).
	\] 
	Similarly, if such $\xi$ exists it implies existence of $\eta$ where
	\[
		\eta_X = \Phi^{-1}_{F(X), G(X)}(\chi_X).
	\] 
\end{lemma}
\begin{proof}
	Let $f : X \to Y$ in $I$. Then we have a commutative diagram 
	\[
	\begin{tikzcd}
		& L \circ F(X)
		\arrow{r}{\eta_{X}}
		\arrow{d}{L \circ F(f)}
		& G(X)
		\arrow{d}{G(f)} \\
		& L \circ F(Y)
		\arrow{r}{\eta_{Y}}
		& G(Y).
	\end{tikzcd}
	\] 
	Let $\Phi_{X, Y}$ denote the adjugant. Then we get morphisms
	\[
	\begin{tikzcd}[column sep = huge]
		& F(X)
		\arrow{r}{\Phi_{F(X), G(X)}(\eta_{X})}
		\arrow{d}{F(f)}
		& R \circ G(X)
		\arrow{d}{R \circ G(f)} \\
		& F(Y)
		\arrow{r}{\Phi_{F(Y), G(Y)}(\eta_{Y})}
		& R \circ G(Y),
	\end{tikzcd}
	\]
	and naturality of adjunction tells us that the square commutes. The other
	direction follows in the same way.
\end{proof}

Now let's state and prove the theorem.

\begin{theorem}
	Let $F : I \to \mathcal{C}$ be a functor, and $L : \mathcal{C} \to
	\mathcal{D}$, $R : \mathcal{D} \to \mathcal{C}$ be a left/right-adjoint
	functor pair. Then $L(\colim(F)) = \colim(L \circ F)$.
\end{theorem}
\begin{proof}
	By the definition of colimits, we have a natural transformation $\tau : F
	\Rightarrow \Delta_{\colim(F)}$, and it follows that we have a natural
	transformation $L \circ F \Rightarrow L \circ \Delta_{\colim(F)} =
	\Delta_{L(\colim(F))}$. This is true whether $L$ is a left adjoint or not.
	What remains to be shown is that $L(\colim(F))$ is initial among all objects
	$K \in \mathcal{D}$ with natural transformations $L \circ F \Rightarrow
	\Delta_K$. \\

	Suppose that $K \in \mathcal{D}$ is an object such that there is a natural
	transformation $\eta : L \circ F \Rightarrow \Delta_K$. Then we have a
	natural transformation
	\[
		\xi : F \rightarrow \Delta_{R(K)}
	\]
	given by 
	\[
		\xi_X = \Phi_{F(X), K}(\eta_X).
	\] 
	As $\colim(F)$ is the initial object in $\mathcal{C}$ with respect to this
	property, it follows that we have a unique morphism $h : \colim(F) \to
	R(K)$ such that for any $X \in I$ we have the following commutative diagram
	\[
	\begin{tikzcd}
		& F(X)
		\ar{rr}{\tau_X}
		\ar{dr}{\xi_X}
		& & \colim(F)
		\ar{dl}{h}
		\\
		& & R(K).
	\end{tikzcd}
	\] 
	By naturality of adjoints we get the following commutative diagram
	\[
	\begin{tikzcd}
		& L \circ F(X)
		\ar{rr}{L(\tau_X)}
		\ar{dr}{\Phi_{F(X), K}^{-1}(\xi_X)}
		& & L(\colim(F))
		\ar{dl}{\Phi_{F(X), K}^{-1}(h)}
		\\
		& & K
	\end{tikzcd}
	\]
	and as
	\[
		\xi_X = \Phi_{F(X), K}(\eta_X),
	\] 
	we have
	\[
		\Phi_{F(X), K}^{-1}(\xi_X) = \eta_X,
	\] 
	and the previous diagram can be simplified to
	\[
	\begin{tikzcd}
		& L \circ F(X)
		\ar{rr}{L(\tau_X)}
		\ar{dr}{\eta_X}
		& & L(\colim(F))
		\ar{dl}{\Phi_{F(X), K}^{-1}(h)}
		\\
		& & K
	\end{tikzcd}
	\] 
	whence we see that $L(\tau) : L \circ F \Rightarrow \Delta_{L(\colim(F))}$
	is initial among all constant functors with natural transformations from $L \circ F$,
	so $L(\colim(F))$ is the colimit of $L \circ F$.
\end{proof}



\subsection*{Exercise 8}

$\Z$ is a PID, so flat $\Z$-modules are the same thing as torsionfree
$\Z$-modules. $\Q$ is a torsionfree $\Z$-module whilst $\Q/\Z$ isn't. 

\subsection*{Exercise 9}

\subsubsection*{(1)}
Let $f : A \to B$ be an injective $R$-module morphism. Then as $M$ is flat $f
\otimes \id_M$ is injective, and as $N$ is flat
\[
	(f \otimes \id_M) \otimes \id_N = f \otimes (\id_M \otimes \id_N) = f \otimes \id_{M \otimes N}
\]
is injective, so $M \otimes N$ is flat. 

\subsubsection*{(2)}

We begin with a lemma.
\begin{lemma}
	Let $A, M, N$ be $\R$-modules, and $a \in A, \phi \in \Hom(M, N)$. Then
	\[
		a \otimes \phi = 0
	\]
	if and only if
	\[
		a \otimes \phi(m) = 0
	\]
	for all $m \in M$.
\end{lemma}
\begin{proof}
	An element $a \otimes \phi$ in $A \otimes \Hom(M, N)$ is $0$ if and only if
	every bilinear map out of $A \times \Hom(M, N)$ vanishes at $(a, \phi)$, so
	let's define some bilinear maps out of $A \times \Hom(M, N)$. For any $m
	\in M$, define the map 
	\[
		f_m : A \times \Hom(M, N) \to A \otimes N
	\] 
	by
	\[
		f_m : a \times \phi \mapsto a \otimes \phi(m).
	\] 
	This map is indeed bilinear as 
	\[
		f_m(a + ra', \phi) 
		= 
		(a + ra') \otimes \phi
		=
		a	\otimes \phi
		+
		r(a' \otimes \phi)
	\] 
	and
	\[
		f_m(a, \phi + r\phi')
		=
		a \otimes (\phi + r\phi')(m)
		=
		a \otimes (\phi(m) + r\phi'(m))
		=
		a \otimes \phi(m)
		+
		r(a \otimes \phi'(m)).
	\] 
	The first direction now follows,
	\[
		a \otimes \phi = 0 
		\Rightarrow
		0 = f_m(a, \phi) = a \otimes \phi(m).
	\]
	Can't prove the other direction for now.
\end{proof}

Let $f : A \to B$ be an injective $R$-module morphism, and $a \in A, \phi \in
\Hom(M, N)$ be such that $f(a) \otimes \phi = 0$. Then $f(a) \otimes \phi(m) = 0$
for all $m \in M$, and in particular 
\[
	(f \otimes \id_N) : a \otimes \phi(m)
	\mapsto 
	f(a) \otimes \phi(m) = 0
\] 
and flatness of $N$ implies that $a \otimes \phi(m) = 0$ for all $m \in M$,
whence $a \otimes \phi = 0$, and $f \otimes \id_{\Hom(M, N)}$ is injective.

\section*{List 6}

\subsection*{Exercise 2}

A submodule of a torsionfree module must be torsionfree.

\subsection*{Exercise 3}

Let $f : \Z/2\Z \to \Z/4\Z$ be given by $f(1) = 2$. Then $f$ is injective
whilst
\[
	(f \otimes \id_{(2)})
	(1 \otimes 2)
	=
	(2 \otimes 2)
	=
	(1 \otimes 4)
	=
	0.
\] 
Note that here the fact that we're tensoring by the submodule $(2)$ instead of
the full module $\Z/4\Z$ means that $(1 \otimes 2)$ is non-zero since we can't
shift the $2$ to the left.

\subsection*{Exercise 6}
\subsubsection*{(1)}

No, there is no way to factor $\id$ through $2$ in the diagram
\[
\begin{tikzcd}
	(2) \ar[hook]{r}{2}\ar{d}{\id} & \Z/4\Z \\
	(2)
\end{tikzcd}
\] 

\subsubsection*{(2)}

There is no way to factor the inclusion through multiplication by $x$
in the following diagram,
\[
\begin{tikzcd}
	\Z[x] \ar[hook]{r}{\cdot x}\ar[hook]{d}{i} & \Z[x] \\
	\Q[x].
\end{tikzcd}
\]

\subsubsection*{(3)}

Yes, $R$ is a PID since $\Q$ is a field, and $\Q(x)$ is divisible.

\subsection*{Exercise 8}

Suppose that $f_{\bullet} : C_{\bullet} \to D_{\bullet}$ is an epimorphism in
$\Ch(\mathcal{A})$. Then let $g, g' : D_n \to X$ be two morphisms in
$\mathcal{A}$ such that $gf_n = g'f_n$. These $g, g'$ can be turned into chain
maps $g_{\bullet}, g'_{\bullet} : D_{\bullet} \to X$ via 
\[
\begin{tikzcd}
	\ldots 
	\ar{r}
	& D_{n + 1}
	\ar{r}
	\ar{d}{0}
	& D_{n}
	\ar{r}
	\ar{d}{g}
	& D_{n - 1}
	\ar{r}
	\ar{d}{0}
	& \ldots \\
	\ldots 
	\ar{r}
	& 0
	\ar{r}
	& X
	\ar{r}
	& 0
	\ar{r}
	& \ldots
\end{tikzcd}
\] 
As $gf_n = g'f_n$ it immediately follows that $g_{\bullet} f_{\bullet} =
g'_{\bullet} f_{\bullet}$, and since $f_{\bullet}$ is an epimorphism, it
follows that $g_{\bullet} = g'_{\bullet}$ whence $g = g'$ and $f_n$ is an
epimorphism. \\

Now suppose instead that $f_{\bullet} : C_{\bullet} \to D_{\bullet}$ is a chain
map such that each $f_n$ is an epimorphism. Then let $g_{\bullet}, g'_{\bullet}
: D_{\bullet} \to X_{\bullet}$ be maps such that $g_{\bullet}f_{\bullet} =
g'_{\bullet}f_{\bullet}$. Two chain maps are equal if and only if they are
equal componentwise, and so for each $n$, we have $g_nf_n = g'_n f_n$.
As $f_n$ is an epimorphism, $g_n = g'_n$ and as this holds for every $n$,
$g_{\bullet} = g'_{\bullet}$ and $f_{\bullet}$ is an epimorphism.

\subsection*{Exercise 9}
The long exact sequence becomes 
\[
\begin{tikzcd}
	\ldots
	\ar{r}
	& 0
	\ar{r}
	& H_{n}(C_{\bullet})
	\ar{r}
	& 0
	\ar{r}
	& \ldots
\end{tikzcd}
\] 
whence $H_n(C_{\bullet}) = 0$.


\subsection*{Exercise 11}

First we prove exactness at $C'_n$. An element $c' \in C'_n$ is in the kernel
of $(f'_n, i_n)$ if and only if it's in $\ker(f'_n) \cap \ker(i_n)$. Suppose
$c'$ is such an element. By exactness of the top row, there is a $c'' \in
C''_{n + 1}$ which maps to $c'$. As $f'_n(c') = 0$, commutativity yields that
$\delta'_{n + 1} f''_{n + 1} (c'' ) = 0$, and so exactness tells us that there
is an element $d \in D_{n + 1}$ which maps to $f''_{n + 1}(c'') \in D''_{n +
1}$. We now have that $c' = \partial_n (f''_{n + 1})^{-1} q_{n+1}(d)$ and so 
\[
	\ker(c') \subseteq \im(\partial_n (f''_{n + 1})^{-1} q_{n+1}).
\] 
The other inclusion is immediate by exactness and commutativity. \\

For exactness at $D_n$, let $d \in \ker(\partial_n (f''_{n + 1})^{-1}
q_{n+1})$. Then by exactness, there is $c \in C_n$ such that
\[
	p_n(c) 
	=
	(f''_{n + 1})q_{n + 1}(d).
\]
It follows that $f_n(c) - d \in \ker(q_n + 1) = \im(j_{n})$, and so there is
$d' \in D'_{n}$ such that $j_{n}(d') = f_n(c) - d$, hence $d = j_n(-d') -
f_n(-c)$ and the kernel lies in the image. The other direction is immediate by
commutativity and exactness. \\

Finally, let's show exactness at $D'_n \oplus C_n$. Let $(d', c) \in D'_n
\oplus C_n$ be such that $j_n(d') = f_n(c)$. Then
\[
	p_{n}(c)
	=
	(f'')^{-1}_n q_n j_n(d'_n)
	=
	0
\]
and  there is $c' \in C'_n$ which maps to $c$. Commutativity then yields
that $f'_n(c') - d' \in \ker(j_n) = \im(\delta_{n + 1})$, and so we
have $d'' \in D''_{n + 1}$ which maps to $f'_n(c') - d'$.
But then if we let $\widetilde{c'} = \partial_{n+1}(f''_{n+1})^{-1}(d'')$
we have
\[
	f'_n(c' - \widetilde{c'}) 
	=
	f'_n(c) - f'_n(\partial_{n+1}(f''_{n+1})^{-1}(d''))
	=
	\delta_{n + 1}(d'') + d'
	f'_n(\partial_{n+1}(f''_{n+1})^{-1}(d''))
	=
	d'
\]
and we are done (as the other inclusion is trivial).


\subsection*{Exercise 12}

First of, we have long exact sequences 
\[
\begin{tikzcd}
	\ldots
	\ar{r}
	& H_{n+1}(C)
	\ar{r}
	& H_{n+1}(C'')
	\ar{r}{\delta^{C}_{n + 1}}
	& H_{n}(C')
	\ar{r}
	& H_{n}(C)
	\ar{r}
	& \ldots \\
	\ldots
	\ar{r}
	& H_{n+1}(D)
	\ar{r}
	& H_{n+1}(D'')
	\ar{r}{\delta^{D}_{n + 1}}
	& H_{n}(D')
	\ar{r}
	& H_{n}(D)
	\ar{r}
	& \ldots,
\end{tikzcd}
\] 
and by a remark in Section 4.1 of the lecture notes, the passage to long exact sequences 
is natural in the sense that we have a chain map
\[
\begin{tikzcd}
	\ldots
	\ar{r}
	& H_{n+1}(C)
	\ar{d}{H_{n+1}(f)}
	\ar{r}
	& H_{n+1}(C'')
	\ar{r}{\delta^{C}_{n + 1}}
	\ar{d}{H_{n+1}(f'')}
	& H_{n}(C')
	\ar{r}
	\ar{d}{H_{n}(f')}
	& H_{n}(C)
	\ar{r}
	\ar{d}{H_{n}(f)}
	& \ldots \\
	\ldots
	\ar{r}
	& H_{n+1}(D)
	\ar{r}
	& H_{n+1}(D'')
	\ar{r}{\delta^{D}_{n + 1}}
	& H_{n}(D')
	\ar{r}
	& H_{n}(D)
	\ar{r}
	& \ldots,
\end{tikzcd}
\]
and as it is assumed that $H_n(f'')$ is an isomorphism for every $n$, we can
apply our results from Exercise 11 from which the desired statement follows
immediately.

\subsection*{Exercise 13}

\section*{List 7}
\subsection*{Exercise 1}
We have Let $P_{\bullet} \to N$ be a projective resolution of $N$. Then
\begin{align*}
	\Tor_{n}^{R}(M, N)
	&=
	H(M \otimes_{R} P_{n}) \\
	&=
	H(P_{n} \otimes_{R} M) \\
	&=
	\Tor_{n}^{R}(N, M)
\end{align*}
where we used balancing of $\Tor$ for the last equality.

\subsection*{Exercise 4}

As $R$ is a domain, a projective resolution of $R/(a)$ is given by
\[
\begin{tikzcd}
	0
	\ar{r}
	& R
	\ar{r}{\_ \cdot a}
	& R
	\ar{r}{\pi}
	& R/(a)
	\ar{r}
	& 0.
\end{tikzcd}
\]
Applying $\Hom_{\RMod}(\_, M)$ to the deleted resolution yields
\[
\begin{tikzcd}
	0
	\ar{r}
	& \Hom_{\RMod}(R, M)
	\ar{r}{(\_ \cdot a)^{*}}
	\ar{d}{\cong}
	& \Hom_{\RMod}(R, M)
	\ar{r}
	\ar{d}{\cong}
	& 0 \\
	& M
	\ar{r}{\_ \cdot a}
	& M
\end{tikzcd}
\]
So we have 
\begin{align*}
	\Ext_{R}^{0}(R, M)
	&=
	\{m \in M : am = 0\} \\
	\Ext_{R}^{1}(R, M)
	&=
	M / aM	
\end{align*}
and $\Ext_{R}^{n}(R, M) = 0$ for $n > 1$.


\subsection*{Exercise 5}

If $M$ is flat, then $\_ \otimes_{R} M$ is an exact functor and it's $n$-th
derived functor (I.e $\Tor_{n}^{R}(M, \_)$) is $0$ for all $n > 0$. \\

Now suppose that $\Tor_{n}^{R}(M, \_) = 0$ for all $n > 0$. Let $f : A \to B$
be an injective morphism of $R$-modules. Then we have an short exact sequence 
\[
\begin{tikzcd}
	0
	\ar{r}
	& A 
	\ar{r}{f}
	& B 
	\ar{r}
	& \coker(f)
	\ar{r}
	& 0,
\end{tikzcd}
\] 
and a long exact sequence 
\[
\begin{tikzcd}[column sep=small]
	\ldots 
	\ar{r}
	& \Tor_{1}^{R}(M, \coker(f)) = 0
	\ar{r}
	& \Tor_{0}^{R}(M, A) 
	\ar{r}
	& \Tor_{0}^{R}(M, B) 
	\ar{r}
	& \Tor_{0}^{R}(M, \coker(f)) 
	\ar{r}
	& 0,
\end{tikzcd}
\] 
hence an injection
\[
	M \otimes_R A
	=
	\Tor^{R}_{0}(M, A) 
	\xrightarrow{\Tor^{R}_{0}(M)(f)}
	\Tor^{R}_{0}(M, B) 
	=
	M \otimes_R B.
\]
Now, choosing projective resolutions $P_{\bullet} \to A, Q_{\bullet} \to B$
yields the following commutative diagram
\[
\begin{tikzcd}
	M \otimes_{R} P_1
	\ar{r}
	\ar{d}{\id_M \otimes \phi_1}
	& M \otimes_{R} P_0
	\ar{r}
	\ar{d}{\id_M \otimes \phi_0}
	& M \otimes_R A
	\ar{d}{\id_M \otimes f}
	\ar{r}
	& 0
	\\
	M \otimes_{R} Q_1
	\ar{r}
	& M \otimes_{R} Q_0
	\ar{r}
	& M \otimes_R B
	\ar{r}
	& 0,
\end{tikzcd}
\] 
where $\phi_0, \phi_1$ are the maps granted by the comparison theorem. 
As $M \otimes_R$ is right exact, the rows are exact and we may insert cokernels 
as follows
\[
\begin{tikzcd}
	M \otimes_{R} P_1
	\ar{r}
	\ar{d}{\id_M \otimes \phi_1}
	& M \otimes_{R} P_0
	\ar{r}
	\ar{d}{\id_M \otimes \phi_0}
	& \coker(\id_{M} \otimes d^{P}_{1})
	\ar{r}{\cong}
	& M \otimes_R A
	\ar{d}{\id_M \otimes f}
	\\
	M \otimes_{R} Q_1
	\ar{r}
	& M \otimes_{R} Q_0
	\ar{r}
	& \coker(\id_{M} \otimes d^{Q}_{1})
	\ar{r}{\cong}
	& M \otimes_R B.
\end{tikzcd}
\]
But $\coker(\id_{M} \otimes d^{P}_{1})$ is exactly $\Tor_0^{R}(M, A)$, and the same for $B$.
Moreover, $\Tor^{R}_{0}(M)(f)$ is the map induced by $\id_{M} \otimes \phi_0$ on the cokernels,
and so we have the following two commutative diagrams
\[
\begin{tikzcd}
	M \otimes_{R} P_0
	\ar{r}
	\ar{d}{\id_M \otimes \phi_0}
	& \coker(\id_{M} \otimes d^{P}_{1})
	\ar{d}{\Tor^{R}_{0}(M)(f)}
	\ar{r}{\cong}
	& M \otimes_R A
	\\
	M \otimes_{R} Q_0
	\ar{r}
	& \coker(\id_{M} \otimes d^{Q}_{1}),
	\ar{r}{\cong}
	& M \otimes_R B.
\end{tikzcd}
\]
and
\[
\begin{tikzcd}
	M \otimes_{R} P_0
	\ar{r}
	\ar{d}{\id_M \otimes \phi_0}
	& \coker(\id_{M} \otimes d^{P}_{1})
	\ar{r}{\cong}
	& M \otimes_R A
	\ar{d}{\id_M \otimes f}
	\\
	M \otimes_{R} Q_0
	\ar{r}
	& \coker(\id_{M} \otimes d^{Q}_{1})
	\ar{r}{\cong} 
	& M \otimes_R B,
\end{tikzcd}
\]
whence surjectivity onto the cokernels yields the following commutative diagram
\[
\begin{tikzcd}
	\coker(\id_{M} \otimes d^{P}_{1})
	\ar{d}{\Tor^{R}_{0}(M)(f)}
	\ar{r}{\cong}
	& M \otimes_R A
	\ar{d}{\id_M \otimes f}
	\\
	\coker(\id_{M} \otimes d^{Q}_{1})
	\ar{r}{\cong}
	& M \otimes_R B.
\end{tikzcd}
\]
Finally, injectivity of $\Tor^{R}_0(M)(f)$ now yields injectivity of $\id_M
\otimes f$ whence $f$ is flat.


\subsection*{Exercise 6}

If $M$ is projective, then $\Hom(M, \_)$ is an exact functor, and so all $n$-th
derived functors (I.e $\Ext^{n}_{R}(M, \_)$) for $n > 0$ are $0$. \\

Now suppose that $\Ext^{n}_{R}(M, \_) = 0$ for all $n > 0$, and let 
\[
\begin{tikzcd}
0
\ar{r}
& A
\ar{r}{f}
& B
\ar{r}{g}
& C
\ar{r}
& 0
\end{tikzcd}
\] 
be a short exact sequence. As $\Ext^{n}(M, \_)$ vanishes for non-zero 
indices, the long exact sequence becomes 
\[
\begin{tikzcd}[column sep = huge]
0
\ar{r}
& \Ext^{0}_{R}(M, A)
\ar{r}{\Ext^{0}_R(M, f)}
& \Ext^{0}_{R}(M, B)
\ar{r}{\Ext^{0}_R(M, g)}
& \Ext^{0}_{R}(M, C)
\ar{r}
& 0
\end{tikzcd}
\] 
and $\Ext^{0}_{R}(M, \_)$ is an exact functor. As $\Hom_{\RMod}(M, \_)$ is
left exact, $\Ext^{0}_{R}(M, \_)$ is naturally isomorphic to $\Hom_{\RMod}(M,
\_)$, and $\Hom_{\RMod}(M, \_)$ is then also exact, whence $M$ is projective.


\subsection*{Exercise 7}

Three steps
\begin{enumerate}
	\item Every element of $\colim M_i$ comes from some $M_i$. In other words
		there is a surjection $\bigoplus M_i \to \colim M_i$: We have maps
		$\phi_j : M_j \to \colim M_i$ for all $j \in I$, hence there is a
		unique map $\phi : \bigoplus M_i \to \colim M_i$ that factors all these
		$\phi_j$. Suppose now that $X \in \RMod$ and $f, g : \colim M_i \to X$
		are such that $f\phi = g \phi$. Then $f\phi, g\phi$ are two cocones of
		$i \mapsto M_i$, and as they are the same, there is a unique map
		$\colim M_i \to X$ which they factor through. Hence both $f$ and $g$
		must be equal to this unique map, and in particular they are equal to
		each other.
	\item Show that $\colim$ is left exact, in other words preserves preserves
		kernels and 
	\item Show that $\colim$ is surjective. Automatic since colimits commute
		with colimits (?), and $\coker$ is a colimit.
\end{enumerate}

\subsection*{Exercise 8}

We begin with a lemma.
\begin{lemma}
	Let $\mathcal{A} \xrightarrow{F} \mathcal{B} \xrightarrow{U} \mathcal{C}$
	be additive functors of abelian categories, and suppose that $U$ is exact.
	Then
	\[
		L_n(U \circ F)
		\cong
		U(L_n(F)).
	\] 
\end{lemma}
\begin{proof}
		
\end{proof}

Let $M$ be an $\R$-module, $I$ be a filtered category and $A : I \to \RMod$ be
an $I$-shaped diagram. We are asked to show that 
\[
	\Tor^{R}_{n}(M, \colim(A))
	\cong
	\colim(\Tor^{R}_{n}(M, \_) \circ A).
\] 
Breaking the left hand side into factors we get
\begin{align*}
	L_{n}(M \otimes_{R} \_)(\colim(A))
	&= 
	L_{n}(M \otimes_{R} \_) \circ \colim(A) \\
	&=
	H_n \circ \HK(M \otimes_{R} \_) \circ P \circ \colim(A).
\end{align*}
We begin by showing that $P \circ \colim(A) \cong \colim(P \circ A)$ where $P :
\RMod \to \HK(\RMod)$ takes modules to (deleted) projective resolutions in the
homotopy category, and $\colim : \RMod^{I} \to \RMod$ takes $I$-shaped diagrams
in $\RMod$ to their colimit.

\subsection*{Exercise 10}
\subsubsection*{(1)}

We will show that the localization
\[
	\phi : M \times K \to S^{-1}M,
\]
with $S = R^{*}$ and 
\[
	\phi : (m, a/b) \to am/b
\] 
satisfies the universal property of the tensor product of $M$ and $K$. \\

$\phi$ is $R$-bilinear and so factors through the tensor product $\phi(m, a/b)
= h(m \otimes_{R} a/b)$. We claim that $h$ is an isomorphism. For surjectivity,
we have $m/s = h(m \otimes_R 1/s)$. For injectivity, $am/b = 0$ if and only if
$sam - 0b = 0$ for some $s \in S = R^{*}$. But if $sam = 0$ then $m \otimes_r
a/b = sam \otimes_r 1/sb = 0$, so $h$ is injective, hence an isomorphism. \\

As shown above, an element $m/s \in M$ is zero if and only if there is some $s'
\in S$ such that $m$ is $s'$ torsion. Hence $M \otimes_R K = S^{-1}M$ is $0$ if
and only if $M$ is $S = R^{*}$-torsion.

\subsubsection*{(2)}

Let $P_{\bullet} \to M$ be a projective resolution of $M$. Then
\begin{align*}
	\Tor_{i}^{R}(M, N) \otimes K
	&=
	H_{i}(P_{\bullet} \otimes N) \otimes K \\
	&= 
	\frac{\ker(d^{P}_i \otimes \id_N)}{\im(d^{P}_{i + 1} \otimes \id_N)} \otimes K,
\end{align*} 
and as $K$ is torsionfree, $K$ is flat, and the short exact sequence 
\[
	\begin{tikzcd}[column sep = small]
	0
	\ar{r}
	& \im(d^{P}_{i+1} \otimes \id_N)
	\ar{r}
	& \ker(d^{P}_{i} \otimes \id_N)
	\ar{r}
	& \frac{\ker(d^{P}_i \otimes \id_N)}{\im(d^{P}_{i + 1} \otimes \id_N)}
	\ar{r}
	& 0
	\end{tikzcd}
\] 
remains eact after being tensored with $K$ so
\[
	\begin{tikzcd}[column sep = small]
	0
	\ar{r}
	& \im(d^{P}_{i+1} \otimes \id_N) \otimes K
	\ar{r}
	& \ker(d^{P}_{i} \otimes \id_N) \otimes K
	\ar{r}
	& \frac{\ker(d^{P}_i \otimes \id_N)}{\im(d^{P}_{i + 1} \otimes \id_N)} \otimes K
	\ar{r}
	& 0
	\end{tikzcd}
\]
from which it follows that
\begin{align*}
	\frac{\ker(d^{P}_i \otimes \id_N)}{\im(d^{P}_{i + 1} \otimes \id_N)} \otimes K
	\cong
	\frac{\ker(d^{P}_i \otimes \id_N) \otimes K}{\im(d^{P}_{i + 1} \otimes \id_N) \otimes K}.
\end{align*}
Finally, again as $K$ is flat it $\_ \otimes_R K$ is exact and commutes with kernels and cokernels,
whence 
\begin{align*}
	\frac{\ker(d^{P}_i \otimes \id_N) \otimes K}{\im(d^{P}_{i + 1} \otimes \id_N) \otimes K}
	\cong
	\frac{\ker(d^{P}_i \otimes \id_N \otimes \id_K)}{\im(d^{P}_{i + 1} \otimes \id_N \otimes \id_K)}.
\end{align*}
From our calculations, it follows that 
\[
	\Tor_{i}^{R}(M, N) \otimes K
	\cong
	\Tor_{i}^{R}(M, N \otimes K),
\] 
but $N \otimes K$ is torsionfree, hence 
\[
	\Tor_{i}^{R}(M, N) \otimes K
	\cong
	\Tor_{i}^{R}(M, N \otimes K)
	=
	0,
\] 
and $\Tor_{i}^{R}(M, N)$ is torsion. 


\section*{List 8}
\subsection*{Exercise 1}
\end{document}
