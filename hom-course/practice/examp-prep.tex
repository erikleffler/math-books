\documentclass{article}
\usepackage[utf8]{inputenc}

\usepackage{rotating}
\usepackage{mathtools}
\usepackage{algpseudocode}
\usepackage{amsfonts}
\usepackage{amsmath}
\usepackage{amssymb}
\usepackage{amsthm}
\usepackage{bm}
\usepackage{listings}
\usepackage{float}
\usepackage{fancyvrb}
\usepackage{xcolor}
\usepackage{tikz-cd}

\hbadness = 10000
\vbadness = 10000

\newcommand\restr[2]{{% we make the whole thing an ordinary symbol
  \left.\kern-\nulldelimiterspace % automatically resize the bar with \right
  #1 % the function
  \vphantom{\big|} % pretend it's a little taller at normal size
  \right|_{#2} % this is the delimiter
  }}

% Default fixed font does not support bold face
\DeclareFixedFont{\ttb}{T1}{txtt}{bx}{n}{12} % for bold
\DeclareFixedFont{\ttm}{T1}{txtt}{m}{n}{12}  % for normal
% Custom colors

\usepackage{color}
\definecolor{deepblue}{rgb}{0,0,0.5}
\definecolor{deepred}{rgb}{0.6,0,0}
\definecolor{deepgreen}{rgb}{0,0.5,0}

% Python style for highlighting
\newcommand\pythonstyle{\lstset{
language=Python,
basicstyle=\ttm,
morekeywords={self},              % Add keywords here
keywordstyle=\ttb\color{deepblue},
emph={MyClass,__init__},          % Custom highlighting
emphstyle=\ttb\color{deepred},    % Custom highlighting style
stringstyle=\color{deepgreen},
frame=tb,                         % Any extra options here
showstringspaces=false
}}
\lstnewenvironment{python}[1][]
{
\pythonstyle
\lstset{#1}
}
{}

\theoremstyle{definition}

\newtheorem{theorem}{Theorem}[subsection]
\newtheorem{definition}[theorem]{Definition}
\newtheorem{corollary}[theorem]{Corollary}
\newtheorem{lemma}[theorem]{Lemma}

\newcommand{\Z}{\mathbb{Z}}
\newcommand{\Q}{\mathbb{Q}}
\newcommand{\R}{\mathbb{R}}
\newcommand{\C}{\mathbb{C}}
\newcommand{\K}{\mathbb{K}}
\renewcommand{\P}{\mathbb{P}}
\newcommand{\F}{\mathbb{F}}
\newcommand{\N}{\mathbb{N}}
\newcommand{\A}{\mathbb{A}}

\newcommand{\x}{\bm{x}}
\newcommand{\Kx}{\K[\bm{x}]}
\newcommand{\KP}[2]{\K[#1_1, #1_2, \ldots, #1_{#2}]}

\newcommand{\oo}{\mathcal{O}}
\newcommand{\osp}[1]{\oo_{\Spec\left(#1\right)}}
\newcommand{\rospu}[2]{\restr{\osp{#1}}{#2}}
\newcommand{\oop}[2]{\oo_{\P^{#1}_{#2}}}
\newcommand{\ox}{\mathcal{O}_X}

\renewcommand{\AA}[1]{\A^{#1}}
\newcommand{\An}{\A^n}
\newcommand{\Am}{\A^m}

\newcommand{\PP}[1]{\P^{#1}}
\newcommand{\Pn}{\P^n}
\newcommand{\Pm}{\P^m}

\newcommand{\Hom}{\text{Hom}}
\newcommand{\Aut}{\text{Aut}}
\newcommand{\End}{\text{End}}
\newcommand{\Iso}{\text{Iso}}

\newcommand{\lm}{\text{lm}}
\newcommand{\nr}{\text{nilrad}}
\newcommand{\Spec}{\text{Spec}}
\newcommand{\Proj}{\text{Proj}}
\newcommand{\proj}{\Proj}
\newcommand{\spec}{\Spec}
\newcommand{\codim}{\text{codim}}
\newcommand{\coker}{\text{coker}}
\newcommand{\colim}{\text{colim}}
\newcommand{\ann}{\text{ann}}
\newcommand{\im}{\text{im}}
\newcommand{\id}{\text{id}}
\newcommand{\height}{\text{height}}

\newcommand{\Tor}{\text{Tor}}
\newcommand{\Ext}{\text{Ext}}
\newcommand{\HK}{\mathbf{K}}

\newcommand{\catname}[1]{{\normalfont\textbf{#1}}}
\newcommand{\Set}{\catname{Set}}
\newcommand{\CRing}{\catname{CRing}}
\newcommand{\Ch}{\catname{Ch}}
\newcommand{\Top}{\catname{Top}}
\newcommand{\Rep}{\catname{Rep}}
\newcommand{\rep}{\catname{rep}}
\newcommand{\Mod}{\catname{Mod}}
\newcommand{\RMod}{\,_{R}\catname{Mod}}
\newcommand{\ModR}{\catname{Mod}_{R}}
\newcommand{\op}{\catname{op}}
\newcommand{\Ab}{\catname{Ab}}

\setlength{\parindent}{0pt}




\begin{document}

\section*{Problem 1}

If $I$ is an injective ideal, the S.E.S
\[
\begin{tikzcd}
	0
	\ar{r}
	& I
	\ar{r}
	& R
	\ar{r}
	& R/I
	\ar{r}
	& 0
\end{tikzcd}
\] 
splits, and $R = I \oplus R/I$.

\subsection*{(1)}
By the splitting above, $I$ is a direct summand of the free $R$-module $R$,
hence projective

\subsection*{(2)}
By the splitting above, $R/I$ is a direct summand of the free $R$-module $R$,
hence projective and in particular, flat.

\section*{Problem 2}

Let $\pi^{*}M, \pi^{*}N$ be the $R$-modules obtained from $M, N$ via
restriction of scalars along $\pi : R \to R/I$. We are asked to show that the
functor is an isomorphism on hom sets,
\[
	\pi^{*} : \Hom_{R/I}(M, N) \xrightarrow{\cong} \Hom_{R}(\pi^{*}M, \pi^{*}N).
\] 

Let $\phi : \pi^{*}M \to \pi^{*}N$ be an $R$-module morphism. As $\pi^{*}M,
\pi^{*}N$ are equal to $M, N$ as sets, we have that $\phi$ induces a funtion of
sets $\overline{\phi} : M \to N$ via 
\[
	\overline{\phi} : m \to \phi(m).
\] 
Furthermore, $\overline{\phi}$ is $R/I$-linear as 
\[
	\overline{\phi}([r]m)
	=
	\phi(\pi(r)m)
	=
	\pi(r)\phi(m)
	=
	[r]\overline{\phi}(m).
\] 
As both $\overline{\cdot}$ and $\pi^{*}$ do nothing to change the underlying
functions, they just change categories, it's immediate that they are mutually
inverse isomorphisms of abelian groups.

\section*{Problem 3}

The middle sequence is exact so it's homologies are $0$, and as the outer two
sequences have $d = 0$, taking homologies doesn't change the underlying
modules. The only thing that remains to calculate is the connecting morphism.
But we don't even need to calculate it as the long exact sequence looks like
\[
\begin{tikzcd}
	\ldots
	\ar{r}
	& 0
	\ar{r}
	& \Z/2
	\ar{r}{\partial_{n}}
	& \Z/2
	\ar{r}
	& 0
	\ar{r}
	& \ldots,
\end{tikzcd}
\] 
hence $\partial_{n}$ is an isomorphism $\Z/2 \to \Z/2$, and there is only one
such isomorphism, the identity morphism.

\section*{Problem 4}

Both $\Hom$ and $\otimes$ are addititive funtors in either argument and so the
derived functors $\Tor$ and $\Ext$ are as well.

\subsection*{(1)}
We have, 
\[
	\Ext^{n}_{\Z/3\Z}(\Z/3\Z, \Z/3\Z \oplus \Z/3\Z)
	\cong
	\Ext^{n}_{\Z/3\Z}(\Z/3\Z, \Z/3\Z)
	\oplus
	\Ext^{n}_{\Z/3\Z}(\Z/3\Z, \Z/3\Z),
\] 
but $\Z/3\Z$ is free hence projective, and so $\Ext^{n}_{\Z/3\Z}(\Z/3\Z, M) =
0$ for all $n > 0$ and any $\Z/3\Z$-module $M$. As $\Ext^{0}_{R}(M, N) \cong \Hom_{R}(M, N)$
we have that 
\[
	\Ext^{0}_{\Z/3\Z}(\Z/3\Z, \Z/3\Z)
	\cong
	\Hom_{\Z/3\Z}(\Z/3\Z, \Z/3\Z)
	\cong
	\Z/3\Z
\]
hence
\[
	\Ext^{n}_{\Z/3\Z}(\Z/3\Z, \Z/3\Z \oplus \Z/3\Z)
	=
	\begin{cases}
		\Z/3\Z \oplus \Z/3\Z \text{ if } n = 0, \\	
		0 \text{ else.}
	\end{cases}
\] 

\subsection*{(2)}

We have
\[
	\Tor_{n}^{\Z}(\Z \oplus (\Z/4), \Q)
	\cong
	\Tor_{n}^{\Z}(\Z, \Q)
	\oplus
	\Tor_{n}^{\Z}(\Z/4, \Q).
\] 
As $\Z$ is projective,
\[
	\Tor_{n}^{\Z}(\Z, \Q)
	\cong
	\begin{cases}
		\Z \otimes_{\Z} \Q \cong \Q \text{ if } n = 0, \\	
		0 \text{ else.}
	\end{cases}
\] 
Meanwhile, a (deleted) projective resolution $P_{\bullet} \to \Z/4$
is given by
\[
\begin{tikzcd}
	\ldots
	\ar{r}
	& 0
	\ar{r}
	& \Z
	\ar{r}{4}
	& \Z
	\ar{r}
	& 0,
\end{tikzcd}
\] 
and tensoring with $\Q$ we get
\[
\begin{tikzcd}
	\ldots
	\ar{r}
	& 0
	\ar{r}
	& \Z \otimes_{\Z} \Q/\Z \cong \Q/\Z
	\ar{r}{4}
	& \Z \otimes_{\Z} \Q\Z \cong \Q/\Z
	\ar{r}
	& 0.
\end{tikzcd}
\]
Now, if $a/b \in \Q/\Z$, then $a/4b \in \Q/\Z$ whence multplication by $4 :
\Q/\Z \to \Q/\Z$ is surjective. It's not injective however as $4a/b \in \Z
\Leftrightarrow b | 4$, hence it has kernel generated by $1/4\Q/\Z$. We get
\[
	\Tor_{0}^{\Z}(\Z/4, \Q/\Z) \cong \frac{\ker(0_{\Q/\Z})}{\im(4_{\Q/\Z})} = 0,
\] 
and
\[
	\Tor_{1}^{\Z}(\Z/4, \Q/\Z) \cong \frac{\ker(4_{\Q/\Z})}{\im(0_{\Q/\Z})} = 1/4\Q/\Z,
\]
Thus
\[
	\Tor_{n}^{\Z}(\Z \oplus (\Z/4), \Q)
	\cong
	\begin{cases}
		\Q \text{ if } n = 0, \\
		1/4\Q/\Z \text{ if } n = 1, \\
		0 \text{ else.}
	\end{cases}
\] 

\subsection*{(3)}

A (deleted) projective resolution of $\Q[x, y]/(x)$ is given by
\[
\begin{tikzcd}
	\ldots
	\ar{r}
	& 0
	\ar{r}
	& \Q[x, y]
	\ar{r}{x}
	& \Q[x, y]
	\ar{r}{0}
	& 0
\end{tikzcd}
\] 
afterwhich applying $\Hom(\_, \Q[x, y]/(y))$ yields
\[
\begin{tikzcd}
	0
	\ar{r}
	& \Hom_{\Q[x, y]}(\Q[x, y], \Q[x, y]/(y))
	\ar{r}{x^{*}}
	& \Hom_{\Q[x, y]}(\Q[x, y], \Q[x, y]/(y))
	\ar{r}{0}
	& 0
	\ar{r}
	& \ldots
\end{tikzcd}
\]
Now, for any ring $R$ and $R$-module $M$ we have that the functor $\Hom_{R}(R, \_)$
naturaly isomorphic to the identity functor via
\[
	\Hom_{R}(R, M) \ni \phi \mapsto \phi(1) \in M,
\]
and
\[
	\Hom(\Hom_{R}(R, M), \Hom_{R}(R, N)) \ni f \mapsto (m \mapsto (f(1 \mapsto m))(1) \in \Hom(M, N).
\] 
Applying this to our resolution yields
\[
\begin{tikzcd}
	0
	\ar{r}
	& \Q[x, y]/(y)
	\ar{r}{x}
	& \Q[x, y]/(y)
	\ar{r}{0}
	& 0
	\ar{r}
	& \ldots,
\end{tikzcd}
\]
afterwhich we can read homology groups according to
\[
	\Ext_{\Q[x, y]}(\Q[x, y]/(x), \Q[x, y]/(y))
	\cong
	\begin{cases}
		0 \text{ if } i = 1, \\
		(\Q[x, y]/(y))/(x\Q[x, y]/(y)) \cong \Q \text{ if } i = 1, \\
		0 \text{ else.}
	\end{cases}
\] 

\section*{Problem 5}

The short exact sequence 
\[
\begin{tikzcd}
	0
	\ar{r}
	& I
	\ar{r}
	& R
	\ar{r}
	& R/I
	\ar{r}
	& 0,
\end{tikzcd}
\] 
induces a long exact sequence 
\[
\begin{tikzcd}[column sep = small]
	0
	\ar{r}
	& \Hom_{R}(R/I, N)
	\ar{r}
	& \Hom_{R}(R, N)
	\ar{r}
	& \Hom_{R}(I, N)
	\ar{r}
	& \Ext^{1}_{R}(R/I, N) \cong 0
	\ar{r}
	& \ldots,
\end{tikzcd}
\]	
and in particular the restriction map $\Hom_{R}(R, N) \to \Hom_{R}(I, N)$ is
surjective. As this is true for all left ideals $I$, it follows from Baer's
Criterion that $N$ is injective.

\section*{Problem 6}
\subsection*{(1)}

If $Q_{\bullet} \xrightarrow{g_0} M$ is a projective resolution of $M$,
it follows that
\[
	\begin{tikzcd}
		Q_1
		\ar{r}
		& Q_0
		\ar{r}{f_n g_0}
		& P_n
		\ar{r}
		& \ldots
		\ar{r}
		& P_0
	\end{tikzcd}
\] 
is a projective resolution of $N$, afterwhich the desired result is immediate.

\subsection*{(2)}

Let $M$ be an arbitrary $R$-module and $P_{\bullet} \to M$ a projective
resolution. Then by part (1), $L_{k + n}(M) = L_{k}(P_n) = 0$ for all $n > 0$.

\section*{Problem 7}

\subsection*{(1)}

As we're in a PID, multiplication by $r$ is injective, and so the 
only non zero homology is given by
\begin{align*}
	H_0(K(r)) &= R/rR.
\end{align*} 

\subsection*{(2)}

By the Kunneth formula, we have a split exact sequence 
\[
\begin{tikzcd}[column sep = small]
	0
	\ar{r}
	& \bigoplus_{i + j = n} H_i(K(r)) \otimes H_j(C)
	\ar{r}
	& H_n(K(r) \otimes C)
	\ar{r}
	& \bigoplus_{i + j = n - 1} \Tor_{1}(H_i(K(r), H_j(C))
	\ar{r}
	& 0
\end{tikzcd}
\] 
Furthermore, as $H_i(K(r)) = 0$ for $i \not = 0$, the sequence becomes
\[
\begin{tikzcd}[column sep = small]
	0
	\ar{r}
	& 
	R/rR \otimes H_n(C)
	\ar{r}
	& H_n(K(r) \otimes C)
	\ar{r}
	& 
	\Tor_{1}(R/rR, H_n(C))
	\ar{r}
	& 0
\end{tikzcd}
\]
Now whenever $R$ is a pid and $M$ is an $R$-module,
\[
	\Tor_1^{R}(R/rR, M)
	=
	M_r
	=
	\{m \in M : rm = 0\},
\] 
and as $r$ is injective on $H_n(C)$ by hypothesis, $\Tor_1^{R}(R/rR, M = 0)$.
Finally our desired result now follows from
\[
	R/rR \otimes H_n(C)
	\cong
	H_n(C)/rH_n(C).
\] 

\section*{Problem 8}

\subsection*{(1)}
We can show something even stronger, that 
if 
\[
	\begin{tikzcd}
	& \Hom(C, D)
	\ar{r}{g^{*}}
	& \Hom(B, D)
	\ar{r}{f^{*}}
	& \Hom(A, D)
	\end{tikzcd}
\]
is exact for every $R$-module $D$ (a weaker hypothesis as we do not require
$g^{*}$ injective), then
\[
	\begin{tikzcd}
	& A
	\ar{r}{f}
	& B
	\ar{r}{g}
	& C
	\end{tikzcd}
\] 
is exact. To see this, let $\pi : B \to \coker(f)$ be the cokernel of $f$
and set $D = \coker(f)$. Then $f^{*}(\pi) = \pi f = 0$ so there is $\sigma$...
yada yada did already.

\subsection*{(2)}

As $F$ is a left-adjoint, $F$ preserves colimits, and in particular 
cokernels. The desired result now follows from the followin lemma.

\begin{lemma}
	Let $F$ be an additve functor which preserves cokernels. Then $F$ is right
	exact.
\end{lemma}
\begin{proof}
	Let 
	\[
	\begin{tikzcd}
	A
	\ar{r}{f}
	& B
	\ar{r}{g}
	& C
	\ar{r}
	& 0
	\end{tikzcd}
	\] 
	be an exact sequence. Then $\coker(g) = 0$ and so $\coker(F(g)) =
	F(\coker(g)) = 0$ whence $F(g)$ is surjective. Moreover, $F(f) F(g) = F(fg)
	=F(0) = 0$. It remains to show that $\ker(F(g)) \subseteq \im(F(f)) =
	\ker(\coker(F(f))$. But $\coker(F(f)) = F(\coker(f)) = F(g)$ and the the
	lemma statement follows.
\end{proof}


\end{document}

