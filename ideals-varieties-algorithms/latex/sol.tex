\documentclass{article}
\usepackage[utf8]{inputenc}

\usepackage{mathtools}
\usepackage{algpseudocode}
\usepackage{amsfonts}
\usepackage{amsmath}
\usepackage{amssymb}
\usepackage{amsthm}
\usepackage{bm}
\usepackage{listings}
\usepackage{float}
\usepackage{fancyvrb}
\usepackage{xcolor}
\usepackage{tikz-cd}

\hbadness = 10000
\vbadness = 10000

\newcommand\restr[2]{{% we make the whole thing an ordinary symbol
  \left.\kern-\nulldelimiterspace % automatically resize the bar with \right
  #1 % the function
  \vphantom{\big|} % pretend it's a little taller at normal size
  \right|_{#2} % this is the delimiter
  }}

% Default fixed font does not support bold face
\DeclareFixedFont{\ttb}{T1}{txtt}{bx}{n}{12} % for bold
\DeclareFixedFont{\ttm}{T1}{txtt}{m}{n}{12}  % for normal
% Custom colors

\usepackage{color}
\definecolor{deepblue}{rgb}{0,0,0.5}
\definecolor{deepred}{rgb}{0.6,0,0}
\definecolor{deepgreen}{rgb}{0,0.5,0}

% Python style for highlighting
\newcommand\pythonstyle{\lstset{
language=Python,
basicstyle=\ttm,
morekeywords={self},              % Add keywords here
keywordstyle=\ttb\color{deepblue},
emph={MyClass,__init__},          % Custom highlighting
emphstyle=\ttb\color{deepred},    % Custom highlighting style
stringstyle=\color{deepgreen},
frame=tb,                         % Any extra options here
showstringspaces=false
}}

\lstnewenvironment{python}[1][]
{
\pythonstyle
\lstset{#1}
}
{}

\theoremstyle{definition}

\newtheorem{theorem}{Theorem}[section]
\newtheorem{definition}[theorem]{Definition}
\newtheorem{corollary}[theorem]{Corollary}
\newtheorem{lemma}[theorem]{Lemma}

\newcommand{\Z}{\mathbb{Z}}
\newcommand{\Q}{\mathbb{Q}}
\newcommand{\R}{\mathbb{R}}
\newcommand{\C}{\mathbb{C}}
\newcommand{\K}{\mathbb{K}}
\renewcommand{\P}{\mathbb{P}}
\newcommand{\F}{\mathbb{F}}
\newcommand{\N}{\mathbb{N}}
\newcommand{\A}{\mathbb{A}}

\newcommand{\x}{\bm{x}}
\newcommand{\Kx}{\K[\bm{x}]}
\newcommand{\KP}[2]{\K[#1_1, #1_2, \ldots, #1_{#2}]}

\renewcommand{\AA}[1]{\A^{#1}}
\newcommand{\An}{\A^n}
\newcommand{\Am}{\A^m}

\newcommand{\PP}[1]{\P^{#1}}
\newcommand{\Pn}{\P^n}
\newcommand{\Pm}{\P^m}

\newcommand{\Hom}{\text{Hom}}
\newcommand{\Aut}{\text{Aut}}
\newcommand{\End}{\text{End}}
\newcommand{\Iso}{\text{Iso}}

\newcommand{\HF}{\text{HF}}
\newcommand{\HS}{\text{HS}}
\newcommand{\HP}{\text{HP}}


\newcommand{\lm}{\text{lm}}
\newcommand{\nr}{\text{nilrad}}
\newcommand{\spec}{\text{spec}}
\newcommand{\codim}{\text{codim}}
\newcommand{\ann}{\text{ann}}
\newcommand{\im}{\text{im}}
\newcommand{\id}{\text{id}}
\newcommand{\height}{\text{height}}

\newcommand{\lt}{\text{lt}}
\newcommand{\Lt}{\text{Lt}}
\newcommand{\Lm}{\text{Lm}}

\newcommand{\catname}[1]{{\normalfont\textbf{#1}}}
\newcommand{\Set}{\catname{Set}}
\newcommand{\CRing}{\catname{CRing}}
\newcommand{\Top}{\catname{Top}}
\newcommand{\op}{\catname{op}}

\setlength{\parindent}{0pt}




\begin{document}

\subsection*{Ex 2.1.5}

\subsubsection*{(a)}

Picking two non-negative integers $a, b$ such that their sum $a + b \leq m$ is
the same as picking three non-negative integers $a + b + c$ such that the sum
$a + b + c = m$. To see this, just note that $c$ is completely determined by
$a, b$. I.e the amount of monomials $x^ay^b$ with total degree $\leq m$, is the
same as the amount of ordered $3$-integer partitions of $m$, namely
$\binom{m + 2}{2}$.

\subsubsection*{(b)}

The amount of $f, g$-monomials $f(t)^a g(t)^b$ with $a, b \leq m$ is given by
$\binom{m+2}{2}$ by the previous exercise. These will all be polynomials in
$\K[t]$ of degree $\leq nm$. The $\K$-subspace of $\K[t]$ consisting of
polynomials of degree $\leq nm$ has dimension $nm + 1$, and since we pick $m$
large enough that $(m + 2)(m + 1) / 2 > nm + 1$, it follows that the $f,
g$-monomials of degree $\leq m$ for such large enough $m$ are linearly
dependent.

\subsubsection*{(c)}

If we pick $m$ large enough as described in (b), the resulting linear
dependence on $f, g$-monomials can be seen as an algebraic dependence $F$ on
$f, g$.

\subsubsection*{(d)}

There are $\binom{k + m}{k}$ $f_1, \ldots, f_k$-monomials of degree $\leq m$,
whilst the there are $\binom{nm + k - 1}{k - 1}$ $t_1, \ldots, t_{k -
1}$-monomials of degree $\leq nm$. Given any $n$, for large enough $m$ we have
$\binom{k + m}{k} > \binom{nm + k - 1}{k - 1}$, since the former product has
$k$ factors whilst the latter have $k - 1$ factors.


\end{document}
