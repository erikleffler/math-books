\documentclass{article}
\usepackage[utf8]{inputenc}

\usepackage{mathtools}
\usepackage{algpseudocode}
\usepackage{amsfonts}
\usepackage{amsmath}
\usepackage{amssymb}
\usepackage{amsthm}
\usepackage{bm}
\usepackage{listings}
\usepackage{float}
\usepackage{fancyvrb}
\usepackage{xcolor}
\usepackage{tikz-cd}

\hbadness = 10000
\vbadness = 10000

\newcommand\restr[2]{{% we make the whole thing an ordinary symbol
  \left.\kern-\nulldelimiterspace % automatically resize the bar with \right
  #1 % the function
  \vphantom{\big|} % pretend it's a little taller at normal size
  \right|_{#2} % this is the delimiter
  }}

% Default fixed font does not support bold face
\DeclareFixedFont{\ttb}{T1}{txtt}{bx}{n}{12} % for bold
\DeclareFixedFont{\ttm}{T1}{txtt}{m}{n}{12}  % for normal
% Custom colors

\usepackage{color}
\definecolor{deepblue}{rgb}{0,0,0.5}
\definecolor{deepred}{rgb}{0.6,0,0}
\definecolor{deepgreen}{rgb}{0,0.5,0}

% Python style for highlighting
\newcommand\pythonstyle{\lstset{
language=Python,
basicstyle=\ttm,
morekeywords={self},              % Add keywords here
keywordstyle=\ttb\color{deepblue},
emph={MyClass,__init__},          % Custom highlighting
emphstyle=\ttb\color{deepred},    % Custom highlighting style
stringstyle=\color{deepgreen},
frame=tb,                         % Any extra options here
showstringspaces=false
}}

\lstnewenvironment{python}[1][]
{
\pythonstyle
\lstset{#1}
}
{}

\theoremstyle{definition}

\newtheorem{theorem}{Theorem}[section]
\newtheorem{definition}[theorem]{Definition}
\newtheorem{corollary}[theorem]{Corollary}
\newtheorem{lemma}[theorem]{Lemma}

\newcommand{\Z}{\mathbb{Z}}
\newcommand{\Q}{\mathbb{Q}}
\newcommand{\R}{\mathbb{R}}
\newcommand{\C}{\mathbb{C}}
\newcommand{\K}{\mathbb{K}}
\newcommand{\F}{\mathbb{F}}
\newcommand{\N}{\mathbb{N}}
\newcommand{\A}{\mathbb{A}}

\newcommand{\x}{\bm{x}}
\newcommand{\Kx}{\K[\bm{x}]}
\newcommand{\An}{\A^n}
\newcommand{\Am}{\A^m}

\newcommand{\Hom}{\text{Hom}}
\newcommand{\Aut}{\text{Aut}}
\newcommand{\End}{\text{End}}
\newcommand{\Iso}{\text{Iso}}


\newcommand{\lm}{\text{lm}}
\newcommand{\nr}{\text{nilrad}}
\newcommand{\spec}{\text{spec}}
\newcommand{\codim}{\text{codim}}
\newcommand{\ann}{\text{ann}}
\newcommand{\im}{\text{im}}
\newcommand{\id}{\text{id}}

\newcommand{\catname}[1]{{\normalfont\textbf{#1}}}
\newcommand{\Set}{\catname{Set}}
\newcommand{\CRing}{\catname{CRing}}
\newcommand{\Top}{\catname{Top}}
\newcommand{\op}{\catname{op}}

\setlength{\parindent}{0pt}
\begin{document}

\section*{Chapter 5}

\subsection*{Exercise 5.1}

We have 
\begin{align*}
	(\alpha^{2} + \alpha + 1)(\alpha^{2} + \alpha)
	&=
	(\alpha^{3} + \alpha^{2} + \alpha)(\alpha + 1) \\
	&=
	(\alpha^{3} + \alpha^{2} + \alpha + 2)(\alpha + 1)
	-
	2(\alpha + 1) \\
	&=
	2\alpha + 2,
\end{align*}
and
\begin{align*}
	(\alpha - 1) (a\alpha^{2} + b\alpha + c)
	&=
	a\alpha^{3} + (b - a)\alpha^{2} + (c - b)\alpha - c,
\end{align*}
\begin{align*}
	b - a = a &\Leftrightarrow b = 2a, \\
	c - b = a &\Leftrightarrow c = 3a, \\
	-c = 2a + 1,
\end{align*}
whence $a = -1/5$, and if we factor out $a$, we get
\begin{align*}
	a(\alpha - 1) (\alpha^{2} + 2\alpha + 3)
	&=
	a(\alpha^{3} + 2\alpha^{2} + 3 \alpha - \alpha^{2} - 2 \alpha  - 3) \\
	&=
	a(\alpha^{3} + \alpha^{2} + \alpha - 3) \\
	&=
	a(-5) \\
	&=
	1
\end{align*}
so
\[
	(\alpha - 1)^{-1} = \frac{-1}{5} \alpha^{2} + \frac{-2}{5} \alpha + \frac{-3}{5}.
\] 

\subsection*{Exercise 5.2} 

We have that $[E:F] = [F(\alpha):F(\alpha^{2})][F(\alpha^{2}):F]$ by Prop
5.1.2, so $[F(\alpha):F(\alpha^{2})]$ must be odd. But $X^{2} - \alpha^{2} \in
F(\alpha^{2})[X]$ has $\alpha$ as a root, so $[F(\alpha):F(\alpha^{2})]$ is
both odd and less than or equal to $2$, hence equal to $1$ and $E = F(\alpha) =
F(\alpha^{2})$.

\subsection*{Exercise 5.3} 

Let $g_1, g_2 \in F(\alpha)[X]$ be such that $g_1g_2 = g$. Then at least one of
$g_1(\beta) = 0$ or $g_2(\beta) = 0$. Suppose $g_1(\beta) = 0$. Then
$[F(\alpha, \beta): F(\alpha)] = \deg(g_1)$. But
\begin{align*}
	\deg(g_1)\deg(f)
	&=
	[F(\alpha, \beta): F(\alpha)][F(\alpha): F] \\
	&=
	[F(\alpha, \beta): F] \\
	&=
	[F(\alpha, \beta): F(\beta)][F(\beta): F] \\
	&=
	[F(\alpha, \beta): F(\beta)]\deg(g),
\end{align*}
and since $(\deg(g), \deg(f)) = 1$, we must have $\deg(g) | \deg(g_1)$. But
$\deg(g_1) \leq \deg(g)$ since $g_1 | g$, so $\deg(g_1) = \deg(g)$ and $g =
g_1$. Hence $g$ is irreducible in $F(\alpha)[X]$.

\subsection*{Exercise 5.4} 

Let $Q(\alpha) \supsetneq L \supsetneq Q$, and $f = X^{4} - 2$ be the minimal
polynomial of $\alpha$ in $Q$. It follows from Prop 5.1.2 that $[L:Q] =  2$.
Then let $g_L$ be the minimal polynomial of $\alpha$ in $L$. Then $\deg g_L =
2$, $g_L | f$, and since $L[X]$ is a UFD, we have that $f = (x - \beta_1)(x -
\beta_2)g_L$ for some $\beta_1, \beta_2 \in Q(\alpha)$. Furthermore, since $L
\subset R$ is real, we must have $g_L = (x - \alpha)(x + \alpha)$, since the
other roots of $f$ are $\pm i \alpha$, and no other combination of the roots of
$f$ yields a real polynomial. We have $g_L = x^{2} - \sqrt{2}$, hence $L
\supseteq Q(\sqrt{2})$, but since $[Q(\alpha):Q(\sqrt{2})] = [Q(\sqrt{2}): Q] =
2$, we have $L = Q(\sqrt{2})$ and this is the only field which lies strictly
between $Q$ and $Q(\alpha)$.

\subsection*{Exercise 5.5} 

Let $\alpha$ be a root of $f(X) = X^{6} + X^{3} + 1$. Then $f(X)(X^{3} - 1) =
X^{9} - 1$, so $\alpha$ is a root of $g(X) = X^{9} - 1$ as well. In other
words, $\alpha$ is a $9$-th root of unity which isn't a $3$-rd root of unity. \\

Any field homomorphism which has a domain which contains $Q$ must fix $Q$,
hence any $\sigma: Q(\alpha) \to C$ can be seen as an embedding of $Q(\alpha)$
over $Q$ into $C$. It the follows from Proposition 2.7 that the number of such
$\sigma$ is $6$, since there are $6$ $9$-th roots of unity which aren't $3$-rd
roots of unity.

\subsection*{Exercise 5.6} 
First of, we have $\alpha = \sqrt{2} + \sqrt{3} \in Q(\sqrt{2})(\sqrt{3})$, so
$\alpha$ has at most degree $4$. Moreover, we have 
\[
	\alpha \frac{\sqrt{3} - \sqrt{2}}{5} = 1,
\] 
so
\[
	\sqrt{2} = \frac{\alpha - 5\alpha^{-1}}{2},\ 
	\sqrt{3} = \frac{\alpha + 5\alpha^{-1}}{2},\ 
\] 
and $\sqrt{2}, \sqrt{3} \in Q(\alpha)$. Finally, note that $\sqrt{2} \not \in
Q(\sqrt{3})$ since $(a\sqrt{3} + b)^{2} = 9a^{2} + 2\sqrt{3}ab + b^{2}$ can
never equal $2$ for rational $a, b$. Indeed, $\sqrt{3}$ is irrational, so we'd
need $b = 0$ (as $a = 0$ isn't an option), but then $a = \sqrt{9/2} =
3/\sqrt{2}$ which is also irrational. It follows that $Q(\sqrt{2} + \sqrt{3}) =
Q(\sqrt{2})(\sqrt{3}) \supsetneq Q(\sqrt{3})$ has degree $4$.

\subsection*{Exercise 5.7}

We have $[EF:k] = [EF:E][E:k]$, so we need to prove that $[EF:E] \leq [F:k]$
with whenever $([F:k], [E:k]) = 1$. Let $a_1, \ldots, a_n$ be a $k$-basis for
$F$. Then since $E \supseteq k$, we have that $a_1, \ldots, a_n$ spans $EF$ in
$E$, which shows $[EF:E] \leq [F:k]$. Now consider the case when $([F:k],
[E:k]) = 1$. Then since $[EF:k] = [EF:F][F:k]$, we have that both $[F:k]$ and
$[E:k]$ divide $[EF:k]$, and the equlity follows.

\subsection*{Exercise 5.8}

Let $f = g_1 g_2 \ldots g_m$ be a decomposition into irreducible polynomials,
and $K_i$ be the splitting field of $g_i$ in $K_{i - 1}$ where $K_0 = k, K_m =
K$. Then $[K_{i + 1} : K_{i}]$
Then $K$

We proceed by induction on $n$. If $n = 1$, then the statement is clear. Now
suppose it holds for all degrees $< n$. 

Let $\alpha$ be a root of $f$, and 
$\hat{f} = f/(X - \alpha) \in K[X]$. Then $\deg \hat{f} = n - 1$, and 
$\deg$

Let $\alpha_1, \alpha_2, \ldots, \alpha_n$ be the roots of $f$ in $K$. Define
$K_0 = k$, and inductively $K_{i + 1} = K_{i}(\alpha_{i + 1})$. Let $g_{i + 1}$
be the minimal polynomial of $\alpha_{i + 1}$ in $K_i$, and $d_i = \deg g_{i +
1}$. Then $[K:k] = \prod d_i$, and we are done if we can show that $\prod d_i |
n!$. First note that $d_1 \leq n$ since $g_1 | f$. Moreover, if


Let $\alpha_1, \alpha_2, \ldots, \alpha_n$ be the roots of $f$ in $K$, and set
$\hat{f} = f/(X - \alpha_1)$. 

Then $\hat{K} = k(\alpha_2, \alpha_3, \ldots,
\alpha_n)$ is a splitting field of $\hat{f}$. Since $\deg \hat{f} = n - 1$
it follows by our inductive hypothesis that $[\hat{K} : k] | (n - 1)!$. Moreover,
the minimal polynomial of $\alpha_1$ in $\hat{K}$ divides 

Let $g_1$ be the minimal polynomial of
$\alpha_1$ in $K_0 = K$, and let $K_1 = K_0(\alpha_1)$. Then $[K_1: K_0] = \deg
g_1$, and $\deg g_1 | n!$ since $g_1 | f$. Let $f_1 = f / (X - \alpha_1)$. 

,\\
and $g$ be its minimal polynomial.
Then $g | f$ and so $\deg g < n \Rightarrow \deg g | n!$. It follows
that $[]$
Let $k^{a}$ be an algebraic closure of $K$. Then $K$ contains all the roots of
$f$ in $k^{a}$, call them $\alpha_1, \alpha_2, \ldots, \alpha_n$. Hence
\[
	k(\alpha_1, \alpha_2, \ldots, \alpha_n) \subseteq K.
\] 
For any $\alpha_i$, we have that $f$ is it's minimal polynomial and 
$[k(\alpha_i): k] = n$
 
\end{document}
