\documentclass{article}
\usepackage[utf8]{inputenc}

\usepackage{rotating}
\usepackage{mathtools}
\usepackage{algpseudocode}
\usepackage{amsfonts}
\usepackage{amsmath}
\usepackage{amssymb}
\usepackage{amsthm}
\usepackage{bm}
\usepackage{listings}
\usepackage{float}
\usepackage{fancyvrb}
\usepackage{xcolor}
\usepackage{tikz-cd}

\hbadness = 10000
\vbadness = 10000

\newcommand\restr[2]{{% we make the whole thing an ordinary symbol
  \left.\kern-\nulldelimiterspace % automatically resize the bar with \right
  #1 % the function
  \vphantom{\big|} % pretend it's a little taller at normal size
  \right|_{#2} % this is the delimiter
  }}

% Default fixed font does not support bold face
\DeclareFixedFont{\ttb}{T1}{txtt}{bx}{n}{12} % for bold
\DeclareFixedFont{\ttm}{T1}{txtt}{m}{n}{12}  % for normal
% Custom colors

\usepackage{color}
\definecolor{deepblue}{rgb}{0,0,0.5}
\definecolor{deepred}{rgb}{0.6,0,0}
\definecolor{deepgreen}{rgb}{0,0.5,0}

% Python style for highlighting
\newcommand\pythonstyle{\lstset{
language=Python,
basicstyle=\ttm,
morekeywords={self},              % Add keywords here
keywordstyle=\ttb\color{deepblue},
emph={MyClass,__init__},          % Custom highlighting
emphstyle=\ttb\color{deepred},    % Custom highlighting style
stringstyle=\color{deepgreen},
frame=tb,                         % Any extra options here
showstringspaces=false
}}

\lstnewenvironment{python}[1][]
{
\pythonstyle
\lstset{#1}
}
{}

\theoremstyle{definition}

\newtheorem{theorem}{Theorem}[section]
\newtheorem{definition}[theorem]{Definition}
\newtheorem{corollary}[theorem]{Corollary}
\newtheorem{lemma}[theorem]{Lemma}

\newcommand{\Z}{\mathbb{Z}}
\newcommand{\Q}{\mathbb{Q}}
\newcommand{\R}{\mathbb{R}}
\newcommand{\C}{\mathbb{C}}
\newcommand{\K}{\mathbb{K}}
\renewcommand{\P}{\mathbb{P}}
\newcommand{\F}{\mathbb{F}}
\newcommand{\N}{\mathbb{N}}
\newcommand{\A}{\mathbb{A}}


\newcommand{\x}{\bm{x}}
\newcommand{\Kx}{\K[\bm{x}]}
\newcommand{\KP}[2]{\K[#1_1, #1_2, \ldots, #1_{#2}]}

\newcommand{\oo}{\mathcal{O}}
\newcommand{\osp}[1]{\oo_{\Spec\left(#1\right)}}
\newcommand{\rospu}[2]{\restr{\osp{#1}}{#2}}
\newcommand{\oop}[2]{\oo_{\P^{#1}_{#2}}}
\newcommand{\ox}{\mathcal{O}_X}

\renewcommand{\AA}[1]{\A^{#1}}
\newcommand{\An}{\A^n}
\newcommand{\Am}{\A^m}

\newcommand{\PP}[1]{\P^{#1}}
\newcommand{\Pn}{\P^n}
\newcommand{\Pm}{\P^m}

\newcommand{\Hom}{\text{Hom}}
\newcommand{\Aut}{\text{Aut}}
\newcommand{\End}{\text{End}}
\newcommand{\Iso}{\text{Iso}}
\newcommand{\Mor}{\text{Mor}}

\newcommand{\lm}{\text{lm}}
\newcommand{\nr}{\text{nilrad}}
\newcommand{\Spec}{\text{Spec}}
\newcommand{\Proj}{\text{Proj}}
\newcommand{\proj}{\Proj}
\newcommand{\spec}{\Spec}
\newcommand{\codim}{\text{codim}}
\newcommand{\ann}{\text{ann}}
\newcommand{\im}{\text{im}}
\newcommand{\id}{\text{id}}
\newcommand{\height}{\text{height}}
\newcommand{\rad}{\text{rad}}

\newcommand{\catname}[1]{{\normalfont\textbf{#1}}}
\newcommand{\Set}{\catname{Set}}
\newcommand{\CRing}{\catname{CRing}}
\newcommand{\Top}{\catname{Top}}
\newcommand{\op}{\catname{op}}

\setlength{\parindent}{0pt}




\begin{document}

\section*{Ch 1}

\subsection*{1.1}

Suppose that $ab$ maps to $1$ in $A/I$. Then there is some nilpotent element
$r$ with $r^n = 0$ such that $ab = 1 - r$. It follows that 
\[
	ab(1 + r + r^{2} + \ldots + r^{n - 1})
	=
	(1 - r)(1 + r + r^{2} + \ldots + r^{n - 1})
	=
	1 - r^{n}
	=
	1
\] 
hence $a$ has inverse $b(1 + r + \ldots + r^{n - 1})$.

\subsection*{1.2}

Suppose that $I_1 \times \ldots \times I_n$ is a prime ideal of $A_1 \times
\ldots \times A_n$. Then
\[
	(A_1 \times \ldots \times A_n) / (I_1 \times \ldots \times I_n)
	\cong 
	A_1/I_1 \times \ldots \times A_n / I_n
\]
is a domain, and as non-trivial cartesian products always have zero divisors,
we must have $A_i = I_i$ for all but one $i = j$. It follows that 
\[
	(A_1 \times \ldots \times A_n) / (I_1 \times \ldots \times I_n)
	\cong 
	A_j / I_j
\]
is a domain and so $I_j$ is prime as well. \\

The other direction is imediate.

\subsection*{1.3}
\subsubsection*{(a)}

Let $\mathfrak{m}$ be maximal in $B$. Then 
\[
	A/f^{-1}(\mathfrak{m})
	\cong 
	B/\mathfrak{m}
\] 
is a field so $\mathfrak{m}$ is maximal in $A$. Hence if $a \in \rad(A) \subset
f^{-1}(\mathfrak{m})$ then $f(a) \in \mathfrak{m}$, and as $\mathfrak{m}$ is an
arbitrary maximal ideal of $B$, $f(a) \in \rad(B)$. \\

An example where the inclusion is strict is given by $\Z \to \Z / p^2$, as
$\rad(\Z) = \emptyset$ whilst $\rad(\Z / p^{2}) = (p)$.

\subsubsection*{(b)}

Suppose $A$ is semi-local and $b \in \rad(B)$ and $a = f^{-1}(b)$. Then $a$ is
in all maximal ideals of $A$ which contain $\ker(f)$. Now let $\mathfrak{m}'$
be a maximal ideal of $A'$ which does not contain $\ker(f)$. Then $\ker(f) +
\mathfrak{m}' = (1)$ and so there is some $m' \in \mathfrak{m}'$ such that $m'
= 1 + c$ with $c \in \ker(f)$. Let $M$ be the set of maximal ideals of $A$
containing $\ker(f)$ and $M'$ be the set of maximal ideals of $A$ not
containing $M$. It follows by the proof of Theorem 1.3 that
\[
	\rad(A)
	=
	\left(
		\bigcap_{\mathfrak{m} \in M} \mathfrak{m}
	\right)	
	\left(
		\prod{\mathfrak{m}' \in M'} \mathfrak{m}'
	\right)
\] 
hence 
\[
	a' = a(1 + c_1)(1 + c_2) \ldots (1 + c_k) \in \rad(A)
\] 
for some $c_i \in \ker(f)$ and $f(a') = b$.

\subsection*{1.4}

Suppose $A$ is a UFD and $a \in A$ is irreducible. Then $a$ must have only one
prime factor by irreducibility, and so $a$ is prime thus $(a)$ is prime. Now 
suppose towards a contradiction
\[
	(a_0) \subseteq (a_1) \subset \ldots
\] 
is an infinite ascending chain of ideals of $A$. Suppose $a_0 = p_0p_1\ldots
p_k$ is a factorisation of $a_0$ into prime elements with repetitions allowed.
Then as $a_0 \in (a_1)$, every prime which divides $a_1$ divides $a_0$, and so
the prime factors of $a_1$ is some subset of the $p_i$. Strict inclusions
correspond to omision of prime factors, which clearly can only happen finitely
many times. \\

For the other direction, suppose that $A$ is a domain where irreducible
elements are prime, and where principal ideals satisfy the ascending chain
condition. Let $a \in A$, and $S$ be the set of prinicipal ideals containing
$A$. By hypothesis, $S$ contain a maximal element (it is non-empty as it
contains $(a)$), and so let $M \subset S$ be the set of maximal elements of
$S$. Let $(m) \in M$. Any principal ideal containing $(m)$ also must contain
$a$, thus $(m)$ is maximal among all principal ideals whence $m$ is
irreducible, and also prime by hypothesis. Let $a_0 = a$, and $(m_0) \in M$
and let $a_1 = m_0/a_0$ if $a_1$ isn't a unit, let $(m_1) \in M \setminus (m_0)$.
Then as $m_0 a_1 = c m_1$, and $m_1 \not | m_0$ since $(m_0) \not (m_1)$,
we have $m_1 | a_1$, and we can write $a_2 = a_1/m_1$. Continuing this way
yields an ascending chain
\[
	(a_0) \subsetneq (a_1) \subsetneq (a_2) \subsetneq \ldots
\] 
which must terminate at some $a_k$ by hypothesis. It follows that 
\[
	a = m_0 m_1 \ldots m_k
\] 
is a prime factorisation of $a$.

\subsection*{1.5}

Let $P = \bigcap_{\lambda \in \Lambda} P_{\lambda}$. Then $P$ is an ideal, and
if $x \not \in P, y \not \in P$, then there is some $P_x, P_y$ such that $x
\not \in P_x, y \not \in P_y$. As the family is totally ordered, we have either
$P_x \subset P_y$ or $P_y \subset P_x$, and we can assume the former. Then $y
\not \in P_x$, hence $xy \not \in P_x$ by primality, whence $xy \not \in P$. \\

Minimal primes of $A$ containing $I$ are precisely minimal primes of $A/I$
subject to no restrictions. We showed above that chains of prime ideals in any
ring have lower bounds. Hence $A/I$ has minimal primes by Zorn's Lemma.

\subsection*{1.6}

For each $i \in [1..r]$, let $x_i \in I \setminus P_i$. Then let $x =
x_1x_2\ldots x_r$.


\end{document}
