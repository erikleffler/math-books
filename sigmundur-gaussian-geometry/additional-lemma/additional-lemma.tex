\documentclass{article}
\usepackage[utf8]{inputenc}

\usepackage{rotating}
\usepackage{mathtools}
\usepackage{algpseudocode}
\usepackage{amsfonts}
\usepackage{amsmath}
\usepackage{amssymb}
\usepackage{amsthm}
\usepackage{bm}
\usepackage{listings}
\usepackage{float}
\usepackage{fancyvrb}
\usepackage{xcolor}
\usepackage{tikz-cd}

\hbadness = 10000
\vbadness = 10000

\newcommand\restr[2]{{% we make the whole thing an ordinary symbol
  \left.\kern-\nulldelimiterspace % automatically resize the bar with \right
  #1 % the function
  \vphantom{\big|} % pretend it's a little taller at normal size
  \right|_{#2} % this is the delimiter
  }}

% Default fixed font does not support bold face
\DeclareFixedFont{\ttb}{T1}{txtt}{bx}{n}{12} % for bold
\DeclareFixedFont{\ttm}{T1}{txtt}{m}{n}{12}  % for normal
% Custom colors

\usepackage{color}
\definecolor{deepblue}{rgb}{0,0,0.5}
\definecolor{deepred}{rgb}{0.6,0,0}
\definecolor{deepgreen}{rgb}{0,0.5,0}

% Python style for highlighting
\newcommand\pythonstyle{\lstset{
language=Python,
basicstyle=\ttm,
morekeywords={self},              % Add keywords here
keywordstyle=\ttb\color{deepblue},
emph={MyClass,__init__},          % Custom highlighting
emphstyle=\ttb\color{deepred},    % Custom highlighting style
stringstyle=\color{deepgreen},
frame=tb,                         % Any extra options here
showstringspaces=false
}}

\lstnewenvironment{python}[1][]
{
\pythonstyle
\lstset{#1}
}
{}

\theoremstyle{definition}

\newtheorem{theorem}{Theorem}[section]
\newtheorem{definition}[theorem]{Definition}
\newtheorem{corollary}[theorem]{Corollary}
\newtheorem{lemma}[theorem]{Lemma}

\newcommand{\Z}{\mathbb{Z}}
\newcommand{\Q}{\mathbb{Q}}
\newcommand{\R}{\mathbb{R}}
\newcommand{\C}{\mathbb{C}}
\newcommand{\K}{\mathbb{K}}
\renewcommand{\P}{\mathbb{P}}
\newcommand{\F}{\mathbb{F}}
\newcommand{\N}{\mathbb{N}}
\newcommand{\A}{\mathbb{A}}


\newcommand{\x}{\bm{x}}
\newcommand{\Kx}{\K[\bm{x}]}
\newcommand{\KP}[2]{\K[#1_1, #1_2, \ldots, #1_{#2}]}

\newcommand{\oo}{\mathcal{O}}
\newcommand{\osp}[1]{\oo_{\Spec(#1)}}
\newcommand{\rospu}[2]{\restr{\oo_{\Spec(#1)}}{#2}}
\newcommand{\oop}[2]{\oo_{\P^{#1}_{#2}}}

\renewcommand{\AA}[1]{\A^{#1}}
\newcommand{\An}{\A^n}
\newcommand{\Am}{\A^m}

\newcommand{\PP}[1]{\P^{#1}}
\newcommand{\Pn}{\P^n}
\newcommand{\Pm}{\P^m}

\newcommand{\Hom}{\text{Hom}}
\newcommand{\Aut}{\text{Aut}}
\newcommand{\End}{\text{End}}
\newcommand{\Iso}{\text{Iso}}
\newcommand{\Mor}{\text{Mor}}

\newcommand{\lm}{\text{lm}}
\newcommand{\nr}{\text{nilrad}}
\newcommand{\Spec}{\text{Spec}}
\newcommand{\Proj}{\text{Proj}}
\newcommand{\proj}{\Proj}
\newcommand{\spec}{\Spec}
\newcommand{\codim}{\text{codim}}
\newcommand{\ann}{\text{ann}}
\newcommand{\im}{\text{im}}
\newcommand{\id}{\text{id}}
\newcommand{\height}{\text{height}}

\newcommand{\pdx}{\frac{\partial}{\partial x}}
\newcommand{\pddx}{\frac{\partial^2}{\partial x^2}}
\newcommand{\pdy}{\frac{\partial}{\partial y}}
\newcommand{\pddy}{\frac{\partial^2}{\partial y^2}}

\newcommand{\catname}[1]{{\normalfont\textbf{#1}}}
\newcommand{\Set}{\catname{Set}}
\newcommand{\CRing}{\catname{CRing}}
\newcommand{\Top}{\catname{Top}}
\newcommand{\op}{\catname{op}}

\setlength{\parindent}{0pt}




\begin{document}

\begin{lemma}
	Let $M$ be a regular surface, $\gamma : I \to M$ be a differentiable curve
	on $M$, and $p \in \gamma(I)$. Furthermore, let $X : U \to X(U) \subset M$
	be a local parameterization at $p \in X(U)$. Then there exist some interval
	$J \subseteq I$, and differentiable curve $\alpha : J \to U$ on $U$ such
	that $\restr{\gamma}{J} = X \circ \alpha$ and $p \in \alpha(J)$.
\end{lemma}
\begin{proof}
	We assume without loss of generality that $\gamma(0) = X(0) = p$. Denote
	the components of $X$ by $X(u, v) = (f(u, v), g(u, v), h(u, v))$. As $X$ is
	regular, we have $X_u(0) \times X_v(0) \not = 0$, hence the differential 	
	\[
		dX(0)
		=
		\begin{bmatrix}
			f_u(0) & f_v(0) \\
			g_u(0) & g_v(0) \\
			h_u(0) & h_v(0) \\
		\end{bmatrix}
	\] 
	has rank $2$. It follows that there is some $2 \times 2$ minor of $dX(0)$
	which has non-zero determinant. Suppose without loss of generality that one
	such minor is given by
	\[
		\begin{bmatrix}
			f_u(0) & f_v(0) \\
			g_u(0) & g_v(0) \\
		\end{bmatrix},
	\]
	and let $F(u, v) = (f(u, v), g(u, v))$. Then let $p' = F(0, 0)$. As $dF(0)$
	has full rank, $F$ has a $C^{1}$ inverse about some neighbourhood $U_{p'}$ of
	$p'$. Now let $\pi_{x, y} : \R^{3} \to \R^{2}$ be the projection onto the
	first two coordinates. Then let $J$ be a small enough interval such that
	$\pi(\gamma(J)) \subset U_{p'}$. Then $0 \in J$ since $\pi(\gamma(0)) =
	\pi(p) = p'$. Now let $\alpha : J \to U$ be given by 
	\[
		\alpha
		=
		F^{-1}
		\circ 
		\pi
		\circ
		\restr{\gamma}{J}.
	\] 
	Then $\alpha$ is $C^1$ as it's a composition of $C^1$ functions. Moreover,
	as $F^{-1} \circ \pi$ coincides with $X^{-1}$ on $\pi^{-1}(U_q)$ it follows
	that
	\begin{align*}
		X \circ \alpha
		&=
		X
		\circ
		\restr{F}{U_q}^{-1}
		\circ 
		\pi
		\circ
		\restr{\gamma}{J} \\
		&=
		\restr{\gamma}{J}.
	\end{align*}
\end{proof}

\begin{corollary}
	Let $M_1, M_2$ be regular surfaces, and $\phi : M_1 \to M_2$ be a
	differentiable map. Let $\gamma : I \to M_1$ be a differentiable curve on
	$M_1$. Then $\phi \circ \gamma$ is a differentiable
	curve on $M_2$.
\end{corollary}
\begin{proof}
	Let $p \in \gamma(I)$ and $q = \phi(p)$. Let $X : U \to X(U) \subset M_1, Y
	: V \to Y(V) \subset M_2$ be two local parameterizations such that
	$\phi(X(U)) \subset Y(V)$ and $p \in X(U)$. Let $\alpha$ be the
	factorization of $\restr{\gamma}{J}$ through $X$. Then
	\[
		\phi \circ \restr{\gamma}{J}
		=
		Y
		\circ
		Y^{-1}
		\circ
		\phi
		\circ
		X
		\circ
		\alpha
	\] 
	and as $Y, Y^{-1} \circ \phi \circ X, \alpha$ are all $C^1$, so is their
	composition $\phi \circ \restr{\gamma}{J}$. We've verified that $\phi \circ
	\gamma$ is $C^1$ at the arbitrary point $\gamma^{-1}(p)$, whence it's $C^1$
	at all of its domain.
\end{proof}

\end{document}
