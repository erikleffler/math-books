\documentclass{article}
\usepackage[utf8]{inputenc}

\usepackage{rotating}
\usepackage{mathtools}
\usepackage{algpseudocode}
\usepackage{amsfonts}
\usepackage{amsmath}
\usepackage{amssymb}
\usepackage{amsthm}
\usepackage{bm}
\usepackage{listings}
\usepackage{float}
\usepackage{fancyvrb}
\usepackage{xcolor}
\usepackage{tikz-cd}

\hbadness = 10000
\vbadness = 10000

\newcommand\restr[2]{{% we make the whole thing an ordinary symbol
  \left.\kern-\nulldelimiterspace % automatically resize the bar with \right
  #1 % the function
  \vphantom{\big|} % pretend it's a little taller at normal size
  \right|_{#2} % this is the delimiter
  }}

% Default fixed font does not support bold face
\DeclareFixedFont{\ttb}{T1}{txtt}{bx}{n}{12} % for bold
\DeclareFixedFont{\ttm}{T1}{txtt}{m}{n}{12}  % for normal
% Custom colors

\usepackage{color}
\definecolor{deepblue}{rgb}{0,0,0.5}
\definecolor{deepred}{rgb}{0.6,0,0}
\definecolor{deepgreen}{rgb}{0,0.5,0}

% Python style for highlighting
\newcommand\pythonstyle{\lstset{
language=Python,
basicstyle=\ttm,
morekeywords={self},              % Add keywords here
keywordstyle=\ttb\color{deepblue},
emph={MyClass,__init__},          % Custom highlighting
emphstyle=\ttb\color{deepred},    % Custom highlighting style
stringstyle=\color{deepgreen},
frame=tb,                         % Any extra options here
showstringspaces=false
}}

\lstnewenvironment{python}[1][]
{
\pythonstyle
\lstset{#1}
}
{}

\theoremstyle{definition}

\newtheorem{theorem}{Theorem}[section]
\newtheorem{definition}[theorem]{Definition}
\newtheorem{corollary}[theorem]{Corollary}
\newtheorem{lemma}[theorem]{Lemma}

\newcommand{\Z}{\mathbb{Z}}
\newcommand{\Q}{\mathbb{Q}}
\newcommand{\R}{\mathbb{R}}
\newcommand{\C}{\mathbb{C}}
\newcommand{\K}{\mathbb{K}}
\renewcommand{\P}{\mathbb{P}}
\newcommand{\F}{\mathbb{F}}
\newcommand{\N}{\mathbb{N}}
\newcommand{\A}{\mathbb{A}}


\newcommand{\x}{\bm{x}}
\newcommand{\Kx}{\K[\bm{x}]}
\newcommand{\KP}[2]{\K[#1_1, #1_2, \ldots, #1_{#2}]}

\newcommand{\oo}{\mathcal{O}}
\newcommand{\osp}[1]{\oo_{\Spec(#1)}}
\newcommand{\rospu}[2]{\restr{\oo_{\Spec(#1)}}{#2}}
\newcommand{\oop}[2]{\oo_{\P^{#1}_{#2}}}

\renewcommand{\AA}[1]{\A^{#1}}
\newcommand{\An}{\A^n}
\newcommand{\Am}{\A^m}

\newcommand{\PP}[1]{\P^{#1}}
\newcommand{\Pn}{\P^n}
\newcommand{\Pm}{\P^m}

\newcommand{\Hom}{\text{Hom}}
\newcommand{\Aut}{\text{Aut}}
\newcommand{\End}{\text{End}}
\newcommand{\Iso}{\text{Iso}}
\newcommand{\Mor}{\text{Mor}}

\newcommand{\lm}{\text{lm}}
\newcommand{\nr}{\text{nilrad}}
\newcommand{\Spec}{\text{Spec}}
\newcommand{\Proj}{\text{Proj}}
\newcommand{\proj}{\Proj}
\newcommand{\spec}{\Spec}
\newcommand{\codim}{\text{codim}}
\newcommand{\ann}{\text{ann}}
\newcommand{\im}{\text{im}}
\newcommand{\id}{\text{id}}
\newcommand{\height}{\text{height}}

\newcommand{\pdx}{\frac{\partial}{\partial x}}
\newcommand{\pddx}{\frac{\partial^2}{\partial x^2}}
\newcommand{\pdy}{\frac{\partial}{\partial y}}
\newcommand{\pddy}{\frac{\partial^2}{\partial y^2}}

\newcommand{\catname}[1]{{\normalfont\textbf{#1}}}
\newcommand{\Set}{\catname{Set}}
\newcommand{\CRing}{\catname{CRing}}
\newcommand{\Top}{\catname{Top}}
\newcommand{\op}{\catname{op}}

\setlength{\parindent}{0pt}




\begin{document}

The following question concerns the proof of Proposition 4.28. I have trouble
following the argument "as $X : U \to X(U)$ is a diffeomorphism, it follows
that every differentiable curve $\gamma : I \to X(U)$ with $\gamma(0) = p$ can
be obtained this way". I interpret this as follows:
\begin{enumerate}
	\item Let $\gamma : I \to U$ be a differentiable curve on $U$.
	\item Then as $X : U \to X(U)$ is a diffeomorphism, so is $X^{-1} : X(U)
		\to U$, hence $\alpha = X^{-1} \circ \gamma$ is a differentiable curve
		on $U$ such that $\alpha(0) = X^{-1}(\gamma(0)) = X^{-1}(p) = 0$.
	\item Then $\gamma = X \circ \alpha$, and every curve arises this way.
\end{enumerate}

I don't follow step (2). I agree that $X^{-1}$ is differentiable as a map
between regular surfaces, but this is not enough to guarantee that $\alpha$ is
smooth as a map into $\R^3$? Or am I mistaken? \\

For example, let $M = \R^{2} \times \{0\} \subset \R^3$, and let $X : \R^2 \to
M$ be a local parameterization at $(0, 0, 0)$ given by $X : (x, y) \mapsto
(x^3, y^3, 0)$. Then the corresponding local chart $X^{-1} : (x, y, z) \mapsto
(x^{1/3}, y^{1/3})$ is not differentiable at $(0, 0, 0)$ as a map $\R^3 \to
\R^2$. In particular, the differentiable curve $\gamma : \R \to M$ given by
$\gamma : x \mapsto (x, 0, 0)$ can't be decomposed as a differentiable curve
into the local coordinates postcomposed with $X$, since $X^{-1} \circ \gamma :
x \mapsto (x^{1/3}, 0, 0)$ isn't a differentiable curve. \\

I tried to read Pressley instead to understand better, but they just make the
same claim (Equation (4.2), Section 4.4), and say that it will be proved in
Section 5.6, but then I can't find a proof in Section 5.6. \\

I've hade some troubles with differentiablity in general, and am wondering if
I'm missunderstanding something fundamental here?

\end{document}
