\documentclass{article}
\usepackage[utf8]{inputenc}

\usepackage{rotating}
\usepackage{mathtools}
\usepackage{algpseudocode}
\usepackage{amsfonts}
\usepackage{amsmath}
\usepackage{amssymb}
\usepackage{amsthm}
\usepackage{bm}
\usepackage{listings}
\usepackage{float}
\usepackage{fancyvrb}
\usepackage{xcolor}
\usepackage{tikz-cd}

\hbadness = 10000
\vbadness = 10000

\newcommand\restr[2]{{% we make the whole thing an ordinary symbol
  \left.\kern-\nulldelimiterspace % automatically resize the bar with \right
  #1 % the function
  \vphantom{\big|} % pretend it's a little taller at normal size
  \right|_{#2} % this is the delimiter
  }}

% Default fixed font does not support bold face
\DeclareFixedFont{\ttb}{T1}{txtt}{bx}{n}{12} % for bold
\DeclareFixedFont{\ttm}{T1}{txtt}{m}{n}{12}  % for normal
% Custom colors

\usepackage{color}
\definecolor{deepblue}{rgb}{0,0,0.5}
\definecolor{deepred}{rgb}{0.6,0,0}
\definecolor{deepgreen}{rgb}{0,0.5,0}

% Python style for highlighting
\newcommand\pythonstyle{\lstset{
language=Python,
basicstyle=\ttm,
morekeywords={self},              % Add keywords here
keywordstyle=\ttb\color{deepblue},
emph={MyClass,__init__},          % Custom highlighting
emphstyle=\ttb\color{deepred},    % Custom highlighting style
stringstyle=\color{deepgreen},
frame=tb,                         % Any extra options here
showstringspaces=false
}}

\lstnewenvironment{python}[1][]
{
\pythonstyle
\lstset{#1}
}
{}

\theoremstyle{definition}

\newtheorem{theorem}{Theorem}[section]
\newtheorem{definition}[theorem]{Definition}
\newtheorem{corollary}[theorem]{Corollary}
\newtheorem{lemma}[theorem]{Lemma}

\newcommand{\Z}{\mathbb{Z}}
\newcommand{\Q}{\mathbb{Q}}
\newcommand{\R}{\mathbb{R}}
\newcommand{\C}{\mathbb{C}}
\newcommand{\K}{\mathbb{K}}
\renewcommand{\P}{\mathbb{P}}
\newcommand{\F}{\mathbb{F}}
\newcommand{\N}{\mathbb{N}}
\newcommand{\A}{\mathbb{A}}


\newcommand{\x}{\bm{x}}
\newcommand{\Kx}{\K[\bm{x}]}
\newcommand{\KP}[2]{\K[#1_1, #1_2, \ldots, #1_{#2}]}

\newcommand{\oo}{\mathcal{O}}
\newcommand{\osp}[1]{\oo_{\Spec(#1)}}
\newcommand{\rospu}[2]{\restr{\oo_{\Spec(#1)}}{#2}}
\newcommand{\oop}[2]{\oo_{\P^{#1}_{#2}}}

\renewcommand{\AA}[1]{\A^{#1}}
\newcommand{\An}{\A^n}
\newcommand{\Am}{\A^m}

\newcommand{\PP}[1]{\P^{#1}}
\newcommand{\Pn}{\P^n}
\newcommand{\Pm}{\P^m}

\newcommand{\Hom}{\text{Hom}}
\newcommand{\Aut}{\text{Aut}}
\newcommand{\End}{\text{End}}
\newcommand{\Iso}{\text{Iso}}
\newcommand{\Mor}{\text{Mor}}

\newcommand{\lm}{\text{lm}}
\newcommand{\nr}{\text{nilrad}}
\newcommand{\Spec}{\text{Spec}}
\newcommand{\Proj}{\text{Proj}}
\newcommand{\proj}{\Proj}
\newcommand{\spec}{\Spec}
\newcommand{\codim}{\text{codim}}
\newcommand{\ann}{\text{ann}}
\newcommand{\im}{\text{im}}
\newcommand{\id}{\text{id}}
\newcommand{\height}{\text{height}}

\newcommand{\pdx}{\frac{\partial}{\partial x}}
\newcommand{\pddx}{\frac{\partial^2}{\partial x^2}}
\newcommand{\pdy}{\frac{\partial}{\partial y}}
\newcommand{\pddy}{\frac{\partial^2}{\partial y^2}}

\newcommand{\catname}[1]{{\normalfont\textbf{#1}}}
\newcommand{\Set}{\catname{Set}}
\newcommand{\CRing}{\catname{CRing}}
\newcommand{\Top}{\catname{Top}}
\newcommand{\op}{\catname{op}}

\setlength{\parindent}{0pt}




\begin{document}

\section*{Ch 2}

\subsection*{Ex 2.1}

We can describe the curve as a sum of two different curves, $\gamma_1(t) = r(t,
1)$ and $\gamma_2(t) = r (\sin(-t), -\cos(-t))$. The first map parameterises a
line parallel through the $x$-axis at height $y = r$. The second map
parameterises a circle of radius $r$, but in the negative direction, and
starting at $(0, -r)$ when $t = 0$. \\

As $(\gamma_2)_x' \leq r$, and $(\gamma_1)_x' = r$, we expect the sum $\gamma =
\gamma_1 + \gamma_2$ to have positive $x$ derivative, never "travels left". We
also see that $\gamma$ has vanishing $x$-derivatives whenever $t = \pi/2 + k
\pi$ for integers $k$. Moreover, $\gamma_1$ and $\gamma_2$ have the same
arclength $|\gamma_1| = |\gamma_2| = r$ over any $t$-interval, and by thinking
about this for a bit, we can convince ourselves that the curve can be drawn by
attaching a marker to a wheel of radius $r$, and then letting that wheel roll
on the $x$-axis. \\

Both the $x$- and $y$-derivatives vanish at $\pi/2 + k\pi$ for $k \in \Z$,
hence the curve is not regular (the curve has singularities wherever the
"marker touches the ground"). \\

The arclength is given by 
\begin{align*}
	\sigma(2\pi)
	&=
	\int_{0}^{2\pi}
	|\gamma'(t)| dt \\
	&=
	\int_{0}^{2\pi}
	\sqrt{(r - r\cos(-t))^2 + (r\sin(-t))^2} dt \\
	&=
	\int_{0}^{2\pi}
	\sqrt{r^2 - 2r^2\cos(-t) + r^2\cos^2(-t) + r^2\sin^2(-t)} dt \\
	&=
	\int_{0}^{2\pi}
	\sqrt{2r^2 - 2r^2\cos(-t)} dt \\
	&=
	r
	\int_{0}^{2\pi}
	\sqrt{2(1 - \cos(t))} dt \\
	&=
	r
	\int_{0}^{2\pi}
	\sqrt{2(2\sin^2(t/2))} dt \\
	&=
	2r
	\int_{0}^{2\pi}
	\sin(t/2) dt \\
	&=
	2r(-2\cos(\pi) + 2\cos(0)) \\
	&=
	8r
\end{align*} 

\subsection*{Ex 2.2}

We can again decompose $\gamma$ into a sum of $\gamma_1(t) = 3r(\cos(t),
\sin(t))$ and $\gamma_2(t) = r(\cos(-3t), \sin(-3t))$. So we have some small
circular motion added to a bigger circular motion. Given the previous exercise,
we hypothesise that this is what you get when you attach a marker to the edge
of a coin of radius $r$, and let it roll around another coin of radius $2r$. In
this scenario, $\gamma_1(t)$ models exactly the motion of the center of the
smaller coin, and in the time it'd take the radius of the center of the smaller
coin to make a full lap around the big coin, we'd expect the smaller coin to
make three full revolutions about it's own center. Indeed, two revolutions
would come from just rolling the length of the diameter of the bigger coin
$2\pi (2r)$, and then another revolution would come from "bending" that length
around the edge of the bigger coin. Also, the smaller coin would have to spin
in the opposite direction, as opposed to its radius. I.e we get exactly
$\gamma_1$ and $\gamma_2$. \\

As before, we expect singularities whenever the marker touches the rolling
surface (the bigger coin), and we can see this analytically as 
\begin{align*}
	\gamma'(t)
	=
	r(-3\sin(t) + 3\sin(-3t), 3\cos(t) - 3\cos(-3t))
\end{align*}
which vanishes whenever $t = 0 + k_2\pi$ for example. Hence the curve is not
regular. \\


The arclength of $\gamma$ for $t \in [0..2\pi]$ is given by 
\begin{align*}
	\sigma(2\pi) 
	&=
	r
	\int_{0}^{2\pi}
	\sqrt{(-3\sin(t) + 3\sin(-3t))^2 + (3\cos(t) - 3\cos(-3t))^2} dt \\
	&=
	3r
	\int_{0}^{2\pi}
	\sqrt{(-\sin(t) + \sin(-3t))^2 + (\cos(t) - \cos(-3t))^2} dt \\
	&=
	3r
	\int_{0}^{2\pi}
	\sqrt{\sin^2(t) -2\sin(t)\sin(-3t) + \sin^2(-3t) + \cos^2(t) -2\cos(t)\cos(-3t) + \cos^2(-3t)} dt \\
	&=
	3r
	\int_{0}^{2\pi}
	\sqrt{2 -2\sin(t)\sin(-3t) -2\cos(t)\cos(-3t)} dt \\
	&=
	3r
	\int_{0}^{2\pi}
	\sqrt{2 + 2\sin(t)\sin(3t) - 2\cos(t)\cos(3t)} dt \\
	&=
	3r
	\int_{0}^{2\pi}
	\sqrt{2 + 2\sin(t)(3\sin(t) - 4\sin^{3}(t)) - 2\cos(t)(4\cos^{3}(t) - 3\cos(t))} dt \\
	&=
	3r
	\int_{0}^{2\pi}
	\sqrt{2 + 6\sin^2(t) - 8\sin^{4}(t) - 8\cos^4(t) + 6\cos^2(t)} dt \\
	&=
	3r
	\int_{0}^{2\pi}
	\sqrt{2 + 6 - 8\sin^{4}(t) - 8\cos^4(t)} dt \\
	&=
	3r
	\int_{0}^{2\pi}
	\sqrt{8 (1 - \sin^{4}(t) - \cos^4(t))} dt \\
	&=
	6r
	\int_{0}^{2\pi}
	\sqrt{8 (2 \sin^{2}(t) \cos^2(t))} dt \\
	&=
	24r
	\int_{0}^{2\pi}
	|\sin(t) \cos(t)| dt \\
	&=
	12r
	\int_{0}^{2\pi}
	|\sin(2t)| dt \\
	&=
	12r 2 \\
	&=
	24r.
\end{align*}

\subsection*{Ex 2.3}

We begin by calculating the curvature $\kappa_1$ of $\gamma_1$, and we don't
want to bother with reparameterisations, hence we use Proposition 2.12.
We have that
\[
	\gamma_1' = ra (-\sin(at), \cos(at)),
\]
and 
\[
	\gamma_1'' = -ra^2 (\cos(at), \sin(at)),
\]
hence 
\begin{align*}
	\kappa_1
	&=
	\frac{\det[\gamma_1', \gamma_2'']}{|\gamma_1'|^3} \\
	&=
	\frac{r^2a^3 \sin^2(at) + r^2a^3 \cos^2(at)}{r^3a^3}\\
	&=
	\frac{1}{r}. \\
\end{align*} 
As $\kappa_1$ is independent of $a$, it follows that $\kappa_2 = \kappa_1$.
Note that $(\gamma_1)_x = (\gamma_2)_x$ and $(\gamma_1)_y = -(\gamma_2)_y$.
Hence $\Phi$ is just flipping $\R^2$ about the $x$-axis. $\Phi$ is not 
orientation preserving as it has determinant $-1$. 

\subsection*{Ex 2.4}

Let $\gamma_{r}(t) = \gamma(t) + r N(t)$. Then 
the first and second derivatives of $\gamma_r$ are given by
\[
	\gamma_r'(t) = T(t) + r N'(t) = (1 - r \kappa(t)) T(t) = (1 - r \kappa(t)) \gamma'(t)
\] 
and
\[
	\gamma_r''(t)
	=
	-r \kappa'(t)\gamma'(t) + (1 - r \kappa(t)) \gamma''(t)
\] 
where we used Theorem 2.8 to substitute $rN'(t)$ for $-rT(t)$. In particular,
$\gamma_r$ is not necesarilly parameterized by arclength. Thus, we use
Proposition 2.12 to calculate the curvature. We get
\begin{align*}
	\kappa_r(t)
	&=
	\frac{1}{|\gamma_r'(t)|^3}
	\det[\gamma'_r(t), \gamma''_r(t)] \\
	&=
	\frac{1}{|(1 - r \kappa(t)) \gamma'(t)|^3}
	\det[(1 - r \kappa(t))\gamma'(t), (1 - r \kappa(t))\gamma''(t) - r\kappa'(t)\gamma'(t)] \\
	&=
	\frac{1}{(1 - r \kappa(t)) |\gamma'(t)|^3}
	\det\left[\gamma'(t), \gamma''(t) - \frac{r \kappa'(t)}{1 - r\kappa(t)}\gamma'(t)\right] \\
	&=
	\frac{1}{(1 - r \kappa(t)) |\gamma'(t)|^3}
	\det\left[\gamma'(t), \gamma''(t)\right] \\
	&=
	\frac{\kappa(t)}{1 - r \kappa(t)},
\end{align*}
where we used the fact that the determinant is invariant to row operations, such as adding/subtracting 
scalar multiples of columns to other columns.

\subsection*{Ex 2.5}

\begin{proof}[Proof of Proposition 2.12]
	First of all, the curvature of any curve $\varphi$ in $\R^2$ parameterized by
	arclength is given by
	\begin{align*}
		\kappa_{\varphi}(s)
		&=
		\langle
			T_{\varphi}'(s),
			N_{\varphi}(s)
		\rangle \\
		&=
		\langle
			(\varphi''(s)_x, \varphi''(s)_y),
			(-\varphi'(s)_y, \varphi'(s)_x)
		\rangle \\
		&=
		(\varphi'(s)_x, \varphi'(s)_y)
		\times
		(\varphi''(s)_x, \varphi''(s)_y) \\
		&= 
		\det
		\begin{bmatrix}
			\varphi'(s)_x & \varphi'(s)_y \\
			\varphi''(s)_x & \varphi''(s)_y
		\end{bmatrix} \\
		&=
		\det[\varphi'(s), \varphi''(s)]
	\end{align*}	

	Now, let $t(s) : J \to I$ be a strictly increasing surjective $C^{2}$
	function such that $\widetilde{\gamma} = \gamma \circ t$ is parameterized
	by arclength, and let $s(t) : J \to I$ be its inverse. Then the curvature
	of $\gamma$ is given by
	\begin{align*}
		\kappa(t(s))
		&=
		\widetilde{\kappa}(s) \\
		&=
		\det[\dot{\widetilde{\gamma}}(s), \ddot{\widetilde{\gamma}}(s)] \\
		&=
		\det[\gamma'(t(s)) \dot{t}(s), \gamma''(t(s)) \dot{t}(s)^2 + \gamma'(t(s))\ddot{t}(s)] \\
		&=
		\det[\gamma'(t(s)) \dot{t}(s), \gamma''(t(s)) \dot{t}(s)^2] \\
		&=
		(\dot{t}(s))^3\det[\gamma'(t(s)), \gamma''(t(s))] \\
		&=
		\frac{1}{(s'(t(s)))^3}\det[\gamma'(t(s)), \gamma''(t(s))],
	\end{align*}
	where we used the fact that the determinant is invariant to row operations
	for the third step, and the inverse function theorem for the last step.
	Composing with $s(t)$ yields
	\[
		\kappa(t)
		=
		\frac{1}{(s'(t))^3}\det[\gamma'(t), \gamma''(t)],
	\] 
	so it remains to show that $|\gamma'(t)|^{3} = (s'(t))^3$. This is
	immediate from
	\begin{align*}
		1
		&=
		|\dot{\widetilde{\gamma}}(s)| \\
		&=
		|\dot{\gamma(t(s))}| \\
		&=
		|\gamma'(t)\dot{t}(s)| \\
		&=
		\left|\frac{\gamma'(t)}{s'(t)}\right|
	\end{align*}
	and the fact that $s'$ is positive due to $s$ being strictly increasing (as
	$t$ is).
\end{proof}


\subsection*{Ex 2.6}

We have $\gamma'(t) = (\cos(t), 2 \cos(2t))$. As $\cos$ vanishes exactly at
$\pi/2 + k\pi$ for all $k \in \Z$, and none of these values differ by a factor
of $2$ (indeed, these are all the halves of odd integer multiples of $\pi$, so
if one of them differed by a factor of $2$, we'd have two odd numbers differing
by a factor of $2$, which is impossible), so $\gamma$ is regular. \\

The curve is closed as it has period $2\pi$. Finally, it's not simple, as it
has multiple zeros, $\pi/2, 3\pi/2$, in $[0, 2\pi)$.

\subsection*{Ex 2.7}

TODO: Finnish later

\subsection*{Ex 3.1}

We have
\begin{align*}
	\gamma_1'(t) 
	&= 
	(-ar \sin(at), ar \cos(at), ba) \\
	\gamma_1''(t)
	&=
	(-a^2r \cos(at), -a^2r \sin(at), 0) \\
	\gamma_1'''(t)
	&=
	(a^3r \sin(at), -a^3r \cos(at), 0) \\
\end{align*}
and 
\begin{align*}
	\gamma_1'(t) \times \gamma_1''(t)
	&=
	(
	ba^3r \sin(at),
	-ba^3r \cos(at), 
	a^3r^2 \sin^2(at) + a^3r^2 \cos^2(at)
	) \\
	&=
	(
	ba^3r \sin(at),
	-ba^3r \cos(at), 
	a^3r^2, 
	) \\
	&=
	a^3r(b \sin(at), -b \cos(at), r) \\
	|\gamma_1'(t) \times \gamma_1''(t)|
	&=
	a^3r \sqrt{b^2 + r^2}
\end{align*}
whence
\begin{align*}
	\kappa_1(t)
	&=
	\frac{a^3r \sqrt{r^2 + b^2}}{a^3 \sqrt{r^2 + b^2}^3} \\
	&=
	\frac{r}{r^2 + b^2}.
\end{align*}
Moreover, $\det[u,v,w] = (u \times v) \cdot w$ for $u,v,w \in \R^3$, hence 
\begin{align*}
	\tau_1(s)
	&=
	\frac
	{(\gamma_1'(s) \times \gamma_1''(s)) \cdot \gamma_1'''(s)}
	{|\gamma_1'(s) \times \gamma_1''(s)|^2} \\
	&=
	\frac
	{a^3r(b \sin(at), -b \cos(at), r) \cdot (a^3r \sin(at), -a^3r \cos(at), 0)}
	{a^6r^2(r^2 + b^2)} \\
	&=
	\frac
	{(b \sin(at), -b \cos(at), r) \cdot (\sin(at), - \cos(at), 0)}
	{r^2 + b^2} \\
	&=
	\frac
	{b \sin^2(at) + b \cos^2(at)}
	{r^2 + b^2} \\
	&=
	\frac{b}{r^2 + b^2}.
\end{align*}
For $\gamma_2$, note that $\gamma_2(t)$ and $\gamma_2(-t)$ have the same
curvature and torsion since the two curve only differ in parameterization. The
curves $\gamma_1(t)$ and $\gamma_2(-t)$ have the same second and third
derivatives, and the first derivative differs only in sign in the first
coordinate. We have
\begin{align*}
	\gamma_2'(-t) \times \gamma_2''(-t)
	&=
	(
	-ba^3r \sin(at),
	ba^3r \cos(at), 
	a^3r^2 \sin^2(at) + a^3r^2 \cos^2(at)
	) \\
	&=
	(
	-ba^3r \sin(at),
	ba^3r \cos(at), 
	a^3r^2, 
	) \\
	&=
	a^3r(b \sin(at), -b \cos(at), r), \\
	|\gamma_2'(-t) \times \gamma_2''(-t)|
	&=
	a^3r \sqrt{b^2 + r^2} \\
	&=
	|\gamma_1'(t) \times \gamma_1''(t)|, \\
	(\gamma_2'(-t) \times \gamma_2''(-t)) \cdot \gamma_2'''(-t)
	&=
	-a^6r^2(b \sin^2(at) + b \cos^2(at)) \\
	&=
	-
	(\gamma_1'(-t) \times \gamma_1''(-t)) \cdot \gamma_1'''(-t)
\end{align*}
Combining all these results with formulas for curvature and torsion, we see that 
\[
	\kappa_2 = \kappa_1, \tau_2 = -\tau_1.
\]
As the two curves differ only by sign in the last coordinate, 
\[
	\Phi 
	=
	\begin{bmatrix}
		1 & 0 & 0 \\	
		0 & 1 & 0 \\	
		0 & 0 & -1 \\	
	\end{bmatrix}
\] 
is a Euclidean motion that maps between them. This map is not orientation
preserving, as it has determinant $-1$.

\subsection*{Ex 3.2}

The curvature and torsion of the helix from Exercise 3.1 were both constant,
and indeed, it seems intuitive that any curve of constant curvature and torsion
would be a helix. So, let $\gamma(t) = (r \sin(t), r \sin(t), bt)$ (we are free
to ignore $a$ from Exercise 3.1, since this variable only affects the
parameterization). Then we calculated in Exercise 3.1 that
\[
	\kappa = \frac{r}{r^2 + b^2},\, 
	\tau = \frac{b}{r^2 + b^2}.
\] 
From this we obtain two linear equations $\kappa r + \tau b = 1$ and $b\kappa =
r\tau$, and as these equations are symmetric with respect to $r,b$ versus
$\kappa, \tau$, we see that the solution is given by
\[
	b = \frac{\tau}{\kappa^2 + \tau^2},\, 
	r = \frac{\kappa}{\tau^2 + \kappa^2}, 
\] 
and these values of $b, r$ induce a helix curve with the desired curvature and torsion.

\subsection*{Ex 3.4}

We have
\[
	\gamma'(t)
	=
	(3t^2 + 2t, 3t^2 - 1, 2t + 1),
	=
	(3t(t + 2), (\sqrt{3} t - 1)(\sqrt{3} t + 1), 2t + 1),
\] 
and from the factorized expressions it follows readily that the $x,y,z$
coordinates never simultaneously vanish. Hence the curve is regular. \\

Any curve that is parameterized by a polynomial of degree $\geq 2$ in one of
its coordinates will never lie in a plane or line. The reason is that the first
three derivatives in that coordinate will be three polynomials of different
degrees, which are all linearly independent.

\subsection*{Ex 4.1}

We begin with showing that $M_1$ is not regular. Suppose towards a
contradiction that $M_1$ is regular, and let $U_{0}, X_{0}$ be a local
parameterization around $0 = (0, 0, 0) \in M_1$. Then let $q \in U_0$ be the
preimage of $0$. As $X_0(U_0)$ is an open set about $0$, it contains points
$a^{+}$ and $a^{-}$ where $a^{+}_z > 0$ and $a^{-}_z < 0$. Let $q^{+}, q^{-}$
be their respective preimages in $U_0$. Then as $U_0$ is open in $\R^2$, it
contains an open ball about $q$, and as it's also simply connected, there exist
a path from $q^{+}$ to $q^{-}$ which avoids $q$. But then $X_0(\tau)$ is a path
in $M_1$ from $a^{+}$ to $a^{-}$ which avoids $0$. This is impossible since $0$
is the only point of $M_1$ which has a vanishing $z$ coordinate. \\


We will use the implicit function theorem to show that $M_2$ is regular. Let $f
: \R^{3} \to \R$ be given by 
\[
	f : (x, y, z) \mapsto x^2 + y^2 - z^2.
\] 
Then $df = (2x, 2y, -2z)$ and $1$ is a regular point of $f(\R^{3})$. Hence $M_2
= f^{-1}(1)$ is a regular surface. \\

We will show that $M_3$ is a parameterized regular surface of revolution. Let
$\gamma(t) = (\cosh(t), 0, t)$. Then $\gamma'(t) = (\sinh(t), 0, 1)$ never vanishes,
hence $\gamma$ is regular and can be parameterized by arclength. Moreover, 
\[
	\gamma_x^2 + \gamma_y^{2} = \cosh^2(\gamma_{z}),
\] 
so $\gamma$ lies on $M_3$, and it's easy see that $M_3$ is the surface of
revolution of $\gamma$, whence it is regular by Example 4.13. \\

We will use the implicit function theorem to show that $M_4$ is regular. Let $f
: \R^{3} \to \R$ be given by 
\[
	f : (x, y, z) \mapsto x \sin(z) - y \cos(z)
\] 
Then $df = (\sin(z), -\cos(z), x \cos(z) + y \sin(z))$ never vanishes and $0$ is
a regular point of $f(\R^{3})$. Hence $M_4 = f^{-1}(0)$ is a regular surface.
\\

\subsection*{Ex 4.2}

We are given $N$ in terms of the local coordinates on $T^2$. I.e
\[
	N \circ X_{T^2}
	:
	(u, v)
	\mapsto
	\begin{bmatrix}
		\cos(u)\cos(v) \\
		\cos(u)\sin(v) \\
		\sin(v)
	\end{bmatrix}.
\]
Composing this with the chart maps
from Example 4.3 yields
\[
	X_{N_2}^{-1} \circ N \circ X_{T^2}
	:
	(u, v)
	\mapsto
	\frac{\cos(v)}{1 - \sin(v)}
	(\cos(u), \sin(u))
\]
and
\[
	X_{N_2}^{-1} \circ N \circ X_{T^2}
	:
	(u, v)
	\mapsto
	\frac{\cos(v)}{1 + \sin(v)}
	(\cos(u), \sin(u))
\]
which are clearly differentiable where they are defined, and their areas of
definition "cover" $T^2$.

\subsection*{Ex 4.3}

This exercise is somewhat underdefined, as we haven't defined what it means for
a map $M \to \R^2$ to be differentiable. We will interpret it as if $U = \{(x,
y, 0) : (x,y,z) \in \R^3)\}$ is a regular surface embedded in $\R^3$

\begin{proof}[Proof of Proposition 4.19]
	The chart on $U$ is just the projection onto the first two coordinates,
	and the composition
	\[
		\pi_{x, y} \circ X^{-1} \circ X
		=
		\id_{x, y}
	\] 
	is clearly differentiable.
\end{proof}

\subsection*{Ex 4.4}

\begin{proof}[Proof of Corollary 4.20]

	Suppose that $\phi$ is differentiable at $p$ and that $q = \phi(p)$.
	Furthermore, let $X_p, : U_p \to M_1, X'_p : U'_p \to M_1$ and $X_q : U_q
	\to M_2, X'_{q} : U'_q \to M_2$ be two different parameterizations of $M_1$
	and $M_2$ at $p$ and $q$ respectively. Furthermore let
	\[
		W = X_p^{-1}(X_p(U_p) \cap \phi^{-1}(X_{q}^{-1}(V_q))),\, 
		W' = (X'_p)^{-1}(X'_p(U'_p) \cap \phi^{-1}(X'_{q}(V'_q))),\, 
	\] 
	and $F : W \to V_q, F': W' \to V'q$ be given by 
	\[
		F = X_q^{-1} \circ \phi \circ X_p, \,
		F' = (X'_q)^{-1} \circ \phi \circ X'_p.
	\] 
	We are supposed to show that $F$ is differentiable at $X_p^{-1}(p)$ if and
	only if $F'$ is differentiable at $(X'_p)^{-1}(p)$.  \\

	This follows immediately by picking a small enough neighbourhood $N$ around
	$(X'_p)^{-1}(p)$ such that
	\[
		\restr{F'}{N} = (X'_{q})^{-1} \circ X_q \circ F \circ X_p^{-1} \circ X'_p,
	\] 
	and the fact that local charts and parameterizations are differentiable (as maps 
	between regular surfaces).
\end{proof}

\subsection*{Ex 4.5}

\begin{proof}[Proof of Proposition 4.21]
	A map $U \to \R^3$ from an open set $U \subset \R^{3}$ is differentiable if
	and only if it's $C^{1}$. As the local charts and atlases, and
	$\restr{\phi}{M_1}$ are all $C^1$, so is their composition.
\end{proof}

\subsection*{Lemma for Proposition 4.28 and 4.31}

\begin{lemma}
	Every chart map of a regular surface is locally the restriction
	of a differentiable map $\R^3 \to \R^2$. \\

	More specifically, let $M$ be a regular surface, and $X : U \to X(U)
	\subseteq M$ a local parameterization. Then for every point $q \in X(U)$,
	there exists an open neighbourhood $V_q$ of $q$ $\R^3$, and a function
	$\phi_q : V_q \to \R^2$ which restricts to $X^{-1}$ on the open set $X(U)
	\cap V_q$ in $X(U)$.
\end{lemma}
\begin{proof}
	Let $p \in U$ and $q = X(p)$. Then $X_u(p) \times X_v(p) \not = 0$,
	hence the differential 	
	\[
		dX(p)
		=
		\begin{bmatrix}
			f_u(p) & f_v(p) \\
			g_u(p) & g_v(p) \\
			h_u(p) & h_v(p) \\
		\end{bmatrix}
	\] 
	has rank $2$. It follows that there is some $2 \times 2$ minor of $dX(p)$
	which has non-zero determinant. Suppose without loss of generality that one
	such minor is given by
	\[
		\begin{bmatrix}
			f_u(p) & f_v(p) \\
			g_u(p) & g_v(p) \\
		\end{bmatrix},
	\]
	and let $F(u, v) = (f(u, v), g(u, v))$. Then let $q' = F(p)$. As $dF(p)$
	has full rank, $F$ has a $C^{1}$ inverse $G_q$ about some neighbourhood
	$V_{q'}$ of $q'$ (in the topology of $\R^3$). Now let $\pi_{x, y} : \R^{3}
	\to \R^{2}$ be the projection onto the first two coordinates. Note that $q'
	= \pi(q)$. Then let $V_q = \pi^{-1}(V_{q'})$. As $\pi$ is continuous, $V_q$
	is an open set in $\R^3$. Moreover, for any $(u, v) \in X^{-1}(V_q)$ we
	have
	\[
		G_q \circ \pi \circ X(u, v)
		=
		G_q \circ (f(u, v), g(u, v))
		=
		G_q \circ F(u, v)
		=
		(u, v).
	\] 
	Hence $X^{-1}$ coincides with $G_q \circ \pi$ on $X(U) \cap V_q$,
	and we can set $\phi_q = G_q \circ \pi$.
\end{proof}

Many nice corollaries follow from this.

\begin{corollary}
	Any differentiable map between regular surfaces is locally the restriction
	of a differentiable map $\R^3 \to \R^3$. \\

	More specifically, let $M_1, M_2$ be regular surfaces and $\phi : M_1 \to
	M_2$ be a differentiable map. Then for every $q \in M_1$, there exist an
	open neighbourhood $V_q$ of $q$ in $\R^3$, and a differentiable function
	$f_q : V_q \to \R^3$ which restricts to $\phi$ on the open set $V_q \cap
	M_1$ in $M_1$.
\end{corollary}
\begin{proof}
	Let $Y : W \to Y(W) \subset M_2$ be a local parameterization of $M_2$ which
	contains $\phi(q)$. Let $X : U \to X(U)$ be a local parameterization of
	$M_1$ which contains $q$ such that $\phi(X(U)) \subset Y(W)$. Let $V_q$ be
	an open neighbourhood of $q$ in $\R^3$ such that there exists a
	differentiable functions $G_q : V_{q} \to \R^{2}$ which restricts to
	$X^{-1}$ on $M_1 \cap V_q$. Then define $f_{q} : V_q \to \R^3$ by
	\[
		f_q
		=
		\phi
		\circ
		X
		\circ
		G.
	\] 
	As $G$ restricts to $X^{-1}$ on $M_1 \cap V_q$, we see that $f_q$ restricts to
	$\phi$ on $M_1 \cap V_q$. Moreover, $f_q$ is $C^1$ as it's the composition
	\[
		f_q
		=
		Y
		\circ
		\left(Y^{-1}
		\circ
		\phi
		\circ
		X\right)
		\circ
		G
	\] 
	of the $C^1$ maps $Y,\, Y^{-1} \circ \phi \circ X, G$.
\end{proof}

\begin{corollary}

	Let $M_1, M_2$ be regular surfaces, and $\phi : M_1 \to M_2$ be a
	differentiable map. Let $\gamma : I \to M_1$ be a differentiable curve on
	$M_1$. Then $\phi \circ \gamma$ is a differentiable
	curve on $M_2$.

\end{corollary}
\begin{proof}

	Let $p \in I$ and $q = \gamma(p)$. Let $f_q : V_q \to \R^3$ be a $C^1$ map
	which restricts to $\phi$. Then $\phi \circ \gamma$ is locally given by
	$f_q \circ \restr{\gamma}{\gamma^{-1}(V_q)}$, which is $C^1$ at $p$. As $p$
	was an arbitrary point in $\gamma(I)$, it follows that $\phi \circ \gamma$
	is $C^1$ on all of its domain.

\end{proof}

\subsection*{Ex 4.7}

Let $f : \R^3 \to \R^3$ be the differentiable map given by
\[
	f : (x, y, z) \mapsto \left(x, \frac{1}{\sqrt{2}}y, \frac{1}{\sqrt{3}} z\right).
\] 
Then $\im(\restr{f}{S^2}) = M$, hence $\phi = \restr{f}{S^2}$ is a
differentiable map $S^2 \to M$. Moreover, as $f$ is injective, so is $\phi$,
and as $\phi$ is surjective, it is a diffeomorphism.

\subsection*{Ex 4.8}

We will show that $X^{-1}$ isn't continuous using only topological arguments.
We have that $X^{-1}$ is continuous if and only $X(V)$ is open in $M$ for every
open set $V \subseteq U$. Now, every open neighbourhood of $0 \in M$ contains
values of each quadrant of the $x, y$-lane. Meanwhile, if we let $V = \{(u, v)
\in U : -\epsilon < 0 < \epsilon \}$ for $\epsilon \leq \pi/2$, we have that
$X(V)$ only contains values in the upper right and lower left quadrants of the
$x, y$-plane. Hence $X^{-1}$ is not continuous. \\

To see that $M$ is not a regular surface, note that the tangent plane in
$(0,0,1/2)$ is $3$-dimensional.

\subsection*{Ex 4.9}

\begin{proof}[Proof of Proposition 4.33]

	Let $X : U \to X(U) \subset M_1$ be a local parameterization of $M_1$ at $p
	= X(0)$, and $Y : V \to Y(V) \subset M_2$ be a local parameterization of
	$M_2$ at $q = Y(0)$. Then let $F = Y^{-1} \circ \phi \circ X$. Then we saw
	in the proof of Proposition 4.32 that any vector $dX (a, b) \in T_p M_1$ is
	mapped by $d\phi_p$ to $T_p M_2$ according to
	\[
		d\phi_p
		:
		dX (a, b)
		\mapsto 
		dY(0) dF(0) (a b).
	\] 
	Thus $d\phi_p$ bijective only when $dF(0)$ is. Hence there is some
	neighbourhoods $U_0 \subset U, V_0 \subset V$ of $0$ where $F$ has a $C^1$
	inverse $F^{-1} : V_0 \to U_0$. We then obtain neighbourhoods $X(U_0),
	Y(V_0)$ where $\phi$ is invertible by
	\[
		\phi^{-1} = X \circ F^{-1} \circ Y^{-1},
	\] 
	and $\phi^{-1}$ is differentiable here as $F^{-1}$ is.

\end{proof}

\subsection*{Ex 4.10}

We have 
\[
	X_{\alpha}(r, \theta)
	=
	r \left(
		\sin(\alpha)
		\cos\left(\frac{\theta}{\sin(\alpha)}\right),
		\sin(\alpha)
		\sin\left(\frac{\theta}{\sin(\alpha)}\right),
		\cos(\alpha)
	\right),
\] 
and
\begin{align*}
	(X_{\alpha})_r
	&=
	\left(
		\sin(\alpha)
		\cos\left(\frac{\theta}{\sin(\alpha)}\right),
		\sin(\alpha)
		\sin\left(\frac{\theta}{\sin(\alpha)}\right),
		\cos(\alpha)
	\right) \\
	(X_{\alpha})_{\theta}
	&=
	\left(
		-\sin\left(\frac{\theta}{\sin(\alpha)}\right),
		\cos\left(\frac{\theta}{\sin(\alpha)}\right),
		0
	\right) \\
\end{align*}
It then follows that 
\begin{align*}
	E 
	&= 
	\sin^2(\alpha)
	\cos^2\left(\frac{\theta}{\sin(\alpha)}\right)
	+
	\sin^2(\alpha)
	\sin^2\left(\frac{\theta}{\sin(\alpha)}\right)
	+
	\cos^2(\alpha) \\
	&=
	\sin^2(\alpha)
	+
	\cos^2(\alpha) \\
	&=
	1, \\
	F
	&=
	-\sin(\alpha)
	\cos\left(\frac{\theta}{\sin(\alpha)}\right)
	\sin\left(\frac{\theta}{\sin(\alpha)}\right)
	+
	\sin(\alpha)
	\sin\left(\frac{\theta}{\sin(\alpha)}\right)
	\cos\left(\frac{\theta}{\sin(\alpha)}\right)
	+
	0 \\
	&=
	0, \\
	G
	&=
	\sin^2\left(\frac{\theta}{\sin(\alpha)}\right)
	+
	\cos^2\left(\frac{\theta}{\sin(\alpha)}\right)
	=
	1.
\end{align*}

Hence, the first fundamental form pulls back to $U$ as
\[
	ds^2 = du^2 + dv^2.
\] 

By staring at the formula for $X_{\alpha}$, we can convince ourselves that it
parameterises a punctured cone with angle $\alpha$ to the $z$-axis. Hence we
expect the $M$ to be the zero locust of 
\[
	f_{\alpha}(x, y, z)
	=
	\cos(\alpha)\sqrt{x^2 + y^2} - \sin(\alpha) z
\] 
for $f_{\alpha} : \R^3 \setminus \{0\} \to \R$.

\subsection*{Ex 4.11}

For distances to be preserved, intuitively it seems that we need the plane to
"wrap" around the cylinder multiple times. I.e, we guess that an isometric
parameterization $X : \R^2 \to M$ will have the form
\[
	X : (\theta, z) \mapsto (\cos(\omega \theta), \sin(\omega \theta), z)
\] 
for some $\omega \in \R$. Differentiating yields 
\begin{align*}
	X_{\theta}(\theta, z)
	&=
	( -\omega \sin(\omega \theta), \omega \cos(\omega \theta), 0), \\
	X_z(\theta, z)
	&=
	(0,0,1),
\end{align*}
and we see that
\[
	X_{\theta} \cdot X_{\theta} = 1, \,
	X_{\theta} \cdot X_{z} = 0, \,
	X_z \cdot X_z = 1.
\]
precisely when $|\omega| = 1$. \\

Now, $X$ is not bijective, but it is bijective on open strips of $\R^2$ that
are thinner than $2\pi$, and it has a continuous inverse on these strips.

\subsection*{Ex 4.13}

Let $\phi : \R^3 \to \R^3$ be a Euclidean motion
Then $\phi$ can be written as 
\[
	\phi(Z) = AZ + B
\] 
with $A^t A = I$. Consider $\phi$ as a differentiable map between regular
surfaces $\phi : M_1 \to M_2$. Then let $\gamma : I \to M_1$ be a
differentiable curve on $M_1$ with $\gamma(0) = p$ and $\gamma'(0) = (a, b)$.
Then 
\[
	d\phi_p
	:
	(a, b)
	=
	\gamma'(0)
	\mapsto
	\frac{d}{dt}
	\restr{(\phi \circ \gamma)}{0}
	=
	d\phi(\gamma(0)) \circ \gamma'(0)
	=
	A (a, b),
\] 
i.e $d\phi_p : (a, b) \mapsto A (a, b)$. We now have 
\begin{align*}
	\langle d\phi_p Z, d\phi_p Z \rangle 
	&=
	\langle AZ, AZ \rangle \\
	&=
	(AZ)^T AZ \\
	&=
	Z^T A^T A Z \\ 
	&=
	Z^T I Z \\
	&=
	Z^T Z \\
	&=
	\langle Z, Z \rangle.
\end{align*}

\subsection*{Ex 4.14}

We solve the exercise for the Catenoid only. We are given 
\[
	X : (u, v)
	\mapsto
	(\cosh(u) \cos(v), \cosh(u) \sin(v), u),
\]
and thus
\begin{align*}
	X_u
	&=
	(\sinh(u)\cos(v), \sinh(u)\sin(v), 1), \\
	X_v
	&=
	(-\cosh(u) \sin(v), \cosh(u) \cos(v), 0).
\end{align*} 
It follows that 
\begin{align*}
	\langle X_u, X_u \rangle
	&=
	\sinh^2(u) + 1 \\
	&=
	\cosh^2(u), \\
	\langle X_u, X_v \rangle
	&=
	-\sinh(u)\cos(v)\cosh(u)\sin(v)
	+
	\sinh(u)\sin(v)\cosh(u)\cos(v)
	+
	0 \\
	&=
	0, \\
	\langle X_v, X_v \rangle
	&=
	\cosh^2(u),
\end{align*}
hence $ds^2 = \cosh^2(u)(du^2 + dv^2)$. The Catenoid is rotationally symmetric in 
the $x,y$-plane, and the radius is given by $\cosh(u) = \cosh(z)$. It follows 
that it's the zero locust of 
\[
	f(x, y, z)
	=
	\sqrt{x^2 + y^2} - \cosh(z)
\] 

\subsection*{Ex 4.16}

Let $X : \to T^2$ be the parameterization given by 
\[
	X
	:
	(u, v)
	\mapsto
	\begin{bmatrix}
		\cos(v) & - \sin(v) & 0 \\
		\sin(v) & \cos(v) & 0 \\
		0 & 0 & 1 \\
	\end{bmatrix}
	\begin{bmatrix}
		R + r \cos(u) \\
		0 \\
		r\sin(u)
	\end{bmatrix}
	=
	\begin{bmatrix}
		\left(R + r \cos(u)\right)(\cos(v)) \\
		\left(R + r \cos(u)\right)(\sin(v)) \\
		r\sin(u)
	\end{bmatrix}.
\] 
Then 
\begin{align*}
	X_u
	&=
	\begin{bmatrix}
		- r \sin(u)\cos(v) \\
		- r \sin(u)\sin(v) \\
		r\cos(u)
	\end{bmatrix}, \\
	X_v
	&=
	\begin{bmatrix}
		-\left(R + r \cos(u)\right)\sin(v) \\
		\left(R + r \cos(u)\right)\cos(v) \\
		0
	\end{bmatrix},
\end{align*}
and
\begin{align*}
	\langle X_u, X_u \rangle
	&=
	r^2, \\
	\langle X_u, X_v \rangle
	&=
	0, \\
	\langle X_u, X_v \rangle
	&=
	\left(R + r \cos(u)\right)^2.
\end{align*}
Hence the area of $T^2$ is given by
\begin{align*}
	A(T^2)
	&=
	\int_{0}^{2\pi}
	\int_{0}^{2\pi}
	\sqrt{
		r^2\left(R + r \cos(u)\right)^2
	}
	dudv \\
	&=
	r
	\int_{0}^{2\pi}
	\int_{0}^{2\pi}
	\left|R + r \cos(u)\right|
	dudv \\
	&=
	2\pi r
	\int_{0}^{2\pi}
	\left|R + r \cos(u)\right|
	du \\
	&=
	2\pi r
	\int_{0}^{\pi}
	R + r \cos(u)
	du \\
	&=
	2\pi r
	\int_{0}^{\pi}
	R
	du \\
	&=
	4\pi^2 rR,
\end{align*}
where we used that $R > r$, hence $R + r\cos(u) > 0$.

\subsection*{Ex 5.1}

We will rely on some intuition from multivariate calculus, which tells us that
the gradient of a function $\R^3 \to \R$ at a point is orthogonal to the
tangent plane of the level surface at that point. I.e "gradients point out of
level surfaces." \\

Let $p \in M$, and $\gamma : I \to M$ be a differentiable curve on $M$
with $\gamma(0) = p$. Then $f(\gamma(0)) = q$, hence 
\[
	0
	=
	\restr{(f(\gamma(t)))'}{t = 0}
	=
	df(\gamma(0)) \cdot \gamma'(0).
\] 
I.e $df(p) \cdot v = 0$ for all $v \in T_{p}M$. Hence $N(p) =
\frac{df(p)}{|df(p)|}$ is a map $M \to S^2$, and it's differentiable since it's
the restriction of the $\R^3 \to \R^3$ $C^1$-map $p \mapsto
\frac{df(p)}{|df(p)|}$.

\subsection*{Ex 5.2}

First of all, let $\lambda_Z, \lambda_W$ be the eigenvalues 
\[
	S_p(Z) = \lambda_Z Z, \, 
	S_p(W) = \lambda_W W.
\] 
Then 
\begin{align*}
	\kappa_{n}(\theta)
	&=
	\langle S_p(Z(\theta)), Z(\theta) \rangle \\
	&=
	\langle \lambda_Z \cos(\theta) Z + \lambda_W \sin(\theta) W, \cos(\theta) Z + \sin(\theta) W \rangle \\
	&=
	\langle
		\lambda_Z \cos(\theta) Z,
		\cos(\theta) Z
	\rangle
	+
	\langle
		\lambda_Z \cos(\theta) Z,
		\sin(\theta) W
	\rangle
	+
	\langle
		\lambda_W \sin(\theta) W,
		\cos(\theta) Z 
	\rangle
	+
	\langle
		\lambda_W \sin(\theta) W,
		\sin(\theta) W
	\rangle \\
	&=
	\lambda_Z \cos^2(\theta)
	+
	\lambda_W \sin^2(\theta),
\end{align*}
and 
\begin{align*}
	\frac{1}{2\pi}
	\int_{0}^{2\pi}
	\kappa_n(\theta)
	d \theta
	&=
	\frac{1}{2\pi}
	\int_{0}^{2\pi}
	\lambda_Z \cos^2(\theta)
	+
	\lambda_W \sin^2(\theta),
	d \theta \\
	&=
	\frac{1}{2\pi}\left(
		\lambda_Z 
		\left[
			\frac{x}{2}
			+
			\frac{\sin(2x)}{4}
		\right]_{0}^{2\pi}
		+
		\lambda_W 
		\left[
			\frac{x}{2}
			-
			\frac{\sin(2x)}{4}
		\right]_{0}^{2\pi}
	\right) \\
	&=
	\frac{\lambda_Z + \lambda_W}{2}.
\end{align*}
Now $H(p)$ is half the trace of $S_p$, which is half the sum of the eigenvalues
of $S_p$, I.e $H(p) = \frac{\lambda_Z + \lambda_W}{2}$ and we are done.

\subsection*{Ex 5.3}

First of all, $K(p) = \det(S_p)$ is the product of the eigenvalues of $S_p$,
and $H(p) = \frac{1}{2} \text{trace}(S_p)$ is half the sum of the eigenvalues
of $S_p$. Now, the normal to any point of the sphere is that same point
normalized. I.e
\[
	S_p : x \mapsto \frac{x}{|x|} = \frac{x}{r},
\]
and it follows that $S_p = \frac{1}{r} I$, so both eigenvalues of $S_p$ are
$1/r$. Thus $H(p) = 1/r$ and $K(p) = 1/r^2$.

\subsection*{Ex 5.6}

Let $\Phi(x) = Ax + B$ be a rigid Euclidean motion. Then $\det(A) = 1$ and $A^t
A = I$. \\

Let $M$ be a regular surface, $p \in M$. Then $T_{\Phi(p)}\Phi(M) = AT_{p}M$
and $N_{\Phi(M)}(\Phi(p)) = A N_{M}(p)$ (here we use the fact that $\det(A) =
1$, since we need the new normal to have magnitude $1$). It follows for any
$\gamma : I \to M$ with $\gamma(0) = p$ and $\gamma'(0) = Z$ that
$\Phi(\gamma(t))' = AZ$ and
\begin{align*}
	S_{\Phi(M), \Phi(p)}(AZ)
	&=
	-\restr{\left(\frac{d}{dt}N_{\Phi(M)}(\Phi(\gamma(t)))\right)}{t = 0} \\
	&=
	-\restr{\left(\frac{d}{dt}AN_{M}(\gamma(t))\right)}{t = 0} \\
	&=
	-A\restr{\left(\frac{d}{dt}N_{M}(\gamma(t))\right)}{t = 0} \\
	&=
	-A dN_M(p) \\
	&=
	A S_p(\gamma).
\end{align*} 
If we now let $Z, W \in T_p M$ we get
\begin{align*}
	II_{\Phi(M), \Phi(p)}(\Phi(Z), \Phi(W))
	&=
	\langle S_{\Phi(M), \Phi(p)}(\Phi(Z)), \Phi(W)\rangle \\
	&=
	\langle S_{\Phi(M), \Phi(p)}(\Phi(Z)), \Phi(W)\rangle \\
	&=
	\langle AS_{M, p}Z, AW \rangle \\
	&=
	(S_{M, p}(Z))^t A^t A W \\
	&=
	(S_{M, p}(Z))^t W \\
	&=
	\langle S_{M, p}Z, W \rangle \\
	&=
	II_{M, p}(Z, W).
\end{align*}

\subsection*{Ex 5.7}


We have
\begin{align*}
	X_u(u, v)
	&=
	(\sinh(u)\cos(v), \sinh(u)\sin(v), 1) \\
	X_v(u, v)
	&=
	(-\cosh(u)\sin(v), \cosh(u)\cos(v), 0),
\end{align*}
and
\begin{align*}
	Y_u(u, v)
	&=
	(\cosh(u)\cos(v), \cosh(u)\sin(v), 0) \\
	Y_v(u, v)
	&=
	(-\sinh(u)\sin(v), \sinh(u)\cos(v), 1).
\end{align*}
For the normals we get,
\begin{align*}
	N_{X}(u, v)
	&=
	\frac{X_u \times X_v}{|X_u \times X_v|} \\
	&=
	\frac{(-\cosh(u)\cos(v), -\cosh(u)\sin(v), \cosh(u)\sinh(u))}{|X_u \times X_v|} \\
	&=
	\frac{(-\cosh(u)\cos(v), -\cosh(u)\sin(v), \cosh(u)\sinh(u))}{\sqrt{\cosh(u)(1 + \sinh^2(u))}} \\
	&=
	\frac{(-\cosh(u)\cos(v), -\cosh(u)\sin(v), \cosh(u)\sinh(u))}{\sqrt{\cosh^2(u)(1 + \sinh^2(u))}} \\
	&=
	\frac{(-\cosh(u)\cos(v), -\cosh(u)\sin(v), \cosh(u)\sinh(u))}{\sqrt{\cosh^4(u)}} \\
	&=
	\frac{(-\cos(v), -\sin(v), \sinh(u))}{\cosh(u)} \\
	&=
	\frac{1}{\cosh(u)}
	\begin{bmatrix}
		-\cos(v) \\
		-\sin(v) \\
		\sinh(u)
	\end{bmatrix},
\end{align*}
and
\begin{align*}
	N_{Y}(u, v)
	&=
	\frac{Y_u \times Y_v}{|Y_u \times Y_v|} \\
	&=
	\frac{(\cosh(u)\sin(v), -\cosh(u)\cos(v), \cosh(u)\sinh(u))}{|Y_u \times Y_v|} \\
	&=
	\frac{(\sin(v), -\cos(v), \sinh(u))}{|\cosh(u)|} \\
	&=
	\frac{1}{\cosh(u)}
	\begin{bmatrix}
		\sin(v) \\
		-\cos(v) \\
		\sinh(u)
	\end{bmatrix},
\end{align*}
We then have
\begin{align*}
	dN_X(u, v)
	&=
	\begin{bmatrix}
		\frac{\cos(v)\sinh(u)}{\cosh^2(u)} & \frac{\sin(v)}{\cosh(u)} \\
		\frac{\sin(v)\sinh(u)}{\cosh^2(u)} & \frac{-\cos(v)}{\cosh(u)} \\
		\frac{\cosh(u)^2 - \sinh^2(u)}{\cosh^2(u)} & 0
	\end{bmatrix} \\
	&=
	\frac{1}{\cosh^2(u)}
	\begin{bmatrix}
		\cos(v)\sinh(u) & \sin(v)\cosh(u) \\
		\sin(v)\sinh(u) & -\cos(v)\cosh(u) \\
		1 & 0
	\end{bmatrix} \\
	dN_Y(u, v)
	&=
	\frac{1}{\cosh^2(u)}
	\begin{bmatrix}
		-\sin(v)\sinh(u) & \cos(v)\cosh(u) \\
		\cos(v)\sinh(u) & \sin(v)\cosh(u) \\
		1 & 0
	\end{bmatrix} \\
\end{align*}
With $p = (p_u, p_v)$, we get
\begin{align*}
	S_{X, p}(a X_u(p) + b X_v(p))
	&=
	\frac{1}{\cosh^2(p_u)}
	\begin{bmatrix}
		\cos(p_v)\sinh(p_u) & \sin(p_v)\cosh(p_u) \\
		\sin(p_v)\sinh(p_u) & -\cos(p_v)\cosh(p_u) \\
		1 & 0
	\end{bmatrix}
	\begin{bmatrix}
		a \\
		b
	\end{bmatrix} \\
	&=
	\begin{bmatrix}
		a\cos(p_v)\sinh(p_u) b\sin(p_v)\cosh(p_u) \\
		a\sin(p_v)\sinh(p_u) -b\cos(p_v)\cosh(p_u) \\
		a
	\end{bmatrix} \\
	&=
	aX_u(p) - bX_v(p),
\end{align*}
hence in the basis of $X_u, X_v$, we can write $S_{X, p}$ as the matrix
$\begin{bmatrix} 1 & 0 \\ 0 & -1 \end{bmatrix}$. The case for $Y$ is similar with
\begin{align*}
	S_{Y, p}(a Y_u(p) + b Y_v(p))
	&=
	\frac{1}{\cosh^2(p_u)}
	\begin{bmatrix}
		-\sin(p_v)\sinh(p_u) & \cos(p_v)\cosh(p_u) \\
		\cos(p_v)\sinh(p_u) & \sin(p_v)\cosh(p_u) \\
		1 & 0
	\end{bmatrix}
	\begin{bmatrix}
		a \\
		b
	\end{bmatrix} \\
	&=
	\begin{bmatrix}
		-a\sin(p_v)\sinh(p_u) + b\cos(p_v)\cosh(p_u) \\
		a\cos(p_v)\sinh(p_u)  + b\sin(p_v)\cosh(p_u) \\
		a
	\end{bmatrix} \\
	&=
	aY_v(p) + bY_u(p),
\end{align*}
so in the basis $Y_u, Y_v$, we can express $S_{Y, p}$ as $\begin{bmatrix} 0 & 1
\\ 1 & 0 \end{bmatrix}$, which in turn has the eigenvalues $-1, 1$ and
eigenvectors $[-1\, 1],\, [1\, 1]$. \\

The principal curvatures of both $X$ and $Y$ are $1, -1$.

\subsection*{Ex 5.8}

If $M$ is compact, it is also bounded. Let
\[
	R
	=
	\sup_{p \in M}(|p|).
\]
Then $M$ is contained in $S^2_R$ except for some points which coincide with
$S^2_R$. At these points, no curve on $M$ has any tangent with a component in
the radial direction, hence the tangent planes at these points must also
coincide with the tangent planes of $S^2_R$. As all of $S^2_R$, hence all of
$M$, lies on one side of these tangent planes, it follows that $K$ is positive
here.

\subsection*{Ex 5.9}

We have 
\begin{align*}
	X_s 
	&=
	\gamma'(s)
	+
	r(\cos(\theta) n'(s) + \sin(\theta) b'(s)) \\
	&=
	t(s)
	+
	r\cos(\theta)(-\kappa(s) t(s) + \tau(s)b(s))
	-
	r\sin(\theta)\tau(s)n(s) \\
	&=
	(1 - r\cos(\theta)\kappa(s))t(s)
	-
	r\sin(\theta)\tau(s)n(s)
	+
	r\cos(\theta)\tau(s)b(s) \\
	X_{\theta}
	&=
	-r\sin(\theta)n(s) + r\cos(\theta)b(s),
\end{align*} 
from which it follows that the outer normal,
\begin{align*}
	N(X(s, \theta))
	&=
	-\frac{X_\theta(s, \theta) \times X_{s}(s, \theta)}
	{|X_\theta(s, \theta) \times X_{s}(s, \theta)|} \\
	&=
	\frac{X_s(s, \theta) \times X_{\theta}(s, \theta)}
	{|X_s(s, \theta) \times X_{\theta}(s, \theta)|} \\
	&=
	\frac{
		-(1 - r\cos(\theta)\kappa(s))
		r\sin(\theta)
		t(s) \times n(s)
		+
		(1 - r\cos(\theta)\kappa(s))
		r\cos(\theta)
		t(s) \times b(s)
		-
		r\sin(\theta)\tau(s)
		r\cos(\theta)
		n(s) \times b(s)
		-
		r\cos(\theta)\tau(s)
		r\sin(\theta)
		b(s) \times n(s) 
	}
	{|X_s(s, \theta) \times X_{\theta}(s, \theta)|} \\
	&=
	\frac{
		-(1 - r\cos(\theta)\kappa(s))
		r\sin(\theta)
		b(s)
		-
		(1 - r\cos(\theta)\kappa(s))
		r\cos(\theta)
		n(s)
		-
		r\sin(\theta)\tau(s)
		r\cos(\theta)
		t(s)
		+
		r\cos(\theta)\tau(s)
		r\sin(\theta)
		t(s)
	}
	{|X_s(s, \theta) \times X_{\theta}(s, \theta)|} \\
	&=
	\frac{
		-(1 - r\cos(\theta)r\kappa(s))
		\left(\cos(\theta)n(s) + \sin(\theta)b(s)\right)
	}
	{|X_s(s, \theta) \times X_{\theta}(s, \theta)|} \\
	&=
	\cos(\theta)n(s) + \sin(\theta)b(s),
\end{align*}
points "out of the tube", in the radial direction with respect to $\theta$.
Intuitively, we then expect $X_{\theta}$ to be a principal direction of the
shape operator, since as we move along a given circle for a fixed $s$, mk, the
normal points in exactly the same direction as $X$, and both the normal and $X$
have constant magnitude, so we expect their derivatives to point in the same
directions as well. From this it would follow that $t(s)$ is the other
principal direction, as it is orthogonal to both $n$ and $X_{\theta}$. We now
calculate the principal curvatures in the $t(s)$ and $X_\theta$ directions, and
will along the way formally verify that these are indeed principal directions.
\\

Fix some point $p = (p_s, p_\theta) \in \R^2$ in parameter space. To calculate
the principal curvatures at $X(p) \in M$, we need to compute $S_p(t(p)) =
-dN(p)t(p)$ and $S_p(X_\theta(p)) = -dN(p)X_{\theta}(p)$. We don't want to
choose a basis and figure out how to represent $-dN(p)$ in this basis, so
instead we'll make use of the definition of the differential of differentiable
map, and produce curves $\phi_p, \psi_p : \R \to M$ which satisfy $\phi_p(0) =
\psi_p(0) = p$ and $\phi_p'(0) = X_\theta(p),\, \psi_p'(0) = t(p_s)$, and make
use of the definition of $dN(p)$ and calculate $\restr{(N \circ \phi_p)'}{0}$
and $\restr{(N \circ \psi_p)'}{0}$. Moreover, as we only have a formula for $N
\circ X$, it will help to produce factorizations $\alpha_p, \beta_p : \R \to
\R^2$ such that 
\[
	\phi_p = X \circ \alpha_p, \,
	\psi_p = X \circ \beta_p.
\] \\

We begin with the easier curve. Let $\alpha_p : \R \to \R^2$ be given by 
\[
	\alpha_p : t \to \left(p_s, p_{\theta} + t\right).
\] 
Then let $\phi_p = X \circ \alpha_p$, which is given by 
\[
	\phi_p 
	: 
	t 
	\mapsto 
	\phi(p_s) 
	+ 
	r\left(
		\cos\left(p_{\theta} + t\right)N(p_s)
		+
		\sin\left(p_{\theta} + t\right)B(p_s)
	\right).
\]
Then
\begin{align*}
	\phi_p'(0) 
	&= 
	0
	+
	r
	\left(
		-\sin(p_{\theta})N(p_s)
		+
		\cos(p_{\theta})B(p_s)
	\right) 
	=
	X_{\theta}(p),
\end{align*}
and 
\begin{align*}
	S_{p}(X_{\theta})
	&=
	-
	\restr{
		\left(
			N(\phi_p(t))
		\right)'
	}{0} \\
	&=
	-
	\restr{
		\left(
			N(X(\alpha_p(t)))
		\right)'
	}{0} \\
	&=
	-
	\restr{
		\left(
			\cos\left(p_{\theta} + t \right)N(p_s)
			+
			\sin\left(p_{\theta} + t \right)B(p_s)
		\right)'
	}{0} \\
	&=
	-
	\left(
		-\sin(p_{\theta})N(p_s)
		+
		\cos(p_{\theta})B(p_s)
	\right) \\
	&=
	\frac{-1}{r} X_{\theta}(p),
\end{align*}
and we see that $X_{\theta}(p)$ is indeed an eigenvector to $S_p$, hence a
principal direction at $p$, and the corresponding principal curvature is
$-1/r$. \\

The other curve is harder, as we need to do some algebra to figure out what
$\beta_p$ should be. Let $a, b \in \R$ and $\beta_p : \R \to \R^2$ be given by 
\[
	\beta_p : t \mapsto (p_s + as, p_{\theta} + bs),
\] 
and $\psi_p = X \circ \beta_p$. Then 
\begin{align*}
	\psi_p'(0)
	&=
	a\gamma'(p_s)
	+
	r\left(
		a \cos(p_\theta)N'(p_s)
		+
		a \sin(p_{\theta})B'(p_s)
		-
		b \sin(p_\theta)N(p_s)
		+
		b \cos(p_\theta)B(p_s)
	\right) \\
	&=
	a\gamma'(p_s)
	+
	r\left(
		a \cos(p_\theta)(-\kappa(p_s)T(p_s) + \tau(p_s)B(p_s))
		+
		a \sin(p_{\theta})(-\tau(p_s)N(p_s))
		-
		b \sin(p_\theta)N(p_s)
		+
		b \cos(p_\theta)B(p_s)
	\right) \\
	&=
	T(p_s)a(1 - r\cos(p_\theta)\kappa(p_s))
		+
	N(p_s)
	r\left(
		-a \sin(p_{\theta}) \tau(p_s)
		-
		b \sin(p_\theta)
	\right)
	+
	B(p_s) 
	r\left(
		a\cos(p_\theta)\tau(p_s)
		+
		b\cos(p_{\theta})
	\right),
\end{align*} 
and we see that for $\psi'(0)$ to point in the $T(p_s)$ direction, we need 
\begin{align*}
	a \sin(p_{\theta}) \tau(p_s) + b \sin(p_\theta)
	&= 
	0, \\
	a\cos(p_\theta)\tau(p_s)
	+
	b\cos(p_{\theta})
	&=
	0.
\end{align*}
We simplify life for ourselves by setting $a = 1, b = -\tau(p_s)$, and
observing that $\psi'_p(0) \not = t(p_s)$, but instead we now have $\psi'_p(0)
= (1 - r\cos(p_\theta)\kappa(p_s))t(p_s)$. Now note that
\begin{align*}
	S_{p}((1 - r\cos(p_\theta)\kappa(p_s))t(p_s))
	&=
	-
	\restr{
		\left(
			N(\psi_p(t))
		\right)'
	}{0} \\
	&=
	-
	\restr{
		\left(
			N(X(\beta_p(t)))
		\right)'
	}{0} \\
	&=
	-
	\restr{
		\left(
			\cos\left(p_{\theta} - \tau(p_s)t \right) N(p_s + t)
			+
			\sin\left(p_{\theta} - \tau(p_s)t \right) B(p_s + t)
		\right)'
	}{0} \\
	&=
	-
	\left(
		\tau(p_s)\sin(p_{\theta})N(p_s)
		+
		\cos(p_\theta)\dot{N}(p_s)
		-
		\tau(p_s)\cos(p_{\theta})B(p_s)
		+
		\sin(p_{\theta})\dot{B}(p_s)
	\right) \\
	&=
	-
	\left(
		\tau(p_s)\sin(p_{\theta})N(p_s)
		-
		\kappa(p_s)\cos(p_\theta)T(s)
		+
		\tau(p_s)\cos(p_\theta)B(p_s)
		-
		\tau(p_s)\cos(p_{\theta})B(p_s)
		-
		\tau(p_s)\sin(p_{\theta})N(s)(p_s)
	\right) \\
	&=
	\kappa(p_s)\cos(p_{\theta})T(s),
\end{align*}
and we see that the second principal curvature is
$\frac{\kappa(p_s)\cos(p_\theta)}{1 - r\cos(p_\theta)\kappa(p_s)}$. \\


\subsection*{Ex 5.10}

If we assume that $N \circ X$ is injective, this is just the definition of a
surface integral.

\subsection*{Ex 5.11}

Suppose that $M$ is a regular surface without a boundary contained in $U$. Then
let $k \in \R$ be the minimum real number with the property that $M$ intersects
$k + \partial U$. Then $M \subset k + \overline{U}$, where $\overline{U} = U
\cup \partial U$. In the points where $M$ intersects $k + \partial U$, we have
that $M$ is fully contained on one side of $k + \partial U$, and hence has the
same tangent planes as $\partial U$ here. But $\partial U$ has positive
Gaussian curvature, hence non-zero mean curvature. 

\subsection*{Ex 6.1}

In Example 5.11, it is shown that given a surface of revolution $M$
of a curve parameterized by arclength $\gamma(t) = (r(t), 0, z(t))$,
the curvature of $M$ is given by 
\[
	K = - \frac{\ddot{r}(u)}{r(u)}.
\] 

The surface given here may be given as a surface of revolution of the curve
$\gamma(t) = t(\sin(\alpha), 0, \cos(\alpha))$, which has a vanishing second
derivative. Hence $K = 0$.

\subsection*{Ex 6.2}

Let $X : U \to X(U) \subset M$ be an orthogonal local parameterization.
Then an orthonormal basis for its tangent plane is given by
\[
	Z = \frac{X_u}{\sqrt{E}},\, 
	W = \frac{X_v}{\sqrt{G}}.
\] 
I.e, there are $a, b : U \to \R$ only depending on $E, G$ such that $Z = a
X_u,\, W = b X_W$. No let $N : U \to S^2$ be a local Gauss map given by
\[
	N 
	= 
	\frac{X_u \times X_v}{|X_u \times X_v|}
	=
	Z \times W.
\] 
Applying Lemma 6.3 Yields that 
\[
	K 
	=
	\frac{\langle Z_u, W \rangle_v}{\sqrt{EG}}
	-
	\frac{\langle Z_v, W \rangle_u}{\sqrt{EG}},
\] 
and expanding the first of these terms gives
\begin{align*}
	\langle Z_u, W \rangle
	&=
	\langle a_uX_u + aX_{uu}, bX_v \rangle \\
	&=
	\langle a_uX_u, bX_v \rangle +
	\langle aX_{uu}, bX_v \rangle \\
	&=
	\langle aX_{uu}, bX_v \rangle,
\end{align*}
whilst simultaneously
\[
	F_u - \frac{1}{2}E_v
	=
	\langle X_{uu}, X_v\rangle
	+
	\langle X_{u}, X_{vu} \rangle
	-
	\langle X_uv, X_u \rangle 
	=
	\langle X_{uu}, X_v\rangle,
\] 
hence 
\[
	\langle Z_u, W \rangle
	=
	\frac{-ab}{2} E_v.
\] 
Similarly, 
\begin{align*}
	\langle Z_v, W \rangle
	&=
	\langle a_vX_u + aX_{uv}, bX_v \rangle \\
	&=
	\langle a_vX_u, bX_v \rangle +
	\langle aX_{uv}, bX_v \rangle \\
	&=
	\langle aX_{uv}, bX_v \rangle \\
	&=
	\frac{ab}{2} G_u.
\end{align*}
Finally, combining all of these expressions and using $a = 1/\sqrt{E},\, b =
1/\sqrt{G}$, we get
\begin{align*}
	K
	&=
	\frac{\langle Z_u, W \rangle_v}{\sqrt{EG}}
	-
	\frac{\langle Z_v, W \rangle_u}{\sqrt{EG}} \\
	&=
	\frac{-1}{2 \sqrt{EG}}
	\left(
		\left(\frac{E_v}{\sqrt{EG}}\right)_v
		+
		\left(\frac{G_u}{\sqrt{EG}}\right)_u
	\right),
\end{align*}
and we are done.

\subsection*{Ex 6.3}

This is a direct consequence of Exercise 6.3. If $E = G$, the formula above becomes
\begin{align*}
	K
	&=
	\frac{-1}{2 \sqrt{EG}}
	\left(
		\left(\frac{E_v}{\sqrt{EG}}\right)_v
		+
		\left(\frac{G_u}{\sqrt{EG}}\right)_u
	\right) \\
	&=
	\frac{-1}{2 E}
	\left(
		\left(\frac{E_v}{E}\right)_v
		+
		\left(\frac{E_u}{E}\right)_u
	\right) \\
	&=
	\frac{-1}{2 E}
	\left(
		\left(\log(E)_v\right)_v
		+
		\left(\log(E)_u\right)_u
	\right) \\
	&=
	\frac{-1}{2 E}
	\left(
		\log(E)_{vv}
		+
		\log(E)_{uu}
	\right).
\end{align*}

Now suppose that $E = \frac{4}{(1 + u^2 + v^2)^2}$. Then
\begin{align*}
	K
	&=
	\frac{-(1 + u^2 + v^2)^2}{8}
	\left(
		\log\left(
			\frac{4}{(1 + u^2 + v^2)^2}
		\right)_{uu}
		+
		\log\left(
			\frac{4}{(1 + u^2 + v^2)^2}
		\right)_{vv}
	\right) \\
	&= 
	\frac{-(1 + u^2 + v^2)^2}{8}
	\left(
		\left(
			\frac{(1 + u^2 + v^2)^2}{4}
			\cdot
			\frac{-16u}{(1 + u^2 + v^2)^3}
		\right)_{u}
		+
		\left(
			\frac{(1 + u^2 + v^2)^2}{4}
			\cdot
			\frac{-16v}{(1 + u^2 + v^2)^3}
		\right)_{v}
	\right) \\
	&=
	\frac{-(1 + u^2 + v^2)^2}{8}
	\left(
		\left(
			\frac{-4u}{1 + u^2 + v^2}
		\right)_{u}
		+
		\left(
			\frac{-4v}{1 + u^2 + v^2}
		\right)_{v}
	\right) \\
	&=
	\frac{-(1 + u^2 + v^2)^2}{8}
	\left(
		\frac{-4(1 + u^2 + v^2) + 8u^2}{(1 + u^2 + v^2)^2}
		+
		\frac{-4(1 + u^2 + v^2) + 8v^2}{(1 + u^2 + v^2)^2}
	\right) \\
	&=
	(1 + u^2 + v^2) - u^2 - v^2 \\
	&=
	1.
\end{align*}

Now suppose that $E = \frac{4}{(1 - u^2 - v^2)^2}$. Then
\begin{align*}
	K
	&=
	\frac{-(1 - u^2 - v^2)^2}{8}
	\left(
		\log\left(
			\frac{4}{(1 - u^2 - v^2)^2}
		\right)_{uu}
		+
		\log\left(
			\frac{4}{(1 - u^2 - v^2)^2}
		\right)_{vv}
	\right) \\
	&= 
	\frac{-(1 - u^2 - v^2)^2}{8}
	\left(
		\left(
			\frac{(1 - u^2 - v^2)^2}{4}
			\cdot
			\frac{16u}{(1 - u^2 - v^2)^3}
		\right)_{u}
		+
		\left(
			\frac{(1 - u^2 - v^2)^2}{4}
			\cdot
			\frac{16v}{(1 - u^2 - v^2)^3}
		\right)_{v}
	\right) \\
	&=
	\frac{-(1 - u^2 - v^2)^2}{8}
	\left(
		\left(
			\frac{4u}{1 - u^2 - v^2}
		\right)_{u}
		+
		\left(
			\frac{4v}{1 - u^2 - v^2}
		\right)_{v}
	\right) \\
	&=
	\frac{-(1 - u^2 - v^2)^2}{8}
	\left(
		\frac{4(1 - u^2 - v^2) + 8u^2}{(1 - u^2 - v^2)^2}
		+
		\frac{4(1 - u^2 - v^2) + 8v^2}{(1 - u^2 - v^2)^2}
	\right) \\
	&=
	-(1 - u^2 - v^2) - u^2 - v^2 \\
	&=
	-1.
\end{align*}
Finally, suppose that $E = 1/u^2$. Then
\begin{align*}
	K
	&=
	-\frac{u^2}{2}
	\left(
		\log\left(
			1/u^2
		\right)_{uu}
		+
		\log\left(
			1/u^2
		\right)_{vv}
	\right) \\
	&= 
	-\frac{u^2}{2}
	\log\left( 1/u^2 \right)_{uu} \\
	&=
	-\frac{u^2}{2}
	\left(\frac{-2}{u^3} u^2\right)_u \\
	&=
	-\frac{u^2}{2}
	\left(\frac{-2}{u}\right)_u \\
	&=
	-\frac{u^2}{2}
	\frac{2}{u^2} \\
	&=
	-1.
\end{align*}

\subsection*{Ex 6.4}

The differential of $X$ is given by
\[
	dX
	=
	\begin{bmatrix}
		-\sin(u) & 0 \\
		\cos(u) & 0 \\
		0 & -\sin(u) \\
		0 & \cos(u) \\
	\end{bmatrix},
\]
and as $dX dX^T = I$, we have for all $v, w \in \R^2$
\[
	\langle dX v, dX v \rangle
	=
	v^T dX^T dX v
	=
	v^T v
	=
	\langle v, v \rangle.
\] 
Hence $X$ is an isometry. \\

This tells us nothing about the Gaussian curvature of $M$, since we haven't
defined Gaussian curvature for surfaces embedded in $\R^4$. Assuming such a
definition, we find that either Theorema Egregium or Theorem 5.13 doesn't apply
to Gaussian curvature of surfaces embedded in $\R^4$. \\

\subsection*{Ex 7.1}

We have that $M$ is a surface of revolution of the curve $\gamma(s) = (r(s),
z(s))$ with $r(s) = 1, z(s) = s$. Then $\gamma$ is parameterized by arclength,
and it follows that with the parameterization
\[
	X : (s, \theta) \mapsto 
	\begin{bmatrix}
		\cos(\theta) \\
		\sin(\theta) \\
		s
	\end{bmatrix},
\] 
we have
\[
	E = 1, \,
	F = 0, \,
	G = r(s)^2 = 1, \,
\]
and so the Geodesic equations become
\begin{align*}
	\ddot u
	&=
	0, \\
	\ddot v
	&=
	0.
\end{align*}
I.e all Geodesics on $M$ are of the form $\gamma = X \circ \alpha$ with $\alpha
= (at + b, ct + d)$, hence are all lines wrapped around the cylinder.

\subsection*{Ex 7.2}

First, any straight line is a geodesic, since $\ddot \gamma = 0$ for all
straight lines $\gamma$. Now note that the lines $y = z$ and $y = -z$ in the
plane $x = 1$ line on $M$, and hence are geodesics through $p$. \\ 

Two more geodesics are yielded by noting that $M$ is a surface of revolution,
and $r = 1$ is a critical point of the radius $r(s)$. Hence both the meridian
$s \mapsto (s^2 + 1, 0, s^2)$ and the parallel $\theta \mapsto (\cos(\theta),
\sin(\theta), 0)$ are geodesics.

\subsection*{Ex 7.3}

First, any straight line is a geodesic, since $\ddot \gamma = 0$ for all
straight lines $\gamma$. Now note that the $x$- and $y$-axes lie on $M$, as
well as the lines in the $xy$-plane with $x = y$ and $x = -y$. All these lines
are geodesics, and they all pass through the origin.


\subsection*{Ex 7.5}

The first and second derivatives of $\gamma_{\theta}$ are given by
\begin{align*}
	\gamma_\theta'(t)
	&=
	\left(
		\cos(\theta), 
		\sin(\theta),
		\cos(\theta)
		\cos(t \cos(\theta))
		\sin(t \sin(\theta))
		+
		\sin(\theta)
		\sin(t \cos(\theta))
		\cos(t \sin(\theta))
	\right), \\
	\gamma_\theta''(t)
	&=
	\left(
		0, 
		0,	
		2\cos(\theta)
		\sin(\theta)
		\cos(t \cos(\theta))
		\cos(t \sin(\theta))
		-
		\cos^2(\theta)
		\sin(t \cos(\theta))
		\sin(t \sin(\theta))
		-
		\sin^2(\theta)
		\sin(t \cos(\theta))
		\sin(t \sin(\theta))
	\right) \\
	&=
	\left(
		0, 
		0,	
		2\cos(\theta)
		\sin(\theta)
		\cos(t \cos(\theta))
		\cos(t \sin(\theta))
		-
		\sin(t \cos(\theta))
		\sin(t \sin(\theta))
	\right).
\end{align*}
The geodesic curvature of $\gamma_{\theta}$,
\begin{align*}
	\kappa_g
	&=
	\langle \gamma''_{\theta}, N \times \gamma_\theta' \rangle \\
	&=
	\langle N, \gamma_\theta' \times \gamma''_{\theta} \rangle
\end{align*} 
is zero precisely when $N$ lies in the plane $L_{\theta}(s)$ spanned by $\gamma_\theta',
\gamma_\theta''$ at all points along $\gamma$. \\

We have 
\begin{align*}
	X_u &= (1, 0, \cos(u)\sin(v)), \\
	X_v &= (0, 1, \sin(u)\cos(v)), 
\end{align*} 
and so
\begin{align*}
	X_u \times X_v
	&=
	(-\cos(u)\sin(v), -\sin(u)\cos(v), 1),
\end{align*}
from which we see that $N(\gamma_\theta(s))$ lies in $L_\theta(s)$ if and only if 
there exist $c \in \R \setminus 0$ such that
\begin{align*}
	\cos(\theta)
	&=
	c \cos(t\cos(\theta))\sin(t\sin(\theta)) \\
	\sin(\theta)
	&=
	c \sin(t\cos(\theta))\cos(t\sin(\theta)) \\
\end{align*} 
for all $t$. The expressions on the left are constant, and the expressions on
the right are constant only when $\theta = k\pi/2$ for $k \in \N$. We've shown
that $\gamma_\theta$ is a geodesic only when $\theta$ assume such values, and
to see that all such values give rise to geodesics, $\gamma_\theta$ is a
straight line for such values of $\theta$, and straight lines are geodesics on
any surface they lie on.

\subsection*{Ex 7.8}

This exercise can be simplified by realising that $M$ is a Mobius strip, and
that the curve is a circle. It follows immediately that $\gamma$ isn't a
geodesic, since $\gamma$ has a constant curvature in the radial direction,
whilst almost all tangent planes of $M$ have are not orthogonal to the radial
direction. Indeed, consider the plane at $p = X(0, 0) = (2, 0, 0)$. Here we
have 
\[
	X_v(u, v) = (\cos(u/2)\cos(u), \cos(u/2)\sin(u), \sin(u/2)), \,
\] 
and
\[
	X_v(0,0)
	=
	(1, 0, 0)
\] 
which is not orthogonal to $p$. \\

It's also immediate that $M$ is not orientable. To see this algebraically,
assume towards a contradiction that $N : M \to S^2$ is a Gauss map. The Gauss
map is determined up to sign, and we can assume without loss of generality that
$N = \frac{X_u \times X_v}{|X_u \times X_v|}$. We will show that $N$ is $4\pi$
periodic in $u$, whilst $X$ is $2\pi$-periodic - a contradiction to $N$ being a
differentiable map between regular surfaces, as such maps send closed curves to
closed curves. \\

Let's now show that $N$ is $4\pi$-periodic. First of, here is $X_u$:
\begin{align*}
	X_u
	=& \\
	 &2(-\sin(u), \cos(u), 0)  \\
	 &+
	\frac{v}{2}\cos(u/2)(0, 0, 1) \\
	 &-
	\frac{v}{2}\sin(u/2)(\cos(u), \sin(u), 0) \\
	 &+
	v \sin(u/2) (-\sin(u), \cos(u), 0).
\end{align*}
Note that both $X_v$ and $X_u$ are $4\pi$-periodic so $N$ must at least be
$4\pi$ periodic. To show that it isn't $2\pi$-periodic, we will show that $N(0,
0) \not = N(2\pi, 0)$. We have 
\begin{align*}
	X_u(0, 0) &= (0, 2, 0) \\
	X_v(0, 0) &= (1, 0, 0) \\
	X_u(2\pi, 0) &= (0, 2, 0) \\
	X_v(2\pi, 0) &= (-1, 0, 0) \\
\end{align*} 
Hence 
\begin{align*}
	X_u(0, 0) \times X_v(0, 0)
	&=
	(0, 0, -2) \\
	X_u(2\pi, 0) \times X_v(2\pi, 0)
	&=
	(0, 0, 2), \\
\end{align*}
and these clearly don't normalize to the same vector, and we are done.

\subsection*{Ex 7.11}

We are given a torus and are asked to find the minimal radial distance of a
geodesic $\gamma$ satisfying certain initial conditions. As the torus is a
surface of revolution, we are free to use the theorem of Clairaut, which says
that $r(s)\sin(\theta(s)) = C \in \R$, where $r(s)$ is the radial distance of
$\gamma(s)$, and $\theta(s)$ is the angle between $\dot\gamma(s)$ and the
meridian through $\gamma(s)$. In our case, we are given the initial conditions
$r(0) = 3$ and $\sin(\theta(0)) = 1/\sqrt{2}$, hence $r(s)\sin(\theta(s)) =
3/\sqrt{2}$ for all $s$. \\

We now claim that a curve on the torus will always either
\begin{itemize}
	\item pass through the inner equator, or
	\item be tangentially orthogonal to a meridian at some point.
\end{itemize}

But there is no $\theta$ such that $2 = \frac{3}{\sin(\theta)\sqrt{2}}$, hence
$\gamma$ will be tangentially orthogonal to a meridian at some point. It
follows from the formula that $r(s)$ is minimized when $\sin(\theta(s))$ is
maximized. In our case, the minimum of $\sin$ is obtained at the point where
$\dot\gamma$ is orthogonal to a meridian, hence $\inf(r(s)) = 3/\sqrt{2}$, and
$\inf(r(s)^2) = 9/2$.


\subsection*{Ex 8.1}

We will calculate the integral of the curvature in the region where $x \geq 0$,
and then multiply our result by $2$. This patch of $M$ is enclosed by two
half-circle parallels at $z = -1$ and $z = 1$, and two meridians at $x = 0$.
The two meridians are geodesics and do not contribute to the Gauss-Bonnet
integral. What does contribute are the two parallels as parameterized by arclength, 
\begin{align*}
	\gamma_1 
	&: 
	(0, \pi)  
	\to 
	M \\
	\gamma_1 
	&: 
	\theta 
	\mapsto 
	\left(
		-\sqrt{2}\cos\left(\frac{\theta}{\sqrt{2}}\right), 
		\sqrt{2}\sin\left(\frac{\theta}{\sqrt{2}}\right), 
		1
	\right),
\end{align*} 
and
\begin{align*}
	\gamma_2 
	&: 
	(0, \pi)  
	\to 
	M \\
	\gamma_2 
	&: 
	\theta 
	\mapsto 
	\left(
		\sqrt{2}\cos\left(\frac{\theta}{\sqrt{2}}\right), 
		\sqrt{2}\sin\left(\frac{\theta}{\sqrt{2}}\right), 
		-1
	\right),
\end{align*}
and we wish to compute their geodesic curvatures $\kappa_1, \kappa_2$. \\

To do this, first note that the $M = f^{-1}(\{1\})$ where $f = x^2 + y^2 -
z^2$, and so the gradient $\Delta f = (2x, 2y, -2z)$ is a normal to $M$. Thus,
\[
	N : (x, y, z) \mapsto \frac{1}{\sqrt{x^2 + y^2 + z^2}}(x, y, -z)
\] 
is a Gauss map on $M$, and we see that this choice of Gauss map makes $\gamma$
positively oriented. Now,
\begin{align*}
	\gamma_1'(\theta) 
	&= 
	\left(
		\sin\left(\frac{\theta}{\sqrt{2}}\right), 
		\cos\left(\frac{\theta}{\sqrt{2}}\right), 
		0
	\right), \\
	\gamma_1''(\theta) 
	&= 
	\frac{1}{\sqrt{2}}\left(
		\cos\left(\frac{\theta}{\sqrt{2}}\right), 
		-\sin\left(\frac{\theta}{\sqrt{2}}\right), 
		0
	\right), \\
\end{align*}
and so 
\begin{align*}
	\kappa_1
	&=
	\langle \gamma_1'', N \times \gamma_1' \rangle \\
	&=
	\left\langle
		\frac{1}{\sqrt{2}}\left(
			\cos\left(\frac{\theta}{\sqrt{2}}\right), 
			-\sin\left(\frac{\theta}{\sqrt{2}}\right), 
			0
		\right), 
		\frac{1}{\sqrt{3}}\left(
			\sqrt{2}\cos\left(\frac{\theta}{\sqrt{2}}\right), 
			\sqrt{2}\sin\left(\frac{\theta}{\sqrt{2}}\right), 
			-1
		\right)
		\times
		\left(
			\sin\left(\frac{\theta}{\sqrt{2}}\right), 
			\cos\left(\frac{\theta}{\sqrt{2}}\right), 
			0
		\right),
	\right\rangle \\
	&=
	\frac{1}{\sqrt{6}}
	\left\langle
		\left(
			\cos\left(\frac{\theta}{\sqrt{2}}\right), 
			-\sin\left(\frac{\theta}{\sqrt{2}}\right), 
			0
		\right),
		\left(
			\cos(\frac{\theta}{\sqrt{2}}), 
			-\sin(\frac{\theta}{\sqrt{2}}), 
			\sqrt{2}
		\right)
	\right\rangle \\
	&=
	\frac{1}{\sqrt{6}}
\end{align*}
By similar arguments, we see that $\kappa_2 = \sqrt{\frac{1}{6}}$. Hence 
\begin{align*}
	\int_{M} K dA
	&=
	2\pi - 2\pi
	- 2 \cdot 2 \int_0^{\pi}\frac{1}{\sqrt{6}} ds\\
	&=
	\frac{2\sqrt{2}\pi}{\sqrt{3}}.
\end{align*}
are being integrated 

\subsection*{Ex 8.3}

\end{document}
