\documentclass{article}
\usepackage[utf8]{inputenc}

\usepackage{rotating}
\usepackage{mathtools}
\usepackage{algpseudocode}
\usepackage{amsfonts}
\usepackage{amsmath}
\usepackage{amssymb}
\usepackage{amsthm}
\usepackage{bm}
\usepackage{listings}
\usepackage{float}
\usepackage{fancyvrb}
\usepackage{xcolor}
\usepackage{tikz-cd}

\hbadness = 10000
\vbadness = 10000

\newcommand\restr[2]{{% we make the whole thing an ordinary symbol
  \left.\kern-\nulldelimiterspace % automatically resize the bar with \right
  #1 % the function
  \vphantom{\big|} % pretend it's a little taller at normal size
  \right|_{#2} % this is the delimiter
  }}

% Default fixed font does not support bold face
\DeclareFixedFont{\ttb}{T1}{txtt}{bx}{n}{12} % for bold
\DeclareFixedFont{\ttm}{T1}{txtt}{m}{n}{12}  % for normal
% Custom colors

\usepackage{color}
\definecolor{deepblue}{rgb}{0,0,0.5}
\definecolor{deepred}{rgb}{0.6,0,0}
\definecolor{deepgreen}{rgb}{0,0.5,0}

% Python style for highlighting
\newcommand\pythonstyle{\lstset{
language=Python,
basicstyle=\ttm,
morekeywords={self},              % Add keywords here
keywordstyle=\ttb\color{deepblue},
emph={MyClass,__init__},          % Custom highlighting
emphstyle=\ttb\color{deepred},    % Custom highlighting style
stringstyle=\color{deepgreen},
frame=tb,                         % Any extra options here
showstringspaces=false
}}

\lstnewenvironment{python}[1][]
{
\pythonstyle
\lstset{#1}
}
{}

\theoremstyle{definition}

\newtheorem{theorem}{Theorem}[section]
\newtheorem{definition}[theorem]{Definition}
\newtheorem{corollary}[theorem]{Corollary}
\newtheorem{lemma}[theorem]{Lemma}

\newcommand{\Z}{\mathbb{Z}}
\newcommand{\Q}{\mathbb{Q}}
\newcommand{\R}{\mathbb{R}}
\newcommand{\C}{\mathbb{C}}
\newcommand{\K}{\mathbb{K}}
\renewcommand{\P}{\mathbb{P}}
\newcommand{\F}{\mathbb{F}}
\newcommand{\N}{\mathbb{N}}
\newcommand{\A}{\mathbb{A}}


\newcommand{\x}{\bm{x}}
\newcommand{\Kx}{\K[\bm{x}]}
\newcommand{\KP}[2]{\K[#1_1, #1_2, \ldots, #1_{#2}]}

\newcommand{\oo}{\mathcal{O}}
\newcommand{\osp}[1]{\oo_{\Spec(#1)}}
\newcommand{\rospu}[2]{\restr{\oo_{\Spec(#1)}}{#2}}
\newcommand{\oop}[2]{\oo_{\P^{#1}_{#2}}}

\renewcommand{\AA}[1]{\A^{#1}}
\newcommand{\An}{\A^n}
\newcommand{\Am}{\A^m}

\newcommand{\PP}[1]{\P^{#1}}
\newcommand{\Pn}{\P^n}
\newcommand{\Pm}{\P^m}

\newcommand{\Hom}{\text{Hom}}
\newcommand{\Aut}{\text{Aut}}
\newcommand{\End}{\text{End}}
\newcommand{\Iso}{\text{Iso}}
\newcommand{\Mor}{\text{Mor}}

\newcommand{\lm}{\text{lm}}
\newcommand{\nr}{\text{nilrad}}
\newcommand{\Spec}{\text{Spec}}
\newcommand{\Proj}{\text{Proj}}
\newcommand{\proj}{\Proj}
\newcommand{\spec}{\Spec}
\newcommand{\codim}{\text{codim}}
\newcommand{\ann}{\text{ann}}
\newcommand{\im}{\text{im}}
\newcommand{\id}{\text{id}}
\newcommand{\height}{\text{height}}

\newcommand{\pdx}{\frac{\partial}{\partial x}}
\newcommand{\pddx}{\frac{\partial^2}{\partial x^2}}
\newcommand{\pdy}{\frac{\partial}{\partial y}}
\newcommand{\pddy}{\frac{\partial^2}{\partial y^2}}

\newcommand{\catname}[1]{{\normalfont\textbf{#1}}}
\newcommand{\Set}{\catname{Set}}
\newcommand{\CRing}{\catname{CRing}}
\newcommand{\Top}{\catname{Top}}
\newcommand{\op}{\catname{op}}

\setlength{\parindent}{0pt}




\begin{document}

\section*{Ch 2}

\subsection*{Ex 2.1}

We can describe the curve as a sum of two different curves, $\gamma_1(t) = r(t,
1)$ and $\gamma_2(t) = r (\sin(-t), -\cos(-t))$. The first map parameterises a
line parallel through the $x$-axis at height $y = r$. The second map
parameterises a circle of radius $r$, but in the negative direction, and
starting at $(0, -r)$ when $t = 0$. \\

As $(\gamma_2)_x' \leq r$, and $(\gamma_1)_x' = r$, we expect the sum $\gamma =
\gamma_1 + \gamma_2$ to have positive $x$ derivative, never "travels left". We
also see that $\gamma$ has vanishing $x$-derivatives whenever $t = \pi/2 + k
\pi$ for integers $k$. Moreover, $\gamma_1$ and $\gamma_2$ have the same
arclength $|\gamma_1| = |\gamma_2| = r$ over any $t$-interval, and by thinking
about this for a bit, we can convince ourselves that the curve can be drawn by
attaching a marker to a wheel of radius $r$, and then letting that wheel roll
on the $x$-axis. \\

Both the $x$- and $y$-derivatives vanish at $\pi/2 + k\pi$ for $k \in \Z$,
hence the curve is not regular (the curve has singularities wherever the
"marker touches the ground"). \\

The arclength is given by 
\begin{align*}
	\sigma(2\pi)
	&=
	\int_{0}^{2\pi}
	|\gamma'(t)| dt \\
	&=
	\int_{0}^{2\pi}
	\sqrt{(r - r\cos(-t))^2 + (r\sin(-t))^2} dt \\
	&=
	\int_{0}^{2\pi}
	\sqrt{r^2 - 2r^2\cos(-t) + r^2\cos^2(-t) + r^2\sin^2(-t)} dt \\
	&=
	\int_{0}^{2\pi}
	\sqrt{2r^2 - 2r^2\cos(-t)} dt \\
	&=
	r
	\int_{0}^{2\pi}
	\sqrt{2(1 - \cos(t))} dt \\
	&=
	r
	\int_{0}^{2\pi}
	\sqrt{2(2\sin^2(t/2))} dt \\
	&=
	2r
	\int_{0}^{2\pi}
	\sin(t/2) dt \\
	&=
	2r(-2\cos(\pi) + 2\cos(0)) \\
	&=
	8r
\end{align*} 

\subsection*{Ex 2.2}

We can again decompose $\gamma$ into a sum of $\gamma_1(t) = 3r(\cos(t),
\sin(t))$ and $\gamma_2(t) = r(\cos(-3t), \sin(-3t))$. So we have some small
circular motion added to a bigger circular motion. Given the previous exercise,
we hypothesise that this is what you get when you attach a marker to the edge
of a coin of radius $r$, and let it roll around another coin of radius $2r$. In
this scenario, $\gamma_1(t)$ models exactly the motion of the center of the
smaller coin, and in the time it'd take the radius of the center of the smaller
coin to make a full lap around the big coin, we'd expect the smaller coin to
make three full revolutions about it's own center. Indeed, two revolutions
would come from just rolling the length of the diameter of the bigger coin
$2\pi (2r)$, and then another revolution would come from "bending" that length
around the edge of the bigger coin. Also, the smaller coin would have to spin
in the opposite direction, as opposed to its radius. I.e we get exactly
$\gamma_1$ and $\gamma_2$. \\

As before, we expect singularities whenever the marker touches the rolling
surface (the bigger coin), and we can see this analytically as 
\begin{align*}
	\gamma'(t)
	=
	r(-3\sin(t) + 3\sin(-3t), 3\cos(t) - 3\cos(-3t))
\end{align*}
which vanishes whenever $t = 0 + k_2\pi$ for example. Hence the curve is not
regular. \\


The arclength of $\gamma$ for $t \in [0..2\pi]$ is given by 
\begin{align*}
	\sigma(2\pi) 
	&=
	r
	\int_{0}^{2\pi}
	\sqrt{(-3\sin(t) + 3\sin(-3t))^2 + (3\cos(t) - 3\cos(-3t))^2} dt \\
	&=
	3r
	\int_{0}^{2\pi}
	\sqrt{(-\sin(t) + \sin(-3t))^2 + (\cos(t) - \cos(-3t))^2} dt \\
	&=
	3r
	\int_{0}^{2\pi}
	\sqrt{\sin^2(t) -2\sin(t)\sin(-3t) + \sin^2(-3t) + \cos^2(t) -2\cos(t)\cos(-3t) + \cos^2(-3t)} dt \\
	&=
	3r
	\int_{0}^{2\pi}
	\sqrt{2 -2\sin(t)\sin(-3t) -2\cos(t)\cos(-3t)} dt \\
	&=
	3r
	\int_{0}^{2\pi}
	\sqrt{2 + 2\sin(t)\sin(3t) - 2\cos(t)\cos(3t)} dt \\
	&=
	3r
	\int_{0}^{2\pi}
	\sqrt{2 + 2\sin(t)(3\sin(t) - 4\sin^{3}(t)) - 2\cos(t)(4\cos^{3}(t) - 3\cos(t))} dt \\
	&=
	3r
	\int_{0}^{2\pi}
	\sqrt{2 + 6\sin^2(t) - 8\sin^{4}(t) - 8\cos^4(t) + 6\cos^2(t)} dt \\
	&=
	3r
	\int_{0}^{2\pi}
	\sqrt{2 + 6 - 8\sin^{4}(t) - 8\cos^4(t)} dt \\
	&=
	3r
	\int_{0}^{2\pi}
	\sqrt{8 (1 - \sin^{4}(t) - \cos^4(t))} dt \\
	&=
	6r
	\int_{0}^{2\pi}
	\sqrt{8 (2 \sin^{2}(t) \cos^2(t))} dt \\
	&=
	24r
	\int_{0}^{2\pi}
	|\sin(t) \cos(t)| dt \\
	&=
	12r
	\int_{0}^{2\pi}
	|\sin(2t)| dt \\
	&=
	12r 2 \\
	&=
	24r.
\end{align*}

\subsection*{Ex 2.3}

We begin by calculating the curvature $\kappa_1$ of $\gamma_1$, and we don't
want to bother with reparameterisations, hence we use Proposition 2.12.
We have that
\[
	\gamma_1' = ra (-\sin(at), \cos(at)),
\]
and 
\[
	\gamma_1'' = -ra^2 (\cos(at), \sin(at)),
\]
hence 
\begin{align*}
	\kappa_1
	&=
	\frac{\det[\gamma_1', \gamma_2'']}{|\gamma_1'|^3} \\
	&=
	\frac{r^2a^3 \sin^2(at) + r^2a^3 \cos^2(at)}{r^3a^3}\\
	&=
	\frac{1}{r}. \\
\end{align*} 
As $\kappa_1$ is independent of $a$, it follows that $\kappa_2 = \kappa_1$.
Note that $(\gamma_1)_x = (\gamma_2)_x$ and $(\gamma_1)_y = -(\gamma_2)_y$.
Hence $\Phi$ is just flipping $\R^2$ about the $x$-axis. $\Phi$ is not 
orientation preserving as it has determinant $-1$. 

\subsection*{Ex 2.4}


\end{document}
