\documentclass{article}
\usepackage[utf8]{inputenc}

\usepackage{mathtools}
\usepackage{hyperref}
\hypersetup{
    colorlinks=true,
    linkcolor=blue,
    filecolor=magenta,      
    urlcolor=cyan,
    pdftitle={Overleaf Example},
    pdfpagemode=FullScreen,
    }
\usepackage{algpseudocode}
\usepackage{amsfonts}
\usepackage{amsmath}
\usepackage{amssymb}
\usepackage{amsthm}
\usepackage{listings}
\usepackage{float}
\usepackage{fancyvrb}
\usepackage{xcolor}
\usepackage{tikz-cd}

\let\temp\phi
\let\phi\varphi
\let\varphi\temp

\newcommand\restr[2]{{% we make the whole thing an ordinary symbol
  \left.\kern-\nulldelimiterspace % automatically resize the bar with \right
  #1 % the function
  \vphantom{\big|} % pretend it's a little taller at normal size
  \right|_{#2} % this is the delimiter
  }}

\hbadness = 10000
\vbadness = 10000

\theoremstyle{definition}

\newtheorem{theorem}{Theorem}[section]
\newtheorem{definition}[theorem]{Definition}
\newtheorem{corollary}[theorem]{Corollary}
\newtheorem{lemma}[theorem]{Lemma}

\newcommand{\Oh}{\mathcal{O}}
\newcommand{\N}{\mathbb{N}}
\newcommand{\Z}{\mathbb{Z}}

\newcommand{\K}{\mathbb{K}}
\newcommand{\F}{\mathbb{F}}
\newcommand{\A}{\mathbb{A}}

\newcommand{\varfont}{\texttt}
\newcommand{\typefont}{\texttt\textbf}

\algnewcommand\Var[1]{\varfont{#1}}
\algnewcommand\Type[1]{\texttt{\textbf{#1}}}


\newcommand{\lm}{\text{lm}}
\newcommand{\nr}{\text{nilrad}}
\newcommand{\spec}{\text{spec}}
\newcommand{\ann}{\text{ann}}
\newcommand{\im}{\text{im}}
\newcommand{\id}{\text{id}}

\newcommand{\catname}[1]{{\normalfont\textbf{#1}}}
\newcommand{\Set}{\catname{Set}}
\newcommand{\CRing}{\catname{CRing}}
\newcommand{\Top}{\catname{Top}}
\newcommand{\op}{\catname{op}}



\newcommand{\pc}{\circ}

\setlength{\parindent}{0pt}




\begin{document}

\section*{Ch 1}
\subsection*{Ex 1.1}
Let $n \in \Z$ be such that $x^n = 0$. Now consider repeating the conjugate
rule according to, $(1 - x)(1 + x) = 1 - x^2$, $(1 - x)(1 + x)(1 + x^2) = 1 -
	x^4$, and on until we get $$ (1 + x)(1 - x)\prod_{k = 1}^{n}(1 + x^{2^{k}}) = 1
	- x^{2^{n + 1}} = 1 - 0 \cdot x^{2^{n + 1} - n} = 1 $$ so $1 + x$ is a unit. \\

The same proof for an arbitrary unit $u$ yields $$ (u + x)(u - x)\prod_{k =
		1}^{n}(u + x^{2^{k}}) = u^{2^{n+1}} - x^{2^{n + 1}} = u^{2^{n+1}}, $$ which is
also whence $u + x$ is.

\subsection*{Ex 1.2}
\subsubsection*{i)}

For the $\Leftarrow$ direction, let $k$ be an integer such that all $a_i^k = 0$
for $i > 0$. Then every term in every coefficient $(f - a_0)^{kn}$ must contain
at least $k$ factors of some $\not = a_0$ coefficient of $f$. Hence $(f -
	a_0)^{kn} = 0$ and $f - a_0$ is nilpotent. But $a_0$ is a unit, so $f$ is a
unit by Ex 1.1. \\

For the $\Rightarrow$ direction, we assume that $f$ is a unit and prove the
given statement using induction.The case when $r = 0$ follow directly from the
fact that the leading coefficient of the product is the product of the leading
coefficients of the terms. \\

Now assume that $a^{r + 1}_n b_{m - r}$ is true for all $r < k$, and consider
the $n + m - k$-th coefficient, $c_{n + m - k}$, of $g = a_{n}^k f f^{-1}$.
It's given as $$ c_{n + m - k} = a_{n}^k \sum_{i + j = n + m - k} a_i b_j, $$
but all terms with $j > m - k$ are $0$ since then $a_{n}^k b_j = 0$, so the
only term which remains is $a_{n}^{k + 1} b_{n - k} = 0$, and $a^{r + 1}_n b_{m
			- r}$ is true for all $r \in [0..m]$. \\

But then it must be that $a^{m + 1}_n = 0$ since $b_0$ is a unit, so $a_n$ is
nilpotent. Then we also have that $a_n x^n$ is nilpotent, and $f - a_n x^n$ is
a unit by Ex 1.1. Continuing reduction of $f$ like this, term by term this way
until we get to the constant term proves the desired result.

\subsubsection*{ii)}

The $\Leftarrow$ direction is analogous to the previous subproblem, just that
we don't need to subtract by $a_0$. \\

For the $\Rightarrow$ direction, we know that $f + 1$ is a unit by Ex 1.1,
which in turn leads to all $a_i$ for $i > 0$ being nilpotent. Now, since
nilpotent elements form an ideal, $a_i x^i$ is nilpotent for every $i > 0$, and
we can subtract them all from $f$ and remain in the nilradical, hence $a_0$ is
nilpotent as well.

\subsubsection*{iii)}
Let $g = b_0 + \ldots + b_m x^m$ be a non-zero polynomial of minimal degree $m$ which annihilates $f$. Then $a_n b_m = 0$. But $a_n g$ annihilates $f$ as well, and $\deg(a_n g) < m$, so $a_n g = 0$. Now we have $(f - a_nx^n)g = 0$ as well, so $a_{n-1}b_m = 0$ and $a_{n-1}g = 0$. This process can be continued all the way down, showing that $b_m f = 0$.

\subsubsection*{iv)}
First, $(c_1, \ldots, c_l) \subseteq (a_1, \ldots, a_n)$ which immediately gives us the $\Rightarrow$ direction. \\

For the $\Leftarrow$ direction, let $A_i$ be a solution to $\sum a_i A_i = 1$,
and $B_i$ similarly. Then we have
\begin{align*}
	1
	 & = \left(\sum a_i A_i \right)  \left(\sum b_i B_i \right)                \\
	 & = \sum_{a_i} \sum_{b_i} a_i A_i b_i B_i                                 \\
	 & = \sum_{k = 0}^{n + m} \sum_{i = 0}^{n + m} a_i A_i b_{k - i} B_{k - i} \\
	 & = \sum_{k = 0}^{n + m} \sum_{c_k} \sum_{i = 0}^{n + m} A_i B_{k - i},
\end{align*}
and we are done.

\subsection*{Ex 1.3}
\subsubsection*{i)}
The $\Leftarrow$ direction of Ex 1.2.i) made no assumptions on the dimension of the polynomials. \\

For the other direction, assume that the statement is true for $n < k$ and
consider $A[x_1, \ldots, x_k]$. This ring can be considered as single variable
polynomial ring with scalars in $A[x_1, \ldots, x_{k - 1}]$. By Ex 1.2.i), we
then have $f$ nilpotent $\Rightarrow$ $c_i(f)$ nilpotent for $i > 0$ and a unit
for $i = 0$. This immediately yields all coefficients for terms with factors of
$x_n$ being nilpotent. The remainder of the polynomial is a unit and the
statements regarding these coefficients follows by induction.

\subsubsection*{ii)}
Again, we get the $\Leftarrow$ direction from Ex 1.2.ii).

We'll use the same induction as above for the $\Rightarrow$ direction. By Ex
1.2.i), we then have $f \in A[x_1, \ldots, x_{k - 1}][x_k]$ nilpotent
$\Rightarrow$ $c_i(f)$ nilpotent for all $i$, and the coefficients are
polynomials of degree $k - 1$ so their coefficients are nilpotent by induction.

\subsubsection*{iii)}

Ex 1.3.iii) made no assumptions of degree on the polynomial ring. Replace
$\deg$ with total degree, and $x^n$ with $\lm(f)$.

\subsubsection*{iv)}
Again, if we organise terms by total degree, we get the same proof as in Ex 1.2.iv).

\subsection*{Ex 1.4}
Let $f$ be in the Jacobson radical. Then $1 - fg$ is a unit for all $g \in
	A[x]$. We will show that $f$ is nilpotent. We can pick $g = x$, and $h = 1 -
	fx$ is a unit by Ex 1.1. Then all coefficients of non-constant terms of $h$ are
nilpotent by Ex 1.2.i). But these are precisely the coefficients of $f$, so $f$
is nilpotent by Ex 1.2.ii)

\subsection*{Ex 1.6}

\begin{lemma}
	The Jacobson radical does not contain any non-zero idempotents.
\end{lemma}
\begin{proof}
	Let $e$ be in the Jacobson radical such that $ee = e$. Then $1 - e$ and $1 + e$
	are both units by Prop 1.7 (?). Let their inverses be named $u, u'$. Then $uu'$
	is the inverse of the product $(1 - e)(1 + e) = 1 - e^2 = 1 - e$. We have
	$uu'(1 - e) = 1$, but we also have $u(1 - e) = 1$, so $u' = 1$, $1 - e = 1$ and
	$e = 0$.
\end{proof}

\subsection*{Ex 1.7}
Let $P$ be a prime ideal, and consider $x + P \in A / P$. Then $x^n + P = x +
	P$ and $x^n - x + P = P$, so $(x^{n-1} - 1 + P)(x + P) = P$. But $A/P$ is a
domain, so either $x + P = P$ or $x^{n-1} - 1 + P = P$. Ignoring the trivial
case, we get that $x^{n-1} + P = 1 + P$, so $x + P$ has the inverse $x^{n - 2}
	+ P$, and $A/P$ is a field, whence $P$ is maximal.

\subsection*{Ex 1.8}
Let $P$ be the set of prime ideals in $A$, and consider a chain $C$
\[
	\mathfrak{p}_{0} \supseteq  
	\mathfrak{p}_{1} \supseteq  
	\ldots
\] 
of ideals $\mathfrak{p}_{i} \in P$. We claim that $\mathfrak{p} =
\bigcap_{\mathfrak{p}_i \in C} \mathfrak{p}_i$ is a prime ideal. Indeed, given
$ab \in \mathfrak{p}$ we have that $ab \in \mathfrak{p}_i$ for all
$\mathfrak{p}_i \in C$, and if there is some $n \in \N$ such that $a \not \in
\mathfrak{p}_n$, then $n \not \in \mathfrak{p}_i$ for all $i \geq n$ since $i
\geq n \Rightarrow \mathfrak{p}_i \subseteq \mathfrak{p}_{n}$. It follows that
$b \in \mathfrak{p}_i$ for all $i \geq n$. Then $b \in \mathfrak{p}_i$ for all
$\mathfrak{p}_i \in C$, as otherwise, the argument above could be repeated with
$b$ instead, leading to a contradiction. It follows that $\mathfrak{p} \in P$,
and $\mathfrak{p}$ is an upper bound to $C$, whence $P$ admits minimal elements
by Zorn's lemma.


\subsection*{Ex 1.9}
The $\Rightarrow$ direction is immediate from Prop 1.14. \\

Let $I$ be an ideal which is the intersection of prime ideals in $A$, and let
$x^n \in I$ for some $x \in A, n \in \Z$. Any prime ideal which contains $x^n$
necessarily contains $x$, so all prime ideals which intersect to $I$ contain
$x$, whence $I$ do as well.

\subsection*{Ex 1.10}
i) $\Rightarrow$ ii) Given i), Prop 1.8 tells us that $\nr(A)$ is prime. Let $x
	\in A$ be a non-unit. Let $I$ be the maximal ideal containing $x$ by Cor 1.5.
Maximal ideals are prime, so $I$ is prime. Thus $I = \nr(A)$ and $x \in
	\nr(A)$, whence $x$ is nilpotent. \\

ii) $\Rightarrow$ iii) follows from the fact that all nilpotent elements are
factored out (Prop 1.7) and only units remain in $A/\nr(A)$. \\

iii) $\Rightarrow$ i) Given iii), we know that $\nr(A)$ is maximal, but it's
also the intersection of all prime ideals. Thus all prime ideals equal
$\nr(A)$.

\subsection*{Ex 1.12}

The only unit which is idempotent is $1$, since if $e, e' \in A$ such that $ee
	= e,\ ee' = 1$, then $1 = ee' = eee' = e$. \\

Let $A, m$ be a local ring and suppose towards a contradiction that $e \in A$
is an idempotent not equal to $0, 1$. Then $e \in m$ by Cor 1.5, and since $m$
is the only maximal ideal, it's the Jacobson radical, which by Prop 1.9 means
that $1 - ge$ is a unit for all $g \in A$. In particular, $1 - e$ is a unit,
but it's also a zero divisor since $e(1 - e) = 0$, a contradiction.

\subsection*{Ex 1.13}
We show that $\mathfrak{a} \not = (1)$. First, if $|\Sigma| = 1$ and $f$ is the
sole member of $\Sigma$, we have that $\deg(f) > 0$ since $f$ is irreducible,
so $\mathfrak{a} = (f)$ doesn't contain the constants. \\

Now assume that $|\Sigma| > 1$, and let $f, g \in \Sigma$. Then $f(x_g) \not
\in \Sigma$ since $k[x_g] \cap \Sigma = (g(x_g)) \not \ni f(x_g)$ since $f \not
= g$ and $f$ is irreducible. \\

\subsection*{Ex 1.14}
We will show that $\Sigma$ admits maximal elements by an application of Zorn's
lemma, and to do this, all we need to do is verify that if 
\[
\mathfrak{a}_0 \subseteq \mathfrak{a}_1 \subseteq \dots
\] 
is a chain in $\Sigma$, then $\mathfrak{a} = \bigcup \mathfrak{a}_i$ is an
ideal. Let $a, b \in \mathfrak{a}, r \in A$. Then $a \in \mathfrak{a}_i$ for
some $i$ and $ra \in \mathfrak{a}_i \subseteq \mathfrak{a}$. Even more, since
$i < j \Rightarrow \mathfrak{a}_i \subseteq \mathfrak{a}_j$, then $a \in
\mathfrak{a}_j$ for all $j > i$. It follows that there is some $\mathfrak{a}_k$
which contains both $a, b$, whence it contains $a + b$, and we see that $a + b
\in \mathfrak{a}$. \\

Now, let $\mathfrak{p} \in \Sigma$ be a maximal element, and consider $a, b \in
A$ such that $ab \in \mathfrak{p}$. Then $ab$ is a zero divisor and $abc = 0$
for some $c \in A$. But then $a, b$ are both zero divisors as well and $(a),
(b) \in \Sigma$. Now assume towards a contradiction that $a, b \not \in
\mathfrak{p}$. It follows that $(a) + \mathfrak{p} = (b) + \mathfrak{p} (1)$
since $\mathfrak{p}$ is maximal and $(a) + \mathfrak{p}, (b) + \mathfrak{p}$
both consist soleley of zero divisors. Multiplying the two ideal sums yields
\[
	(1) 
	= 
	(1)^{2} 
	= 
	((a) + \mathfrak{p})((b) + \mathfrak{p}) 
	= 
	(ab) + (a)\mathfrak{p} + (b)\mathfrak{p} + \mathfrak{p}^{2}
	\subseteq
	\mathfrak{p}
\] 
which contradicts $\mathfrak{p}$ being a proper ideal.

\subsection*{Ex 1.15}
\subsubsection*{i)}
Since any ideal which contains $E$ is required to contain $a$ and vice verca,
we have $V(a) = V(E)$. Moreover, if $P \supset a$ is a prime ideal, then
$\nr(a) \subset P$ and $V(a) \subseteq V(r(a))$. The other inclusion is
immediate.

\subsubsection*{ii)}
Any ideal contains $0$ so $V(0)$ is the set of prime ideals. Only $(1)$
contains $1$ so $V(1) = \emptyset$.

\subsubsection*{iii)}
$x \in V(\bigcup E_i)$ means that $x$ is a prime ideal which contain all $E_i$.
In other words, $x \in V(E_i)$ for all $E_i$, so $ x \in \bigcap V(E_i)$. \\

$y \in \bigcap V(E_i)$ means that $y$ is a prime ideal which contain some $E_i$
all $V(E_i)$, and just like above, we see that this is a simple rewording. \\

\subsubsection*{iv)}
By and Ex 1.1.13 i) we have
$$
	V(a \cap b) = V(r(a \cap b)) = V(r(ab)) = V(ab),
$$
Now, if $x \in V(ab)$ then $x$ is a prime ideal which contains $ab$. With the
following lemma, this means that $x$ contains either $a$ or $b$, whence $x \in
V(a) \cup V(b)$.

\begin{lemma}
	Let $P$ be a prime ideal such that $IJ \in P$ for two ideals $I, J$. Then
	either $I \in P$ or $J \in P$.
\end{lemma}
\begin{proof}
	Assume that $J \not \in P$. Then there is some $x \in J \setminus P$. For all
	$i \in I$ we have $xi \in P$, and since $P$ is a prime ideal, $i \in P$.
\end{proof}

\subsection*{Ex 1.17}

Let $U$ be open in $X$. Then $U=V(E)^c$ for some $E \subset A$. Now, $$ U =
	V(E)^c = \left(V\left(\bigcup_{f \in E}f\right)\right)^c = \left(\bigcap_{f \in
		E} V(f)\right)^c = \bigcup_{f \in E} V(f)^c = \bigcup_{f \in E} X_f. $$

\subsubsection*{i)}
We have that any $a \in X_f \cap X_g$ is a prime ideal which doesn't contain $f$, nor $g$. Such an ideal can't contain $fg$ by the lemma in 1.15.iv) so $x \in X_{fg}$. Likewise, $b \in X_{fg}$ means that $b$ is a prime ideal which doesn't contain $fg$, whence it clearly can't contain either $f$ or $g$.

\subsubsection*{ii)}
$X_f = \emptyset \Leftrightarrow V(f) = X$ which means that $f$ is in all prime ideals, so $f \in \nr(A)$.

\subsubsection*{iii)}
Every non-unit is contained in a maximal ideal, which is prime, so if $X_f = X$, then $f$ must be a unit.

\subsubsection*{iv)}
$X_f = X_g \Leftrightarrow V(f) = V(g) \Leftrightarrow V(r(f)) = V(r(g))$ by Ex 1.15.i)

\subsection*{Ex 1.18}

\begin{lemma}
	Every proper ideal $I$ is contained in some maximal ideal.
\end{lemma}
\begin{proof}
	Let $I$ be a proper ideal. Let $\Sigma_I$ be the set of ideals which contain
	$I$. Then $\Sigma_I$ contains $I$ and is not empty. Order $\Sigma_I$ by
	inclusion and apply Zorn's lemma like in the proof of Theorem 1.3 to show that
	$\Sigma_I$ contains a maximal element.
\end{proof}

\subsubsection*{i)}
If $x$ is a closed point, then $x = V(E)$ for some $E \subset A$, which means that $p_x$ is the only prime ideal which covers $E$. Since $E \subset p_x$, we know there is some maximal ideal which covers $E$, which in turn would be prime, so it must be $p_x$. \\

If $p_x$ is maximal, then $V(x) = \{p_x\}$.

\subsubsection*{ii)}
$\overline{\{x\}}$ is the intersection of all closed sets containing $x$, which by Ex 1.15 is
$$
	\bigcap_{E : x \in V(E)} V(E) = V \left( \bigcup_{E : x \in V(E)} E \right).
$$
Now, $E$ is such that $x \in V(E)$ precisely when $x$ contains $E$, so $\bigcup_{E : x \in V(E)} E = x$ and $\overline{\{x\}} = V(x) = V(p_x)$

\subsubsection*{iii)}
Let $y \in \overline{\{x\}}$. By ii) we then have $y \in V(p_x)$ which by definition means that $x \subseteq y$. \\

Let $x \subseteq y$. By definition we have $y \in V(p_x)$ which by ii) means
that $y \in \overline{\{x\}}$.

\subsubsection*{iv)}
Let $x \not = y$. Then we must have either $x \not \subset y$ or $y \not \subset x$. Assume $x \not \subset y$. Then $y \not \in V(p_x)$, but $x \in V(p_x)$.

\subsection*{Ex 1.19}
Let $a, b\in A \setminus \nr(A)$. Then $X_a, X_b \not = \emptyset$ since if
$V(a) = \spec(A)$, we'd have $a$ in every prime ideal, and therefore $a$ in the
nilradical. If $\spec(A)$ is irreducible, we now get that $X_a \cap X_b$ is
non-empty. Since $X_f$ is the set of prime ideals which doesn't contain $f$,
$X_a \cap X_b \not = \emptyset$ means that there is some prime ideal $p$ which
contains neither $a$ nor $b$. Then $p$ can't contain $ab$ on account of being
prime. Since the nilradical is the intersection of all prime ideals, it can't
contain $ab$ either. \\

Now let $A$ be a ring with prime nilradical. Let $a, b \in A$ be such that
$X_a, X_b$ are non-empty. Then neither $a, b$ can be in the nilradical, which
in turn means that $ab$ isn't either and $X_{ab}$ is non-empty. I.e there is
some prime ideal $p$ which doesn't contain $ab$, but then $p$ doesn't contain
$a$ nor $b$, whence $p \in X_a \cap X_b$. Since the $X_f$ form a basis, and we
just showed that any two non-empty basis elements intersect, we conclude that
$\spec(A)$ is irreducible.

\subsection*{Ex 1.20}
\subsubsection*{i)}

Let $U, V$ be two open sets in $\overline{Y}$. Since $U$ is the neighbourhood
of some $x \in Y$, we have that $U \cap Y \not = \emptyset$, and the same for
$V$. But $U \cap Y$, $V \cap Y$ are open in $Y$ and must therefore intersect by
the hypothesis. Thus $U, V$ intersects as well and $\overline{Y}$ is
irreducible.

\subsubsection*{ii)}
Let $Y$ be an irreducible subspace and $\Sigma$ be the set of irreducible subspaces of $X$ which contain $Y$. Let
$$
	Y_1 \subset Y_2 \subset Y_3 \subset \ldots
$$
be some chain in $\Sigma$, and denote their union $M = \bigcup Y_i$. Then $M$ is an open set, and given $U, V \in M$, there must be some least $Y_i$ which contains both, so $U \cap V \not = \emptyset$. Hence the chain has a maximal element, whence $\sigma$ does as well by Zorn's lemma.

\subsubsection*{iii)}
They are clearly closed by i). They cover $X$ since for any $y \in X$ we have that the subspace $\{y\}$ is irreducible. If the space is Hausdorff, then these are precisely the irreducible components of $X$, since if $Z$ is a subspace of $X$ and $x, y \in Z$, then we have $U_x, U_y \in X$ which don't intersect, and $U_x \cap U_y \cap Z$ is empty as well.

\subsubsection*{iv)}
Let $p_x$ be a prime ideal, and $Y = V(p_x)$ be given the subspace topology. Consider two non-empty basis elements $X_a \cap Y, X_b \cap Y$. If $X_a \cap Y = (V(a))^c \cap Y$ is non empty, this means that there is some ideal in $V(p_x)$ which doesn't contain $a \in Y$. Let $J_a, J_b$ be those ideals. Both $a, b$ contain $p_x$ since they lie in $Y = V(p_x)$. So we have two ideals $J_a \in X_a \cap Y$, $J_a \in X_a \cap Y$, and neither of them contain $p_x$, so the two basis elements intersect and $V(p_x)$ is irreducible. \\

Now let $V(E)$ be some closed irreducible set which contains $V(p_x)$ for a
minimal prime ideal $p_x$. We have $p_x \in V(p_x) \subseteq V(E)$. Since
$V(E)$ is irreducible, $x$ is dense in $V(E)$, and by Ex 1.18.ii),
$\overline{\{x\}} = V(p_x)$ so $V(E) = V(p_x)$ is an irreducible component.

\subsection*{Ex 1.21}

\subsubsection*{i)}
To be explicit, $X_f$ is the set of prime ideals in $A$ which don't contain $f$. The map $\phi^{*-1}$ sends ideals in $A$ to ideals in $B$ which contain $\ker \phi$ by Prop 1.1. Any ideal $p$ in $A$ which doesn't contain $f$ must clearly be mapped to an ideal $\phi(p)$ which doesn't contain $\phi(f)$, so $\phi^{*-1}(X_f) \subset Y_{\phi(f)}$ (Note that if $p \in X_f$ is such that $\phi(p)$ isn't prime, then $\phi^*$ doesn't map anything to $X_f$ and $\phi^{*-1}(X_f) = \emptyset$). If $p \in Y_{\phi(f)}$, then $\phi^*(p)$ is a prime ideal in $X$ and clearly $f \not \pi \phi^*(p)$, so $\phi^{*-1}(X_f) = Y_{\phi(f)}$.

\subsubsection*{ii)}
Let $b \in \phi^{*-1}(V(a))$. Then $b$ is a prime ideal in $B$ such that $\phi^{-1}(b)$ is a prime ideal in $A$ containing $a$. Then $\phi \phi^{-1} b = b$ contains $\phi a$. But if $b$ is a prime ideal containing $\phi(a)$, then it necessarily contains $a^e$, the smallest ideal generated by $\phi(a)$, so $b \in V(a^e)$. \\

Let $b \in V(a^e)$. Then $b$ is a prime ideal containing the extension $a^e$ of
$\phi(a)$. Since $a^e \supseteq \phi(a)$ we have $\phi^{-1}(b) \supseteq
	\phi^{-1}(a^e) \supseteq \phi^{-1}(\phi(a)) = a$ so $b \in \phi^{*-1}(V(a))$.
\\

\subsection*{Ex 1.28}
It's shown in the exercise statement that any regular function $\phi : X \to Y$ can be
used to induce a map $\overline{\phi} : P(Y) \to P(X)$. We first show that this map
is a $K$-algebra homomorphism. \\

Let $f, g \in P(Y),\ a, b \in K$. Then
\begin{align*}
	&\phi(af + bg) = (af + bg) \circ \phi = a (f \circ \phi) + b (g \circ \phi)
	= a\overline{\phi}(f) + b \overline{\phi}(g) \\
	&\phi(fg) = fg \circ \phi = (f \circ \phi) (g \circ \phi) =
	\overline{\phi}(f)\overline{\phi}(g),
\end{align*}
and the constant polynomial functions on $P(Y)$ ignore their arguments, whence
$\overline{\phi}$ restricts to the identity on $K$. \\

To see that the correspondence is injective, let $\phi, \psi : X \to Y$ be two
regular maps such that $\overline{\phi} = \overline{\psi}$. Let $x \in X$ be an
arbitrary point on the variety, and write $a = \phi(x), b = \psi(x)$. Then $a,
b \in Y$. Since $\overline{\phi} = \overline{\psi}$, we have that for all $g
\in P(Y)$, $g(a) = g(b)$, which in implies that $a = b$, whence $\phi(x) =
\psi(x)$ for all $x \in X$, and finally $\phi = \psi$. \\

We now show that it is surjective. Let $\mu : P(Y) \to P(X)$ be a $K$-algebra
homomorphism. Let $\xi_i \in P(Y)$ be the $i$-th coordinate function in
$P(Y)$ and $\hat{\xi}_i = \mu(\xi_i) \in P(X)$, and pick arbitrary
representatives $f_i \in \hat{\xi}_i$ for every $i$. We claim that $\phi =
\restr{(f_1, f_2, \ldots f_n)}{X}$ is such that $\overline{\phi} = \mu$. \\ 

We begin with showing that $\phi$ is a regular map $X \to k^{m}$. Let $x \in X$.
Then 
\begin{align*}
	\phi(x) 
	&= 
	(f_1(x), f_2(x), \ldots f_n(x)) \\
	&= 
	(\mu(\xi_1)(x), \mu(\xi_2)(x), \ldots, \mu(\xi_n)(x)) \in k^{m}.
\end{align*}
It follows from arguments in previous parts of the exercise that
$\overline{\phi}$ is a $K$-algebra homomorphism $X \to k[t_1, t_2, \ldots t_n]$. To see that $\overline{\phi} =
\mu$, note that in $P(X)$ we have
\[
\overline{\phi}(\xi_i) = \xi_i \circ \phi = f_i = \hat{\xi}_i = \mu(\xi_i),
\] 
and any homomorphism is determined by where it maps generators. It follows that
$\im(\phi) = \im(\mu) \subseteq Y$ and $\phi$ can be identified with a regular map $X \to
Y$.


\section*{Ch 2}

\subsection*{Ex 1}
Let $a \otimes b \in \Z_m \otimes_{\Z} \Z_n$. Then as $m, n$ are co-prime, we have $a \in (m, n) = \Z$, and
there exist $k_1, k_2 \in \N$ where $a = k_1 m + k_1 n$. It follows that 
\[
a \otimes b 
= 
(k_1 m + k_1 n) \otimes b
= 
k_1 m \otimes b + k_1 n \otimes b
= 
k_1 n \otimes b
= 
k_1 \otimes (n b)
= 
0.
\] 

\subsection*{Ex 2}
We follow the given advise and tensor the exact sequence
\[
	\begin{tikzcd}
		\mathfrak{a} \arrow[r] & 
		A \arrow[r] & 
		A / \mathfrak{a} \arrow[r] & 
		0
	\end{tikzcd}
\] 
with $M$ to get the sequence
\[
	\begin{tikzcd}
		M \otimes \mathfrak{a} \arrow[r] & 
		M \otimes A \arrow[r] & 
		M \otimes A / \mathfrak{a} \arrow[r] & 
		0,
	\end{tikzcd}
\]
which is also exact by Prop 2.18. Now consider the map $f: M \otimes A \to M/\mathfrak{a}M$ given by
$m \otimes a \to am + \mathfrak{a}M$. This map has kernel $M \otimes \mathfrak{a}$, so the sequence
\[
	\begin{tikzcd}
		M \otimes \mathfrak{a} \arrow[r] & 
		M \otimes A \arrow{r}{f} & 
		M / \mathfrak{a}M \arrow[r] & 
		0,
	\end{tikzcd}
\]
is exact as well, and the two modules are isomorphic.

\subsection*{Ex 3}
Let $m$ be the maximal ideal in $A$, $k = A/m$, and  let $M_k = k \otimes M$ be
the $k$-module obtained by extension of scalars. By Ex 2, we have $M_k \cong
M/mM$. If we had $M_k = 0$, we'd have $M = 0$, since $M / mM = M_k = 0
\Rightarrow M = mM$ whence $M = 0$ follows by Nakayama. The same is true of
$N_k$. \\

We've shown that it's enough to prove the statement with $M_k, N_k$, since
that would yield the following chain of implications
\[
	M \otimes_{A} N = 0 
	\Rightarrow
	M_k \otimes_{k} N_k = 0 
	\xRightarrow{\text{remains to prove}}
	N_k = 0 \lor M_k = 0 
	\Rightarrow
	N = 0 \lor M = 0.
\]

Now, $M_k, N_k$ are $k$-vector spaces, and given two non-zero vector spaces, we
can pick one basis element in each $e_m, e_n$, and define the non-zero bilinear
map $(m, n) = (m \cdot e_m)(n \cdot e_n)$. So their tensor product must be
non-zero by the universal property.


\subsection*{Ex 4}

First assume that $M$ is flat, and consider two $A$-modules $N', N$ and an
injective map $f : N' \to N$. Since $M$ is flat, we have that $1_M \otimes f$
is injective. It follows that the restriction $1_{M'} \otimes f$ must be
injective as well, and $M_i$ is flat by Prop 2.19. \\

For the other direction, we need two lemmas. 

\begin{lemma}
	Let $M_i$ be a set of $A$-modules indexed by the (potentially infinite) 
	set $J$, and $N$ another $A$-module. Then
	\[
		N 
		\otimes 
		\left(
			\bigoplus_{i \in J} M_i
		\right)
		\cong
		\bigoplus_{i \in J} 
		N 
		\otimes 
		M_i
	\] 
\end{lemma}
\begin{proof}
	Let 	
	\[
		f : 
		N 
		\times 
		\left(
			\bigoplus_{i \in J} M_i
		\right)
		\to
		\bigoplus_{i \in J} 
		N 
		\otimes 
		M_i
	\]
	be given by $f\left(n, \sum_{i \in S} m_i\right) = \sum_{i \in S}
	\left(n \otimes m_i\right)$. This map is $A$-bilinear and as such,
	induces an $A$-linear map
	\[
		f' : 
		N 
		\otimes 
		\left(
			\bigoplus_{i \in J} M_i
		\right)
		\to
		\bigoplus_{i \in J} 
		N 
		\otimes 
		M_i
	\]
	where $f'\left(n \otimes \sum_{i \in S} m_i\right) = \sum_{i \in S} \left(n
	\otimes m_i\right)$. \\

	Now define $g_i : N \times M_i \to N \otimes M $ by $g_i(n, m_i) = n
	\otimes m_i$ Then $g_i$ is $A$-bilinear and induces an $A$-linear map $g_i'
	: N \otimes M_i \to N \otimes M$. Now define the map 
	\[
		h :
		\bigoplus_{i \in J} 
		N 
		\otimes 
		M_i
		\to
		N 
		\otimes 
		\left(
			\bigoplus_{i \in J} M_i
		\right)
	\] 
	where
	\[
		f'\left(\sum_{i \in S} \left(n \otimes m_i\right)\right) 
		= 
		\sum_{i \in S} 
		g_i'\left(n \otimes m_i\right)
		= 
		\sum_{i \in S} 
		n \otimes m_i
		= 
		n \otimes
		\left(
		\sum_{i \in S} 
		m_i
		\right).
	\]
	This function is well-defined, as $S$ always is a finite set. The two maps
	$f', h$ are inverse each other, and the two modules are isomorphic.
\end{proof}

The second lemma is given as follows

\begin{lemma}
	Let $J$ be a potentially infinite indexing set, and 
	\[
		\begin{tikzcd}
			\ldots \arrow{r} &
			M'_{i} \arrow{r}{f_i} &
			M_{i} \arrow{r}{h_i} &
			M''_{i} \arrow{r} &
			\ldots
		\end{tikzcd}, \ \ i \in J
	\] 
	be a set of exact sequences. Then
	\[
		\begin{tikzcd}
			\ldots \arrow{r} &
			M' \arrow{r}{f} &
			M \arrow{r}{h} &
			M'' \arrow{r} &
			\ldots
		\end{tikzcd}
	\] 
	is exact where $M = \bigoplus_{i \in J} M_{i}$, and $M', M''$ are defined
	analogously.
\end{lemma}
\begin{proof}
	Let $m \in \ker(h)$. Then we can write $m$ as a finite sum $m = \sum_{i \in
	S} m_i$, and its image is given as $0 = \sum_{i \in S} h(m_i)$. But since
	$h$ is defined as the sum of $h_i$ on the relevant coordinates, we have $0
	= \sum_{i \in S} h_i(m_i)$, and each $h_i(m_i) \in M''_i$ so if their sum is
	zero we must have that for every $i \in S$, $h_i(m_i) = 0$ and we have that
	$m_i \in \ker f_i = \im(f_i)$ whence $m \in \im(f)$. \\

	The other inclusion can be shown analogously. 
\end{proof}

Now, as every $M_i$ is flat, we can direct sum all the corresponding tensored
exact sequences, and use distributivity in reverse to yield an exact sequence
of $M$ tensored with the module $N$.

\subsection*{Ex 5}

First of, $A$ is a flat $A$-module since $A \otimes_{A} M \cong M$. Moreover,
$A[x]$ is the direct sum of the $A$-modules, $(x^i) : i \in \N$, and each such
$(x^i)$ is generated by one element and therefore isomorphic to $A$. Since all
of the direct summands of $A[x]$ are flat, it follows that $A[x]$ itself is
flat.


\subsection*{Ex 6}
Let $\hat{f} : A[x] \times M \to M[x]$ be given by $\hat{f}(a(x), m) = a(x)m$.
This is a $A$-bilinear mapping, and as such introduces a $A$-linear map $f :
A[x] \otimes M \to M[x]$ given by $f(a(x) \otimes m) = a(x)m$. Even more, this
map is $A[x]$-linear since $a_1(x)f(a_2(x) \otimes m) = a_1(x)a_2(x)m =
f(a_1(x)a_2(x) \otimes m)$. \\

Now consider the map $g : M[x] \to A[x] \otimes M$ given by the $A$-linear
extension of $g(mx^i) = x^i \otimes m$. This map is also $A[x]$ linear since 
$g(a(x)mx^i) = a(x)x^i \otimes m$. \\

We have that $g \circ f = \id$ and the two modules are isomorphic.

\subsection*{Ex 7}
$\mathfrak{p}[x]$ being prime is equivalent to $A[x]/\mathfrak{p}[x]$ not
having zero divisors. To show this, we will first show that
$A[x]/\mathfrak{p}[x] \cong (A / \mathfrak{p})[x]$. To see this, consider the
map $f(ax^i + \mathfrak{p}[x]) = (a + \mathfrak{p})x^i$, which we extend
by the ring axioms to all of $A[x]/\mathfrak{p}[x]$. First note that
it's well-defined on the elements above, as if $ax^i - bx^i \in
\mathfrak{p}[x]$, then $a - b \in \mathfrak{p}$, and $f(ax^i + \mathfrak{p}[x])
= f(bx^i + \mathfrak{p}[x])$. It's also easy to see that we can extend $f$
to a ring homomorphism. Moreover, it's injective since if $a(x) \in \ker(f)$
only when every coefficient of $a$ lies in $\mathfrak{p}$, whence $a(x) \in
\mathfrak{p}[x]$. It's surjective since the $(a + \mathfrak{p})x^i$ generate
the image for all $i \in \N, a \in A$. \\

It's easy to see that $(A/\mathfrak{p})[x]$ is an integral domain, since the
lowest degree term in any product $f(x)g(x)$ is the product of the lowest
degree terms in $f(x), g(x)$. It follows that $A[x]/\mathfrak{p}[x]$ is 
an integral domain and we are done. \\

For the second question, the answer is no. Let $n \not \in \mathfrak{m}$, then
$nx \not \in \mathfrak{m}[x]$ so $nx + \mathfrak{m}[x]$ is a non-trivial
element in $A[x]/\mathfrak{m}[x]$ without an inverse. Indeed we could not have
$f(x)nx = 1 + m(x)$ for some $m(x) \in \mathfrak{m}[x]$, since $f(x)nx$ doesn't
have a constant term, and $m(x)$ can't have $-1$ as a constant term since $-1
\not \in \mathfrak{m}$.

\subsection*{Ex 8}
\subsubsection*{i)}
Given some exact sequence
\[
\begin{tikzcd}
	0 \arrow{r} &
	W' \arrow{r} &
	W \arrow{r} &
	W'' \arrow{r} &
	0,
\end{tikzcd}
\] 
we have that 
\[
\begin{tikzcd}
	0 \arrow{r} &
	M \otimes W' \arrow{r} &
	M \otimes W \arrow{r} &
	M \otimes W'' \arrow{r} &
	0,
\end{tikzcd}
\] 
is exact, after which it follows that 
\[
\begin{tikzcd}
	0 \arrow{r} &
	N \otimes (M \otimes W') \arrow{r} &
	N \otimes (M \otimes W) \arrow{r} &
	N \otimes (M \otimes W'') \arrow{r} &
	0,
\end{tikzcd}
\] 
is exact, and our desired result follows from Prop 2.14.
\subsubsection*{ii)}
First of, by Prop 2.14 we have that $N \cong B \otimes_B N$, and given some
$A$-module $M$, Prop 2.15 yields $M \otimes_A (B \otimes_B N) \cong (M
\otimes_A B) \otimes_B N$, after which $A$-flatness follows by $A$-flatness of
$B$ followed by $B$-flatness of $N$, just like in the previous sub-exercise.

\subsection*{Ex 9}

Let $m'_i$ finitely generate $M'$ and $m''_i$ finitely generate $M''$. Then let
$M_i = g^{-1}(m''_i)$. For each $M_i$ pick one representative $m_i \in M_i$.
Now, we have that $\im(f) = \ker(g)$ is finitely generated as well, and we let
$n_i$ be a finite generating set for $\ker(g)$. Our claim is that the $m_i$ and
$n_i$ together generate $M$. \\

To see this, consider any element $m \in M$. Then we can write $g(m) = \sum_{i
\in I} m''_i a_i$ for some finite index set $I$ and a set of scalars $a_i \in
A$. It follows that 
\[
g(m) = g\left(\sum_{i \in I} a_i m_i\right),
\] 
and $m - \sum_{i \in I} a_i m_i \in \ker(g)$. We can then write

\[
	m - \sum_{i \in I} a_i m_i = \sum_{i \in J} b_i n_i,
\] 
and finally,
\[
	m = \sum_{i \in I} a_i m_i + \sum_{i \in J} b_i n_i.
\] 

\subsection*{Ex 10}

If the induced homomorphism is surjective, we have that for every $n \in N$,
there is some $m \in M$ such that $n - u(m) \in \mathfrak{a}N$. In other words,
$u(M) + \mathfrak{a}N = N$, and we have $u(M) = N$ by Corollary 2.7.

\subsection*{Ex 14}
I couldn't find a quetion in the exercise description.

\subsection*{Ex 15}
For the first statement, let $m \in M$. Then $m = \mu(m')$ for some $M' \in C$.
Since $C$ is the direct sum of $M_i$ we have
\[
	m' = \sum_{i \in J} a_i m_i
\] 
where $m_i \in M_i,\ a_i \in A$ and $J \subseteq I$ is a finte index set. It follows
that 
\[
	m 
	= 
	\mu\left(\sum_{i \in J} a_i m_i\right)
	= 
	\sum_{i \in J} \mu(a_i m_i)
	= 
	\sum_{i \in J} \mu_i(a_i m_i).
\] 
Since $J$ is finite, we can inductively invoke the defining property of a
directed set to conclude that there exist some $k \in I$ such that $i \leq k$
for all $i \in J$. By the defining relations of $M$, we have that 
\[
	\sum_{i \in J} \mu_i(a_i m_i) 
	=
	\sum_{i \in J} \mu_k( \mu_{ik}(a_i m_i) )
	=
	\mu_k\left(\sum_{i \in J}  \mu_{ik}(a_i m_i) \right),
\] 
and we see that $m = \mu_k(m_k)$ where $m_k = \sum_{i \in J}  \mu_{ik}(a_i
m_i)$ \\

For the second part of the question, note that $\mu_i(m_i) = 0$ implies that
$m_i \in D$ (since $\mu_i$ can be factored as an injection followed by a
quotient of $D$), but we also have $m_i - \mu_{ij}(m_i) \in D$ for all $j \geq
i$. It follows that $\mu_{ij}(m_i) \in D$ which means that $\mu_{ij}(m_i) = 0$
in $M$ and $M_j$ for all $j \geq i$ (and there is always at least one such $j$
with $j = i$). 


\subsection*{Ex 16}
We begin by showing that the direct limit exhibits the given property. Let
$N, \alpha_{i}$ be given as in the exercise description. Then let $\alpha : M
\to N$ be given as $\alpha(m) = \alpha_{i}(m_i)$ where $i \in I,\ m_i \in M_i$
are such that $m = \mu_i(m_i)$. We begin by showing that this map is well
definied. If $\mu_j(m_j) = \mu_i(m_i)$, then there exists $k \in I$ where $i
\leq k,\ j \leq k$. Using this $k$ we get $\mu_j(m_j) = \mu_k(\mu_{jk}(m_j))$ and
$\mu_i(m_i) = \mu_k(\mu_{ik}(m_i))$, so $\mu_k(\mu_{ik}(m_i) - \mu_{jk}(m_j)) =
0$, and by Ex 2.15, there exist $l \in I, l \geq k$ such that
$\mu_{kl}(\mu_{ik}(m_i) - \mu_{jk}(m_j)) = 0$. We now have
\begin{align*}
	\alpha_i(m_i)
	&=
	\alpha_k(\mu_{ik}(m_i)) \\
	&=
	\alpha_l(\mu_{kl}(\mu_{ik}(m_i))) \\
	&=
	\alpha_l(\mu_{kl}(\mu_{jk}(m_j))) \\
	&=
	\alpha_k(\mu_{jk}(m_j)) \\
	&=
	\alpha_j(m_j),
\end{align*} 
which shows that $\alpha$ is well defined. It's clear that $\alpha_i = \alpha
\circ \mu_i$, To see that $\alpha$ is a homomorphism, let $m, m \in m$, where
$m = \mu_i(m_i),\ n = \mu_j(n_j)$, and let $k \geq i, j$, $m_k = \mu_{ik}(m_i),
n_k = \mu_{jk}(n_k)$. Then $\alpha(m) = \alpha_i(m_i) = \alpha_k(\mu_{ik}(m_i))
= \alpha_k(m_k)$, and similarly $\alpha(n) = \alpha_k(n_k)$, whence
homomorphism properties of $\alpha$ follows by homomorphism properties from
$\alpha_k$. Finaly, to see that the homomorphism is unique, let $\beta : M \to
N$ be an arbitrary map which factors the $\alpha_i$. Then for any $m \in M$ we
have that ther exist $i \in I, m_i \in M_i$ such that $m = \mu_i(m_i)$ and we
see that
\[
	\beta(m)
	=
	\beta(\mu_i(m_i))
	=
	\alpha_i(m_i)
	=
	\alpha(\mu_i(m_i))
	=
	\alpha(m)
\] 
so $\beta = \alpha$. \\

We now show that the universal property uniquely determines the direct limit up
to isomorphism. Let $M, \mu_i$, and $M', \mu'_i$ be two modules and families of
homomorphisms which exhibit the universal property. Then both modules can take
the place of $N$ in the definition of the universal property, and invoking the
universal property for each module, we see that there exists two homomorphisms
\begin{align*}
	\alpha : M \to M', &\ \mu'_i = \alpha \circ \mu_i, \\
	\alpha' : M' \to M, &\ \mu_i = \alpha' \circ \mu'_i.
\end{align*}
But then $\phi = \alpha' \circ \alpha$ is a homorphism $M \to M$
such that $\phi \circ \mu_i = \mu_i$, and by the universal property, 
there can only be one such map. We conlcude that $\phi$ is the identity map 
and $M \cong M'$.


\section*{Ch 3}

\subsection*{Ex 1}
Let $s \in S$ be such that $sm = 0$ for all $m \in M$. Then $(m, 1) = (0, 1)$
since $s(m - 0) = 0$, and it follows that $S^{-1}M = 0$.

For the other direction, let $M$ be an $A$-module finitely generated by $x_i$,
and $S$ be such that $S^{-1}M = 0$. We know that for every $m/s \in S^{-1}M$
there is $u \in S$ such that $um = 0$, in particular, let $u_i \in S$ be such
that $u_ix_i = 0$. Then the product of all $u_i$ annihilates all generators,
and therefor all of $M$.

\subsection*{Ex 2}

For any $a \in \mathfrak{a}$, $x \in A$, we have that $-ax \in \mathfrak{a}$
so $1 - ax \in S$ is a unit in $S^{-1}A$, which in turn means $a$ is 
in the Jacobson radical of $S^{-1}A$. \\

Now let $M$ be a finitely generated $A$-module and $\mathfrak{a} \subseteq A$
be an ideal such that $\mathfrak{a}M = M$. Now let $S$ be as above. We have
that $(S^{-1}\mathfrak{a})(S^{-1}A) = S^{-1}(\mathfrak{a}A) = S^{-1}A$. Now we
can apply Nakayama's lemma in and see that $S^{-1}A = 0$. But by the previous
exercise, this happens only when there is $s \in S$ s.t $sA = 0$, and it's true
for any $s \in S$ that $s \equiv 1 \mod \mathfrak{a}$, whence we are done.

\subsection*{Ex 3}

Let $f : A \to S^{-1}A,\ g : S^{-1}A \to U^{-1}(S^{-1}A)$ be the canonical maps
and consider $g \circ f$. Then if $st \in ST$, we have
\[
	g(f(st)) 
	= 
	g(uf(s))
	= 
	g(u)g(f(s))
\]
a unit since $g(u), f(s)$ are units. Moreover, $g(f(a)) = 0$ implies that there exists
$u \in U$ s.t. $uf(a) = 0$. But then $0 = uf(a) = f(t)f(a) = f(at)$ so there exists 
$s \in S$ s.t $sta = 0$. Finally, every element in $U^{-1}(S^{-1}A)$ is of the form
\begin{align*}
g(f(a)f^{-1}(s))g(u)^{-1}
&=
g(f(a)f^{-1}(s))g(f(t))^{-1} \\
&=
g(f(a))g(f(s))^{-1}g(f(t))^{-1} \\
&=
g(f(a))g(f(st))^{-1},
\end{align*} 
and Corollary 3.2 delivers our desired result.


\subsection*{Ex 4}

We first give an account of how these modules are defined. We have that $A$
acts on $B$ via $ab = f(a)b$, $a \in A, b \in B$. Similarly, $s \in S$ acts on
$B$ via $f$, and $S^{-1}A$ acts on $B$ also via $f$. So, the elements in
$S^{-1}B$ are of the form $b/f(s)$, just like the elements in $T^{-1}B$ which
are all of the form $b/t = b/f(s)$. It's immediate that the two modules are
isomorphic, they are the same module!


\subsection*{Ex 5}
Let $A$ be such that for each prime ideal $\mathfrak{p} \subseteq A$, we have
that $A_{\mathfrak{p}}$ has no nilpotents. Assume towards a contradiction that
$A$ has some nilpotent element $a^r = 0$. Then the map $e_r : A \to A^r$ has a
non-trivial kernel $a \in \ker(e_r)$. It follows from Proposition 3.9 that
there must be some prime ideal $\mathfrak{p} \subseteq A$ such that
$(e_r)_\mathfrak{p}$ has non-trivial kernel, but this is contradiction to our
hypothesis, so $A$ must have an empty radical. \\

For the second question, the answer is no. Let $k$ be a field and consider the
ring $R = k \times k$. This is not a domain since it has the zero divisors $(1,
0) \cdot (0, 1) = 0$. $R$ admits only two ideals, $((1, 0)), ((0, 1))$. Let $I
= ((1, 0))$ and consider $R_I$.  We have that $(a, b)/(c, d) = (1, 0)(a, b)/(c,
0) = (b, 0)/(c, 0) = (bc^{-1}, 0)/(1, 0)$ for all $a, b, c, d \in k$, so 
in other words $R_I \cong R$, and $R_I$ is a field, and a domain in particular.

\subsection*{Ex 7}
\subsubsection*{i)}

\textbf{$\Rightarrow$ direction}:

First of all, we show that if $a \in A - S$, then $(a/1)$ is a non-trivial
ideal in $S^{-1}A$. To see this, first note that all units lie in $S$
by virtue of $S$ beeing saturated and containing $1$. So $a$ is 
not a unit in $A$. If $a/1$ were a unit in $S^{-1}A$, then we'd have 
$b/c$ such that $(a/1)(b/c) = (ab)/c = 1/1$, which in turn means that
$(ab - c)s = 0$ for some $s \in S$. But since $c \in S$,
this would mean that $ab \in S$, which is imposible since $S$ is saturated 
and doesn't contain $a$. \\

Let $\mathfrak{p}_{a} = \mathfrak{m}_{a/1}^{c}$ where $\mathfrak{m}_{a/1}$ is
the maximal ideal in $S^{-1}A$ containing the principal ideal $(a/1)$. Such 
exist by Corollary 1.3 and our argument in the previous paragraph. It
now follows from Prop 3.11 iv) that $\mathfrak{p}_{a}$ is prime. Moreover it
contains $a$ since $f(a)$ is contained in $\mathfrak{m}_{a/1}$. Then $A - S =
\bigcup_{a \in A - s} \mathfrak{m}_a$, and we see that $A - S$ is a union of
prime ideals. \\

\textbf{$\Leftarrow$ direction}:
Let $S$ be a subset of $A$ such that $A - S$ is a union of  ideals $A - S
= \bigcup_{\mathfrak{a}_{i}, i \in J} \mathfrak{a}_{i}$. Consider two elements $a,
b \in A$ such that $ab \in S$. Then $ab$ doesn't lie in any $\mathfrak{a}_{i}$.
But then neither $a$ nor $b$ can lie in any $\mathfrak{a}_{i}$ since that would
imply $ab \in \mathfrak{a}_{i}$.

\subsubsection*{ii)}
Let $\overline{S}$ be the intersection of all saturated multiplicative subsets
containing $S$. Then $\overline{S} = \{a \in A : \exists b \in A, ab \in S \}$,
since if $ab \in A$ then $a$ must be contained in any saturated multiplicative
set containing $a$, and if $ab \in \overline{A}$, then $abc = a(bc) = b(ac) \in
A$ for some $c \in A$, whence it follows that $a, b \in \overline{A}$. \\

From i) we know that $\overline{S}$ is a union of prime ideals which don't meet
$\overline{S} \supseteq S$. It remains to show that any prime ideal which
doesn't meet $S$ also won't meet $\overline{S}$. To do so, consider a prime
ideal $\mathfrak{p}$ which meets $\overline{S}$ then it must meet $S$, since if
$a \in \mathfrak{p} \cap \overline{S}$, then either $a \in S$, or $a \not \in
S$, whence there must be some $b \in A$ such that $ab \in S$ (by our
constructive definition of $\overline{S}$), but $ab \in \mathfrak{p}$ as well. \\

For the second quetion, let $\mathfrak{p}$ be prime ideal which dosen't meet $1
+ \mathfrak{a}$. Then $\mathfrak{p} + \mathfrak{a}$ is a non-trivial ideal, as
otherwise we'd be able to write $1 = xp + ya$, whence $xp = 1 - ya \in 1 +
\mathfrak{a}$. It follows that there exist some maximal ideal $\mathfrak{m}$
which contains $\mathfrak{p} + \mathfrak{a}$, so every ideal which doesn't meet
$1 + \mathfrak{a}$ is contained in a maximal ideal containing $\mathfrak{a}$.
As no proper ideal containing $\overline{a}$ can meet $\overline{a} + 1$, we
deduce that $\overline{1 + \mathfrak{a}}$ is the union of all maximal ideals
containing $\overline{a}$.

\subsection*{Ex 8}

\textbf{i) $\Rightarrow$ ii):}
$\phi$ bijective means we can construct its inverse $\psi = \phi^{-1}$, and as
$t/1$ is a unit in $T^{-1}(A)$, we see that $\psi(t/1) = t/1$ is a unit in
$S^{-1}A$. \\

\textbf{ii) $\Rightarrow$ iii):}
Let $u/1$ be the inverse of $t/1$, then $ut/1 \in S$ \\

\textbf{iii) $\Rightarrow$ iv):}
Let $xt \in S$, then $x, t \in \overline{S}$ by definition. \\

\textbf{iv) $\Rightarrow$ v):}
we have that $S \subseteq T$ and $T \subseteq \overline{S}$, so $\overline{T} =
\overline{S}$. The saturation of a multiplicatively closed set is the
complement of all prime ideals not meeting the set (Ex 7.ii)), so if
$\mathfrak{p} \cap S = 0$, then $\mathfrak{p} \subseteq \overline{S}^{c} =
\overline{T}^{c} \Rightarrow \mathfrak{p} \cap T = 0$. The other implication
follows analogously. \\

\textbf{v) $\Rightarrow$ i):}
Let $a/s \in S^{-1}A$ be such that $a/s = 0$ as an element in $T$. Then there
exists some $t \in T$ such that $ta = 0$. But then since $t$ is a unit in
$T^{-1}A$, we either have $T^{-1}A = 0$, or $a = 0$. Now, every prime ideal
which meets $S$ also meets $T$, so if $S^{-1}A$ contains non-trivial prime
ideals, then so does $T^{-1}A$ by Prop 3.11.iv), and since the only rings which
don't contain prime ideals are either the zero ring or fields, we see that
$T^{-1}A = 0$ exactly when $S^{-1}A = 0$. It follows that $\phi$ is bijective.



\subsection*{Ex 21}

Just to recap, $\spec(A)$ is the set of all prime ideals in $A$, and the closed
sets are of the form $V(E)$ for arbitrary $E \subseteq A$ where $V(E)$ is the
set of all prime ideals containing $E$. $V(E) = V(r(\mathfrak{q}))$ where
$\mathfrak{a}$ is the smallest ideal containing $E$. The basic open sets are of
the form $X_f$, where $X_f$ is the complement of $V(f)$, I.e the set of all
prime ideals not containing $f$.


\subsubsection*{i)}

Let $X = \spec(A), Y = \spec(S^{-1}A)$. Let $f \in A$ and $\mathfrak{p} \in
\phi^{*-1}(X_f)$. Then $\mathfrak{p}$ is a prime ideal in $S^{-1}B$, and
$\phi(f) \not \in \mathfrak{p}$, since otherwise we'd have $f \in
\phi^{-1}(\mathfrak{p}) = \phi^*(\mathfrak{p}) \in X_f$. Hence $Y_{\phi(f)}
\supseteq \phi^{*-1}(X_f)$. For the other inclusion, suppose $\mathfrak{p}$
is a prime ideal in $Y$ not containing $\phi(f)$. Then $\phi^{-1}(\mathfrak{p})
= \phi^{*}(\mathfrak{p})$ doesn't contain $\phi^{-1}(\phi(f)) = f$. We've shown
that $\phi^{*}$ is continuous. It's immediate from Prop 3.11.iv) that
$\phi^{*}$ is injective, so $\phi^{*}$ is a homomorphism onto its image.


\subsubsection*{ii)}

We draw the setup in diagrams. First, we have homomorphisms between the rings as follows
\[ 
	\begin{tikzcd}
		A \arrow{r}{f} \arrow[swap]{d}{s_a} & B \arrow{d}{s_b} \\
		S^{-1}A \arrow{r}{S^{-1}f}& S^{-1}B,
	\end{tikzcd}
\]
where we define $S^{-1}f$ so the diagram commutes $a/s \mapsto f(a)/f(s)$.
This induces the following diagram on the respective spectra
\[ 
	\begin{tikzcd}
		S^{-1}Y \arrow{r}{S^{-1}f^{*}} \arrow[swap]{d}{i_y} & S^{-1}X \arrow{d}{i_x} \\
		Y \arrow{r}{f^{*}}& X,
	\end{tikzcd}
\]
and our task is to show that the diagram commutes, I.e that $S^{-1}f^{*}$ is
the restriction of $f^{*}$, and that $f^{*-1}(S^{-1}X) = S^{-1}Y$. We will
show that $\spec$ is a contravariant functor, from which it will follow that
the diagram commutes. We showed in i) that $\spec$ maps morphisms $f$ in the
$\CRing^{\op}$ to morphisms $f^{*}$ in $\Top$. Now let $f, g$ be two composable
ring homomorphisms. Then $(f \circ g)^{*} = (f \circ g)^{-1} = g^{-1} \circ
f^{-1} = g^{*} \circ f^{*}$, so the functor preserves composition in a
contravariant we've shown the first statement. \\

Our setup differs slightly from that in the exercise in the sense that we've
not identified the spectra of the fraction rings with their inclusions in to
the spectra of the original rings. So, when the exercise asks us to show that
$f^{*-1}(S^{-1}X) = S^{-1}Y$, in our setup, this translates to showing that
$f^{*-1}(i_x(S^{-1}X)) = i_y(S^{-1}Y)$. \\

Let $\mathfrak{p} \in f^{*-1}(i_x(S^{-1}X))$. Then $f^{*}(\mathfrak{p})
\subseteq i_x(S^{-1}X)$. In the original rings, $i_x$ corresponds to
contracting ideals from $S^{-1}A$ back to $A$, so $f^{*}(\mathfrak{p})$ is a
prime ideal in $A$ which doesn't touch $S$. Moreover, $\mathfrak{p}$ is a prime
ideal in $B$ since it lies in the inverse image of $f^{*}$, and it doesn't meet
$f(S)$ since $f^{*}(\mathfrak{p}) = f^{-1}(\mathfrak{p})$ doesn't meet $S$.
It follows that $\mathfrak{p} \in i_y(S^{-1}Y)$ and we are done (the other
inclusion is trivial).

\subsubsection*{iii)}
We have a commutative diagram
\[ 
	\begin{tikzcd}
		A \arrow{r}{f} \arrow[swap]{d}{q_a} & B \arrow{d}{q_b} \\
		A/\mathfrak{a} \arrow{r}{\overline{f}}& B/\mathfrak{b},
	\end{tikzcd}
\]
and the $\spec$ functor induces the following commutative diagram
\[ 
	\begin{tikzcd}
		\spec(B/\mathfrak{b}) \arrow{r}{\overline{f}^{*}} \arrow[swap]{d}{i_y} & \spec(A/\mathfrak{a}) \arrow{d}{i_x} \\
		Y \arrow{r}{f^{*}}& X,
	\end{tikzcd}
\]
which since $i_y, i_x$ are injections, tells us that $\overline{f}^{*}$ is the
restriction of $f^{*}$ to $V(\mathfrak{b}) = i_x(\spec(B/\mathfrak{b}))$. \\

(the following is not part of iii), but used in iv)) Just like in i), we also
show that $f^{*-1}(i_x(\spec(A/\mathfrak{a}))) = i_y(\spec(B/\mathfrak{b}))$.
Let $\mathfrak{p} = f^{*-1}(i_x(\spec(A/\mathfrak{a})))$. Then
$f^{*}(\mathfrak{p}) \in i_x(\spec(A/\mathfrak{a}))$, so $f^{*}(\mathfrak{p})$
is an ideal in $A$ which contains $\mathfrak{a}$. Moreover, $\mathfrak{p}$ is a
prime ideal in $B$ since it lies in the inverse image of $f^{*}$, and it
doesn't meet $\mathfrak{b} = \mathfrak{a}^{e}$ since $f^{*}(\mathfrak{p}) =
f^{-1}(\mathfrak{p})$ doesnt meet $\mathfrak{a} = f^{-1}(\mathfrak{a}^{e})$.
It follows that $\mathfrak{p} \in i_y(\spec(B/\mathfrak{b}))$.


\subsubsection*{iv)}

We first draw two diagrams of our situation. We
have the following diagram of rings
\[ 
	\begin{tikzcd}
		A \arrow{r}{f} \arrow[swap]{d}{l_a} & 
		B \arrow{d}{l_b} \\
		A_{\mathfrak{p}} \arrow{r}{\overline{\overline{f}}} \arrow[swap]{d}{q_a} &
		(f(A) - \mathfrak{p})^{-1} B \arrow[swap]{d}{q_a} \\
		k(\mathfrak{p}) \arrow{r}{\overline{f}} &
		(f(A) - \mathfrak{p})^{-1} B / (\mathfrak{p}B),
	\end{tikzcd}
\]
which after passing through $\spec$ turns into
\[ 
	\begin{tikzcd}
		X & 
		Y 
			\arrow{l}{f^{*}} \\
		\spec(A_{\mathfrak{p}}) \arrow[hookrightarrow, u, "i_a"] &
		\spec((f(A) - \mathfrak{p})^{-1} B) 
			\arrow[hookrightarrow, u, "i_b"] 
			\arrow{l}{\overline{\overline{f^{*}}}} \\
		\spec(k(\mathfrak{p})) 
			\arrow[hookrightarrow, u, "i_a'"] &
		\spec((f(A) - \mathfrak{p})^{-1} B)/(\mathfrak{p}B) 
			\arrow[hookrightarrow, u, "i_b'"] 
			\arrow{l}{\overline{f^{*}}}.
	\end{tikzcd}
\]
We have that $\spec(k(\mathfrak{p})) = \{(0)\}$, and after the two inclusions $i_a', i_a$, the
zero ideal in $\spec(k(\mathfrak{p}))$ is mapped to  $\mathfrak{p}$ in $A$. From ii) and iii) it
also follows that $f^{*-1}(i_a(i_a'(\spec(k(\mathfrak{p}))))) =
i_b(i_b'(\spec((f(A) - \mathfrak{p})^{-1} B/(\mathfrak{p}B))))$. Combining these two facts tells 
us that $f^{*-1}(\mathfrak{p}) = i_b(i_b'(\spec((f(A) - \mathfrak{p})^{-1}
B/(\mathfrak{p}B))))$, so the two spaces are homeomorphic in a natural way. \\

It remains to show that $\spec((f(A) - \mathfrak{p})^{-1} B/(\mathfrak{p}B)) =
\spec(k(\mathfrak{p}) \otimes_{A} B)$. We will show that the corresponding
rings are isomorphic. To see this, let 
\[
	\phi : (f(A) - f(\mathfrak{p}))^{-1}B \to k(\mathfrak{p}) \otimes_{A} B
\] 
be given by $\phi(b/f(a)) = \frac{1 + \mathfrak{p}}{a + \mathfrak{p}} \otimes_{A} b$.
Then $\phi$ is well-defined, since if $b_1/f(a_1) = b_2/f(a_2)$, then there exists
$s \in A - \mathfrak{p}$ such that $f(s)(b_1f(a_2) - b_2f(a_1)) = 0$, whence 
\begin{align*}
	\phi(b_1/f(a_1)) - \phi(b_2/f(a_2))
	&=
	\frac{1 + \mathfrak{p}}{a_1 + \mathfrak{p}} \otimes_A b_1
	-
	\frac{1 + \mathfrak{p}}{a_2 + \mathfrak{p}} \otimes_A b_2 \\
	&=
	\frac{sa_2 + \mathfrak{p}}{sa_1a_2 + \mathfrak{p}} \otimes_A b_1
	-
	\frac{sa_1 + \mathfrak{p}}{sa_1a_2 + \mathfrak{p}} \otimes_A b_2 \\
	&=
	\frac{1 \mathfrak{p}}{sa_1a_2 + \mathfrak{p}} \otimes_A f(s)f(a_2)b_1
	-
	\frac{1 + \mathfrak{p}}{sa_1a_2 + \mathfrak{p}} \otimes_A f(s)f(a_1)b_2 \\
	&=
	\frac{1 + \mathfrak{p}}{sa_1a_2 + \mathfrak{p}} \otimes_A f(s)(f(a_2)b_1 - f(a_1)b_2) \\
	&=
	0.
\end{align*}
Moreover, $\phi$ is surjective since for any $\frac{a_1 + \mathfrak{p}}{a_2 + \mathfrak{p}} \otimes_A b$,
we have 
\[
\frac{a_1 + \mathfrak{p}}{a_2 + \mathfrak{p}} \otimes_A b
=
\frac{1 + \mathfrak{p}}{a_2 + \mathfrak{p}} \otimes_A f(a_1)b
=
f(f(a_1)b/f(a_2)).
\] 
Finally, the kernel of $\phi$ are all $b/f(a)$ such that $b$ can be written as
$b = b'f(a')$ where $a' \in \mathfrak{p}$, in other words, $\ker(\phi) = (f(A)
- \mathfrak{p})^{-1}Bf(\mathfrak{p})$, and the isomorphism follows from the
First Homomorphism Theorem.


\section*{Ch 4}


\subsection*{Ex 4}
To see that $\mathfrak{m}$ is maximal, note that $\mathbb{Z}[t]/\mathfrak{m}
\cong \mathbb{Z}/2\mathbb{Z}$ is a field. To see that $\mathfrak{q}$ is
$\mathfrak{m}$-primary, note that $2 \in r(\mathfrak{q})$ and $t \in
r(\mathfrak{q})$, and since $\mathfrak{m}$ is maximal $r(\mathfrak{q}) =
\mathfrak{m}$. Also, $t \not \in \mathfrak{m}^{k}$ when $k > 1$, and $2 \not
\in \mathfrak{q}$, so $\mathfrak{m}^k \not = \mathfrak{q}$ for any $k \geq 1$.

\subsection*{Ex 5}

$\mathfrak{p}_1 \cap \mathfrak{p}_2 \cap \mathfrak{m}^{2}$ is a primary
decomposition since all of the ideals being intersected are primary
($\mathfrak{m}^{2}$ is so by Prop 4.2, and the others since they're prime). \\

To see that it's reduced, the first criterion is satisfied since the two prime
ideals and $r(\mathfrak{m}^{2}) = \mathfrak{m}$ are all pairwise distinct, and 
the second criterion is fulfilled since
\begin{align*}
	z^{2} &\in \mathfrak{p}_2 \cap \mathfrak{m}^{2} \setminus \mathfrak{p}_1, \\
	y^{2} &\in \mathfrak{p}_1 \cap \mathfrak{m}^{2} \setminus \mathfrak{p}_2, \\
	x &\in \mathfrak{m}^{2} \cap \mathfrak{p}_{1} \setminus \mathfrak{p}_2.
\end{align*}
To see that the decomposition is equal to $\mathfrak{a}$, note that 
\begin{align*}
	\mathfrak{p}_1 \cap \mathfrak{p}_2 \cap \mathfrak{m}^{2}
	&=
	(x, y) \cap (x, z) \cap (x, y, z)^{2} \\
	&=
	(x, yz) \cap (x^{2}, xy, xz, y^{2}, yz, z^{2}) \\
	&=
	(x^{2}, xy, xz, yz) \\
	&=
	(x, y) (x, z) \\
	&=
	\mathfrak{a}.
\end{align*} 
We have that $\mathfrak{p_1}, \mathfrak{p_2}$ are minimal since they are both
contained in $\mathfrak{m}$, but don't contain each other. $\mathfrak{m}$ is
embedded. \\

\subsection*{Ex 7}
\subsubsection*{i)}

Let $f \in \mathfrak{a}^{e}$. Then we can write 
\[
	f = \sum_{i = 0}^{n} h_i(x) a_i
\] 
for some set of elements $a_i \in \mathfrak{a}$ and polynomials $h_i(x) \in
A[x]$. But we have $h_i(x) a_i \in \mathfrak{a}[x]$ for each $i$, so 
$f \in \mathfrak{a}[x]$. \\

For the other inclusion, note that the monomials $a x^k, a \in
\mathfrak{a}, k \in \N$ generate $\mathfrak{a}[x]$ as an $A$-module, and
each $a x^k \in \mathfrak{a}^{e}$ for each $a \in \mathfrak{a}, k \in \N$,
so $\mathfrak{a}[x] \subseteq \mathfrak{a}^{e}$.

\subsubsection*{ii)}

Let $\phi : A[x] \to (A/\mathfrak{p})[x]$ be the homorphism which sends $a_n
x^n + \ldots + a_1 x + a_0$ to $\overline{a_n}x^n + \ldots + \overline{a_1} x +
\overline{a_0}$ where $\overline{a} = a + \mathfrak{p}$. Then it's easy to see
that $\ker(\phi) = \mathfrak{p}[x]$, so $A[x]/\mathfrak{p}[x] \cong
(A/\mathfrak{p})[x]$. The fact that $\mathfrak{p}[x]$ is prime now follows from
the fact that the polynomial ring over an integral domain is an integral domain
(seen easily by looking at degree $0$ terms during polynomial multiplication). 

\subsubsection*{iii)}

That $r(\mathfrak{q}[x]) = \mathfrak{p}[x]$ follows from the following lemma. 
\begin{lemma}
	Let $\mathfrak{a}$ be an ideal in $A$. Then $r(\mathfrak{a}[x]) = (r(\mathfrak{a}))[x]$.
\end{lemma}
\begin{proof}
	Since $\mathfrak{a} \subset \mathfrak{a}[x]$, we have $r(\mathfrak{a})
	\subset r(\mathfrak{a}[x])$, and as $(r(\mathfrak{a}))[x]$ is the extension
	of $r(\mathfrak{a})$ (hence the minimal ideal of $A[x]$ containing
	$r(\mathfrak{a}$), we have $(r(\mathfrak{a}))[x] \subseteq
	r(\mathfrak{a}[x])$. \\

	For the other inclusion, let $g \in A[x]$ be a polynomial of degree $d$
	such that $g^k \in \mathfrak{a}[x]$. Let $a_i$ be the $i$-th coefficient 
	of $g$ and $b_i$ be the $i$-th coefficient of $g^k$. Then
	\[
		b_{ik}
		=
		a_{i}^{k} + H
	\] 
	where $H$ is a sum of terms each having a factor of some $a_j$ for $j < i$.
	Since $b_0 = a_0^k$ we have that $a_0 \in r(\mathfrak{a})$, and it now
	follows from induction that $a_i \in r(\mathfrak{a})$ for all $i \leq d$.
	We see that $g \in (r(\mathfrak{a}))[x]$ and we are done.
\end{proof}

It remains to show that $\mathfrak{q}[x]$ is a primary ideal. This is
equivalent to $A[x]/\mathfrak{q}[x] \cong (A/\mathfrak{q})[x]$ beeing non-zero,
and having all zero-divisors being nilpotent. We know it's non-zero since $a
\not \in \mathfrak{a} \Rightarrow a \not \in \mathfrak{q}[x]$. \\

Let $f \in (A/\mathfrak{q})[x]$ be a zero divisor. By Ex 1.2 iii), there exists
some $a \in A/\mathfrak{q}$ such that $af = 0$. But then every coefficient of
$f$ must be a zero-divisor. Since $\mathfrak{q}$ is primary, and the
coefficients lie in $A/\mathfrak{q}$, they must then all be nilpotent. Ex 1.2
ii) now tells us that $f$ is nilpotent.

\subsubsection*{iv)}

It follows from iii) that the primary decomposition of $\mathfrak{q}[x]$ is as
given. \\

Non-containment follows from $q \in \left(\bigcap_{j \not = i} \mathfrak{q}_j\right)
\setminus \mathfrak{q}_i \Rightarrow qx \in \left(\bigcap_{j \not = i}
\mathfrak{q}_j[x]\right) \setminus \mathfrak{q}_i[x]$. \\

Similarly, we have that the prime ideals $\mathfrak{p}_i[x]$ belonging to the
$\mathfrak{q}_i[x]$ are all different from one another by virtue of the
$\mathfrak{p}_i$ being different from one another.

\subsubsection*{v)}

We have that the minimal prime ideals belonging to $\mathfrak{a}[x]$ are the
minimal elements among the set of prime ideals containing $\mathfrak{a}[x]$. \\

It follows from ii) that a minimal prime ideal $\mathfrak{p}$ belonging to
$\mathfrak{a}$ extends to a prime ideal $\mathfrak{p}[x]$ containing
$\mathfrak{a}[x]$. We are done if we can show that all prime ideals containing
$\mathfrak{a}[x]$ are extensions of prime ideals containing $\mathfrak{a}$. \\

Let $\mathfrak{b}$ be a prime ideal containing $\mathfrak{a}[x]$. Then
$\mathfrak{b}^{c} = \mathfrak{b}/(x)$ is a prime ideal in $A$ containing
$\mathfrak{a}$. We also have $\mathfrak{b} \supseteq \mathfrak{b}^{ce} =
(\mathfrak{b}/(x))[x]$. But clearly, $(\mathfrak{b}/(x))[x] \supseteq
\mathfrak{b}$, so $\mathfrak{b}$ is an extension of a prime ideal in $A$ which
contains $\mathfrak{a}$.

\subsection*{Ex 8}

First note that $\mathfrak{p}_i$ is maximal in $A_i = k[x_1, x_2, \ldots, x_i]$
since $A_i/\mathfrak{p}_i \cong A$ is a field. It follows that the powers of
$\mathfrak{p}_i$ are $\mathfrak{p}_i$-primary. Ex 7 tells us that
$\mathfrak{p}_i$ is prime in $A_{i + 1}$, and that it's powers are
$\mathfrak{p}_i$-primary in $A_{i + 1}$. Repeating this argument inductively
shows that this holds for all $A_j, j \geq i$.

\subsection*{Ex 9}

Let $x \in A$ be a zero divisor such that $xa = 0$ for some $a \not = 0$. Then
$x \in (0:a)$, so any prime ideal which contains $(0:a)$ must also contain $x$.
There is at least one prime ideal containing $(0:a)$ since $1 \cdot a \not = 0$
and any non-trivial ideal is contained in a maximal ideal, whence there is a
minimal prime ideal which contains $(0:a) \supseteq \{x\}$. \\

For the other inclusion, let $x \in \mathfrak{p} \in D(A)$. Then there is some
$a \in A$ such that $\mathfrak{p}$ is minimal among the set of prime ideals
which contain $(0:a)$. Note that all elements in $(0:a)$ are zero divisors, so
the set $\mathfrak{p}' = \mathfrak{p} \cap D$ where $D$ are the zero divisors
in $A$, contains $(0:a)$ as well. Moreover, it's an ideal since $D$ is an
ideal. We will show that it's prime as well. If $bc \in \mathfrak{p}'$ then
either $b$ or $c$ is in $\mathfrak{p}$. Say that $b \in \mathfrak{p}$, we also
have that $b$ is a zero divisor, since $bc$ is, so $b \in D$ also. By
minimality of $\mathfrak{p}$, it follows that $\mathfrak{p'} = \mathfrak{p}$
and all elements in $\mathfrak{p}$ are zero divisors. \\

For the second part of the question, let $\mathfrak{p} \in D(S^{-1}A)$. Then
$\mathfrak{p} \in \spec(S^{-1}A)$ by definition. Let $a/s \in S^{-1}A$ be such
that $\mathfrak{p}$ is a minimal prime ideal containing $(0:a/s) = (0:a/1)$.
Then Prop 3.11 iv) tells us that $\mathfrak{p}^{c}$ is a prime ideal. Since
function inverses preserve inclusion, we have that $\mathfrak{p}^{c}$ contains
$(0:a/1)^{c} = (0:a)^{ec} \supseteq (0:a)$. Even more, if $\mathfrak{q}$ is a
prime ideal such that $(0:a) \subseteq \mathfrak{q} \subseteq
\mathfrak{p}^{c}$, then $(0:a/1) \subseteq \mathfrak{q}^{e} \subseteq
\mathfrak{p}^{ce} = \mathfrak{p}$, and $\mathfrak{q}^{e}$ is prime by Prop 3.11
iv). But since $\mathfrak{p}$ is the minimal prime ideal containing, we have
that $\mathfrak{q}^{e} = \mathfrak{p}$ whence we see that $\mathfrak{p}^{c}$ is
a minimal ideal containing $(0:a)$. In summary $\mathfrak{p}^{c} \in D(A) \cap
\spec(S^{-1}A)$. \\

For the other inclusion, let $\mathfrak{p} \in D(A) \cap \spec(S^{-1}A)$. Then
$\mathfrak{p}$ is a prime ideal in $A$, which doesn't meet $S$, and is a
minimal prime ideal containing some $(0:a)$. Then clearly, $\mathfrak{p}^{e}$
is a minimal prime ideal containing $(0:a)^{e} = (0:a/1)$ whence
$\mathfrak{p}^{e} \in D(S^{-1}A)$. \\

For the third part of the question, Prop 4.5 tells us that each
$\mathfrak{p}_i$ are the prime ideals occuring in the set $r(0:a)$ for some $a
\in A$. Furthermore, any prime ideal containing $(0:a)$ must also contain
$r(0:a)$, so if $\mathfrak{p}_i = r(0:a)$ is prime, then $\mathfrak{p}_i$ is
minimal among the prime ideals containing $(0:a)$. Also, if $r(0:a)$, let
$\mathfrak{p}$ be the minimal prime ideal containing $r(0:a)$. The proof of
Prop 4.5 tells us that $r(0:a) = \bigcap \mathfrak{p}_i$, and thus
$\mathfrak{p} \supseteq \bigcap \mathfrak{p}_i$, whence Prop 1.11 tells us that
$\mathfrak{p}$ contains some $\mathfrak{p}_j$, but $\mathfrak{p}_j \supset
(0:a)$, so $\mathfrak{p} = \mathfrak{p}_j$ by minimality of $\mathfrak{p}$.

\subsection*{Ex 10}
\subsubsection*{i)}

Let $x \in S_{\mathfrak{p}}(0)$. Then there exists $a \in A - \mathfrak{p}$
such that $xa = 0$, so $x$ is a zero divisor, and $x \in \mathfrak{p}$ since $0
\in \mathfrak{p}$ but $a \not \in \mathfrak{p}$.

\subsubsection*{ii)}

We have the following chain of equivalences
\begin{align*}
	\text{$\mathfrak{p}$ is a minimal prime ideal}
	&\xLeftrightarrow{\text{Cor 3.13}}
	\text{$A_{\mathfrak{p}}$ is a ring with only one prime ideal} \\
	&\xLeftrightarrow{\text{Prop 1.8}}
	\mathfrak{p}^{e} = \nr(A_{\mathfrak{p}}) \\
	&\Leftrightarrow
	\mathfrak{p} = r(S_{\mathfrak{p}}(0))
\end{align*}

\subsubsection*{iii)}

Let $x \in S_{\mathfrak{p}}(0)$. Then there exists $a \in A - \mathfrak{p}$
such that $xa = 0$. But then $a \in A - \mathfrak{p}' \supseteq A -
\mathfrak{p}$, so $x \in S_{\mathfrak{p'}}(0)$.

\subsubsection*{iv)}

Let $x \in \bigcap_{\mathfrak{p} \in D(A)} S_{\mathfrak{p}}(0)$. Then there
exists an $a_{\mathfrak{p}} \in A - \mathfrak{p}$ for all $\mathfrak{p} \in
D(A)$ such that $xa_{\mathfrak{p}} = 0$. In other words, the anihilator of $x$
is not contained in any of the $\mathfrak{p} \in D(A)$. But any non-trivial
ideal is contained in some prime ideal, so any non-trivial annihilator is
contained in some ideal in $D(A)$. It follows that $(0:x) = (1)$ and $x = 0$.

\subsection*{Ex 13}

\subsubsection*{i)}

First we show that $r(\mathfrak{p}^{(n)}) = \mathfrak{p}$. Let $x^m \in
\mathfrak{p}^{(n)}$. Then there exists some $a \in A - \mathfrak{p}$ and $p \in
\mathfrak{p}^{n}$ such $a(x^{m} - p) = 0 \Leftrightarrow ax^{m} = ap$. As
$ax^{m} = ap \in \mathfrak{p}$, but $a \not \in \mathfrak{p}$, we get $x \in
\mathfrak{p}$. \\

Now we show that the ideal is primary, we have the following chain of
equalities $(\mathfrak{p}^{(n)})^{e} = (\mathfrak{p}^{n})^{ece} =
(\mathfrak{p}^{n})^{e} = (\mathfrak{p}^{e})^{n}$, which shows that
$(\mathfrak{p}^{(n)})^{e}$ is $\mathfrak{p}^{e}$-primary in $A_{p}$, since it's
a power of a maximal ideal. We can now use Prop 4.8 ii) to conclude that
$\mathfrak{p}^{(n)}$ is primary. 


\subsubsection*{ii)}

Let $\mathfrak{a}$ be the smallest $\mathfrak{p}$-primary ideal containing
$\mathfrak{p}^{n}$.

Let $\mathfrak{p}^n = \bigcap \mathfrak{q}_i$ be a minimal primary composition.
Then 

\section*{Ch 5}

\subsection*{Ex 1}

Let $V(E) \in \spec(B)$ be a closed set, and $I$ be the radical ideal $I =
\bigcap_{\mathfrak{p} \in V(E)}$ such that $V(E) = V(I)$. We will show that
$f^{*}(V(E)) = f^{*}(V(I)) = V(f^{-1}(I))$. It's immedieate that $f^{*}(V(I))
\subseteq V(f^{-1}(I)) $ since every prime ideal in $f^{*}(V(I))$ will contain
$f^{-1}(I)$. \\

For the other inclusion, let $\mathfrak{p} \in V(f^{-1}(I))$ be a prime ideal
in $A$. Then $f(\mathfrak{p})$ is a prime ideal in $f(A)$, and Theorem 5.10
tells us that there exists some prime ideal $\mathfrak{q}$ in $B$ such that
$f(\mathfrak{p}) = \mathfrak{q} \cap f(A)$. By applying $f^{-1}$ we see that
$f^{-1}(f(\mathfrak{p})) = f^{*}(\mathfrak{q})$. Now we claim that
$\mathfrak{p} = f^{-1}(f(\mathfrak{p}))$. To see this, note that if $a \in A$
is such that $f(a) = f(p)$ for some $p \in \mathfrak{p}$, then $a - p \in
\ker(f)$, but $\mathfrak{p}$ contains $\ker(f)$ since $\mathfrak{p}$ contains
$f^{-1}(I)$. It follows that $a \in \mathfrak{p}$ and $\mathfrak{p} =
f^{-1}(f(\mathfrak{p})) = f^{*}(\mathfrak{q}) \in f^{*}(V(I))$.


\subsection*{Ex 8}

\subsubsection*{i)}

The hint gives the whole solution, but we write it out in full anyway. Let
$\overline{K}$ be the algebraic closure of the field of fractions of $B$. Then
we can factor $f, g$ in $\overline{K}[x]$ as
\begin{align*}
	f &= \prod (x - \chi_i), \\
	g &= \prod (x - \eta_i).
\end{align*}
Then the $\chi_i, \eta_i$ are roots of $fg$ (which is monic since $f, g$ are),
and they are all integral over $C$. $C$ is integrally closed, so the $\chi_i,
\eta_i$ all lie in $C$, whence the coefficients of the polynomials $f, g$ do as
well, so $f, g \in C[x]$.

\subsubsection*{ii)}

We no longer have a full field of fractions, and we can't construct a field
which contains $B$. What we can do, is to create a bigger ring $B^{+} \supset
B$, such that the polynomials $f, g$ factor into linear factors over $B^{+}$,
after which the remainder of the proof would come along just like above. \\

Let $y_1$ be a new formal indeterminate, and consider $B_1 = B[y_1]/f(y_1)$. In
this ring, we have that $f(y_1) = 0$, so $x - y_1$ is a factor of $f = (x -
y_1)f_1$ (since the polynomials are monic, the division algorithm still works
in the non-field $B$). Now repeat this process starting with $B_1, f_1$, and
continue this way until $f$ is completely factored into linear factors, and
then do the same for $g$. The final result is a ring $B^{+}$ containing $B$,
where $f, g$ splits into linear factors, from which we may proceed like in part
i).


\subsection*{Ex 9}

We proceed as advised by the hint. Let $f(x) \in B[x]$ be integral over $A[x]$ such that 
\[
	f^{n} + f^{n - 1}g_{n-1} + \ldots g_{0} = 0.
\] 
A small side-step, note that we can't apply Ex 8.ii) to 
\[
	f(f^{n - 1} + f^{n - 2}g_{n - 1} + \ldots g_1) = -g_0 \in A[x],
\] 
since the polynomials $f, (f^{n - 1} + f^{n - 2}g_{n - 1} + \ldots g_1)$ need
not be monic. To circumvent this issue, let $f_1 = f - x^{r}$ for some $r \geq
\max(n, \deg(g_i), \deg(f))$. Then $f_1$ is monic and
\[
	(f_1 + x^{r})^{n} + (f + x^{r})^{n - 1}g_{n-1} + \ldots g_{0} = 0.
\] 
Expanding the expression above yields
\[
	f_1^{n} + f_1^{n - 1}h_{n-1} + \ldots h_{0} = 0.
\] 
where $h_0 = \sum_{i = 0}^{n} x^{nr - ir}g_{n - i} \in A[x]$ with $g_n = 1$. We
can now apply Ex 8.ii) to
\[
	f_1 (f_1^{n-1} + f_1^{n - 2}h_{n-1} + \ldots h_{1}) = -h_0 \in A[x],
\] 
since $f_1$ is monic, and $(f_1^{n-1} + f_1^{n - 2}h_{n-1} + \ldots h_{1})$ is
monic as well since we picked $r$ large enough. It follows that $f_1 \in C[x]$,
so $f \in C[x]$ as well and we are done.

\subsection*{Ex 16}

We pick up from the following task in the exercise text: "Show that $x_n$ is
integral over the ring $A' = k[x_1', x_2', \ldots, x_{n-1}']$." \\

It follows from the construction of the $\lambda_i, x'_i$ that $f(x_1' +
\lambda_1 x_n, x_2' + \lambda_2 x_n, \ldots, x_{n-1}' + \lambda_{n-1} x_n, x_n)
= 0$. Moreover, this polynomial has a leading coefficient $\lambda =
H(\lambda_1, \lambda_2, \ldots \lambda_{n-1}, 1) \not = 0 \in k$ when
considered as a polynomial in $k[x_1', x_2', \ldots, x_{n-1'}][x_n]$,
and can therefore easily be made monic, whence $x_n$ is algebraic over 
$k[x_1', x_2', \ldots x_{n-1}']$.

\subsection*{Ex 17}

By reverse inclusion, maximal ideals correspond to minimal varieties, and by
the proof in the exercise text, any proper ideals correspond to non-empty
varieties. So maximal (proper by definition) correspond to minimal non-empty
varieties. Points are non-empty minimal sets, and they are all varieties since
they induce ideals of the form given in the text. It follows that all maximal
ideals are of the given form as well.

\subsection*{Ex 31}
Let $R = \{x \in K^{*} : v(x) \geq 0 \} \cup \{0\}$. Then $R$ is a sub ring of
$K$ since if $a, b \in R$, then
\begin{itemize}
	\item $v(a + b) \geq \min(v(a), v(b)) \geq 0 \Rightarrow a + b \in R$
	\item $v(ab) = v(a) + v(b) \geq 0 \Rightarrow ab \in R$ 
	\item $v(1) = v(1^{m}) = mv(1)$ for all $m \in \N$ so $v(1) = 0$
		and $1 \in R$. We also have $v(-1) = v(1)/2 = 0$.
	\item $v(-a) = v(-1 \cdot a) = v(-1) + v(a) = v(a) \geq 0$ so $-a \in R$.
\end{itemize}

It's an integral domain since it's a sub ring of a field. Finally, it's an
evaluation ring, since $0 = v(1) = v(a a^{-1}) = v(a) + v(a^{-1})$ so if $v(a)
< 0$ then $v(a^{-1}) > 0$ and vice versa (if $v(a) = v(a^{-1}) = 0$ then both
$a, a^{-1} \in R$, which is allowed).

\section*{Ch 6}


\subsection*{Ex 3}

By Prop 6.3 we have $M / N_i$ Noetherian, Cor 6.4 gives $(M / N_1) \oplus (M /
N_2)$ Noetherian, and we are done after noting that $M/(N_1 \cap N_2) \cong (M
/ N_1) \oplus (M / N_2)$ (was an early exercise or proposition or something,
follows by considering the kernel of $m \mapsto (m + N_1, m + N_2)$). 

\subsection*{Ex 8}

No, we can still have strict infinite ascending chains of ideals which aren't
prime. But to aid us in our search for a counterexample, we'll explore if there
is anything useful which must hold in such a scenario first. \\

Let $A$ be a ring such that $\spec(A)$ is Noetherian. Let 
\[
	I_0 \subseteq I_1 \subseteq \ldots
\] 
be a ascending chain of ideals. Then $V(I_i)$ is an ascending chain of closed
subsets in $\spec(A)$, hence it must be stationairy after some index $i = k$.
From the definition of the spectrum toplogy, it follows that all ideals $I_i$
from the point $i = k$ and on, are covered by the same prime ideals. Hence
they have the same radical. Thus every ideal in the tail of the chain 
\[
	I_k \subseteq I_{k + 1} \subseteq \ldots
\] 
lies in $r(I_k)$, and has $r(I_k)$ as its radical. Passing to $A' = A/I_k$ and
writing $J_i = I_{k + i} + I_k \in A / I_k$, we get an infnite ascending chain
\[
	J_0 \subseteq J_1 \subseteq \ldots
\] 
which is contained in the nilradical of $A'$. We will use this to craft a
counter example. \\

Let $k$ be a algebraically closed field of characteristic $0$. Let $B = k[x_1,
x_2, \ldots, ]$ be the polynomial ring in infinitely many indeterminates. Let
$A = B / (x_1^{2}, x_2^{2}, \ldots, )$ be the quotient ring where we've modded
out all powers of $x_i$. Then
\[
	(x_1) \subsetneq (x_1, x_2) \subsetneq (x_1, x_2, x_3) \subsetneq \ldots
\] 
is a strictly ascending infinite chain of ideals. However any prime ideal of
$A$ must contain the nilradical, which contains every $x_i$. But $(x_1, x_2,
\ldots)$ is maximal since we get $k$ if we quotient by it. It follows that $A$
only has one prime ideal, whence $\spec(A)$ is Noetherian. 


\section*{Ch 7}


\subsection*{Ex 2}

Suppose $f$ is nilpotent and $f^k = 0$. Then the the $0$-th coefficient of
$f^k$ is $a_0^k = 0$ so $a_0$ is nilpotent. Denothe the $m$-th coefficient of
$f^{k}$ by $c_m$. Then $c_{km} = a_{m}^{k} + E(k, m)$ where $E(k, m)$ is an sum
where each term has a factor $a_{j}^{i}$ for some $j < m, i \geq k$. By
induction we have that $E(k, m) = 0$, so $a_{m}^{k} = 0$ for all $m$. \\

Now assume that all $a_i$ are nilpotent. We need the following lemma before we
show that $f$ is nilpotent. Then $a_i \in r(0)$ by definition, and Corollary
7.14 tells us that there is some exponent such that $r(0)^{k} = 0$. It follows
that any product of $k$ elements from $a_i$ is $0$. But the coefficients of $f^k$
are sums of $k$-element products of $a_i$, so $f^{k} = 0$.

\subsection*{Ex 11}

No, but before we produce a counterexample, we'll try and find some properties
which such a counterexample must exhibit. \\

Let $A$ be a ring which is locally Noetherian. Let 
\[
	I_1 \subsetneq I_2 \subsetneq \ldots
\] 
be an infinite strictly ascending chain. Let $\mathfrak{p}$ be a prime ideal.
Then the chain of $I_i^{e}$ is stationary in $A_{\mathfrak{p}}$ by hypothesis.
Let $k$ be the index from which all the $I_i$ extend to the same ideal in
$A_{\mathfrak{p}}$. Since the original chain is strictly ascending, there
exists $x \in I_k \setminus I_{k + 1}$. Since $x/1 \in I_{k + 1}^{e}$, there
exist some $y \in I_{k + 1}$ such that $x/s = y/t$ for $s, t \in A -
\mathfrak{p}$ and this is the case if and only if we have $u \in A -
\mathfrak{p}$ such that $u(xt - ys) \in A - \mathfrak{p}$. If the ideal $x, y
\in \mathfrak{p}$, we'd have that $u(xt - ys) \in \mathfrak{p}$ a
contradiction. So there can be no ideal which contains the entire chain, since
such an ideal would be contained in a maximal (I.e prime) ideal. In particular,
the union of the chain can't be an ideal. \\ 

Even more, every ideal in the chain has to be contained in infinitely many
maximal ideals. Suppose towards a contradiciton $I_r$ is some ideal only
contained in the finitely many maximal ideals $\mathfrak{m}_1, \mathfrak{m}_2,
\ldots, \mathfrak{m}_n$. Then there has to be some index $r_1 > r$ such that
$I_{r_1} \not \subset \mathfrak{m}_1$, as otherwise $\mathfrak{m}_1$ would
contain the whole chain. Let $R = \max(r_i)$. Since $I_R$ is an ideal, it's
contained in some maximal ideal $\mathfrak{m}_R$ which isn't equal to any other
$\mathfrak{m}_{r_i}$ since none of them contain $I_R$. But then since $I_R
\supset I_r$, we have that $\mathfrak{m}_R \supset I_r$, a contradiction. \\

Let $k$ be an algebraically closed field of characteristic $0$ and $A =
\prod^{\infty}_{i = 1} k$. First, note that 
\[
	((1, 0, 0, 0, \ldots, ))
	\subsetneq
	((1, 1, 0, 0, \ldots, ))
	\subsetneq
	((1, 1, 1, 0, \ldots, ))
	\subsetneq
	((1, 1, 1, 1, 0, \ldots, ))
\] 
is an infinite chain of strictly increasing ideals in $A$, whence $A$ isn't
Noetherian. \\

The maximal ideals coincide with the prime ideals in $A$ and are precisely the
$\mathfrak{m}_i$ where 
\[
	\mathfrak{m_i} = ((1, 1, 1, \ldots, 1, 0, 1, 1, 1, \ldots))
\] 
where the $0$ is at index $i$. To see this, first note that since $k$ is a
field, any ideal in $A$ can be identified with the indices where it has
non-zero elements. Then suppose $\mathfrak{a}$ is an ideal in $A$ which is zero
at more than one index. Say at least indices $i$ and $j$, $i \not = j$. Then
$A/\mathfrak{a}$ contains the zero divisors $(1_i + \mathfrak{a})(1_j +
\mathfrak{a}) = 0$ (where $1_i$ is the element of all zeros except for a $1$ at
index $i$). So any prime ideal is of the form $\mathfrak{m}_i$. The
$\mathfrak{m}_i$ are maximal since they induce fields as quotient rings. \\

Let's dig a little deeper and investigate what the induced localizations look
like. Let $f : A \to A_{\mathfrak{m}_1}$ be the canonical injection and
consider some elements $x,  y \in A$. Then $f(x) = f(y)$ if and only if there
are $u, s, t, \in A \setminus \mathfrak{m}_1$ such that $u(xt - ys) = 0$. Let
$p : A \to k$ denote the projection onto the first entry. I.e $p(a, b, c,
\ldots,) = a$. Let $i : k \to A$ denote the corresponding injection. The
elements in $A \setminus \mathfrak{m}_{1}$ are precisely those which have a
non-zero entry in the first index, and if $p(x) = X, p(y) = Y$ where $X, Y \not
= 0$, then $i(1)(x i(X^{-1}) - y i(Y^{-1})) = 0$. If both $X = Y = 0$, then
picking $u = s = t = i(1)$ also results in $u(xt - ys) = 0$. If only one of $X,
Y$ are non-zero, then we can't pick $u, s, t \in A \setminus \mathfrak{m}_1$
such that $u(xt - ys) = 1$. So in other words, $f(x) = f(y)$ precisely when
either both $x, y \in \mathfrak{m}_1$ or both $x, y \not \in \mathfrak{m}_1$.
So each $A_{\mathfrak{m}_i}$ is a ring of two elements, hence isomorphic to
$\F_2$, whence they are all clearly Noetherian.



\subsection*{Ex 14}

We flesh out the steps in the hint to show that $r(\mathfrak{a}) \supseteq
I(V(\mathfrak{a}))$. \\

Let $f \in A \setminus r(\mathfrak{a})$. Then since $r(\mathfrak{a})$ is the
intersection of all prime ideals containing $\mathfrak{a}$, there must be some
prime ideal $\mathfrak{p}$ which does not contain $f$. Let $\overline{f}$ be
the image of $f$ in $B = A/\mathfrak{p}$. Let $C = B_{f} = B[1/\overline{f}]$
and let $\mathfrak{m}$ be a maximal ideal of $C$. $C$ as a $k$-algebra is
finitely generated by $1/\overline{f}, \overline{t_1}, \overline{t_2}, \ldots,
\overline{t_n}$, and Corollary 7.9 can be applied to yield $C/\mathfrak{m}
\cong k$. Let $\phi: A \to k$ be given by the composition
\[
\begin{tikzcd}
	A \arrow[r, "a"] & 
	B = A / \mathfrak{p} \arrow[r, "b"] &
	C = B_{f} \arrow[r, "c"] &
	C / \mathfrak{m} \arrow[r, "d"] &
	k.
\end{tikzcd}
\] 
Then $(\phi(t_1), \phi(t_2), \ldots, \phi(t_n))$ defines a point in $k^{n}$.
Moreover, consider any $g \in \mathfrak{a}$. We have that $\phi(g(t_1, t_2,
\ldots, t_n)) = 0$ since $a(g(t_1, t_2, \ldots, t_n)) = 0$. But all of the $a,
b, c, d$ are homomorphisms and commute with algebraic operations (which is what
polynomials really are), so $g(\phi(t_1), \phi(t_2), \ldots, \phi(t_n)) = 0$
and $(\phi(t_1), \phi(t_2), \ldots, \phi(t_n)) \in V(\mathfrak{a})$. We also
have $\phi(f(t_1, t_2, \ldots, t_n)) \not = 0$, since $b(a(f(t_1, t_2, \ldots,
t_n)))$ is a unit, and $k \not = 0$, so $x = (\phi(t_1), \phi(t_2), \ldots,
\phi(t_n))$ is a point in $V(\mathfrak{a})$ such that $f(x) \not = 0$ whence $f
\not \in I(V(\mathfrak{a})).$

\section*{Ch 9}


\subsection*{Ex 2}


First of, every coefficient in $fg$ is a sum of terms of the form $a_i b_j$
with $a_i \in c(f), b_j \in c(g)$ so $c(fg) \subseteq c(f)c(g)$. \\

(the arguments in this paragraph uses lots of results from Prop 9.2 and the
proof of it without reference.) For the other inclusion, first consider the
case when $A$ is a discrete valuation ring. Let $x$ generate the maximal ideal
$\mathfrak{m}$ in $A$, and let $v$ be the valuation on $A$ where $v(a) = k
\Leftrightarrow (a) = (x^{k})$. Then $v(a + b) = \min(v(a) + v(b))$, and there
exist $a_i, b_j$ such that $v(a_ib_j) = v(c(f)c(g))$. But $a_ib_j$ is a term in
the sum which is the $i + j$-th coefficient of $fg$, so $v(c(fg)) \leq
v(a_ib_j)$, and it follows that $c(fg) \supseteq (x^{v(a_ib_j)}) = c(f)c(g)$,
whence $c(fg) = c(f)c(g)$. \\

We give a quick lemma before we generalize to Dedekind domains.

\begin{lemma}
	Let $A$ be a Dedekind domain, $\mathfrak{m}$ a prime ideal and
	$\mathfrak{a}$ an arbitrary ideal. Then the exponent of $\mathfrak{m}$ in
	the prime ideal factorisation of $\mathfrak{a}$ is precisely the exponent
	of $\mathfrak{a}^{e} = (\mathfrak{m}^{e})^{r}$ in the localization
	$A_{\mathfrak{m}}$ (where we allow $r = 0$).
\end{lemma}
\begin{proof}
	Let $\mathfrak{m}$ be a primary ideal belonging to $\mathfrak{a}$, and $r,
	\mathfrak{b}$ be such that $\mathfrak{a} = \mathfrak{m}^{r}\mathfrak{b}$
	and $r$ is maximal. It follows that $\mathfrak{b}$ isn't contained in
	$\mathfrak{m}$. Then by 3.11.ii) and 3.11.iv), we have 
	\begin{align*}
		(A - \mathfrak{m})^{-1}\mathfrak{a}
		&=
		((A - \mathfrak{m})^{-1}\mathfrak{m})^{r}
		(A - \mathfrak{m})^{-1}\mathfrak{b} \\
		&=
		((A - \mathfrak{m})^{-1}\mathfrak{m})^{r}
	\end{align*}
\end{proof}

Now let $A$ be a Dedekind domain. Since Gauss's lemma holds
in discrete valuation rings, we see that $c(f)c(g)$ extends to
the same ideal as $c(fg)$ in every localization. It follows that 
they have the same prime factorization, whence they must be equal.


\subsection*{Ex 9}

First we show that the problem statement is equivalent to the given sequence
being exact. This follows from the fact that $x \in A$ is a solution to the
equation system induced by $\phi(x) = (x_1 + \mathfrak{a}_1, \ldots, x_n +
\mathfrak{a}_n)$, and if the sequence is exact, then any elements in
$\bigoplus_{i=1}^{n} A/\mathfrak{a}_i$ are in in the image of $\phi$ precisely
when they are in the kernel of $\psi$. $(x_1 + \mathfrak{a}_1, \ldots, x_n +
\mathfrak{a}_n)$ is in the kernel of $\psi$ exactly when $x_i - x_j +
\mathfrak{a}_i + \mathfrak{a}_j = 0$, and we see that the two formulations of
the problem are equivalent. \\

Prop 3.8 tells us that exactness is a local property (it also tells us that
it's enough to localize at maximal ideals, so we may skip localizing at $(0)$),
and since all localizations of a Dedekind domain are discrete valuation rings,
we show that the sequence is exact when $A$ is a DVR. \\

Since $A$ is a DVR, we may write $\mathfrak{a}_i = (a)^{k_i}$ where $a$
generates the maximal ideal of $A$. It also follows that $\mathfrak{a}_i +
\mathfrak{a}_j = (a)^{k_{ij}}$ where $k_{ij} = \max(k_i, k_j)$. Now, since all
$x_i$ are in the equivalence class of $x$ in their corresponding quotient
rings, we can write $\phi(x) = (x + (a)^{k_1}, \ldots, x + (a)^{k_n})$. Then we
see that $\psi(\phi(x))$ consists of entries $x - x + (a)^{k_i} + (a)^{k_j} = x
- x + (a)^{k_{ij}} = (a)^{k_{ij}}$, so $\im(\phi) \subseteq \ker(\psi)$. Now
let $x_i \in A$ and $x' = (x_1 + (a)^{k_1}, \ldots, x_n + a^{k_n})$ be such
that $x' \in \ker(\psi)$. Then $x_i - x_j \in (a)^{k_{ij}}$ for every $i \not =
j$. Let $s \in [1..n]$ be such that $k_s$ is maximal. Then every single $x_i$
is congruent to $x_s$ modulo $(a^{k_s})$, and since every other $(a)^{k_i}
\supseteq (a)^{k_s}$, they are all congruent to $x_s$ modulo every $(a)^{k_i}$.
Pick some representative $x \in x_s + (a)^{k_s}$. Then $\phi(x) = x'$, and we
see that $\im(\phi) = \ker(\psi)$.


\section*{Ch 10}

\subsection*{Ex 9}

We follow the hint. Let $\phi : A[x] \to (A/\mathfrak{m})[x]$ be the natural
homomorphism and $g_0, h_0$ be elements in the preimages of $\overline{g},
\overline{h}$. Then $\phi(g_0h_0) = \overline{f}$ by definition. Hence $g_0h_0
- f \in \ker(\phi) = \mathfrak{m}A[x]$. Now inductively assume that we have
$g_{k-1}, h_{k-1}$ in the preimages of $\overline{g}, \overline{h}$ such that
$g_{k-1}h_{k-1} - f \in \mathfrak{m}^{k - 1}A[x]$. We will construct $g_k, h_k$
with analogous properties and the same degrees. We will let $p, q \in A[x]$ be
generic polynomials and set $g_k = g_{k-1} + p, h_k = h_{k - 1} + q$. We then
want to set $p, q$ in such a way that
\begin{align*}
	f - g_kh_k &\in \mathfrak{m}^{k}A[x], \\
	g_k - g_{k - 1} &\in \mathfrak{m}^{k - 1}A[x], \\
	h_k - h_{k - 1} &\in \mathfrak{m}^{k - 1}A[x], \\
	\deg(p) &\leq \deg(g_{k-1}), \\
	\deg(q) &\leq \deg(h_{k-1}).
\end{align*}
If we can do this, then it follows that the $i$-th coefficients of $(g_j)$
forms a Cauchy sequence in $A$, and since $A$ is $\mathfrak{m}$-adically
complete, we have that there is some $g \in A[x]$ which is sent to $(g_j)$ by
the natural map $A[x] \to \hat{A}[x]$. The first condition implies that
$(g_j)(h_j) = f$ in $\hat{A}[x]$, and it follows that $gh = f$. We now turn our
efforts to finding $p, q$ which fulfil the requirements above. \\

Since $f - g_{k-1}h_{k-1} \in \mathfrak{m}^{k - 1}A[x]$, we can write
$f - g_{k-1}h_{k-1} = \sum c_i x^{i}$ where $c_i \in \mathfrak{m}^{k-1}A[x]$.
Our requirements can now be rewritten into
\begin{align*}
	pq + ph_k + qg_k - \sum c_i x^{i} &\in \mathfrak{m}^{k}A[x], \\
	p &\in \mathfrak{m}^{k - 1}A[x], \\
	q &\in \mathfrak{m}^{k - 1}A[x].
\end{align*}
Since $\phi(g_{k-1}), \phi(h_{k-1})$ are monic, they generate the trivial ideal
$(1)$ in $(A/\mathfrak{m})[x]$, and it follows that there exist $a_i, b_i \in
A[x]$ such that $a_i g_{k-1} + b_i h_{k - 1} = x_i + r_i$ where $r_i \in
\mathfrak{m}A[x]$. Moreover, since $A/\mathfrak{m}$ is a field,
it follows that $\deg(a_i) \leq \deg(h_{k-1}), \deg(b_i) \leq \deg(g_{k - 1})$.
If we let $p = \sum b_i c_i , q = \sum a_i c_i$ we get that both $p, q \in
\mathfrak{m}^{k-1}A[x]$, the degrees are sufficiently low, and
\begin{align*}
	pq + ph_k + qg_k - \sum c_i x^{i}
	&=
	pq + \sum c_i (
		b_i h_k
		+
		a_i g_k
		-
		x_i
	) \\
	&=
	pq + \sum c_i r_i \\
	&\in
	\mathfrak{m}^{(k - 1)^{2}}A[x] + \mathfrak{m}^{k-1}\mathfrak{m}A[x] \\
	&=
	\mathfrak{m}^{k}A[x],
\end{align*}
whence we are done.


\section*{Ch 11}

\subsection*{Ex 6}
We begin with showing the inequality $1 + \dim A \leq \dim A[x]$. Let 
\[
	\mathfrak{p}_0 \subsetneq \mathfrak{p}_1 \subsetneq \ldots \subsetneq \mathfrak{p}_n
\] 
be a strictly increasing chain of prime ideals in $A$. Then let $\mathfrak{q}_i
= \mathfrak{p}_i (x)$ be the ideal of polynomials in $A[x]$ with coefficients
in $\mathfrak{p}_i$. Then every $\mathfrak{q}_i$ is a prime ideal, since
$(A/\mathfrak{p}_i)[x] \cong A[x]/\mathfrak{q}_i$, and $A/\mathfrak{p}_i$ is a
domain whence $(A/\mathfrak{p}_{i})[x] = A[x]/\mathfrak{q}_i$ is aswell. We can
define $\mathfrak{q}_{n + 1} = (x) + \mathfrak{q}_n$ to be the ideal of
polynomials where the constant coefficient is in $\mathfrak{p}_n$. This is
prime, since it's the preimage of the prime ideal $\mathfrak{p}_n$ under the
quotient map by $(x)$. Moreover, $\mathfrak{q}_{n + 1} \supsetneq
\mathfrak{q}_n$ and we have a strictly increasing chain of length $n + 1$ in
$A[x]$. It follows that $1 + \dim A \leq \dim A[x]$. \\

We use the hint to show the second inequality. Let $\mathfrak{p}$ be a prime
ideal in $A$. By Ex 3.21.iv), we have that the spectrum of the fiber of
$\mathfrak{p}$, I.e the prime ideals in $A[x]$ which contract to
$\mathfrak{p}$, is homeomorphic to the spectrum of $k(\mathfrak{p}) \otimes_A
A(x)$ which in turn is isomorphic to $k(\mathfrak{p})[x]$, a field. Since
$k(\mathfrak{p})[x]$ has two ideals, $(0), (x)$, it follows that there are
exactly two prime ideals in $A[x]$ which contract back to $\mathfrak{p}$. It's
easy to verify that these are given by $\mathfrak{p}A[x]$ (Ex 4.7.ii)), and
$\mathfrak{p} + (x)$. \\

Any chain of strict inclusions in $A[x]$ must contract back to a chain of (not
necessarily strict) inclusions in $A$. It then follows that after removing
duplicates in the contracted chain, it can never be more than half as short
(modulo odd length chains in $A[x]$, for which we have to round up the length
after halving it), and the second inequality follows.

\subsection*{Ex 7}

Let $\mathfrak{p}$ be a prime ideal of height $m$ in $A$. The hint claims that
we can pick $a_1, a_2, \ldots, a_m \in A$ such that $\mathfrak{p}$ is a minimal
ideal of $(a_1, a_2, \ldots, a_m)$. To see why, note that
$\dim(A_{\mathfrak{p}}) = m$, so by the Dimension Theorem of Noetherian Local
Rings, we have that there is some $\mathfrak{p}^{e}$-primary ideal
$\mathfrak{q}$ in $A_{\mathfrak{p}}$ generated by $m$ elements $\mathfrak{q} =
(\overline{a}_1, \overline{a}_2, \ldots, \overline{a}_m)$. When we contract
back to $A$, this ideal is $\mathfrak{p}$-primary since inverse images of
primary ideals are primary, and by Prop 1.8 \& 3.11.iv) we have 
\begin{align*}
	r(\mathfrak{q}^{c})
	&=
	r(\mathfrak{q})^{c} \\
	&=
	\mathfrak{p}^{ec} \\
	&=
	\mathfrak{p}.
\end{align*}
Moreover, given $a_i \in f^{-1}(\overline{a}_i)$, we have that
$\mathfrak{q}^{c} = (a_1, a_2, \ldots, a_m)$ since 
\[
	a_i \in f^{-1}(\overline{a}_i) \subset f^{-1}(\mathfrak{q}) = \mathfrak{q}^{c} \subseteq \mathfrak{p},
\] 
so no element of $A - \mathfrak{p}$ is a zero-divisor in $A/(a_1, a_2, \ldots,
a_m)$ whence $(a_1, a_2, \ldots, a_m)$ is a contracted ideal. We've shown the
that the first statement proposed in the hint holds. Denote $\mathfrak{a} =
\mathfrak{q}^{c}$. \\

Continuing along with the hint, Ex 4.7.iv) tells us that $\mathfrak{p}A[x]$ is
a minimal ideal of $\mathfrak{a}A[x]$, so the height of $\mathfrak{p}A[x]$ is
$\leq m$ by Corollary 11.16 (the $a_i$ generate $\mathfrak{a}A[x]$ as well). It
follows that $\mathfrak{p}A[x]$ has height $m$. Suppose that we picked
$\mathfrak{p}$ to be a maximal ideal with maximal height in $A$. Then the only
prime ideal in $A[x]$ containing $\mathfrak{p}A[x]$ is $\mathfrak{p} + (x)$,
since we saw in Exercise 6 that this is the only other prime ideal in $A[x]$
which contracts to $\mathfrak{p}$, and $A[x]/(\mathfrak{p} + (x)) =
A/\mathfrak{p}$ is a field, so $\mathfrak{p} + (x)$ is maximal. It follows that
the first bound in Exercise 6 is tight in the Noetherian case and we are done.


\end{document}
