\documentclass{article}
\usepackage[utf8]{inputenc}

\usepackage{rotating}
\usepackage{mathtools}
\usepackage{algpseudocode}
\usepackage{amsfonts}
\usepackage{amsmath}
\usepackage{amssymb}
\usepackage{amsthm}
\usepackage{bm}
\usepackage{listings}
\usepackage{float}
\usepackage{fancyvrb}
\usepackage{xcolor}
\usepackage{tikz-cd}

\hbadness = 10000
\vbadness = 10000

\newcommand\restr[2]{{% we make the whole thing an ordinary symbol
  \left.\kern-\nulldelimiterspace % automatically resize the bar with \right
  #1 % the function
  \vphantom{\big|} % pretend it's a little taller at normal size
  \right|_{#2} % this is the delimiter
  }}

% Default fixed font does not support bold face
\DeclareFixedFont{\ttb}{T1}{txtt}{bx}{n}{12} % for bold
\DeclareFixedFont{\ttm}{T1}{txtt}{m}{n}{12}  % for normal
% Custom colors

\usepackage{color}
\definecolor{deepblue}{rgb}{0,0,0.5}
\definecolor{deepred}{rgb}{0.6,0,0}
\definecolor{deepgreen}{rgb}{0,0.5,0}

% Python style for highlighting
\newcommand\pythonstyle{\lstset{
language=Python,
basicstyle=\ttm,
morekeywords={self},              % Add keywords here
keywordstyle=\ttb\color{deepblue},
emph={MyClass,__init__},          % Custom highlighting
emphstyle=\ttb\color{deepred},    % Custom highlighting style
stringstyle=\color{deepgreen},
frame=tb,                         % Any extra options here
showstringspaces=false
}}

\lstnewenvironment{python}[1][]
{
\pythonstyle
\lstset{#1}
}
{}

\theoremstyle{definition}

\newtheorem{theorem}{Theorem}[section]
\newtheorem{definition}[theorem]{Definition}
\newtheorem{corollary}[theorem]{Corollary}
\newtheorem{lemma}[theorem]{Lemma}

\newcommand{\Z}{\mathbb{Z}}
\newcommand{\Q}{\mathbb{Q}}
\newcommand{\R}{\mathbb{R}}
\newcommand{\C}{\mathbb{C}}
\newcommand{\K}{\mathbb{K}}
\renewcommand{\P}{\mathbb{P}}
\newcommand{\F}{\mathbb{F}}
\newcommand{\N}{\mathbb{N}}
\newcommand{\A}{\mathbb{A}}
\newcommand{\D}{\mathbb{D}}


\newcommand{\x}{\bm{x}}
\newcommand{\Kx}{\K[\bm{x}]}
\newcommand{\KP}[2]{\K[#1_1, #1_2, \ldots, #1_{#2}]}

\newcommand{\oo}{\mathcal{O}}
\newcommand{\osp}[1]{\oo_{\Spec\left(#1\right)}}
\newcommand{\rospu}[2]{\restr{\osp{#1}}{#2}}
\newcommand{\oop}[2]{\oo_{\P^{#1}_{#2}}}
\newcommand{\ox}{\mathcal{O}_X}

\renewcommand{\AA}[1]{\A^{#1}}
\newcommand{\An}{\A^n}
\newcommand{\Am}{\A^m}

\newcommand{\PP}[1]{\P^{#1}}
\newcommand{\Pn}{\P^n}
\newcommand{\Pm}{\P^m}

\newcommand{\Hom}{\text{Hom}}
\newcommand{\Aut}{\text{Aut}}
\newcommand{\End}{\text{End}}
\newcommand{\Iso}{\text{Iso}}
\newcommand{\Mor}{\text{Mor}}

\newcommand{\lm}{\text{lm}}
\newcommand{\nr}{\text{nilrad}}
\newcommand{\Spec}{\text{Spec}}
\newcommand{\Proj}{\text{Proj}}
\newcommand{\proj}{\Proj}
\newcommand{\spec}{\Spec}
\newcommand{\codim}{\text{codim}}
\newcommand{\ann}{\text{ann}}
\newcommand{\im}{\text{im}}
\newcommand{\id}{\text{id}}
\newcommand{\height}{\text{height}}

\newcommand{\catname}[1]{{\normalfont\textbf{#1}}}
\newcommand{\Set}{\catname{Set}}
\newcommand{\CRing}{\catname{CRing}}
\newcommand{\Top}{\catname{Top}}
\newcommand{\op}{\catname{op}}

\setlength{\parindent}{0pt}




\begin{document}

\section*{Ch 1}

\subsection*{Ex 1.1.1}

We plan to use Proposition B.19. Let $g_{r}(\zeta) = f(\zeta, re^{i\theta})$ for
$r \in [0, 1)$. Then let $r_j \in [0, 1)$ be some sequence which converges to
$1$. Each $g_{r_j}$ is holomorphic on $\D$, as $f$ is holomorphic on $\D^{2}$
and $r_j e^{i \theta} \in \D$. The $g_{r_j}$ converge towards $g_1$, and we
need to show that the $g_{r_j}$ (or some subsequence) converge uniformly on
compact subsets of $\D$, after which an application of Proposition B.19 tells
us that $g_1$ is holomorphic. \\

We will use Theorem B.20 for this. As $f$ is continuous on the compact set
$\overline{\D^{2}}$, it is bounded, and attains some maximal modulus $M$. It
follows that the $g_{r_j}$ are uniformly bounded on compact subsets of $\D$,
and Theorem B.20 gives us some some subsequence of $g_{r_j}$ which converges
uniformly on compact subsets of $\D$. By the previous paragraph, it follows
that $g_1$ is holomorphic, and $g_2$ as well by analogous arguments.

\subsection*{Ex 1.1.2}

As $f$ is holomorphic in the first variable on $\Delta_1$, and continuous in
the first variable on $\overline{\Delta}_1$, it follows from the univariate
Cauchy integral formula that
\[
	f(z)
	=
	\frac{1}{2\pi i}
	\int_{\partial \Delta_1}
	\frac{f(\zeta_1, z_2, z_3, \ldots, z_n)}{z_1 - \zeta_1} d\zeta_1.
\] 
But then for a fix $\zeta_1 \in \partial \Delta_1$, we showed above that
$f(\zeta_1, z_2, z_3, \ldots, z_n)$ is holomorphic in the second variable by
Exercise 1.1.1. We can apply Cauchy's integral formula again, and get
\[
	f(z)
	=
	\frac{1}{(2\pi i)^2}
	\int_{\partial \Delta_1}
	\int_{\partial \Delta_2}
	\frac{f(\zeta_1, \zeta_2, z_3, \ldots, z_n)}{(\zeta_1 - z_1)(\zeta_2 - z_2)} d\zeta_1 d\zeta_2.
\]
Continuing this way, we see that we eventually get 
\[
	f(z)
	=
	\frac{1}{(2\pi i)^n}
	\int_{\Gamma}
	\frac{f(\zeta_1, \zeta_2, \ldots, \zeta_n)}{(\zeta_1 - z_1)(\zeta_2 - z_2) \ldots (\zeta_n - z_n)}
	d\zeta_1 \wedge d\zeta_2 \wedge \ldots \wedge d\zeta_n.
\]

\subsection*{Ex 1.1.3}

As $\Gamma$ is compact and $f$ is continuous on $\Gamma$, $f$ attains its
maximal modulus $M$ for some $z_m \in \Gamma$. It now follows from the Cauchy
integral formula that
\begin{align*}
	f(z)
	&=
	\frac{1}{(2\pi i)^n}
	\int_{\Gamma}
	\frac{f(\zeta_1, \zeta_2, \ldots, \zeta_n)}{(\zeta_1 - z_1)(\zeta_2 - z_2) \ldots (\zeta_n - z_n)}
	d\zeta_1 \wedge d\zeta_2 \wedge \ldots \wedge d\zeta_n \\
	& \leq
	\frac{1}{(2\pi i)^n}
	\int_{\Gamma}
	\frac{|f(\zeta_1, \zeta_2, \ldots, \zeta_n)|}{(\zeta_1 - z_1)(\zeta_2 - z_2) \ldots (\zeta_n - z_n)}
	d\zeta_1 \wedge d\zeta_2 \wedge \ldots \wedge d\zeta_n \\
	& \leq
	\frac{1}{(2\pi i)^n}
	\int_{\Gamma}
	\frac{M}{(\zeta_1 - z_1)(\zeta_2 - z_2) \ldots (\zeta_n - z_n)}
	d\zeta_1 \wedge d\zeta_2 \wedge \ldots \wedge d\zeta_n \\
	&=
	M
	\frac{1}{(2\pi i)^n}
	\int_{\Gamma}
	\frac{1}{(\zeta_1 - z_1)(\zeta_2 - z_2) \ldots (\zeta_n - z_n)}
	d\zeta_1 \wedge d\zeta_2 \wedge \ldots \wedge d\zeta_n \\
	&=
	M,
\end{align*}
where
\[
	1
	=
	\frac{1}{(2\pi i)^n}
	\int_{\Gamma}
	\frac{1}{(\zeta_1 - z_1)(\zeta_2 - z_2) \ldots (\zeta_n - z_n)}
	d\zeta_1 \wedge d\zeta_2 \wedge \ldots \wedge d\zeta_n \\
\] 
since the RHS is the expansion of the constant function $(z_1, \ldots, z_n)
\mapsto 1$ via the Cauchy integral formula.

\subsection*{Ex 1.1.4}

Let $f_p(\bm{z}) =  \bm{z} + p$. Then $f_p$ is holomorphic everywhere. By
considering the $f_p$ as a function on real space $\R^{2n} \to \R^2$, we see
that the image of $f_p$ is a $2n$ sphere centered about $p$, which has a
strictly furtherest point $2p$ from the origin. Hence the function attains a
strict maximum at $p$ for values in the unit ball.


\subsection*{Ex 1.1.5}

Fix $y$ and let $f_a(x) = \frac{ax}{x^2 + a^2}$. Then if $a \not = 0$, we have
$f_a$ differentiable everywhere with derivative
\[
	f_a'(x) 
	= 
	\frac{a(x^2 + a^2) - 2ax^2}{(x^2 + a^2)^2}
	= 
	\frac{a^3 - ax^2}{(x^2 + a^2)^2}.
\]
If $a = 0$, then $f_a = 0$, and is infinitely smooth everywhere on $\R$. The
same story holds when we fix $x$ and partial derivatives in $y$. Thus we've
shown that partial derivatives of $f$ exist everywhere. Moreover, $f$ is
locally bounded as $|xy| < |x^2 + y^2|$ for all $x, y \in \R^2$, hence $|f|
\leq 1$.

\subsection*{Ex 1.1.6}

First suppose that $f$ is locally bounded and that $\zeta \mapsto f(\zeta a +
b)$ is holomorphic for every $a,b \in \C^n$ on the set $W$ such that $\zeta \in
W$ whenever $\zeta a + b \in U$. Then picking $a = e_k^{c} = (c_1, c_2, \ldots,
c_{k - 1}, 1, c_{k + 2}, \ldots, c_n)$ and $b = 0$ yields that $f$ is
holomorphic in the $k$-th coordinate when we fix the remaining coordinates to
some $c \in \pi_{k^*}(U)$ (where $\pi_{k^{*}}$ is the projection onto all but
the $k$-th coordinate). Running over all possible choices of $c$ shows that
$f$ is holomorphic in the $k$-th variable on all points of $U$, then running
through all $k$s, shows that $f$ is holomorphic in every variable on $U$.
Combining this with the fact that $f$ is locally bounded yields that $f$ is
holomorphic on $U$. \\

Now suppose that $f$ is holomorphic. Then $f$ is locally bounded by definition.
Moreover, $f \circ (\zeta \mapsto a\zeta + b)$ is clearly continuous and differentiable
as it is a composition of such functions, and the Wirtinger equation is given by
\begin{align*}
	\frac{\partial}{\partial \overline{z}} f (a\zeta + b)
	&=
	\frac{\partial}{\partial x} 
	f (a\zeta + b) 
	+
	i
	\frac{\partial}{\partial x} 
	f (a\zeta + b) \\
	&=
	\sum_{k = 1}^{n}
	f_{x_k} (a\zeta + b) 
	a
	+
	i f_{y_k} (a\zeta + b)
	a \\
	&=
	\sum_{k = 1}^{n}
	\restr{a_k \frac{\partial}{\partial \overline{z}} f}{a\zeta + b} \\
	&=
	0.
\end{align*}


\subsection*{Ex 1.2.5}

First, let's prove the following lemma which allows us to restrct checking normal convergence 
to checking uniform convergence on polydiscs.

\begin{lemma}
	Let $D$ be a domain. Then a sequence of functions $f_n$ is normally
	convergent on $D$ if it converges uniformly on all closed polydiscs
	$\overline{\Delta} \subset D$.
\end{lemma}
\begin{proof}
	Let $K \subset D$ be a compact subset. Then there is some neighbourhood $U
	\subset D$ containing $D$. Write $D$ as a union of open polydiscs. Now pick
	a finite subset of the polydiscs which cover $K$. As $K$ is closed, we may
	pick these polydiscs to be closed as well (either by adding the boundary,
	or by shrinking them). Denote this cover of $K$ by closed polydiscs as
	$\overline{\Delta}_n, n \in [1..N]$\\

	Now, let $\epsilon > 0$. As the $f_n$ converge uniformly on each
	$\overline{\Delta}_n, n \in [1..N]$, to say $g_n$, we can pick $N_n$ for
	each $n \in N$ such that $f_{n} - g_n < \epsilon$ for all $n > N_n$. Let $M
	= \max_{N_n}$. Then $f_n - g < \epsilon$ for all $n > M$ where $g = g_n$ on
	each $\overline{\Delta}_n$, and so $f_n$ converges uniformly on $K$.
\end{proof}

We are now ready to prove the theorem.

\begin{proof}
	Let $\overline{\Delta}$ be a polydisc in $K$. Then 
	\[
		\left|\frac{d f}{d z_k}(z)\right|
		\leq
		\frac{1}{\rho_k} \|f\|_{\Gamma}
	\]
	for all $z \in \overline{\Delta}$. Uniform convergence is equivalent to
	uniform cauchy convergense, and so if we let $F_n = f_{n + 1} - f_n$, we
	see that uniform cauchy convergence of the $f_n$ imply that $F_n
	\rightarrow 0$ uniformly. It follows that 
	\[
		\left|\frac{d f_{n + 1}}{d z_k}(z)
		-
		\frac{d f_n}{d z_k}(z)\right|
		\leq
		\frac{1}{\rho_k} \|f_{n + 1} - f_n\|_{\Gamma}
		\rightarrow
		0
	\]
	uniformly, hence the derivatives are uniformly cauchy convergent,
	and so uniformly convergent. \\

	It now follows that the derivatives of the $f_n$ converge to the 
	derivative of $f$, as in Exercise 1.2.3, and so $f$ is holomorphic
	just as in Exercise 1.2.3. \\

	We may now repeat the procedure, taking derivatives of the sequence of
	derivatives, as many times as we need, to obtain the desired result.
\end{proof}

\subsection*{Ex 1.2.13}

\begin{proof}
	Pick two points, $a, b \in \C^{n}$ and let $c = f(a)$. Consider the
	function which fixes the last $n$ coordinates of $f$ to $a$, namely $z_1
	\mapsto f(z_1, a_2, \ldots, a_n)$. This is a bounded holomorphic function
	$\C \to \C$, hence constant by Liouville's Theorem in one variable, and so
	$f(b_1, a_2, \ldots, a_n) = c$. Now consider the function, $z_2 \mapsto
	f(b_1, z_2, a_3, \ldots, a_n)$. Again, this is a constant function and so
	$f(b_1,b_2,a_3,\ldots,a_n) = c$. Continuing this way we see that $f(a) =
	f(b)$. We've shown that $f$ is constant on two arbitrary points in
	$\C^{n}$, whence $f$ is constant.
\end{proof}

\subsection*{Ex 1.2.15}

\begin{proof}
	First of, it's enough to prove uniform convergence on polydiscs, since any
	compact set can be covered by finitely many polydiscs. \\

	$\C^{n}$ is separable as it's homeomorphic to a countable product of $\R$.
	Thus, by Arzela-Ascoli, it will suffice to show that the sequence is
	equicontinuous. \\

	Let $M$ be a bound on the $f_j$. Then for any $f = f_j$ and $a, b$ in some
	polydisc $\Delta \subset U$, we have
	\begin{align*}
		\left|
		f(a) - f(b)
		\right|
		&=
		\left|
		\frac{1}{(2\pi i)^{n}}
		\int_{\Gamma}
		\frac{f(\zeta)}{\zeta - a}
		d\zeta
		-
		\frac{1}{(2\pi i)^{n}}
		\int_{\Gamma}
		\frac{f(\zeta)}{\zeta - b}
		d\zeta
		\right| \\
		&=
		\left|
		\frac{1}{(2\pi i)^{n}}
		\int_{\Gamma}
		\frac{f(\zeta)(\zeta - b) + f(\zeta)(\zeta - a)}{(\zeta - a)(\zeta - b)}
		d\zeta
		\right| \\
		&=
		\left|
		\frac{1}{(2\pi i)^{n}}
		\int_{\Gamma}
		\frac{f(\zeta)(b - a)}{(\zeta - a)(\zeta - b)}
		d\zeta
		\right| \\
		&\leq
		\left|
		\frac{M(b - a)}{(2\pi i)^{n}}
		\int_{\Gamma}
		\frac{1}{(\zeta - a)(\zeta - b)}
		d\zeta
		\right| \\
		&\leq
		\left|
		\frac{M(b - a)}{\rho_1\rho_2\ldots \rho_n}
		\right|.
	\end{align*} 
	We can pick a neighbourhood $U$ about $a$ which is small enough such that
	for all $b \in U$, we have
	\[
		\left|
		\frac{M(b - a)}{\rho_1\rho_2\ldots \rho_n}
		\right|
		\leq \epsilon
	\] 
	for any $\epsilon > 0$, and so the sequence is equicontinuous.
\end{proof}

\subsection*{Ex 1.2.15}

Suppose that $U$ is disconnected, and that $V$ is a connected component of $U$.
Then we can pick two holomorphic non-zero functions, one which is non-zero on
$V$, and zero on $U \setminus V$, and one which is zero on $V$ and non-zero on
$U \setminus V$, and their product is the zero function. \\

Now suppose that $U$ is connected, and that $f,g \in \mathcal{O}(U)$ are such
that $fg = 0$, or in other words, $V(f) \cup V(g) = U$. If $f \not = 0$, then
the vanishing set of $f$, $V(f) = \{p \in U : f(p) = 0\}$ does not contain any
open set by the uniqueness theorem. It follows that every single open subset of
$U$ intersects $V(g)$, since otherwise it would be contained in $V(f)$ per
$V(f) \cup V(g) = U$. Thus $V(g)$ is dense in $U$. But $V(g)$ is also closed as
$g$ is continuous, hence $V(g) = U$ and $g = 0$.

\subsection*{Ex 1.4.3}

Suppose that $f(\partial U) \subset \partial V$. Then let $\{p_k\}$ be a
sequence of points in $U$ converging to $p \in \partial U$. As $f$ is
continuous, the points $g(p_k) = f(p_k)$ converge to $f(p)$ which lies in
$\partial V$ by hypothesis. Hence $g$ is proper by Lemma 1.4.7. \\

Suppose now that $g$ is proper, and let $p \in \partial U$. Let $\{p_k\}$ be a
sequence of points in $U$ converging to $p$. As $f$ is continuous, the points
$g(p_k) = f(p_k)$ converge to $f(p)$, and as $g$ is proper, $f(p) \in \partial
V$. As $p$ was an arbitrary point in $\partial U$, it follows that $f(\partial
U) \subset \partial V$

\subsection*{Ex 1.4.7}

Suppose that $f : U \to \D$ is proper. We can assume that $f$ has a zero in
$U$, as otherwise we just subtract $f$ by some constant. As $\{0\}$ is a
compact set in $\D$, we have $f^{-1}(0)$ compact. Thus, the following theorem
will allow us to conclude that $n = 1$ is compact.

\begin{theorem}
	Let $f : U \to \C$ be a holomorphic map, $U \subset \C^{n}$ with $n > 1$.
	Then $f^{-1}(0)$ is either empty or non-compact.
\end{theorem}
\begin{proof}
	Suppose that $a \in f^{-1}(0)$. Suppose towards a contradiction that
	$f^{-1}(0)$ non-empty and bounded by $M \in \R$. Let 
	$F : \R^{2n} \to \R$ be given by $|f|$. Then $F$ i

	We will show that $f^{-1}(0)$ is unbounded.
	Suppose that $$	
\end{proof}

\subsection*{Ex 1.4.8}





\end{document}
